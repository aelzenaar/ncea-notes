\documentclass[a4paper,leqno]{article}
\usepackage[utf8]{inputenc}
\usepackage{lmodern}
\usepackage{microtype}
\usepackage[inline]{enumitem}

\usepackage{siunitx}
\usepackage{multirow}
\usepackage{subcaption}

\usepackage[english]{babel}
\usepackage[autostyle, english=british]{csquotes}
\MakeOuterQuote{"}

\usepackage{commath}
\usepackage{amsmath}
\usepackage{amsthm}
\usepackage{amssymb}

\usepackage{pgfplots}
\pgfplotsset{compat=1.11}
\usepgfplotslibrary{fillbetween}
\usetikzlibrary{patterns}

\usepackage{hyperref}

\usepackage[margin=1in]{geometry}
\usepackage{changepage}
\usepackage{titlesec}
\titleformat{\section}{\normalfont\Large\bfseries\centering}{Section~\thesection:}{1em}{}

\def\signed #1{{\leavevmode\unskip\nobreak\hfil\penalty50\hskip2em
  \hbox{}\nobreak\hfil(#1)%
  \parfillskip=0pt \finalhyphendemerits=0 \endgraf}}
\newsavebox\mybox
\newenvironment{aquote}[1]
  {\savebox\mybox{#1}\begin{quote}}
  {\signed{\usebox\mybox}\end{quote}}

% Augmented matrices.
\makeatletter
\renewcommand*\env@matrix[1][*\c@MaxMatrixCols c]{%
  \hskip -\arraycolsep
  \let\@ifnextchar\new@ifnextchar
  \array{#1}}
\makeatother

%--------grstep
% For denoting a Gauss' reduction step.
% Use as: \grstep{\rho_1+\rho_3} or \grstep[2\rho_5 \\ 3\rho_6]{\rho_1+\rho_3}
\newcommand{\grstep}[2][\relax]{%
   \ensuremath{\mathrel{
       {\mathop{\longrightarrow}\limits^{#2\mathstrut}_{
                                     \begin{subarray}{l} #1 \end{subarray}}}}}}
\newcommand{\swap}{\leftrightarrow}


\swapnumbers
\numberwithin{equation}{section}
\newtheorem{thm}[equation]{Theorem}
\newtheorem{lem}[equation]{Lemma}
\newtheorem{cor}[equation]{Corollary}
\newtheorem{prp}[equation]{Proposition}
\theoremstyle{definition}
\newtheorem{defn}[equation]{Definition}
\newtheorem{ex}[equation]{Example}
\newtheorem{exercise}[equation]{Exercise}
\newtheorem{alg}[equation]{Algorithm}
\theoremstyle{remark}
\newtheorem{rem}[equation]{Remark}

\newcommand{\df}[1]{\textbf{#1}}
\newcommand{\T}{\mathrm{T}}
\newcommand{\F}{\mathrm{F}}
\newcommand{\IndSet}{\mathbf{I}}


\title{Down the River: Foundations for Scholarship}
\author{Alex Elzenaar}
\date{\today}

\begin{document}
\maketitle
\tableofcontents
\section*{Preface}
This set of notes is an introduction to the shallower foundations of mathematics, for those students who
are attempting the New Zealand Scholarship examination in Calculus.

I assume knowledge of the breadth of Level 2, but not any material from Level 3. As a guideline, I would
expect any student attempting Scholarship to come into Level 3 with the following skills:
\begin{itemize}
  \item Comfort with basic algebraic manipulations.
  \item Understanding of exponential and logarithmic functions.
  \item Understanding of quadratic equations (especially with the link between parabolae and quadratics, and factoring).
  \item Knowledge of the links between functions and their graphs (in general; particularly, being able to give some kind
        of definition and knowledge of some basic examples).
  \item Comfort with the geometry underlying the derivative (slopes of lines, average slopes of curves, and the ability
        to apply the derivative in geometric situations like finding turning points). [This skill is not explicitly required for these notes.]
\end{itemize}

\titleformat{\section}{\clearpage\titlerule[0.8pt]\vspace{0.5ex}\normalfont\Large\bfseries\centering}{Section~\thesection:}{1em}{}[{\titlerule[0.8pt]}]
\let\oldsection\section
\renewcommand\section{\clearpage\oldsection}
\section{Set Notation}
\subsection{Defining Sets}
We begin by quoting from Paul Halmos' classic book, \textit{Naive Set Theory}.
\begin{aquote}{Halmos}
  A pack of wolves, a bunch of grapes, or a flock of pigeons are all examples of sets of things. The mathematical concept of a set can
  be used as the foundation for all known mathematics...
\end{aquote}
Like Halmos, we will avoid an exact definition of sets so that we do not need to deal with the logical issues that come
with it (does the set of all sets that do not contain themselves contain itself?) --- for us, a set is simply a collection
of objects.

We can write down a set by placing the names of the objects within the set (the \df{elements} of the set) inside curly brackets; for example,
\begin{equation}
  S = \{1, 2, 3\}
\end{equation}
is a set consisting of three elements. If $ x $ is an element of a set $ S $, then we write $ x \in S $ ($ x $ is in $ S $); otherwise,
we can write $ x \notin S $.

There are several `standard' sets:
\begin{itemize}
  \item $ \mathbb{N} = \{1, 2, 3, ...\} $ is the set of natural numbers.
  \item $ \mathbb{N}_0 = \{0, 1, 2, ...\} $ is the set of natural numbers together with zero.
  \item $ \mathbb{Z} = \{..., -2, -1, 0, 1, 2, ...\} $ is the set of integers.\footnote{The letter $ Z $ comes from the German \textit{Zahlen} (number).}
  \item $ \mathbb{Q} $ is the set of rational numbers (that is, all numbers $ x $ which can be written in the form $ a/b $ for $ a $ and $ b $ integers).
  \item $ \mathbb{R} $ is the set of real numbers (that is, all possible lengths that can be measured along an infinite line\footnote{There are more rigorous
        definitions of the real numbers than this, but we do not need to worry too much about defining them properly.}).
  \item $ \emptyset $ is the set containing no elements.
\end{itemize}

If we construct a set using a rule, then we have another notation: if $ E $ is the set containing all even numbers, then we can write
\begin{equation}
  E = \{ n : n\text{ is an even number}\}.
\end{equation}
This notation is called \df{set-builder notation}; if $ S = \{ x \in \mathcal{U} : P(x) \text{ is true}\} $, then we read it
as `$ S $ is the set of all $ x $ in $ \mathcal{U} $ such that $ P(x) $ is true'.

If $ S $ and $ T $ are sets such that every element of $ T $ is also an element of $ S $, then we say that $ T $ is a \df{subset} of $ S $ (or
that $ S $ \df{includes} $ T $), and write $ T \subseteq S $. In particular, we have $ \mathbb{N} \subseteq \mathbb{Z} \subseteq \mathbb{Q} \subseteq \mathbb{R} $.
\begin{exercise}
  Let $ X $ be some set. Prove that $ \emptyset \subseteq X $, and that $ X \subseteq X $.
\end{exercise}

\subsection{Set Operations}
If $ S $ and $ T $ are two sets, then their \df{intersection} is the set
\begin{equation}
  S \cap T = \{ x : x \in S \text{ and } x \in T \}
\end{equation}
and their \df{union} is the set
\begin{equation}
  S \cup T = \{ x : x \in S \text{ or } x \in T \}.
\end{equation}

\begin{ex}
  If $ S = \{2, 4, 6, 8, 10, 12\} $ and $ T = \{1, 4, 9, 16\} $, then $ S \cup T = \{1,2,3,6,8,9,10,12,16\} $ and $ S \cap T = \{4\} $.
\end{ex}

\begin{exercise}[Distributivity]
  Show that, if $ R $, $ S $, and $ T $ are sets, then
  \begin{enumerate}
    \item $ R \cup (S \cap T) = (R \cup S) \cap (R \cup T) $, and
    \item $ R \cap (S \cup T) = (R \cap S) \cup (R \cap T) $.
  \end{enumerate}
  Draw Venn diagrams to show each configuration.
\end{exercise}

Now, suppose $ S $ is a subset of $ \mathcal{U} $. The \df{compliment} of $ S $ in $ \mathcal{U} $ is the set of
all elements of $ \mathcal{U} $ that are not in $ S $; in other words,
\begin{equation}
  \mathcal{U} - S = \{ x \in \mathcal{U} : c \notin S\}.
\end{equation}
If the set $ \mathcal{U} $ is clear from context, we can just write $ S^C $ or $ \overline{S} $ in place of $ \mathcal{U} - S $.

\begin{exercise}[de Morgan's laws]
  Show that, if $ R $ and $ S $ are subsets of $ \mathcal{U} $, then
  \begin{enumerate}
    \item $ \overline{R \cup S} = \overline{R} \cap \overline{S} $, and
    \item $ \overline{R \cap S} = \overline{R} \cup \overline{S} $.
  \end{enumerate}
  Draw Venn diagrams to show each configuration.
\end{exercise}

\subsection{Functions}
We also define the \df{Cartesian product} of $ S $ and $ T $ by
\begin{equation}
  S \times T = \{(s,t) : s \in S, t \in T\}
\end{equation}
where $ (s,t) $ denotes the ordered pair with $ s $ in the first position and $ t $ in the second; i.e. $ (s,t) = (t,s) $ if and only if $ s = t $.

A \df{function} $ f $ with \df{domain} $ D $ and \df{codomain} $ C $ (or a function \df{from} $ D $ \df{(in)to} $ C $) is a subset
of $ D \times T $ such that if $ (x,y_1) \in f $ and $ (x, y_2) \in f $ then $ y_1 = y_2 $. When defining such a function, we can
write that $ f : D \to C $. If $ (x,y) \in f $, we usually write $ f(x) = y $.

The set $ R = \{r \in C : f(x) = r \text{ for some } x \in D\} $ is called the \df{range} of $ f $. If $ R = C $ (that is, for every value $ y $
in the codomain there is some $ x $ in the domain so that $ f(x) = y $), then the function is called \df{surjective} or \df{onto}.

If $ f(x_1) = y $ and $ f(x_2) = y $ implies that $ x_1 = x_2 $ (that is, if every member $ y $ of the range has precisely one corresponding
element $ x $ of the domain so that $ f(x) = y $) then the function is called \df{injective} or \df{one-to-one}.

We say that $ f : A \to B $ and $ g : C \to D $ are \df{equal}, writing $ f = g $, if and only if both of the following conditions are met:
\begin{enumerate}
  \item $ A = C $ and $ B = D $; and
  \item For every $ x \in A $, $ f(x) = g(x) $.
\end{enumerate}

\begin{ex}\leavevmode
  \begin{enumerate}
    \item If $ f : \mathbb{R} \to \mathbb{R} $ is defined by $ f(x) = 3x + 1 $, then $ f $ is one-to-one and onto.
    \item If $ g : \mathbb{N} \to \mathbb{N} $ is defined by $ f(x) = x + 1 $, then $ f $ is one-to-one but not onto, since there is no $ x \in \mathbb{N} $
          such that $ 1 = x + 1 $.
  \end{enumerate}
\end{ex}

\begin{thm}\footnote{\df{Theorem}: A statement which has been proved to be true.}
  Let $ f : D \to C $ be a function, and let $ R \subseteq C $ be the range of $ f $. Then $ f $ is one-to-one and onto if and only if there exists a
  function $ g : C \to D $ so that for every $ x \in D $, $ x = g(f(x)) $ and for every $ y \in C $, $ y = f(g(y)) $. This function $ g $ is unique.
\end{thm}
\begin{proof}\footnote{\df{Proof}: A valid argument, from true premises, to obtain a conclusion. The symbol $ \qed $ denotes the end of a proof.}
  Suppose $ f $ is one-to-one and onto. Since $ f $ is one-to-one, we can define a new function $ g $ so that $ g(y) = x $ if and only if $ f(x) = y $:
  since if $ g(y) = x_1 $ and $ g(y) = x_2 $, we have $ f(x_1) = y = f(x_2) $ and $ x_1 = x_2 $ by the injectivity of $ f $. This new function $ g $
  has domain $ R = C $ (since $ f $ is onto, the two sets are equal) and range $ D $. In particular, if $ x \in D $, $ f(x) = y $ for some $ y \in R $
  and hence $ x = g(y) = g(f(x)) $. Similarly, if $ y \in R $, then there exists $ x \in D $ such that $ f(x) = y $ and so $ g(y) = x $; hence $ f(g(y)) = y $.

  Conversely, suppose such a $ g $ exists. Suppose $ f(x_1) = f(x_2) = y $ for some $ x_1, x_2 \in D $ and $ y \in R $. Then $ g(f(x_1)) = g(y) $, so $ g(y) = x_1 $.
  Similarly, $ g(f(x_2)) = g(y) $ so $ g(y) = x_2 $. Hence $ x_1 = x_2 $ and $ f $ is injective. Let $ y \in C $; then $ y = f(g(y)) $, and so there
  exists $ x = g(y) $ so that $ f(x) = y $; hence $ y \in R $, so $ C = R $ and $ f $ is onto.

  Suppose for uniqueness that both $ g_1 $ and $ g_2 $ satisfy the requirements of the theorem. Then for every $ y \in R $,  $ y = f(g_1(y)) = f(g_2(y)) $. But
  since $ f $ is injective, $ g_1(y) = g_2(y) $; and since $ g_1 $ and $ g_2 $ have the same domain and codomain, and agree at every point, they are equal.
\end{proof}

The unique function $ g $ defined as in the theorem for some function $ f $ is called the \df{inverse} of $ f $, and we usually write $ g = f^{-1} $.

\begin{exercise}
  Show that, if $ f^{-1} $ is the inverse of $ f $, then $ f^{-1} $ has an inverse and that inverse is $ f $.
\end{exercise}

\section{Divisibility}
Let $ a $ and $ b $ be integers. We say that $ a $ \df{divides} $ b $ (or $ a $ is a \df{factor} of $ b $) if there is some integer $ q $ such
that $ b = aq $; we write $ a \mid b $.
\begin{lem}\footnote{\df{Lemma}: A statement which is proved on the way to a more important result. (Plural: lemmata.)}\leavevmode
  Let $ a $, $ b $, and $ c $ be integers. Then:
  \begin{enumerate}
    \item $ a \mid 0 $.
    \item If $ a \mid b $, and $ b \mid c $, then $ a \mid c $.
    \item If $ a \mid b $, and $ a \mid c $, then $ a \mid (b + c) $.
    \item If $ a \mid b $, then $ a \mid bc $.
%     \item If $ a \mid b $, and $ b \mid a $, then $ a = \pm b $.
  \end{enumerate}
\end{lem}
\begin{proof}\leavevmode
  \begin{enumerate}
    \item We must find an integer $ q $ so that $ 0 = qa $. Let us simply take $ q = 0 $; then $ 0 = 0a $ for all $ a $, so $ a \mid 0 $.
    \item Because $ a \mid b $, there exists some integer $ q $ so that $ b = qa $. Similarly, there exists some integer $ r $ so that $ c = rb $.
          Putting these together, $ c = rb = r(qa) = (rq)a $; so $ a \mid c $.
    \item Because $ a \mid b $, there exists some integer $ q $ so that $ b = qa $. Similarly, there exists some integer $ r $ so that $ c = ra $.
          Adding these, we have that $ b + c = qa + ra = (q + r)a $, and so $ a \mid (b+c) $.
    \item Because $ a \mid b $, there exists some integer $ q $ so that $ b = qa $. Hence $ bc = c(qa) = (cq)a $ and so $ a \mid bc $.
%     \item We have $ q $ and $ r $ so that $ b = qa $ and $ a = rb $. Hence $ b = qrb $, and so $ qr = 1 $. Then $ q = \pm 1 $, so $ b = \pm a $.
  \end{enumerate}
\end{proof}

From this point, lower case italic letters will always denote integers unless otherwise stated.

\begin{cor}\footnote{\df{Corollary}: A proof which is a direct result of an earlier proof.}
  If $ d \mid a_1 $, $ d \mid a_2 $, ..., $ d \mid a_n $, then
  \begin{displaymath}
    d \mid (c_1 a_1 + c_2 a_2 + \cdots c_n a_n)
  \end{displaymath}
  for any integers $ c_1 $, ..., $ c_n $.
\end{cor}
\begin{proof}
  By the definition of divisibility, there exist integers $ q_1 $, ..., $ q_n $ such that $ a_1 = q_1 d$, ..., $ a_n = q_n d $. Hence
  \begin{displaymath}
    c_1 a_1 + \cdots + c_n a_n = c_1 q_1 d + \cdots + c_n q_n d = d(c_1 q_1 + \cdots + c_n q_n)
  \end{displaymath}
  and so the result holds.
\end{proof}

\begin{exercise}
  Show that if $ a \mid b $, and $ a $ and $ b $ are positive, then $ a \leq b $. Show that this result
  does not always hold if $ a $ and $ b $ are non-negative.
\end{exercise}

\begin{exercise}
  If $ n $ is an integer, define the \df{absolute value} of $ n $ by
  \begin{displaymath}
    \abs{n} = \begin{cases}
                n & \text{if } n \geq 0\\
                -n & \text{if } n < 0.
              \end{cases}
  \end{displaymath}
  Show that $ \abs{n} \mid n $ and $ n \mid \abs{n} $.
\end{exercise}

\begin{defn}
  If $ a $ and $ b $ are integers, then the integer $ d $ is called the \df{greatest common divisor} of $ a $ and $ b $ (written $ d = (a,b) $)
  if the following two conditions are met:
  \begin{enumerate}
    \item $ d \mid a $ and $ d \mid b $ (that is, $ d $ is a common divisor of both numbers).
    \item If $ d' \mid a $ and $ d' \mid b $, then $ d' \leq d $ (that is, every other common divisor is less than $ d $).
  \end{enumerate}
\end{defn}

\begin{ex}\leavevmode\label{ex:gcd}
  \begin{enumerate}
    \item Since for every $ a $ and $ b $ we have $ 1 \mid a $ and $ 1 \mid b $, it follows that $ (a,b) \geq 1 $.
    \item The greatest common divisor of $ -3 $ and $ 5 $ is $ (-3, 5) = 1 $, since $ 1 \mid -3 $ and $ 1 \mid 5 $ but no integer larger than $ 1 $ divides both.
    \item We also have $ (4, 14) = 2 $ and $ (24, 36) = 12 $.
    \item For all integers $ n $, $ (n, 1) = 1 $ (since the only integers dividing 1 are $ \pm 1 $, and $ 1 > -1 $).
    \item For all integers $ n $, $ (n, 0) = \abs{n} $ (since $ \abs{n} \mid 0 $ and $ \abs{n} \mid n $, so $ (n,0) \geq \abs{n} $; and no divisors
          of $ n $ are greater than $ \abs{n} $, so $ (n, 0) \leq \abs{n} $).
  \end{enumerate}
\end{ex}

\begin{exercise}
  Let $ d $ be an integer. Prove that if $ d $ is a positive integer, then $ (d, nd) = d $; and if $ d $ is negative, then $ (d, nd) = -d $.
\end{exercise}

If $ (a,b) = 1 $, then $ a $ and $ b $ are called \df{coprime} or \df{relatively prime}.

\begin{thm}
  If $ (a,b) = d $, then $ a/d $ and $ b/d $ are coprime.
\end{thm}
This theorem is intuitive: if we divide two numbers out by their greatest common divisor, they no longer have any common divisors greater than 1. This
observation essentially proves itself.
\begin{proof}
  We must show that $ (a/d, b/d) = 1 $. Let $ (a/d, b/d) = c $. Then $ c \mid a/d $ and $ c \mid b/d $; so there exist $ q_1 $ and $ q_2 $ such
  that $ a/d = cq_1 $ and $ b/d = cq_2 $; hence $ a = cq_1 d $ and $ b = c q_2 d $. This implies that $ cd $ is a common divisor of $ a $ and $ b $,
  and so $ cd \leq d $. Since $ d = (a,b) \geq 1 $ (by example \ref{ex:gcd}.1 above), we can divide through by $ d $ and obtain $ c \leq 1 $. Applying
  the same example again, since $ c $ is itself a greatest common divisor by definition, we have $ c \geq 1 $. Combining these two inequalities, $ c = 1 $.
\end{proof}

Our next goal is to find an easy (well, easier) way to compute the greatest common divisor of two numbers. To do this, we must develop the
idea of division with remainder like we learned in primary school. The main theorem which ensures that we can divide properly in the integers
is the following
\begin{thm}[The division algorithm]
  Let $ a $ and $ b \neq 0 $ be positive integers. Then there exist \emph{unique} integers $ q $ and $ r $, so that $ 0 \leq r < b $,
  and so that $ a = qb + r $. The integers $ q $ and $ r $ are respectively called the \df{quotient} and \df{remainder}.
\end{thm}

For the proof of this, we must use the \df{well-ordering principle}:
\begin{equation}
  \begin{minipage}{0.8\textwidth}
    If $ S $ is a non-empty set of integers which is bounded below (i.e. there is some $ b \in \mathbb{Z} $ so that
    if $ n \in S $ then $ n \geq b $), then $ S $ has a smallest element.
  \end{minipage}
\end{equation}


\begin{proof}[Proof of the division algorithm]
  Consider the set $ S $ defined by
  \begin{equation}
    S = \{a - nb : n \in \mathbb{Z} \} \cap \mathbb{N}_0.
  \end{equation}
  This is just the set of all non-negative numbers which can be written in the form $ a - nb $, for some integer $ n $. Since $ a \in S $, the
  set is non-empty; and since every $ x \in S $ satisfies $ x \geq 0 $, $ x $ is bounded below by 0. Hence $ S $ has a least element, $ a - qb $.
  This element is non-negative by definition, and must be less than $ b $: if it were larger than $ b $, we could consider $ a - (q + 1)b = (a - qb) - b $,
  which is in $ S $ and less than $ a - qb $, which is impossible since $ a - qb $ is the least element in $ S $. So if we let $ r = a - qb $,
  then $ 0 \leq r < q $, and $ a = ab + r $. So suitable $ q $ and $ r $ exist.

  For uniqueness, suppose that $ q_1 $, $ r_1 $ and $ q_2 $, $ r_2 $ both satisfy the conclusion of the theorem. Hence $ a = bq_1 + r_1 = b q_2 + r_2 $,
  and so
  \begin{equation}\label{eqn:divn}
    0 = b(q_1 - q_2) + (r_1 - r_2).
  \end{equation}
  Since $ b \mid 0 $ and $ b \mid b(q_1 - q_2) $, it follows that $ b $ divides $ r_1 - r_2 $. But we have $ 0 \leq r_1 < b $ and $ 0 \geq -r_2 > -b $,
  so $ -b < r_1 - r_2 < b $. The only multiple of $ b $ in between $ -b $ and $ b $ is zero, so $ r_1 = r_2 $ and hence (by \ref{eqn:divn}) $ q_1 = q_2 $.
\end{proof}

In order to compute the greatest common divisor easily, we will create a chain of integers so that the pairs of integers next to each other
share the same GCD. The division algorithm allows us to do this, because of the following lemma.

\begin{lem}\label{lem:euclidalg}
  If $ a = bq + r $, then $ (a,b) = (b,r) $.
\end{lem}
\begin{proof}
  By the hypothesis, $ (a,b) \mid r $, and so $ (a,b) $ is a common divisor of $ b $ and $ r $. Hence $ (a,b) \leq (b,r) $. On the other hand, $ (b,r) \mid a $
  (since it divides both numbers on the right-hand side of the equality in the hypothesis) and so $ (b,r) \leq (a,b) $ (since $ (b,r) $ is a common divisor
  of $ a $ and $ b $). Combining both inequalities, $ (b,r) = (a,b) $.
\end{proof}

Finally, we have an algorithm for computing the greatest common divisor.
\begin{thm}[The Euclidean algorithm]
  If $ a $ and $ b \neq 0 $ are positive integers, and
  \begin{align*}
    a &= bq_1 + r_1,\\
    b &= r_1q_2 + r_2,\\
    r_1 &= r_2 q_3 + r_3,\\\
    &\enspace \vdots\\
    r_k &= r_{k + 1} q_{k + 2} + r_{k + 2}
  \end{align*}
  where each $ r_{k + 2} $ is bounded like $ 0 \leq r_{k + 2} < r_{k + 1} $, then for $ k $ large enough
  we have $ r_{k - 1} = r_{k} q_{k + 1} $ (i.e. $ r_{k + 1} = 0 $) and $ (a,b) = r_k $.
\end{thm}
\begin{proof}
  Since each $ r_k $ is non-negative, and each $ r_k $ is less than $ r_{k - 1} $, the sequence $ b > r_1 > \cdots > r_k $
  has a least non-negative element, $ r_k $, which must be zero or we would be able to divide again by the division algorithm.
  Then $ r_{k - 1} = r_{k} q_{k + 1} $, and (by repeated application of lemma \ref{lem:euclidalg}) we have
  \begin{displaymath}
    (a,b) = (b,r_1) = (r_1, r_2) = \cdots = (r_{k - 1}, r_{k}) = (r_{k}, 0) = r_k.
  \end{displaymath}
\end{proof}

\begin{ex}
  Let us compute $ (240, 373) $. We have:
  \begin{align*}
    373 &= 1 \times 240 + 133\\
    240 &= 1 \times 133 + 107\\
    133 &= 1 \times 107 + 26\\
    107 &= 4 \times 26 + 3\\
    26 &= 8 \times 3 + 2\\
    3 &= 1 \times 2 + 1\\
    2 &= 1 \times 1 + 1\\
    1 &= {\color{red}1} \times 1.
  \end{align*}
\end{ex}

\begin{ex}
  Let us compute $ (120, 340) $. We have:
  \begin{align*}
    340 &= 2 \times 120 + 100\\
    120 &= 1 \times 100 + 20\\
    100 &= {\color{red}5} \times 20.
  \end{align*}
\end{ex}

\begin{cor}[B\'ezout's Lemma]
  If $ a $ and $ b $ are positive integers, then there exist integers $ x $ and $ y $ such that $ ax + by = (a,b) $.
\end{cor}
\begin{proof}
  By the proof of the Euclidean algorithm, for some $ k $ we have
  \begin{align*}
    a &= bq_1 + r_1,\\
    b &= r_1q_2 + r_2,\\
    r_1 &= r_2 q_3 + r_3,\\\
    &\enspace \vdots\\
    r_{k - 2} &= r_{k - 1} q_k + r_k,\\
    r_{k - 1} &= r_k q_{k + 1},
  \end{align*}
  and $ r_k = (a,b) $. Rearranging:
  \begin{align*}
    a - bq_1 &= r_1,\\
    b - r_1q_2 &= r_2,\\
    r_1 - r_2 q_3 &= r_3,\\\
    &\enspace \vdots\\
    r_{k - 2} - r_{k - 1} q_k &= r_k = (a,b),
  \end{align*}
  and it is clear that we can substitute each equation into the one below it, eliminating every $ r_i $ until
  we (in the final line) arrive at some equation of the form $ ax + by = (a,b) $.
\end{proof}

B\'ezout's Lemma is surprisingly useful; in fact, we will use it later this year to prove various properties of a particular system
in algebra.\footnote{For those following along from the future, it will be used when we study \df{primitive roots}.}

Since we know that $ (a,b) $ divides $ ax + by $ for all $ x $ and $ y $, the lemma tells us that for every $ n $, the equation
\begin{equation}
  ax + by = n(a,b)
\end{equation}
has an integer solution for $ x $ and $ y $. In fact, it has infinitely many:
\begin{thm}
  If $ (x,y) $ solves $ ax + by = z $, then all solutions are of the form $ (x - bn, an + y) $ for some $ n \in \mathbb{Z} $.
\end{thm}

\begin{proof}
  Clearly $ a(x - bn) + b(an + y) = z $, so all expressions of that form are indeed solutions. On the other hand, suppose $ (x', y') $
  is a solution. Then $ 0 = a(x - x') + b(y - y') $; it follows that $ a \mid (y - y') $. Rearranging, we obtain $ x' = x - b\frac{(y' - y)}{a} $,
  where $ (y' - y)/a = n $ is an integer. Hence $ z = a(x - bn) + by' = ax - abn + by' = ax + b(y' - an) $, and hence $ y' = y + an $.
\end{proof}

\begin{ex}
  The equation $ 18x + 24y = 23 $ has no solutions, since $ (18,24) = 6 $ does not divide $ 23 $.
\end{ex}
\begin{ex}
  We will solve $ 18x + 24y = 36 $ for all the integer solutions $ x, y $. Since $ (18,24) = 6 $, we will begin by solving $ 18x + 24y = 6 $:
  \begin{align*}
    24 = 1 \times 18 + 6\\
    18 = 6 \times 3
  \end{align*}
  Hence $ 6 = 1 \times 24 + (-1) \times 18 $, and $ 36 = 6 \times 24 + (-6) \times 18 $. By the theorem above, all the solutions are
  given by $ (6 - 24n, -6 + 18n) $ for $ n \in \mathbb{Z} $.
\end{ex}

\section{Prime Factorisation}
\begin{defn}
  A \df{prime} is an integer
\end{defn}

\section{Next Steps}
As well as the material in the level three standards (differentiation, integration, algebra, linear systems, linear programming, path analysis,
trigonometry, and conic sections) it is advantageous for Scholarship students to be aware of mathematics as a broader discipline. In the 2017
paper, for example, the opening question part was on number theory.

As such, I have here a list of books and topics which Scholarship students may wish to look over. I have included the University of Auckland
library call number in brackets.
\begin{itemize}
  \item \textbf{Culture:} \emph{A Mathematician's Lament} by Paul Lockhart [510.71 L81].
  \item \textbf{Proofs:} \emph{How to Think Like a Mathematician} by Kevin Houston [510 H84].
  \item \textbf{Number theory:} \emph{Elementary Number Theory} by Underwood Dudley [512.72 D84].
  \item \textbf{Linear algebra:} \emph{Linear Algebra: A Modern Introduction} by David Poole [512.5 P82].
  \item \textbf{Geometry:} \emph{Geometry: A High School Course} by Serge Lang and Gene Murrow [516 L27].
  \item \textbf{Set theory:} \emph{Naive Set Theory} by Paul Halmos [511.322 H19].
  \item \textbf{Calculus:} \emph{Calculus} by Paul Spivak [515 S76].
\end{itemize}

I must stress that a comprehensive understanding of the content in the above books is \emph{not required}; in most cases, a read of the first three
sections will suffice to obtain an overview of the relevant parts of the subject (not that that precludes interested students from reading further).

\end{document}
