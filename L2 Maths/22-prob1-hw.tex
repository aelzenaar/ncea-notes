\documentclass{exam}
\usepackage[utf8]{inputenc}
\usepackage{lmodern}
\usepackage{microtype}

% \usepackage[parfill]{parskip}
\usepackage[dvipsnames]{xcolor}
\usepackage{amsmath}
\usepackage{amsfonts}
\usepackage{amsthm}
\usepackage{siunitx}
\DeclareSIUnit\year{yr}
\DeclareSIUnit\foot{ft}
\DeclareSIUnit\litre{\liter}

\usepackage{skull}

\usepackage{pgfplots}
\usepgfplotslibrary{polar}
\pgfplotsset{compat=1.11}
\usepgfplotslibrary{statistics}
\usepackage{graphicx}
\usepackage{sidecap}
\sidecaptionvpos{figure}{c}
\usepackage{float}
\usepackage{gensymb}
\usepackage{tkz-euclide}
\usetkzobj{all}
\usepackage{commath}
\usepackage{hyperref}
\usepackage{enumitem}
\usepackage{wasysym}
\usepackage{multicol}
\usepackage{mathtools}
\usepackage{tcolorbox}
\usepackage{tabularx}
\usepackage[version=4]{mhchem}
\usepackage{changepage}
\usepackage{listings}
\lstset{basicstyle=\ttfamily\linespread{0.8}\small}

\renewcommand*{\thefootnote}{\fnsymbol{footnote}}

\newtheorem*{thm}{Theorem}
\newtheorem*{iden}{Identity}
\newtheorem*{lemma}{Lemma}
\newtheorem{obs}{Observation}
\theoremstyle{definition}
\newtheorem*{defn}{Definition}
\newtheorem*{ex}{Example}
\newtheorem{con}{Construction}
\newtheorem*{alg}{Algorithm}

\newtheoremstyle{break}
  {\topsep}{\topsep}%
  {\itshape}{}%
  {\bfseries}{}%
  {\newline}{}%
\theoremstyle{break}
\newtheorem*{bthm}{Theorem}

% russian integral
\usepackage{scalerel}
\DeclareMathOperator*{\rint}{\scalerel*{\rotatebox{17}{$\!\int\!$}}{\int}}

% \DeclareMathOperator*{\rint}{\int}

\pgfplotsset{vasymptote/.style={
    before end axis/.append code={
        \draw[densely dashed] ({rel axis cs:0,0} -| {axis cs:#1,0})
        -- ({rel axis cs:0,1} -| {axis cs:#1,0});
    }
}}

% \pointsinrightmargin
\boxedpoints
\pointname{}

\newcommand{\questioA}{\question[\texttt{\textbf{\color{Cerulean} A}}]}
\newcommand{\questioM}{\question[\texttt{\textbf{\color{PineGreen} M}}]}
\newcommand{\questioE}{\question[\texttt{\textbf{\color{WildStrawberry} E}}]}
\newcommand{\questioS}{\question[\texttt{\textbf{\color{Goldenrod} S}}]}
\newcommand{\questioO}{\question[\texttt{\textbf{\color{BurntOrange} O}}]}

\newcommand{\parA}{\part[\texttt{\textbf{\color{Cerulean} A}}]}
\newcommand{\parM}{\part[\texttt{\textbf{\color{PineGreen} M}}]}
\newcommand{\parE}{\part[\texttt{\textbf{\color{WildStrawberry} E}}]}
\newcommand{\parS}{\part[\texttt{\textbf{\color{Goldenrod} S}}]}
\newcommand{\parO}{\part[\texttt{\textbf{\color{BurntOrange} O}}]}

\newcommand{\subparA}{\subpart[\texttt{\textbf{\color{Cerulean} A}}]}
\newcommand{\subparM}{\subpart[\texttt{\textbf{\color{PineGreen} M}}]}
\newcommand{\subparE}{\subpart[\texttt{\textbf{\color{WildStrawberry} E}}]}
\newcommand{\subparS}{\subpart[\texttt{\textbf{\color{Goldenrod} S}}]}
\newcommand{\subparO}{\subpart[\texttt{\textbf{\color{BurntOrange} O}}]}

\newcommand{\mainHeader}[2]{\section*{NCEA Level 2 Mathematics\\#1. #2}}
\newcommand{\mainHeaderHw}[2]{\section*{NCEA Level 2 Mathematics (Homework)\\#1. #2}}
\newcommand{\seealso}[1]{\begin{center}\emph{See also #1.}\end{center}}
\newcommand{\drills}[1]{\begin{center}\emph{Drill problems: #1.}\end{center}}
\newcommand{\basedon}[1]{\begin{center}\emph{Notes largely based on #1.}\end{center}}


\begin{document}

\mainHeaderHw{22}{Probability and Risk}
\subsection*{Reading}
\begin{center}
\begin{tcolorbox}[width=0.8\textwidth,colback={white},title={\textbf{Go and watch... (basic probability)}},colbacktitle=black,coltitle=white]
  \textcolor{black}{\url{https://www.youtube.com/watch?v=Kgudt4PXs28}}
\end{tcolorbox}
\end{center}

\begin{center}
\begin{tcolorbox}[width=0.8\textwidth,colback={white},title={\textbf{Go and watch... (relative risk)}},colbacktitle=black,coltitle=white]
  \textcolor{black}{\url{https://www.youtube.com/watch?v=felIAwyaGFM}}
\end{tcolorbox}
\end{center}


\subsection*{Questions}
\begin{questions}
  \item Watch the first video above. I have six tetrahedral dice. I roll them 1000 times. How many times should I expect to get six of the same
        number (six 1's, six 2's, six 3's, or six 4's) in a roll?
  \item Watch the second video above. Consider the following two tables, which show data from two groups of New Zealander males over the age of fifty.

        \begin{tabular}{|r|c|c|c|}\hline
          A& \textbf{Heart disease} & \textbf{None} & \textbf{Total}\\\hline
          \textbf{Overweight} &24&142&166\\\hline
          \textbf{Not overwt} &14&1706&184\\\hline
          \textbf{Total} &38&312&350\\\hline
        \end{tabular}
        \begin{tabular}{|r|c|c|c|}\hline
          B& \textbf{Heart disease} & \textbf{None} & \textbf{Total}\\\hline
          \textbf{Smoker} &12&65&77\\\hline
          \textbf{Non-smoker} &9&84&93\\\hline
          \textbf{Total} &21&149&170\\\hline
        \end{tabular}

        \begin{parts}
          \part Discuss the following statement (is it correct? why?): The \emph{absolute risk} of heart disease in
                group A was $ 38/350 = 0.109 $, while the absolute risk of heart disease in group B was much greater: $ 0.124 $. Thus,
                smoking is more problematic than obesity when it comes to risks of heart disease.
          \part \textbf{Important lesson:} it is not the \emph{risk} that matters when comparing probabilities, it is the \emph{relative risk}.
                In other words, we don't care about the probability that a person gets heart disease given that they are overweight --- we care
                about whether this probability is higher than the probability that they get heart disease given that they are not overweight.

            \begin{subparts}
              \subpart Suppose a person from group $ A $ is known to be overweight. What is the risk that they develop a heart disease?
              \subpart Suppose a person from group $ A $ is known to be not overweight. What is the risk that they develop a heart disease?
              \subpart How much more likely is a person from group $ A $ to get heart disease if they are overweight?
            \end{subparts}
          \part \emph{According to this study, a 50-year-old male is 60\% more likely to develop heart disease if they are a smoker than if
                they are a non-smoker.} Does the evidence support this statement?
        \end{parts}
\end{questions}

\end{document}
