\documentclass{exam}
\usepackage[utf8]{inputenc}
\usepackage{lmodern}
\usepackage{microtype}

% \usepackage[parfill]{parskip}
\usepackage[dvipsnames]{xcolor}
\usepackage{amsmath}
\usepackage{amsfonts}
\usepackage{amsthm}
\usepackage{siunitx}
\DeclareSIUnit\year{yr}
\DeclareSIUnit\foot{ft}
\DeclareSIUnit\litre{\liter}

\usepackage{skull}

\usepackage{pgfplots}
\usepgfplotslibrary{polar}
\pgfplotsset{compat=1.11}
\usepackage{graphicx}
\usepackage{sidecap}
\sidecaptionvpos{figure}{c}
\usepackage{float}
\usepackage{gensymb}
\usepackage{tkz-euclide}
\usetkzobj{all}
\usepackage{commath}
\usepackage{hyperref}
\usepackage{enumitem}
\usepackage{wasysym}
\usepackage{multicol}
\usepackage{mathtools}
\usepackage{tcolorbox}
\usepackage{tabularx}
\usepackage[version=4]{mhchem}
\usepackage{changepage}
\usepackage{listings}
\lstset{basicstyle=\ttfamily\linespread{0.8}\small}

\renewcommand*{\thefootnote}{\fnsymbol{footnote}}

\newtheorem*{thm}{Theorem}
\newtheorem*{iden}{Identity}
\newtheorem*{lemma}{Lemma}
\newtheorem{obs}{Observation}
\theoremstyle{definition}
\newtheorem*{defn}{Definition}
\newtheorem*{ex}{Example}
\newtheorem{con}{Construction}
\newtheorem*{alg}{Algorithm}

\newtheoremstyle{break}
  {\topsep}{\topsep}%
  {\itshape}{}%
  {\bfseries}{}%
  {\newline}{}%
\theoremstyle{break}
\newtheorem*{bthm}{Theorem}

% russian integral
\usepackage{scalerel}
\DeclareMathOperator*{\rint}{\scalerel*{\rotatebox{17}{$\!\int\!$}}{\int}}

% \DeclareMathOperator*{\rint}{\int}

\pgfplotsset{vasymptote/.style={
    before end axis/.append code={
        \draw[densely dashed] ({rel axis cs:0,0} -| {axis cs:#1,0})
        -- ({rel axis cs:0,1} -| {axis cs:#1,0});
    }
}}

% \pointsinrightmargin
\boxedpoints
\pointname{}

\newcommand{\questioA}{\question[\texttt{\textbf{\color{Cerulean} A}}]}
\newcommand{\questioM}{\question[\texttt{\textbf{\color{PineGreen} M}}]}
\newcommand{\questioE}{\question[\texttt{\textbf{\color{WildStrawberry} E}}]}
\newcommand{\questioS}{\question[\texttt{\textbf{\color{Goldenrod} S}}]}
\newcommand{\questioO}{\question[\texttt{\textbf{\color{BurntOrange} O}}]}

\newcommand{\parA}{\part[\texttt{\textbf{\color{Cerulean} A}}]}
\newcommand{\parM}{\part[\texttt{\textbf{\color{PineGreen} M}}]}
\newcommand{\parE}{\part[\texttt{\textbf{\color{WildStrawberry} E}}]}
\newcommand{\parS}{\part[\texttt{\textbf{\color{Goldenrod} S}}]}
\newcommand{\parO}{\part[\texttt{\textbf{\color{BurntOrange} O}}]}

\newcommand{\subparA}{\subpart[\texttt{\textbf{\color{Cerulean} A}}]}
\newcommand{\subparM}{\subpart[\texttt{\textbf{\color{PineGreen} M}}]}
\newcommand{\subparE}{\subpart[\texttt{\textbf{\color{WildStrawberry} E}}]}
\newcommand{\subparS}{\subpart[\texttt{\textbf{\color{Goldenrod} S}}]}
\newcommand{\subparO}{\subpart[\texttt{\textbf{\color{BurntOrange} O}}]}

\newcommand{\mainHeader}[2]{\section*{NCEA Level 2 Mathematics\\#1. #2}}
\newcommand{\mainHeaderHw}[2]{\section*{NCEA Level 2 Mathematics (Homework)\\#1. #2}}

\begin{document}

\section*{NCEA Level 2 Mathematics\\Preface}
These notes present the NCEA Level 2 mathematics content from a mainly geometric standpoint, and so in some places
are a little nonstandard. For example, the study of quadratic equations is approached by analysing quadratic graphs,
and many of the exercises ask for intuitive and/or geometric explanations of many of the phenomena we study. This is
because I see so many students entering L3 calculus with little to no ability to make use of the interplay between geometric and
algebraic views of the same picture.

For example, I can cite the example of finding turning points of graphs: students remember very well that the way to solve this kind
of exam problem is to ``take the derivative and set it to zero'', but most of them cannot explain why this works! This is foreign to
me, because I don't remember not understanding this (and in fact I think the technique was taught to me geometrically first, which is
really the only way to do it)! I find it hard to believe that teachers of L2 mathematics are so incompetent that they don't mention
geometry at all when they introduce calculus, so maybe the problem is that when studying for exams the students just remember how to
solve problems and neglect the ``big picture'': that memorising how to solve ``types of problems'' for an exam is both counterproductive
and damaging because they don't remember the important ideas (about tangent lines, slopes, approximations, geometry) and then wonder why
the ``skills'' they've learned (symbol pushing on an exam paper) are useless in the ``real world''!

These notes also have another agenda: to introduce students in a calm way to mathematical proof. There is a definite increase in sophistication
required for the later sections, but right from the first problem set students are asked to justify statements mathematically. I make no apology
to those who want to use these notes but avoid forcing students to write proofs: it is simply how mathematics is done (and I don't think many
of the exercises, if any, are out of the reach of the enthusiastic student).

I have tried to address many of these ideas in my student introduction as well.

\subsection*{Guide to the bibliography}
The bibliography is a mixture of further reading and additional problemsets. I have not included many drill-type problems
(like ``solve for $ x $ given $ x^2 + 3x - 20 $'') because they are easily found for those sections that need them: in particular,
Spiegel is a good source of algebra drill problems and Foerster is a good source of trigonometry drill problems. For calculus
and coordinate geometry, I have included Andree --- although I know no good source that covers just the material in calculus needed
for L2 and so it should be used with caution.

In terms of additional reading, most (all?) of the books are suitable for an enthusiastic Y12 student. I particularly recommend
Lauwerier, B\'ona, or any books on graph theory and the four-colour theorem for students interested in computer science and/or programming.

The two books by Ben Goldacre are also highly recommended for any students who will be going into the sciences or medicine.

Many of the titles are popular mathematics books (e.g. the two by Bellos) that cover the material we see this year at a slightly lower
level, and put it in context (although some of the topics should be taken with a grain of salt: Bellos includes chapters on such crank
topics as the Golden Ratio).

\subsection*{List of sections with the standards that they cover}
\subsubsection*{Geometry}
\begin{enumerate}
  \item (2.1) Coordinate Geometry
  \item (2.4) Arcs and Sectors of Circles
  \item (2.4) Trigonometry
\end{enumerate}

\subsubsection*{Algebra}
\begin{enumerate}[resume]
  \item (2.2) Functions
  \item (2.2/2.6) Quadratic Modelling
  \item (2.6/2.14) Simultaneous Equations
  \item (2.6/2.14) Linear Inequations
  \item (2.6) The Quadratic Formula
  \item (2.2/2.6) Exponential and Logarithmic Functions
  \item (2.2/2.6) Negative and Fractional Powers
\end{enumerate}

\subsubsection*{Calculus}
\begin{enumerate}[resume]
  \item (2.7) Slopes and Differentiation
  \item (2.7) Tangent Lines and Approximation
  \item (2.7) Turning Points Optimisation
  \item (2.7) Anti-differentiation
  \item (2.7) Kinematics and Rates of Change
\end{enumerate}

\subsubsection*{Combinatorics}
\begin{enumerate}[resume]
  \item (Stats?) Counting and Combinatorics
  \item (2.3) Number Sequences and Fractals
  \item (2.5) Graphs and Networks
\end{enumerate}

\end{document}
