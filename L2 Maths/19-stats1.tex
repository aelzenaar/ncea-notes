\documentclass{exam}
\usepackage[utf8]{inputenc}
\usepackage{lmodern}
\usepackage{microtype}

% \usepackage[parfill]{parskip}
\usepackage[dvipsnames]{xcolor}
\usepackage{amsmath}
\usepackage{amsfonts}
\usepackage{amsthm}
\usepackage{siunitx}
\DeclareSIUnit\year{yr}
\DeclareSIUnit\foot{ft}
\DeclareSIUnit\litre{\liter}

\usepackage{skull}

\usepackage{pgfplots}
\usepgfplotslibrary{polar}
\pgfplotsset{compat=1.11}
\usepackage{graphicx}
\usepackage{sidecap}
\sidecaptionvpos{figure}{c}
\usepackage{float}
\usepackage{gensymb}
\usepackage{tkz-euclide}
\usetkzobj{all}
\usepackage{commath}
\usepackage{hyperref}
\usepackage{enumitem}
\usepackage{wasysym}
\usepackage{multicol}
\usepackage{mathtools}
\usepackage{tcolorbox}
\usepackage{tabularx}
\usepackage[version=4]{mhchem}
\usepackage{changepage}
\usepackage{listings}
\lstset{basicstyle=\ttfamily\linespread{0.8}\small}

\renewcommand*{\thefootnote}{\fnsymbol{footnote}}

\newtheorem*{thm}{Theorem}
\newtheorem*{iden}{Identity}
\newtheorem*{lemma}{Lemma}
\newtheorem{obs}{Observation}
\theoremstyle{definition}
\newtheorem*{defn}{Definition}
\newtheorem*{ex}{Example}
\newtheorem{con}{Construction}
\newtheorem*{alg}{Algorithm}

\newtheoremstyle{break}
  {\topsep}{\topsep}%
  {\itshape}{}%
  {\bfseries}{}%
  {\newline}{}%
\theoremstyle{break}
\newtheorem*{bthm}{Theorem}

% russian integral
\usepackage{scalerel}
\DeclareMathOperator*{\rint}{\scalerel*{\rotatebox{17}{$\!\int\!$}}{\int}}

% \DeclareMathOperator*{\rint}{\int}

\pgfplotsset{vasymptote/.style={
    before end axis/.append code={
        \draw[densely dashed] ({rel axis cs:0,0} -| {axis cs:#1,0})
        -- ({rel axis cs:0,1} -| {axis cs:#1,0});
    }
}}

% \pointsinrightmargin
\boxedpoints
\pointname{}

\newcommand{\questioA}{\question[\texttt{\textbf{\color{Cerulean} A}}]}
\newcommand{\questioM}{\question[\texttt{\textbf{\color{PineGreen} M}}]}
\newcommand{\questioE}{\question[\texttt{\textbf{\color{WildStrawberry} E}}]}
\newcommand{\questioS}{\question[\texttt{\textbf{\color{Goldenrod} S}}]}
\newcommand{\questioO}{\question[\texttt{\textbf{\color{BurntOrange} O}}]}

\newcommand{\parA}{\part[\texttt{\textbf{\color{Cerulean} A}}]}
\newcommand{\parM}{\part[\texttt{\textbf{\color{PineGreen} M}}]}
\newcommand{\parE}{\part[\texttt{\textbf{\color{WildStrawberry} E}}]}
\newcommand{\parS}{\part[\texttt{\textbf{\color{Goldenrod} S}}]}
\newcommand{\parO}{\part[\texttt{\textbf{\color{BurntOrange} O}}]}

\newcommand{\subparA}{\subpart[\texttt{\textbf{\color{Cerulean} A}}]}
\newcommand{\subparM}{\subpart[\texttt{\textbf{\color{PineGreen} M}}]}
\newcommand{\subparE}{\subpart[\texttt{\textbf{\color{WildStrawberry} E}}]}
\newcommand{\subparS}{\subpart[\texttt{\textbf{\color{Goldenrod} S}}]}
\newcommand{\subparO}{\subpart[\texttt{\textbf{\color{BurntOrange} O}}]}

\newcommand{\mainHeader}[2]{\section*{NCEA Level 2 Mathematics\\#1. #2}}
\newcommand{\mainHeaderHw}[2]{\section*{NCEA Level 2 Mathematics (Homework)\\#1. #2}}


\begin{document}

\mainHeader{19}{Statistical Reports}
Statistics is one of the most important tools in experimental science, and medicine. Even in the everyday world we are
bombarded with statistics given to us by the media, politicians, and the internet. Because of this, being able to interpret
statistics and judging whether or not they support a given argument, result, or point of view is an important skill in
the modern world.

\begin{center}
  \includegraphics[width=0.7\textwidth]{stats}
\end{center}

The statistical inquiry cycle is something you should already know about from as far back as intermediate school.

\begin{enumerate}
  \item We begin with a question we want to answer, and perhaps a hypothesis backed up by some theory.
  \item We decide what data we need to collect to answer our question.
  \item We plan how we will collect our data, and how we will ensure it is accurate and precise.
  \item We plan how we will process our data.
  \item We collect our data. (This is intentionally \emph{after} we plan how we will use it.)
  \item We process our data in accordance with our plan.
  \item We answer our question.
  \item We decide how reliable our process was.
  \item We write a report, detailing \emph{all} of the above steps, such that another person
        could pick it up and try to replicate our findings.
\end{enumerate}

\subsection*{Writing a question, and deciding on data to collect}
A question should be written clearly and precisely, and there should be a straightforward way to answer it.

Some good questions:
\begin{itemize}
  \item Are high school students in Wellington City more likely to cycle to school than students in Upper Hutt?
  \item Do male Y12 students tend to be taller than female Y12 students?
  \item Does this drug lower the risk of heart failure after a stroke?
  \item With what speed does a \SI{0.5}{\kilo\gram} weight hit the ground after being dropped \SI{10}{\metre}?
\end{itemize}

Some bad questions (why?):
\begin{itemize}
  \item Does this drug work?
  \item Will I crash if I drink and drive?
  \item Does the average person support the attempt by the USA to bring freedom and democracy to other places in the world?
  \item Any question you write after you've already collected your data!\footnote{Try googling `data dredging': ``Data dredging (also data fishing, data snooping,
        data butchery, and p-hacking) is the misuse of data analysis to find patterns in data that can be presented as statistically significant when in fact there
        is no real underlying effect. This is done by performing many statistical tests on the data and only paying attention to those that come back with
        significant results, instead of stating a single hypothesis about an underlying effect before the analysis and then conducting a single test for it.

        ``The process of data dredging involves automatically testing huge numbers of hypotheses about a single data set by exhaustively searching—perhaps for
        combinations of variables that might show a correlation, and perhaps for groups of cases or observations that show differences in their mean or in
        their breakdown by some other variable.'' (Wikipedia contributors. (2019, January 13). Data dredging. In Wikipedia, The Free Encyclopedia. Retrieved 22:29, January 20, 2019, from \url{https://en.wikipedia.org/w/index.php?title=Data_dredging&oldid=878260765})}
\end{itemize}

When you write your question, you should come up with a good idea as to what kind of data you will need to gather to answer it. This could
be something like a set of measurements, or responses to a questionnaire, or something else.

\subsection*{Planning how to collect data}
\subsubsection*{Sampling}
Usually, we want to collect data to answer a question about a whole population. Unfortunately, it's not cheap or easy to measure or survey every individual
in a population, so we have to make do with examining a smaller sample, and then process our data to make an educated inference about
the state of the full population from that.

We need to choose our sample to be representative of the population, so that

\subsubsection*{Surveys}

\subsection*{Questions}
\begin{questions}
  \question
\end{questions}

\end{document}
