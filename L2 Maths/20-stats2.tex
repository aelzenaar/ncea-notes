\documentclass{exam}
\usepackage[utf8]{inputenc}
\usepackage{lmodern}
\usepackage{microtype}

% \usepackage[parfill]{parskip}
\usepackage[dvipsnames]{xcolor}
\usepackage{amsmath}
\usepackage{amsfonts}
\usepackage{amsthm}
\usepackage{siunitx}
\DeclareSIUnit\year{yr}
\DeclareSIUnit\foot{ft}
\DeclareSIUnit\litre{\liter}

\usepackage{skull}

\usepackage{pgfplots}
\usepgfplotslibrary{polar}
\pgfplotsset{compat=1.11}
\usepackage{graphicx}
\usepackage{sidecap}
\sidecaptionvpos{figure}{c}
\usepackage{float}
\usepackage{gensymb}
\usepackage{tkz-euclide}
\usetkzobj{all}
\usepackage{commath}
\usepackage{hyperref}
\usepackage{enumitem}
\usepackage{wasysym}
\usepackage{multicol}
\usepackage{mathtools}
\usepackage{tcolorbox}
\usepackage{tabularx}
\usepackage[version=4]{mhchem}
\usepackage{changepage}
\usepackage{listings}
\lstset{basicstyle=\ttfamily\linespread{0.8}\small}

\renewcommand*{\thefootnote}{\fnsymbol{footnote}}

\newtheorem*{thm}{Theorem}
\newtheorem*{iden}{Identity}
\newtheorem*{lemma}{Lemma}
\newtheorem{obs}{Observation}
\theoremstyle{definition}
\newtheorem*{defn}{Definition}
\newtheorem*{ex}{Example}
\newtheorem{con}{Construction}
\newtheorem*{alg}{Algorithm}

\newtheoremstyle{break}
  {\topsep}{\topsep}%
  {\itshape}{}%
  {\bfseries}{}%
  {\newline}{}%
\theoremstyle{break}
\newtheorem*{bthm}{Theorem}

% russian integral
\usepackage{scalerel}
\DeclareMathOperator*{\rint}{\scalerel*{\rotatebox{17}{$\!\int\!$}}{\int}}

% \DeclareMathOperator*{\rint}{\int}

\pgfplotsset{vasymptote/.style={
    before end axis/.append code={
        \draw[densely dashed] ({rel axis cs:0,0} -| {axis cs:#1,0})
        -- ({rel axis cs:0,1} -| {axis cs:#1,0});
    }
}}

% \pointsinrightmargin
\boxedpoints
\pointname{}

\newcommand{\questioA}{\question[\texttt{\textbf{\color{Cerulean} A}}]}
\newcommand{\questioM}{\question[\texttt{\textbf{\color{PineGreen} M}}]}
\newcommand{\questioE}{\question[\texttt{\textbf{\color{WildStrawberry} E}}]}
\newcommand{\questioS}{\question[\texttt{\textbf{\color{Goldenrod} S}}]}
\newcommand{\questioO}{\question[\texttt{\textbf{\color{BurntOrange} O}}]}

\newcommand{\parA}{\part[\texttt{\textbf{\color{Cerulean} A}}]}
\newcommand{\parM}{\part[\texttt{\textbf{\color{PineGreen} M}}]}
\newcommand{\parE}{\part[\texttt{\textbf{\color{WildStrawberry} E}}]}
\newcommand{\parS}{\part[\texttt{\textbf{\color{Goldenrod} S}}]}
\newcommand{\parO}{\part[\texttt{\textbf{\color{BurntOrange} O}}]}

\newcommand{\subparA}{\subpart[\texttt{\textbf{\color{Cerulean} A}}]}
\newcommand{\subparM}{\subpart[\texttt{\textbf{\color{PineGreen} M}}]}
\newcommand{\subparE}{\subpart[\texttt{\textbf{\color{WildStrawberry} E}}]}
\newcommand{\subparS}{\subpart[\texttt{\textbf{\color{Goldenrod} S}}]}
\newcommand{\subparO}{\subpart[\texttt{\textbf{\color{BurntOrange} O}}]}

\newcommand{\mainHeader}[2]{\section*{NCEA Level 2 Mathematics\\#1. #2}}
\newcommand{\mainHeaderHw}[2]{\section*{NCEA Level 2 Mathematics (Homework)\\#1. #2}}


\begin{document}

\mainHeader{20}{Sampling}
Last time, we mainly looked at the broad picture: what we need to think about, in general, when we try to
answer a statistical question. This time we will begin to think about some of the practical issues we need
to overcome.

As we've already discussed, it is usually impractical to measure an entire population. Our goal is therefore
to measure a smaller sample and then extrapolate our findings. This process, known as \emph{statistical inference},
requires us to have a good method for choosing our sample so that it is representative.

We will look at several examples of bad methods of sampling to begin with.

\subsection*{The examples}
\begin{enumerate}
  \item I asked all my friends whether they own a car, and none of them do.
  \item A survey of high-school students samples all the Y13 students at a particular school, and concludes that only
        7\% of students use illegal drugs.
  \item In 1936, a US presidential election poll posted questionnaires to ten million people selected from telephone
        books and club membership lists, and got 2.4 million responses. Based on these, they predicted a decisive
        victory for one candidate (57\% of the popular vote). In reality, the other candidate won by a landslide (62\%).\footnote{Freedman et al., \emph{Statistics}. Section 19.2.}
  \item A psychiatrist finds that practically everyone is neurotic.
  \item A drug trial is performed; the patients were analysed according to the treatment they actually took, rather
        than the treatment they were assigned at the randomisation stage of the trial.
\end{enumerate}

\subsection*{The problems}
\begin{enumerate}
  \item Some people aren't my friends. In addition, most of my friends live in urban areas with frequent public transport,
        and tend to be more affluent.
  \item This doesn't include students who drop out of school, or are homeschooled. It also only measures students at
        a particular school, which might be more or less affluent than average and thus drug use by its students might be more or less
        probable.
  \item Despite the large sample size, the sampling method used tended to screen out the poor (who didn't belong to clubs, or
        have telephones) who were more likely to vote for the other candidate.
  \item The psychiatrist's patients are far from a sample of the population.
  \item This might seem reasonable (if 30\% of participants drop out, they didn't receive the benefit of the treatment and so
        shouldn't be part of the `participated in treatment' group during analysis). However, the problem is that the question
        that should be being answered is `is this treatment effective?' rather than `out of the people who chose to take our
        tablets, is the treatment effective?'. After all, if people don't end up taking the medication after being given it,
        this is philosophically and medically the same as if the medication was ineffective.\footnote{See Ben Goldacre, \emph{Bad Pharma}, pages 200-1.}
\end{enumerate}

READING: http://www.goodmath.org/blog/2016/09/29/polls-and-sampling-errors-in-the-presidental-debate-results/

SURVEY DESIGN: https://www.youtube.com/watch?v=G0ZZJXw4MTA


\subsection*{Questions}

\end{document}

