\documentclass{exam}
\usepackage[utf8]{inputenc}
\usepackage{lmodern}
\usepackage{microtype}

% \usepackage[parfill]{parskip}
\usepackage[dvipsnames]{xcolor}
\usepackage{amsmath}
\usepackage{amsfonts}
\usepackage{amsthm}
\usepackage{siunitx}
\DeclareSIUnit\year{yr}
\DeclareSIUnit\foot{ft}
\DeclareSIUnit\litre{\liter}

\usepackage{skull}

\usepackage{pgfplots}
\usepgfplotslibrary{polar}
\pgfplotsset{compat=1.11}
\usepgfplotslibrary{statistics}
\usepackage{graphicx}
\usepackage{sidecap}
\sidecaptionvpos{figure}{c}
\usepackage{float}
\usepackage{gensymb}
\usepackage{tkz-euclide}
\usetkzobj{all}
\usepackage{commath}
\usepackage{hyperref}
\usepackage{enumitem}
\usepackage{wasysym}
\usepackage{multicol}
\usepackage{mathtools}
\usepackage{tcolorbox}
\usepackage{tabularx}
\usepackage[version=4]{mhchem}
\usepackage{changepage}
\usepackage{listings}
\lstset{basicstyle=\ttfamily\linespread{0.8}\small}

\renewcommand*{\thefootnote}{\fnsymbol{footnote}}

\newtheorem*{thm}{Theorem}
\newtheorem*{iden}{Identity}
\newtheorem*{lemma}{Lemma}
\newtheorem{obs}{Observation}
\theoremstyle{definition}
\newtheorem*{defn}{Definition}
\newtheorem*{ex}{Example}
\newtheorem{con}{Construction}
\newtheorem*{alg}{Algorithm}

\newtheoremstyle{break}
  {\topsep}{\topsep}%
  {\itshape}{}%
  {\bfseries}{}%
  {\newline}{}%
\theoremstyle{break}
\newtheorem*{bthm}{Theorem}

% russian integral
\usepackage{scalerel}
\DeclareMathOperator*{\rint}{\scalerel*{\rotatebox{17}{$\!\int\!$}}{\int}}

% \DeclareMathOperator*{\rint}{\int}

\pgfplotsset{vasymptote/.style={
    before end axis/.append code={
        \draw[densely dashed] ({rel axis cs:0,0} -| {axis cs:#1,0})
        -- ({rel axis cs:0,1} -| {axis cs:#1,0});
    }
}}

% \pointsinrightmargin
\boxedpoints
\pointname{}

\newcommand{\questioA}{\question[\texttt{\textbf{\color{Cerulean} A}}]}
\newcommand{\questioM}{\question[\texttt{\textbf{\color{PineGreen} M}}]}
\newcommand{\questioE}{\question[\texttt{\textbf{\color{WildStrawberry} E}}]}
\newcommand{\questioS}{\question[\texttt{\textbf{\color{Goldenrod} S}}]}
\newcommand{\questioO}{\question[\texttt{\textbf{\color{BurntOrange} O}}]}

\newcommand{\parA}{\part[\texttt{\textbf{\color{Cerulean} A}}]}
\newcommand{\parM}{\part[\texttt{\textbf{\color{PineGreen} M}}]}
\newcommand{\parE}{\part[\texttt{\textbf{\color{WildStrawberry} E}}]}
\newcommand{\parS}{\part[\texttt{\textbf{\color{Goldenrod} S}}]}
\newcommand{\parO}{\part[\texttt{\textbf{\color{BurntOrange} O}}]}

\newcommand{\subparA}{\subpart[\texttt{\textbf{\color{Cerulean} A}}]}
\newcommand{\subparM}{\subpart[\texttt{\textbf{\color{PineGreen} M}}]}
\newcommand{\subparE}{\subpart[\texttt{\textbf{\color{WildStrawberry} E}}]}
\newcommand{\subparS}{\subpart[\texttt{\textbf{\color{Goldenrod} S}}]}
\newcommand{\subparO}{\subpart[\texttt{\textbf{\color{BurntOrange} O}}]}

\newcommand{\mainHeader}[2]{\section*{NCEA Level 2 Mathematics\\#1. #2}}
\newcommand{\mainHeaderHw}[2]{\section*{NCEA Level 2 Mathematics (Homework)\\#1. #2}}
\newcommand{\seealso}[1]{\begin{center}\emph{See also #1.}\end{center}}
\newcommand{\drills}[1]{\begin{center}\emph{Drill problems: #1.}\end{center}}
\newcommand{\basedon}[1]{\begin{center}\emph{Notes largely based on #1.}\end{center}}


\begin{document}

\mainHeaderHw{13}{Tangent Lines and Approximations}
\subsection*{Reading}
\begin{center}
\begin{tcolorbox}[width=0.8\textwidth,colback={white},title={\textbf{Go and watch...}},colbacktitle=black,coltitle=white]
  \textcolor{black}{\url{https://www.youtube.com/watch?v=F5RyVWI4Onk}}
\end{tcolorbox}
\end{center}

\subsubsection*{Polya's four-step approach to problem solving}
\begin{enumerate}
  \item \emph{Preparation:} Understand the problem
    \begin{itemize}
      \item Learn the necessary underlying mathematical concepts
      \item Consider the terminology and notation used in the problem:
        \begin{enumerate}
          \item What sort of a problem is it?
          \item What is being asked?
          \item What do the terms mean?
          \item Is there enough information or is more information needed?
          \item What is known or unknown?
        \end{enumerate}
      \item Rephrase the problem in your own words.
      \item Write down specific examples of the conditions given in the problem.
    \end{itemize}
  \item \emph{Thinking Time:} Devise a plan
    \begin{itemize}
      \item You must start somewhere so try something.
      \item How are you going to attack the problem?
      \item Possible strategies: (i.e. reach into your bag of tricks.)
        \begin{enumerate}
          \item Draw pictures
          \item Use a variable and choose helpful names for variables or unknowns.
          \item Be systematic.
          \item Solve a simpler version of the problem.
          \item Guess and check.  Trial and error.  Guess and test. (Guessing is OK as long as you can back it up later.)
          \item Look for a pattern or patterns.
          \item Make a list.
        \end{enumerate}
      \item Once you understand what the problem is, if you are stumped or stuck, set the problem aside for a while. Your
            subconscious mind may keep working on it.
      \item Moving on to think about other things may help you stay relaxed, flexible, and creative rather than becoming tense,
            frustrated, and forced in your efforts to solve the problem.
    \end{itemize}
  \item \emph{Insight:} Carry out the plan
    \begin{itemize}
      \item Once you have an idea for a new approach, jot it down immediately. When you have time, try it out and see if it
            leads to a solution.
      \item If the plan does not seem to be working, then start over and try another approach. Often the first approach does not
            work. Do not worry, just because an approach does not work, it  does not mean you did it wrong. You actually accomplished
            something, knowing a way does not work is part of the process of elimination.
      \item Once you have thought about a problem or returned to it enough times, you will often have a flash of insight: a new
            idea to try or a new perspective on how to approach solving the problem.
      \item The key is to \textbf{\emph{keep trying until something works}}.
    \end{itemize}
  \item \emph{Verification: } Look back
    \begin{itemize}
      \item Once you have a potential solution, check to see if it works.
        \begin{enumerate}
          \item Dd you answer the question?
          \item Is your result reasonable?
          \item Double check to make sure that all of the conditions related to the problem are satisfied.
          \item Double check any computations involved in finding your solution.
        \end{enumerate}
      \item If you find that your solution does not work, there may only be a simple mistake. Try to fix or modify your current
            attempt before scrapping it. Remember what you tried --- it is likely that at least part of it will end up being useful.
      \item Is there another way of doing the problem which may be simpler? (You need to become flexible in your thinking. There usually is not one right way.)
      \item Can the problem or method be generalized so as to be useful for future problems?
    \end{itemize}
\end{enumerate}
\subsection*{Questions}
\begin{questions}
  \question A cylindrical tube (with open ends) is to be made from a sheet of paper with area \SI{25}{\centi\metre\squared}. What should the dimensions
            of the tube be in order to maximise the volume of the tube? Justify that you have found a maximum.
  \question A function $ f $ is given by $ f(x) = 2x^3 - 3ax^2 + 6bx - 2 $. The function has two turning points, at $ x = 2 $
            and at $ x = 3 $. What are the values of $ a $ and $ b $?
  \question Suppose a wire of length $ \ell $ is cut into two pieces, one of length $ x $ and one of length $ \ell - x $. One
            piece is used to form the circumference of a circle, and the other is used to form the perimeter of a square. How
            long should the length $ x $ be in order to ensure that the total area of the circle and the square is minimised?
            Carefully justify that you have found a minimum.
\end{questions}

\end{document}
