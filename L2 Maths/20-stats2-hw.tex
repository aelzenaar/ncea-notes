\documentclass{exam}
\usepackage[utf8]{inputenc}
\usepackage{lmodern}
\usepackage{microtype}

% \usepackage[parfill]{parskip}
\usepackage[dvipsnames]{xcolor}
\usepackage{amsmath}
\usepackage{amsfonts}
\usepackage{amsthm}
\usepackage{siunitx}
\DeclareSIUnit\year{yr}
\DeclareSIUnit\foot{ft}
\DeclareSIUnit\litre{\liter}

\usepackage{skull}

\usepackage{pgfplots}
\usepgfplotslibrary{polar}
\pgfplotsset{compat=1.11}
\usepackage{graphicx}
\usepackage{sidecap}
\sidecaptionvpos{figure}{c}
\usepackage{float}
\usepackage{gensymb}
\usepackage{tkz-euclide}
\usetkzobj{all}
\usepackage{commath}
\usepackage{hyperref}
\usepackage{enumitem}
\usepackage{wasysym}
\usepackage{multicol}
\usepackage{mathtools}
\usepackage{tcolorbox}
\usepackage{tabularx}
\usepackage[version=4]{mhchem}
\usepackage{changepage}
\usepackage{listings}
\lstset{basicstyle=\ttfamily\linespread{0.8}\small}

\renewcommand*{\thefootnote}{\fnsymbol{footnote}}

\newtheorem*{thm}{Theorem}
\newtheorem*{iden}{Identity}
\newtheorem*{lemma}{Lemma}
\newtheorem{obs}{Observation}
\theoremstyle{definition}
\newtheorem*{defn}{Definition}
\newtheorem*{ex}{Example}
\newtheorem{con}{Construction}
\newtheorem*{alg}{Algorithm}

\newtheoremstyle{break}
  {\topsep}{\topsep}%
  {\itshape}{}%
  {\bfseries}{}%
  {\newline}{}%
\theoremstyle{break}
\newtheorem*{bthm}{Theorem}

% russian integral
\usepackage{scalerel}
\DeclareMathOperator*{\rint}{\scalerel*{\rotatebox{17}{$\!\int\!$}}{\int}}

% \DeclareMathOperator*{\rint}{\int}

\pgfplotsset{vasymptote/.style={
    before end axis/.append code={
        \draw[densely dashed] ({rel axis cs:0,0} -| {axis cs:#1,0})
        -- ({rel axis cs:0,1} -| {axis cs:#1,0});
    }
}}

% \pointsinrightmargin
\boxedpoints
\pointname{}

\newcommand{\questioA}{\question[\texttt{\textbf{\color{Cerulean} A}}]}
\newcommand{\questioM}{\question[\texttt{\textbf{\color{PineGreen} M}}]}
\newcommand{\questioE}{\question[\texttt{\textbf{\color{WildStrawberry} E}}]}
\newcommand{\questioS}{\question[\texttt{\textbf{\color{Goldenrod} S}}]}
\newcommand{\questioO}{\question[\texttt{\textbf{\color{BurntOrange} O}}]}

\newcommand{\parA}{\part[\texttt{\textbf{\color{Cerulean} A}}]}
\newcommand{\parM}{\part[\texttt{\textbf{\color{PineGreen} M}}]}
\newcommand{\parE}{\part[\texttt{\textbf{\color{WildStrawberry} E}}]}
\newcommand{\parS}{\part[\texttt{\textbf{\color{Goldenrod} S}}]}
\newcommand{\parO}{\part[\texttt{\textbf{\color{BurntOrange} O}}]}

\newcommand{\subparA}{\subpart[\texttt{\textbf{\color{Cerulean} A}}]}
\newcommand{\subparM}{\subpart[\texttt{\textbf{\color{PineGreen} M}}]}
\newcommand{\subparE}{\subpart[\texttt{\textbf{\color{WildStrawberry} E}}]}
\newcommand{\subparS}{\subpart[\texttt{\textbf{\color{Goldenrod} S}}]}
\newcommand{\subparO}{\subpart[\texttt{\textbf{\color{BurntOrange} O}}]}

\newcommand{\mainHeader}[2]{\section*{NCEA Level 2 Mathematics\\#1. #2}}
\newcommand{\mainHeaderHw}[2]{\section*{NCEA Level 2 Mathematics (Homework)\\#1. #2}}


\begin{document}

\mainHeaderHw{20}{Sampling}
\subsection*{Reading}
Please read the following article on political polls in New Zealand prior to the last General Election:
\begin{center}
  \url{https://www.stuff.co.nz/national/politics/96793252/qa-what-should-we-make-of-political-poll-results}
\end{center}
Think about how reasonable the sample methods used are. If you were given a couple of million dollars and asked to produce
your own poll based on a sample of the population, how would you go about this?

Optional: Another interesting blog post on the problems with self-selected samples, and correcting for selection bias, can be found here:
\begin{center}
  \url{http://www.goodmath.org/blog/2016/09/29/polls-and-sampling-errors-in-the-presidental-debate-results/}
\end{center}

\subsection*{Questions}
\begin{questions}
  \question Discuss how the data in the following graph might affect your interpretation of a survey, performed by a student in your biology class,
            which finds the average size of a particular species of fish in a given area.
            \begin{center}
              \includegraphics[width=0.7\textwidth]{badsample}
            \end{center}
  \question Watch the following clip from the television series \emph{Yes Prime Minister}: \url{https://www.youtube.com/watch?v=G0ZZJXw4MTA}.
            An upmarket toothpaste manufacturer has asked you to run a survey for a marketing campaign, comparing how many people use their
            toothpaste versus the product of a competitor who produces a cheaper toothpaste.
    \begin{parts}
      \item Design a sampling method and survey questions that are likely to cast the products of the manufacturer in a good light.
      \item Design a sampling method and survey questions that are likely to cast the products of their competitor in a bad light.
      \item Design a sampling method and survey questions to fairly compare the products usage.
    \end{parts}
\end{questions}
\end{document}
