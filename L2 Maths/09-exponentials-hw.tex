\documentclass{exam}
\usepackage[utf8]{inputenc}
\usepackage{lmodern}
\usepackage{microtype}

% \usepackage[parfill]{parskip}
\usepackage[dvipsnames]{xcolor}
\usepackage{amsmath}
\usepackage{amsfonts}
\usepackage{amsthm}
\usepackage{siunitx}
\DeclareSIUnit\year{yr}
\DeclareSIUnit\foot{ft}
\DeclareSIUnit\litre{\liter}

\usepackage{skull}

\usepackage{pgfplots}
\usepgfplotslibrary{polar}
\pgfplotsset{compat=1.11}
\usepgfplotslibrary{statistics}
\usepackage{graphicx}
\usepackage{sidecap}
\sidecaptionvpos{figure}{c}
\usepackage{float}
\usepackage{gensymb}
\usepackage{tkz-euclide}
\usetkzobj{all}
\usepackage{commath}
\usepackage{hyperref}
\usepackage{enumitem}
\usepackage{wasysym}
\usepackage{multicol}
\usepackage{mathtools}
\usepackage{tcolorbox}
\usepackage{tabularx}
\usepackage[version=4]{mhchem}
\usepackage{changepage}
\usepackage{listings}
\lstset{basicstyle=\ttfamily\linespread{0.8}\small}

\renewcommand*{\thefootnote}{\fnsymbol{footnote}}

\newtheorem*{thm}{Theorem}
\newtheorem*{iden}{Identity}
\newtheorem*{lemma}{Lemma}
\newtheorem{obs}{Observation}
\theoremstyle{definition}
\newtheorem*{defn}{Definition}
\newtheorem*{ex}{Example}
\newtheorem{con}{Construction}
\newtheorem*{alg}{Algorithm}

\newtheoremstyle{break}
  {\topsep}{\topsep}%
  {\itshape}{}%
  {\bfseries}{}%
  {\newline}{}%
\theoremstyle{break}
\newtheorem*{bthm}{Theorem}

% russian integral
\usepackage{scalerel}
\DeclareMathOperator*{\rint}{\scalerel*{\rotatebox{17}{$\!\int\!$}}{\int}}

% \DeclareMathOperator*{\rint}{\int}

\pgfplotsset{vasymptote/.style={
    before end axis/.append code={
        \draw[densely dashed] ({rel axis cs:0,0} -| {axis cs:#1,0})
        -- ({rel axis cs:0,1} -| {axis cs:#1,0});
    }
}}

% \pointsinrightmargin
\boxedpoints
\pointname{}

\newcommand{\questioA}{\question[\texttt{\textbf{\color{Cerulean} A}}]}
\newcommand{\questioM}{\question[\texttt{\textbf{\color{PineGreen} M}}]}
\newcommand{\questioE}{\question[\texttt{\textbf{\color{WildStrawberry} E}}]}
\newcommand{\questioS}{\question[\texttt{\textbf{\color{Goldenrod} S}}]}
\newcommand{\questioO}{\question[\texttt{\textbf{\color{BurntOrange} O}}]}

\newcommand{\parA}{\part[\texttt{\textbf{\color{Cerulean} A}}]}
\newcommand{\parM}{\part[\texttt{\textbf{\color{PineGreen} M}}]}
\newcommand{\parE}{\part[\texttt{\textbf{\color{WildStrawberry} E}}]}
\newcommand{\parS}{\part[\texttt{\textbf{\color{Goldenrod} S}}]}
\newcommand{\parO}{\part[\texttt{\textbf{\color{BurntOrange} O}}]}

\newcommand{\subparA}{\subpart[\texttt{\textbf{\color{Cerulean} A}}]}
\newcommand{\subparM}{\subpart[\texttt{\textbf{\color{PineGreen} M}}]}
\newcommand{\subparE}{\subpart[\texttt{\textbf{\color{WildStrawberry} E}}]}
\newcommand{\subparS}{\subpart[\texttt{\textbf{\color{Goldenrod} S}}]}
\newcommand{\subparO}{\subpart[\texttt{\textbf{\color{BurntOrange} O}}]}

\newcommand{\mainHeader}[2]{\section*{NCEA Level 2 Mathematics\\#1. #2}}
\newcommand{\mainHeaderHw}[2]{\section*{NCEA Level 2 Mathematics (Homework)\\#1. #2}}
\newcommand{\seealso}[1]{\begin{center}\emph{See also #1.}\end{center}}
\newcommand{\drills}[1]{\begin{center}\emph{Drill problems: #1.}\end{center}}
\newcommand{\basedon}[1]{\begin{center}\emph{Notes largely based on #1.}\end{center}}


\begin{document}

\mainHeaderHw{9}{Exponential and Logarithmic Functions}
\subsection*{Reading}
\begin{center}
\begin{tcolorbox}[width=0.8\textwidth,colback={white},title={\textbf{Go and watch...}},colbacktitle=black,coltitle=white]
  \textcolor{black}{\url{https://www.youtube.com/watch?v=N-7tcTIrers}}
\end{tcolorbox}
\end{center}

\begin{center}
\begin{tcolorbox}[width=0.8\textwidth,colback={white},title={\textbf{What's it good for?}},colbacktitle=MidnightBlue,coltitle=white]
  People use exponential and logarithmic equations for...
  \begin{itemize}
    \item Chemistry, physics, engineering: whenever the rate of growth or rate of decline of a quantity is proportional to (or inversely proportional
          to) the amount of quantity present, the quantity is an exponential or logarithmic function of time. (This includes rates of chemical reaction,
          rates of capacitor charge/discharge, the position of a damped spring over time, and many other examples.)
  \end{itemize}
\end{tcolorbox}
\end{center}

\subsection*{Questions}
\begin{questions}
  \question Thirty minutes after a patient is administered his first dose of a medication, the amount of medication in his
            bloodstream reaches \SI{224}{\milli\gram}. The amount of the medication in the bloodstream decreases continuously
            by 20\% each hour. The amount of the medication $M$ mg in the patient's bloodstream after it is administered can
            be modelled by the function
            \begin{displaymath}
              M = 224 \times 0.8^{t - 0.5}
            \end{displaymath}
            where $ t $ is the time in hours since the drug was first administered.
    \begin{parts}
      \part Explain what 0.8 means in this function.
      \part Give the initial amount of medication administered.
      \part A second dose of the medication can be administered some time later, and again the amount of the medication in the patient's
            bloodstream from the second dose can be modelled by the same function as that for the first. The total amount of the drug in
            the blood stream must never exceed \SI{300}{\milli\gram}. How long after administering the first dose can the second dose be administered?
    \end{parts}
  \question Here are some revision questions on topics we have already covered.
    \begin{parts}
      \part Rearrange the following formula to make $ x $ the subject: $ \frac{4x}{5} = \frac{y(x + 3)}{2} $.
      \part Show that the solutions of $ x^2 + x - 56 = 0 $ are four times those of $ 4x^2 + x - 14 = 0 $.
      \part Find the relationship between the solutions of the equations $ dx^2 + ex + f = 0 $ and $ x^2 + ex + df = 0 $ where $ d $, $ e $,
            and $ f $ are real numbers.
      \part Consider the equation $ (3x + 1)^2 = -7 $.
        \begin{subparts}
          \subpart Explain why it has no real solutions; explain what this means graphically.
          \subpart Compute the discriminant of the equation, and explain why this further supports your answer to (i).
        \end{subparts}
    \end{parts}
\end{questions}

\end{document}
