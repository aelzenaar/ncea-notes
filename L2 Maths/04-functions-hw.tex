\documentclass{exam}
\usepackage[utf8]{inputenc}
\usepackage{lmodern}
\usepackage{microtype}

% \usepackage[parfill]{parskip}
\usepackage[dvipsnames]{xcolor}
\usepackage{amsmath}
\usepackage{amsfonts}
\usepackage{amsthm}
\usepackage{siunitx}
\DeclareSIUnit\year{yr}
\DeclareSIUnit\foot{ft}
\DeclareSIUnit\litre{\liter}

\usepackage{skull}

\usepackage{pgfplots}
\usepgfplotslibrary{polar}
\pgfplotsset{compat=1.11}
\usepgfplotslibrary{statistics}
\usepackage{graphicx}
\usepackage{sidecap}
\sidecaptionvpos{figure}{c}
\usepackage{float}
\usepackage{gensymb}
\usepackage{tkz-euclide}
\usetkzobj{all}
\usepackage{commath}
\usepackage{hyperref}
\usepackage{enumitem}
\usepackage{wasysym}
\usepackage{multicol}
\usepackage{mathtools}
\usepackage{tcolorbox}
\usepackage{tabularx}
\usepackage[version=4]{mhchem}
\usepackage{changepage}
\usepackage{listings}
\lstset{basicstyle=\ttfamily\linespread{0.8}\small}

\renewcommand*{\thefootnote}{\fnsymbol{footnote}}

\newtheorem*{thm}{Theorem}
\newtheorem*{iden}{Identity}
\newtheorem*{lemma}{Lemma}
\newtheorem{obs}{Observation}
\theoremstyle{definition}
\newtheorem*{defn}{Definition}
\newtheorem*{ex}{Example}
\newtheorem{con}{Construction}
\newtheorem*{alg}{Algorithm}

\newtheoremstyle{break}
  {\topsep}{\topsep}%
  {\itshape}{}%
  {\bfseries}{}%
  {\newline}{}%
\theoremstyle{break}
\newtheorem*{bthm}{Theorem}

% russian integral
\usepackage{scalerel}
\DeclareMathOperator*{\rint}{\scalerel*{\rotatebox{17}{$\!\int\!$}}{\int}}

% \DeclareMathOperator*{\rint}{\int}

\pgfplotsset{vasymptote/.style={
    before end axis/.append code={
        \draw[densely dashed] ({rel axis cs:0,0} -| {axis cs:#1,0})
        -- ({rel axis cs:0,1} -| {axis cs:#1,0});
    }
}}

% \pointsinrightmargin
\boxedpoints
\pointname{}

\newcommand{\questioA}{\question[\texttt{\textbf{\color{Cerulean} A}}]}
\newcommand{\questioM}{\question[\texttt{\textbf{\color{PineGreen} M}}]}
\newcommand{\questioE}{\question[\texttt{\textbf{\color{WildStrawberry} E}}]}
\newcommand{\questioS}{\question[\texttt{\textbf{\color{Goldenrod} S}}]}
\newcommand{\questioO}{\question[\texttt{\textbf{\color{BurntOrange} O}}]}

\newcommand{\parA}{\part[\texttt{\textbf{\color{Cerulean} A}}]}
\newcommand{\parM}{\part[\texttt{\textbf{\color{PineGreen} M}}]}
\newcommand{\parE}{\part[\texttt{\textbf{\color{WildStrawberry} E}}]}
\newcommand{\parS}{\part[\texttt{\textbf{\color{Goldenrod} S}}]}
\newcommand{\parO}{\part[\texttt{\textbf{\color{BurntOrange} O}}]}

\newcommand{\subparA}{\subpart[\texttt{\textbf{\color{Cerulean} A}}]}
\newcommand{\subparM}{\subpart[\texttt{\textbf{\color{PineGreen} M}}]}
\newcommand{\subparE}{\subpart[\texttt{\textbf{\color{WildStrawberry} E}}]}
\newcommand{\subparS}{\subpart[\texttt{\textbf{\color{Goldenrod} S}}]}
\newcommand{\subparO}{\subpart[\texttt{\textbf{\color{BurntOrange} O}}]}

\newcommand{\mainHeader}[2]{\section*{NCEA Level 2 Mathematics\\#1. #2}}
\newcommand{\mainHeaderHw}[2]{\section*{NCEA Level 2 Mathematics (Homework)\\#1. #2}}
\newcommand{\seealso}[1]{\begin{center}\emph{See also #1.}\end{center}}
\newcommand{\drills}[1]{\begin{center}\emph{Drill problems: #1.}\end{center}}
\newcommand{\basedon}[1]{\begin{center}\emph{Notes largely based on #1.}\end{center}}


\begin{document}

\mainHeaderHw{4}{Functions}
\subsection*{Reading}
\begin{center}
\begin{tcolorbox}[width=0.8\textwidth,colback={white},title={\textbf{Go and watch...}},colbacktitle=black,coltitle=white]
  \textcolor{black}{\url{https://www.youtube.com/watch?v=PtLnwvH4kuE}}
\end{tcolorbox}
\end{center}

\begin{center}
\begin{tcolorbox}[width=0.8\textwidth,colback={white},title={\textbf{What's it good for?}},colbacktitle=MidnightBlue,coltitle=white]
  People use functions and mathematical modelling for...
  \begin{itemize}
    \item Statistics, engineering, and the sciences: extrapolation from a set of data and prediction of results from future experiments
          and other situations is a key part of the scientific method.
    \item Theoretical physics: in particular, the mathematics behind quantum mechanics is heavily dependent on linear transformations,
          a particular class of function.
    \item Mathematics: as I mentioned in the notes, functions are a key idea in mathematics because they allow us to describe things
          like curves in space, rates of change, and relationships between objects. Graphs and diagrams of functions are often another
          way to view a concept and let us gain more clarity. Normally, we don't talk about functions in general, but restrict ourselves
          to subclasses of functions with nice properties (preserving distance, or more generally `closeness'; preserving algebraic properties
          like addition or multiplication; etc.).
  \end{itemize}
\end{tcolorbox}
\end{center}

\subsection*{Questions}
[This is from a sample Ministry of Education L2 assessment task for this standard.]

Place cones at the following co-ordinates, in metres, with the positive y axis pointing due north:

\begin{center}
  \begin{tabular}{|c|c||c|c||c|c|}\hline
    \textbf{Cone} & \textbf{Location} & \textbf{Cone} & \textbf{Location} & \textbf{Cone} & \textbf{Location}\\\hline
    $ A $ & $ (-14,1) $ & $ D $ & $ (7,6) $ & $ G $ & $ (7,0) $\\\hline
    $ B $ & $ (-16,6) $ & $ E $ & $ (7,10) $ & $ H $ & $ (17,-3) $\\\hline
    $ C $ & $ (3,6) $ & $ F $ & $ (10,10) $ & $ I $ & $ (7,1) $ \\\hline
  \end{tabular}
\end{center}

Give equations for each of the following curves:
\begin{questions}
  \question Start from a point one metre to the north of cone $A$. Ride in a straight line to a point two metres to the north of cone $B$.
  \question Starting from the end of line 1, weave around cones $B$, $C$, and $D$, such that the maximum distance south of cone $C$ is the
            same as the maximum distance north of cones $B$ and $D$, passing through the point $(5,6)$ following a curve of the form $f(x) = A\sin(x - 5) + C $.
\end{questions}

As revision from L1, expand and simplify the following:
\begin{questions}
  \question $ (x - 2)^2(x + 8) $
  \question $ (5x - 4)(x + 2)(x + 1) $
\end{questions}

\end{document}
