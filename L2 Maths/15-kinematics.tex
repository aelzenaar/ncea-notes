\documentclass{exam}
\usepackage[utf8]{inputenc}
\usepackage{lmodern}
\usepackage{microtype}

% \usepackage[parfill]{parskip}
\usepackage[dvipsnames]{xcolor}
\usepackage{amsmath}
\usepackage{amsfonts}
\usepackage{amsthm}
\usepackage{siunitx}
\DeclareSIUnit\year{yr}
\DeclareSIUnit\foot{ft}
\DeclareSIUnit\litre{\liter}

\usepackage{skull}

\usepackage{pgfplots}
\usepgfplotslibrary{polar}
\pgfplotsset{compat=1.11}
\usepackage{graphicx}
\usepackage{sidecap}
\sidecaptionvpos{figure}{c}
\usepackage{float}
\usepackage{gensymb}
\usepackage{tkz-euclide}
\usetkzobj{all}
\usepackage{commath}
\usepackage{hyperref}
\usepackage{enumitem}
\usepackage{wasysym}
\usepackage{multicol}
\usepackage{mathtools}
\usepackage{tcolorbox}
\usepackage{tabularx}
\usepackage[version=4]{mhchem}
\usepackage{changepage}
\usepackage{listings}
\lstset{basicstyle=\ttfamily\linespread{0.8}\small}

\renewcommand*{\thefootnote}{\fnsymbol{footnote}}

\newtheorem*{thm}{Theorem}
\newtheorem*{iden}{Identity}
\newtheorem*{lemma}{Lemma}
\newtheorem{obs}{Observation}
\theoremstyle{definition}
\newtheorem*{defn}{Definition}
\newtheorem*{ex}{Example}
\newtheorem{con}{Construction}
\newtheorem*{alg}{Algorithm}

\newtheoremstyle{break}
  {\topsep}{\topsep}%
  {\itshape}{}%
  {\bfseries}{}%
  {\newline}{}%
\theoremstyle{break}
\newtheorem*{bthm}{Theorem}

% russian integral
\usepackage{scalerel}
\DeclareMathOperator*{\rint}{\scalerel*{\rotatebox{17}{$\!\int\!$}}{\int}}

% \DeclareMathOperator*{\rint}{\int}

\pgfplotsset{vasymptote/.style={
    before end axis/.append code={
        \draw[densely dashed] ({rel axis cs:0,0} -| {axis cs:#1,0})
        -- ({rel axis cs:0,1} -| {axis cs:#1,0});
    }
}}

% \pointsinrightmargin
\boxedpoints
\pointname{}

\newcommand{\questioA}{\question[\texttt{\textbf{\color{Cerulean} A}}]}
\newcommand{\questioM}{\question[\texttt{\textbf{\color{PineGreen} M}}]}
\newcommand{\questioE}{\question[\texttt{\textbf{\color{WildStrawberry} E}}]}
\newcommand{\questioS}{\question[\texttt{\textbf{\color{Goldenrod} S}}]}
\newcommand{\questioO}{\question[\texttt{\textbf{\color{BurntOrange} O}}]}

\newcommand{\parA}{\part[\texttt{\textbf{\color{Cerulean} A}}]}
\newcommand{\parM}{\part[\texttt{\textbf{\color{PineGreen} M}}]}
\newcommand{\parE}{\part[\texttt{\textbf{\color{WildStrawberry} E}}]}
\newcommand{\parS}{\part[\texttt{\textbf{\color{Goldenrod} S}}]}
\newcommand{\parO}{\part[\texttt{\textbf{\color{BurntOrange} O}}]}

\newcommand{\subparA}{\subpart[\texttt{\textbf{\color{Cerulean} A}}]}
\newcommand{\subparM}{\subpart[\texttt{\textbf{\color{PineGreen} M}}]}
\newcommand{\subparE}{\subpart[\texttt{\textbf{\color{WildStrawberry} E}}]}
\newcommand{\subparS}{\subpart[\texttt{\textbf{\color{Goldenrod} S}}]}
\newcommand{\subparO}{\subpart[\texttt{\textbf{\color{BurntOrange} O}}]}

\newcommand{\mainHeader}[2]{\section*{NCEA Level 2 Mathematics\\#1. #2}}
\newcommand{\mainHeaderHw}[2]{\section*{NCEA Level 2 Mathematics (Homework)\\#1. #2}}


\begin{document}

\mainHeader{15}{Kinematics and Rates of Change}
The original development of calculus was as a tool for physicists to describe motion. The application itself is entirely natural,
and there is no new mathematics involved, so this week should be nice and easy.

Let $ x(t) $ be a function describing the position of some object at some time $ t $. Then the velocity of the object,
measured by the rate of change of position, is just $ v(t) = x'(t) $. The acceleration of the object, which is just the
rate of change of velocity, is $ a(t) = v'(t) $. This is summed up in the following table:
\begin{center}
  \def\arraystretch{1.5}
  \begin{tabular}{|rcc|}\hline
    \textbf{Displacement}, $ x $ && $\rint v(t) \dif{t} $\\
    \textbf{Velocity}, $ v $ & $ \od{x}{t} $ & $ \rint a(t) \dif{t} $\\
    \textbf{Acceleration}, $ a $ & $ \od{v}{t} $ &\\\hline
  \end{tabular}
\end{center}

\begin{center}\itshape
  There is only one fundamental concept in this topic that you must remember: the derivative is just a rate of
  change. Velocity is rate of change of position, and acceleration is rate of change of velocity. If you slow down faster,
  your acceleration is more negative.

  \textbf{Do not try to memorise the above table, understand what it is saying physically.}
\end{center}

Note that if you are taking physics, physicists sometimes have an annoying habit of writing $ \dot x $ instead of $ \od{x}{t} $; so
dotted variables may indicate time derivatives. While this notation is horrible and ugly, it is often forced upon us. I will not use
it myself, and you should not expect to ever see it in a mathematics paper again; but be aware that this notation does exist.

\subsection*{A note about L3 physics}
This topic appears in L2 mathematics for two main reasons: firstly, as an easy historical application of calculus; and secondly, so
that you are well-prepared for level three physics. While calculus is not actually required for physics until university, a solid
understanding of this year's calculus topics will enable you to make connections next year that may be otherwise obscure. If you are
planning to do scholarship physics next year (and I urge you to consider it if you enjoy physics), then knowledge of calculus is a
definite advantage as it can simplify some of the problems!

\subsection*{Questions}
All distances are given in \si{\metre}, and all times in \si{\second}, unless otherwise stated.
\begin{questions}
  \question If a ball is thrown into the air with a velocity of \SI{10}{\metre\per\second},
            its height $ y $ in metres after $ t $ seconds is given by
            \begin{equation}
              y = 10t - 4.9t^2.
            \end{equation}
            Find the vertical velocity of the ball when $ t = 2 $.
  \question The velocity $ v $ \si{\metre\per\second} of an object $ t $ seconds after it passes
            a fixed point can be modelled by the function
            \begin{displaymath}
              v(t) = 4t^3 - t^2 + 2t.
            \end{displaymath}
            Find the equation for the acceleration of the object.
  \question A tank is being filled with water. The height of the water, $ h \si{\centi\metre} $, in the tank at any time $ t $
            minutes after it began filling is given by $ h = t^2 + 2t $. Find the rate that the height of the water is changing
            at three minutes after the tank begins to fill.
  \question A balloon has an initial volume \SI{5}{\centi\metre\cubed}, and is being inflated at a rate given by $ \od{V}{t} = 4t $,
            where $ V $ is the volume of the balloon in cubic centimetres and $ t $ is the time in seconds since the balloon began to
            inflate. Give the volume of the balloon after ten seconds.
  \question A projectile follows a path through space modelled by $ y = 4x - x^2 $. At what distance
            along the ground is it at its maximum height, and what is that height?
  \question The distance, $ s $, of a moving point from a given point $ P $ at a time $ t $ is given by
            \begin{displaymath}
              s(t) = \frac{1}{3}t^3 - 2t^2 - 12t.
            \end{displaymath}
            Fully describe the acceleration of the object beginning from time $ t = 0 $.
  \question A car begins to roll slowly down a sloping driveway, with velocity given by $ v = 0.4t\si{\metre\per\second} $ (where $ t $ is the
            time from the beginning of the car's roll). It is five seconds until someone notices the movement; in this time, how far
            does the car travel?
  \question A child moves a Buzzy Bee\textsuperscript{TM} toy forwards and backwards along a straight line. At time $ t $,
            where $ 0 \leq t \leq 10 $, the toy's position is modelled by $ x = 3t - 1.3t^2 + 0.1t^3 $.
    \begin{parts}
      \part At which time(s) is the toy stationary?
      \part What is the acceleration of the toy at $ t = 3 $?
      \part What displacement is the toy from the origin when the velocity of the toy is most negative?
    \end{parts}
  \question The area of a circle varies if the radius $ r $ of the circle is changed. In this way, the area $ A $ of
            the circle is a function of $ r $ given by $ A(r) = \pi r^2 $.
    \begin{parts}
      \part What is the rate of change of the area of the circle with respect to the radius?
      \part A piece of computer graphics software is drawing a circle so that the radius changes
            with time at a rate $ \od{r}{t} = 3t $. The initial radius of the circle is $ \SI{2}{\centi\metre} $.
        \begin{subparts}
          \subpart Give the radius of the circle after three seconds.
          \subpart After how long will the area of the circle reach \SI{10}{\centi\metre\squared}?
        \end{subparts}
    \end{parts}
  \question A particle is moving through space along an axis. Its displacement from the origin
            at any time $ t > 0 $ is given by $ s(t) = t^5 - 38t^4 + 560t^3 - 3982t^2 + 13599t - 17820 $.
    \begin{parts}
      \part Find an expression for the velocity of the particle at time $ t $, $ v(t) $.
      \part At what time is the particle moving with the most speed towards the origin, and how fast will it be moving at that time?
      \part What is the acceleration of the particle at that time?
      \part How many times does the particle change direction after $ t = 0 $?
    \end{parts}
  \question An object is moving on a straight line; the point $ P $ lies on this line. Initially, the object is at point $ P $ and
            has a velocity of \SI{4}{\metre\per\second}. The acceleration of the object at a time $ t $ seconds after it leaves $ P $
            is given by the function $ a(t) = 2 - 6t $. How far from $ P $ is the object after three seconds?
  \question Look at the following table of trigonometric derivatives.
            \begin{center}
              \begin{tabular}{r|l}
                $ f(x) $ & $ f'(x) $\\
                \hline
                $ \sin(x) $ & $ \cos(x) $\\
                $ \cos(x) $ & $ -\sin(x) $\\
                $ \tan(x) $ & $ \frac{1}{\cos^2(x)} $
              \end{tabular}
            \end{center}
            A spring oscillates such that the position of its end after a length
            of time $ t $ is given by $ x = 2\sin(t) $. What is the approximate
            acceleration of its end at $ t = 5 $?
\end{questions}

\end{document}
