\documentclass{exam}
\usepackage[utf8]{inputenc}
\usepackage{lmodern}
\usepackage{microtype}

% \usepackage[parfill]{parskip}
\usepackage[dvipsnames]{xcolor}
\usepackage{amsmath}
\usepackage{amsfonts}
\usepackage{amsthm}
\usepackage{siunitx}
\DeclareSIUnit\year{yr}
\DeclareSIUnit\foot{ft}
\DeclareSIUnit\litre{\liter}

\usepackage{skull}

\usepackage{pgfplots}
\usepgfplotslibrary{polar}
\pgfplotsset{compat=1.11}
\usepgfplotslibrary{statistics}
\usepackage{graphicx}
\usepackage{sidecap}
\sidecaptionvpos{figure}{c}
\usepackage{float}
\usepackage{gensymb}
\usepackage{tkz-euclide}
\usetkzobj{all}
\usepackage{commath}
\usepackage{hyperref}
\usepackage{enumitem}
\usepackage{wasysym}
\usepackage{multicol}
\usepackage{mathtools}
\usepackage{tcolorbox}
\usepackage{tabularx}
\usepackage[version=4]{mhchem}
\usepackage{changepage}
\usepackage{listings}
\lstset{basicstyle=\ttfamily\linespread{0.8}\small}

\renewcommand*{\thefootnote}{\fnsymbol{footnote}}

\newtheorem*{thm}{Theorem}
\newtheorem*{iden}{Identity}
\newtheorem*{lemma}{Lemma}
\newtheorem{obs}{Observation}
\theoremstyle{definition}
\newtheorem*{defn}{Definition}
\newtheorem*{ex}{Example}
\newtheorem{con}{Construction}
\newtheorem*{alg}{Algorithm}

\newtheoremstyle{break}
  {\topsep}{\topsep}%
  {\itshape}{}%
  {\bfseries}{}%
  {\newline}{}%
\theoremstyle{break}
\newtheorem*{bthm}{Theorem}

% russian integral
\usepackage{scalerel}
\DeclareMathOperator*{\rint}{\scalerel*{\rotatebox{17}{$\!\int\!$}}{\int}}

% \DeclareMathOperator*{\rint}{\int}

\pgfplotsset{vasymptote/.style={
    before end axis/.append code={
        \draw[densely dashed] ({rel axis cs:0,0} -| {axis cs:#1,0})
        -- ({rel axis cs:0,1} -| {axis cs:#1,0});
    }
}}

% \pointsinrightmargin
\boxedpoints
\pointname{}

\newcommand{\questioA}{\question[\texttt{\textbf{\color{Cerulean} A}}]}
\newcommand{\questioM}{\question[\texttt{\textbf{\color{PineGreen} M}}]}
\newcommand{\questioE}{\question[\texttt{\textbf{\color{WildStrawberry} E}}]}
\newcommand{\questioS}{\question[\texttt{\textbf{\color{Goldenrod} S}}]}
\newcommand{\questioO}{\question[\texttt{\textbf{\color{BurntOrange} O}}]}

\newcommand{\parA}{\part[\texttt{\textbf{\color{Cerulean} A}}]}
\newcommand{\parM}{\part[\texttt{\textbf{\color{PineGreen} M}}]}
\newcommand{\parE}{\part[\texttt{\textbf{\color{WildStrawberry} E}}]}
\newcommand{\parS}{\part[\texttt{\textbf{\color{Goldenrod} S}}]}
\newcommand{\parO}{\part[\texttt{\textbf{\color{BurntOrange} O}}]}

\newcommand{\subparA}{\subpart[\texttt{\textbf{\color{Cerulean} A}}]}
\newcommand{\subparM}{\subpart[\texttt{\textbf{\color{PineGreen} M}}]}
\newcommand{\subparE}{\subpart[\texttt{\textbf{\color{WildStrawberry} E}}]}
\newcommand{\subparS}{\subpart[\texttt{\textbf{\color{Goldenrod} S}}]}
\newcommand{\subparO}{\subpart[\texttt{\textbf{\color{BurntOrange} O}}]}

\newcommand{\mainHeader}[2]{\section*{NCEA Level 2 Mathematics\\#1. #2}}
\newcommand{\mainHeaderHw}[2]{\section*{NCEA Level 2 Mathematics (Homework)\\#1. #2}}
\newcommand{\seealso}[1]{\begin{center}\emph{See also #1.}\end{center}}
\newcommand{\drills}[1]{\begin{center}\emph{Drill problems: #1.}\end{center}}
\newcommand{\basedon}[1]{\begin{center}\emph{Notes largely based on #1.}\end{center}}


\begin{document}

\mainHeader{6}{Systems of Equations}
Let's go back to last week, where we had the following example of a quadratic model.
\begin{ex}
  The following table gives the instantaneous rates of reaction for the a particular chemical reaction.
  \begin{center}
    \begin{tabular}{|c|c|c|}\hline
      \textbf{Time} (min) & \textbf{Concentration of reactant} (M) & \textbf{Instantaneous reaction rate} (M/min)\\\hline
      0 & 0.0054 &\\\hline
      10 & 0.0044 & \num{8.0e-5}\\\hline
      26 & 0.0034 & \num{5.0e-5}\\\hline
      44 & 0.0027 & \num{3.1e-5}\\\hline
      70 & 0.0020 & \num{1.8e-5}\\\hline
      120 & 0.0014 & \num{8.0e-6}\\\hline
    \end{tabular}
  \end{center}
  It is known that the reaction rate is modelled by a quadratic function of the reaction concentration. Let's call
  the reaction rate at a particular concentration $ R(C) $; so
  \begin{displaymath}
    R(C) = XC^2 + YC + Z.
  \end{displaymath}
  By using the values in the table above, we have that
  \begin{align*}
    \num{8.0e-5} &= X \cdot (0.0044)^2 + Y (0.0044) + Z\\
    \num{5.0e-5} &= X \cdot (0.0034)^2 + Y (0.0034) + Z\\
    \num{3.1e-5} &= X \cdot (0.0027)^2 + Y (0.0027) + Z.
  \end{align*}
\end{ex}
I'll leave off the end, because the point is we used a computer to solve this system of equations. This week, we'll
learn a \emph{systematic} method for solving such systems: not because it's easier than using a computer, but because
it's interesting to see what's going on geometrically. (In my Y13 notes, there's a lot more detail --- and proofs, which
I'll skip this year.)

The main idea to get your head around this year is that a system of equations is \textit{geometric}.

\begin{ex}
  Consider the system of equations
  \begin{displaymath}
    \begin{cases}
      x + y = 7\\
      5x + 10y = 45.
    \end{cases}
  \end{displaymath}
  If we plot all the values $ (x,y) $ which satisfy these equations, we obtain:
  \begin{center}
    \fbox{\begin{tikzpicture}
      \begin{axis}[
        axis lines = center,
        xlabel = $ x $,
        ylabel = $ y $
      ]
        \addplot[domain = 0:5, color = green, samples=100] {7 - x};
        \addplot[domain = 0:5, color = red, samples=100] {9 - 2*x};
      \end{axis}
    \end{tikzpicture}}
  \end{center}
  The key point here is that the point which satisfies both solutions is simply the geometric point of intersection.

  In order to find this point algebraically, we can solve the first equation for $ y $: $ y = 7 - x $. We can then
  substitute this into the other: $ 5x + 10(7 - x) = 45 $. Simplifying, we have
  \begin{gather*}
    5x + 70 - 10x = 45\\
    25 = 5x\\
    5 = x.
  \end{gather*}
  Let's think about this a little more explicitly: since the same pair of values $ (x,y) $ makes both equations
  true, then if we can find $ y $ in terms of $ x $ using one equation then we can substitute it straight into
  the other one \emph{because the symbol $ y $ represents the same thing in both}.
\end{ex}

In another situation, we might have a line and a parabola.

\begin{ex}
  [Extract from the sample L2 assessment for this topic.]

  The student committee is planning the upcoming Performing Arts Showcase at your school this year. They are trying to determine how much they
  should make the price of adult tickets this year.

  Determine the price they should set the adult tickets in 2012 if they want to make a profit of \$3210 from ticket sales, given that:
  \begin{itemize}
    \item The cost of a child ticket is planned to be \$5 and the cost of an adult ticket is not determined yet.
    \item The relationship between the expected number of child tickets and adult tickets to be sold can be modelled by $ 200x + y^2 = 80000 $,
           where $x$ represents the number of child tickets and $y$ represents the number of adult tickets.
    \item The planning committee wants to make \$3210 from ticket sales.
    \item There needs to be only one possibility for the number of child and adult tickets to be sold.
  \end{itemize}

  \textit{Solution.} The total profit that will be made is $ 5x + my $, where $ m $ is the cost of an adult ticket (yet to be determined);
  we therefore have three important pieces of information: $ 5x + my = 3210 $, $ 200x + y^2 = 80000 $, and the curves of these two equations
  intersect at only one point. Solving the first for $ x $, we have $ x = \frac{3210 - my}{5} $; substituting, we have
  \begin{gather*}
    200\left(\frac{3210 - my}{5}\right) + y^2 = 80000\\
    128400 - 40my + y^2 = 80000\\
    48400 + 40my + y^2 = 0
  \end{gather*}
  When we learned about parabolae, we learned that we could transform them into vertex form. Let's do that here, because our goal
  is to find a value $ m $ such that the vertex of $ 48400 + 40my + y^2 = x $ is sitting exacty on the $ y$-axis --- our parabola is flipped
  sideways, but the idea is the same. In fact, to make it clearer let's look at the `graphy' form:\footnote{The idea of switching from looking
  at something in an `equationy' sense and turning something into a parameter is actually quite a powerful idea that's applicable in many
  different situations.}
  \begin{gather*}
    (y + 20m)^2 + 48400 - 400m^2 = x.
  \end{gather*}
  This parabola is sitting on the $ y$-axis when $ 48400 - 400m^2 $ is zero (because this is the $ x$-shift); hence $ m = 11 $,
  and the committee should sell the adult tickets for \$11 each.
\end{ex}

\subsection*{Questions}
\begin{questions}
  \question Graph and solve the system of equations
            \begin{displaymath}
              \begin{cases}
                1 = x + y,\\
                2 = 2x - y.
              \end{cases}
            \end{displaymath}
  \question Source: J. A. H. Hunter, \emph{Mathematical Brain-teasers}. Dover (1976). Page 18, adapted.
    \begin{adjustwidth}{2em}{0em}
      That croc was a freak,\\\qquad not at all the right shape,\\
      With a tail thrice as long\\\qquad\qquad as its head.\\
      ~\\
      Its body was short,\\\qquad far too short for the rest;\\
      Half as long as its tail,\\\qquad\qquad so they said.\\
      ~\\
      The body and tail,\\\qquad measured four metres point one:\\
      An ugly great brute,\\\qquad\qquad you'll agree.\\
      ~\\
      But I never discovered,\\\qquad the length of its head.\\
      Just how long do you think\\\qquad\qquad it would be?
    \end{adjustwidth}
  \question Let us explore how many solutions we can obtain for simple systems. If
            \begin{displaymath}
              \begin{cases}
                a = bx + cy,\\
                p = qx + ry
              \end{cases}
            \end{displaymath}
            is a system of simultaneous equations in $ x $ and $ y $ then depending on the constants we choose there are
            three possible situations:
            \begin{itemize}
              \item We can have no pairs $ (x,y) $ that satisfy both equations.
              \item We can have precisely one pair $ (x,y) $ that satisfies both equations.
              \item We can have infinitely many pairs $ (x,y) $ that satisfy both equations.
            \end{itemize}
            (We will prove this next year.)
    \begin{parts}
      \item Draw a diagram showing each situation geometrically.
      \item Give an example of a system for each case, taking care to show that your systems
            have the desired number of solutions.
    \end{parts}
  \question Graph (using a computer or a calculator) and solve the system of equations
            \begin{displaymath}
              \begin{cases}
                y = 2x + 3,\\
                x^2 + 2xy - 1 = 0.
              \end{cases}
            \end{displaymath}
  \question {[Extract from the sample L2 assessment for this topic.]} For the performing arts showcase example above, you must
            also provide the planning committee with the change in the adult tickets sales in 2011 compared to 2010. You are
            given that:
            \begin{itemize}
              \item \textbf{2010 Performing Arts Showcase}
                \begin{itemize}
                  \item The total number of tickets sold was 400.
                  \item The relationship between the number of child tickets and adult tickets sold can be modelled by $ x^2 + y = 22750 $, where $x$
                        represents the number of child tickets and $y$ represents the number of adult tickets.
                \end{itemize}
              \item \textbf{2011 Performing Arts Showcase}
                \begin{itemize}
                  \item The cost of a child ticket was \$5 and the cost of an adult ticket was \$10.
                  \item The relationship between the number of child tickets and adult tickets sold can be modelled by $ xy = 1000 + 100x $,
                        where $x$ represents the number of child tickets and $y$ represents the number of adult tickets.
                  \item More than 300 tickets were sold.
                  \item The money generated from ticket sales was \$2050.
                \end{itemize}
            \end{itemize}
  \question For which values of $ c $ does the system of equations
            \begin{displaymath}
              \begin{cases}
                y^2 = 2x^2 + xy\\
                y = cx - 2
              \end{cases}
            \end{displaymath}
            have precisely two solutions?
  \question Find the parabola passing through the three points $ (0,0) $, $ (1,1) $, and $ (3,0) $.
  \question Does there exist a parabola through the three points $ (1,1) $, $ (2,-1) $, and $ (3,3) $? If so, find its equation,
            and calculate its vertex and $ x$--intercepts.
  \question The graph of the equation $ xy = a $ (where $ a $ is a number) is called a \emph{hyperbola}.
    \begin{parts}
      \part Show that $ xy = a $ and $ xy = b $ never intersect if $ a \neq b $.
      \part Find the point of intersection between the curves
            \begin{displaymath}
              \begin{cases}
                x(y+1) = 4\\
                y = 2x - 3.
              \end{cases}
            \end{displaymath}
    \end{parts}
  \question The line $ y = 2x - 3 $ intersects the circle $ x^2 - 6x + y^2 = 0 $ precisely twice.
    \begin{parts}
      \part Find the coordinates of the centre of the circle.
      \part Find the points of intersection.
      \part How could the constant term $ 3 $ of the line be changed such that the line becomes tangent with the circle?
    \end{parts}
\end{questions}

\end{document}
