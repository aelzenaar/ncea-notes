\documentclass{exam}
\usepackage[utf8]{inputenc}
\usepackage{lmodern}
\usepackage{microtype}

% \usepackage[parfill]{parskip}
\usepackage[dvipsnames]{xcolor}
\usepackage{amsmath}
\usepackage{amsfonts}
\usepackage{amsthm}
\usepackage{siunitx}
\DeclareSIUnit\year{yr}
\DeclareSIUnit\foot{ft}
\DeclareSIUnit\litre{\liter}

\usepackage{skull}

\usepackage{pgfplots}
\usepgfplotslibrary{polar}
\pgfplotsset{compat=1.11}
\usepgfplotslibrary{statistics}
\usepackage{graphicx}
\usepackage{sidecap}
\sidecaptionvpos{figure}{c}
\usepackage{float}
\usepackage{gensymb}
\usepackage{tkz-euclide}
\usetkzobj{all}
\usepackage{commath}
\usepackage{hyperref}
\usepackage{enumitem}
\usepackage{wasysym}
\usepackage{multicol}
\usepackage{mathtools}
\usepackage{tcolorbox}
\usepackage{tabularx}
\usepackage[version=4]{mhchem}
\usepackage{changepage}
\usepackage{listings}
\lstset{basicstyle=\ttfamily\linespread{0.8}\small}

\renewcommand*{\thefootnote}{\fnsymbol{footnote}}

\newtheorem*{thm}{Theorem}
\newtheorem*{iden}{Identity}
\newtheorem*{lemma}{Lemma}
\newtheorem{obs}{Observation}
\theoremstyle{definition}
\newtheorem*{defn}{Definition}
\newtheorem*{ex}{Example}
\newtheorem{con}{Construction}
\newtheorem*{alg}{Algorithm}

\newtheoremstyle{break}
  {\topsep}{\topsep}%
  {\itshape}{}%
  {\bfseries}{}%
  {\newline}{}%
\theoremstyle{break}
\newtheorem*{bthm}{Theorem}

% russian integral
\usepackage{scalerel}
\DeclareMathOperator*{\rint}{\scalerel*{\rotatebox{17}{$\!\int\!$}}{\int}}

% \DeclareMathOperator*{\rint}{\int}

\pgfplotsset{vasymptote/.style={
    before end axis/.append code={
        \draw[densely dashed] ({rel axis cs:0,0} -| {axis cs:#1,0})
        -- ({rel axis cs:0,1} -| {axis cs:#1,0});
    }
}}

% \pointsinrightmargin
\boxedpoints
\pointname{}

\newcommand{\questioA}{\question[\texttt{\textbf{\color{Cerulean} A}}]}
\newcommand{\questioM}{\question[\texttt{\textbf{\color{PineGreen} M}}]}
\newcommand{\questioE}{\question[\texttt{\textbf{\color{WildStrawberry} E}}]}
\newcommand{\questioS}{\question[\texttt{\textbf{\color{Goldenrod} S}}]}
\newcommand{\questioO}{\question[\texttt{\textbf{\color{BurntOrange} O}}]}

\newcommand{\parA}{\part[\texttt{\textbf{\color{Cerulean} A}}]}
\newcommand{\parM}{\part[\texttt{\textbf{\color{PineGreen} M}}]}
\newcommand{\parE}{\part[\texttt{\textbf{\color{WildStrawberry} E}}]}
\newcommand{\parS}{\part[\texttt{\textbf{\color{Goldenrod} S}}]}
\newcommand{\parO}{\part[\texttt{\textbf{\color{BurntOrange} O}}]}

\newcommand{\subparA}{\subpart[\texttt{\textbf{\color{Cerulean} A}}]}
\newcommand{\subparM}{\subpart[\texttt{\textbf{\color{PineGreen} M}}]}
\newcommand{\subparE}{\subpart[\texttt{\textbf{\color{WildStrawberry} E}}]}
\newcommand{\subparS}{\subpart[\texttt{\textbf{\color{Goldenrod} S}}]}
\newcommand{\subparO}{\subpart[\texttt{\textbf{\color{BurntOrange} O}}]}

\newcommand{\mainHeader}[2]{\section*{NCEA Level 2 Mathematics\\#1. #2}}
\newcommand{\mainHeaderHw}[2]{\section*{NCEA Level 2 Mathematics (Homework)\\#1. #2}}
\newcommand{\seealso}[1]{\begin{center}\emph{See also #1.}\end{center}}
\newcommand{\drills}[1]{\begin{center}\emph{Drill problems: #1.}\end{center}}
\newcommand{\basedon}[1]{\begin{center}\emph{Notes largely based on #1.}\end{center}}

\begin{document}

\mainHeaderIntg{24}{Formal Definitions of $ \exp $ and $ \ln $}
Today we will write down formal definitions of the natural exponential and logarithm functions.

\subsection*{The Natural Logarithm}
We can integrate $ x^3 $, $ x^2 $, $ x^{-2017} $, and $ x^{0} $ easily using the reverse power rule
(respectively, they become $ x^4/4 $, $ x^3/3 $, $ -x^{-2016}/2016 $, and $ 0 $). However, we cannot
integrate $ x^{-1} $:
\begin{displaymath}
  \rint x^{-1} \dif{x} = \frac{1}{-1 + 1} x^{-1 + 1} = \frac{1}{0} x^0 =\thinspace ???
\end{displaymath}
But we know that the indefinite integral of $ x^{-1} $ must exist, since there is obviously a
finite area beneath the curve $ y = x^{-1} $ over (for example) the interval $ [1,2] $. We therefore
make the following \textbf{definition}:
\begin{displaymath}
  \ln x := \rint^x_1 \frac{1}{t} \dif{t}.
\end{displaymath}

\paragraph{How does one pronounce `$ ln $'?} \texttt{log}.

We can prove the log rules using this definition; as a sampler, we prove that $ \ln x + \ln y = \ln xy $:
\begin{align*}
  \ln x + \ln y &= \rint^x_1 \frac{1}{t} \dif{t} + \rint^y_1 \frac{1}{t} \dif{t}\\
                &= \rint^x_1 \frac{1}{t} \dif{t} + \rint^{xy}_{x} \frac{1}{u/x} \frac{1}{x} \dif{u} \quad\text{($u = tx $)}\\
                &= \rint^x_1 \frac{1}{t} \dif{t} + \rint^{xy}_{x} \frac{1}{u} \dif{u}\\
                &= \rint^x_1 \frac{1}{t} \dif{t} + \rint^{xy}_{x} \frac{1}{t} \dif{t}\\
                &= \rint^{xy}_1 \frac{1}{t} \dif{t}\\
                &= \ln xy.
\end{align*}

\newpage
\subsection*{The Exponential Function}
We define the exponential function to be the function implicitly defined by $ \exp(x) = y \iff \ln(y) = x $. Consider the following (where $ y = \exp(x) $):
\begin{align*}
  \od{y}{x} = \frac{1}{\od{x}{y}} = \frac{1}{1/y} = y
\end{align*}
So $ \od{}{x} \exp(x) = \exp(x). $

We can also write $ \exp(x) \equiv e^x $.

Note that $ e $ is the unique number such that $ \ln(e) =  1 $:
\begin{displaymath}
  \ln(e) = \ln(e^1) = \ln(\exp(1)) = 1.
\end{displaymath}

\subsection*{Questions}
\begin{questions}
  \questioS Prove the following:
    \begin{parts}
      \part $ \ln 1 = 0 $
      \part $ \ln x > 0 $ for all $ x > 1 $.
      \part $ \ln x < 0 $ for all $ 0 < x < 1 $.
      \part $ a\ln x = \ln(x^a) $ (Remember, multiplication is just repeated addition.)
      \part $ \ln x - \ln y = \ln(x/y) $
      \part $ e^{x + y} = e^x e^y $
      \part $ e^{x - y} = e^x / e^y $
      \part $ e^0 = 1 $
      \part $ (e^{x})^y = e^{xy} $
    \end{parts}
\end{questions}

\end{document}

