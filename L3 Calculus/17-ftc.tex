\documentclass{exam}
\usepackage[utf8]{inputenc}
\usepackage{lmodern}
\usepackage{microtype}

% \usepackage[parfill]{parskip}
\usepackage[dvipsnames]{xcolor}
\usepackage{amsmath}
\usepackage{amsfonts}
\usepackage{amsthm}
\usepackage{siunitx}
\DeclareSIUnit\year{yr}
\DeclareSIUnit\foot{ft}
\DeclareSIUnit\litre{\liter}

\usepackage{skull}

\usepackage{pgfplots}
\usepgfplotslibrary{polar}
\pgfplotsset{compat=1.11}
\usepackage{graphicx}
\usepackage{sidecap}
\sidecaptionvpos{figure}{c}
\usepackage{float}
\usepackage{gensymb}
\usepackage{tkz-euclide}
\usetkzobj{all}
\usepackage{commath}
\usepackage{hyperref}
\usepackage{enumitem}
\usepackage{wasysym}
\usepackage{multicol}
\usepackage{mathtools}
\usepackage{tcolorbox}
\usepackage{tabularx}
\usepackage[version=4]{mhchem}
\usepackage{changepage}
\usepackage{listings}
\lstset{basicstyle=\ttfamily\linespread{0.8}\small}

\renewcommand*{\thefootnote}{\fnsymbol{footnote}}

\newtheorem*{thm}{Theorem}
\newtheorem*{iden}{Identity}
\newtheorem*{lemma}{Lemma}
\newtheorem{obs}{Observation}
\theoremstyle{definition}
\newtheorem*{defn}{Definition}
\newtheorem*{ex}{Example}
\newtheorem{con}{Construction}
\newtheorem*{alg}{Algorithm}

\newtheoremstyle{break}
  {\topsep}{\topsep}%
  {\itshape}{}%
  {\bfseries}{}%
  {\newline}{}%
\theoremstyle{break}
\newtheorem*{bthm}{Theorem}

% russian integral
\usepackage{scalerel}
\DeclareMathOperator*{\rint}{\scalerel*{\rotatebox{17}{$\!\int\!$}}{\int}}

% \DeclareMathOperator*{\rint}{\int}

\pgfplotsset{vasymptote/.style={
    before end axis/.append code={
        \draw[densely dashed] ({rel axis cs:0,0} -| {axis cs:#1,0})
        -- ({rel axis cs:0,1} -| {axis cs:#1,0});
    }
}}

% \pointsinrightmargin
\boxedpoints
\pointname{}

\newcommand{\questioA}{\question[\texttt{\textbf{\color{Cerulean} A}}]}
\newcommand{\questioM}{\question[\texttt{\textbf{\color{PineGreen} M}}]}
\newcommand{\questioE}{\question[\texttt{\textbf{\color{WildStrawberry} E}}]}
\newcommand{\questioS}{\question[\texttt{\textbf{\color{Goldenrod} S}}]}
\newcommand{\questioO}{\question[\texttt{\textbf{\color{BurntOrange} O}}]}

\newcommand{\parA}{\part[\texttt{\textbf{\color{Cerulean} A}}]}
\newcommand{\parM}{\part[\texttt{\textbf{\color{PineGreen} M}}]}
\newcommand{\parE}{\part[\texttt{\textbf{\color{WildStrawberry} E}}]}
\newcommand{\parS}{\part[\texttt{\textbf{\color{Goldenrod} S}}]}
\newcommand{\parO}{\part[\texttt{\textbf{\color{BurntOrange} O}}]}

\newcommand{\subparA}{\subpart[\texttt{\textbf{\color{Cerulean} A}}]}
\newcommand{\subparM}{\subpart[\texttt{\textbf{\color{PineGreen} M}}]}
\newcommand{\subparE}{\subpart[\texttt{\textbf{\color{WildStrawberry} E}}]}
\newcommand{\subparS}{\subpart[\texttt{\textbf{\color{Goldenrod} S}}]}
\newcommand{\subparO}{\subpart[\texttt{\textbf{\color{BurntOrange} O}}]}

\newcommand{\mainHeader}[2]{\section*{NCEA Level 2 Mathematics\\#1. #2}}
\newcommand{\mainHeaderHw}[2]{\section*{NCEA Level 2 Mathematics (Homework)\\#1. #2}}

\begin{document}

\mainHeaderIntg{17}{The Fundamental Theorem of Calculus}
\begin{bthm}[First Fundamental Theorem]
  Suppose $ f $ is a continuous function, and suppose $ F $ is any antiderivative of $ f $ (so $ F' = f $). Then:
  \begin{displaymath}
    \rint^b_a f(x) \dif{x} = F(b) - F(a) = \eval{F(x)}_a^b
  \end{displaymath}
\end{bthm}

In other words, the definite integral of a function can be found by evaluating the indefinite integrals at
the endpoints.

\begin{bthm}[Second Fundamental Theorem]
  Suppose $ f $ is a continuous function. Then:
  \begin{displaymath}
    \od{}{x} \rint^x_a f(t) \dif{t} = f(x)
  \end{displaymath}
\end{bthm}

\begin{center}
  \includegraphics[width=0.5\textwidth]{ftc1}
\end{center}

The rate of change of the area under a curve is simply the height of the curve.

We also have the following theorem:
\begin{thm}
  Suppose $ f,g $ are functions and $ \lambda $ is a real constant. Then:
  \begin{enumerate}
    \item $ \lambda \rint_a^b f(x) \dif{x} = \rint_a^b \lambda f(x) \dif{x} $.
    \item $ \rint_a^b f(x) \dif{x} + \rint_a^b g(x) \dif{x} = \rint_a^b f(x) + g(x) \dif{x} $.
    \item $ \rint_a^a f(x) \dif{x} = 0 $.
    \item $ \rint_a^b f(x) \dif{x} + \rint_b^c f(x) \dif{x} = \rint_a^c f(x) \dif{x} $.
  \end{enumerate}
\end{thm}

Note that the areas below a curve are assigned \textbf{\textcolor{RubineRed}{NEGATIVE AREA}}.

\begin{ex}
  Note that the definite integral $ \rint^\pi_0 \cos(x) \dif{x} $ is equal to 0; this is because the (negative) area under the $ x-$axis exactly
  cancels the (positive) area above the $ x-$axis. (Draw a picture.)
\end{ex}

\subsection*{Questions}
\begin{questions}
  \questioA Compute the following definite integrals.
    \begin{parts}
      \part $ \rint_0^1 \dif{x} $
      \part $ \rint_{-1}^{1} e^x \dif{x} $
      \part $ \rint_3^4 x^2 + 3x - 1 \dif{x} $
      \part $ \rint_0^1 x^n \dif{x} $ for integer values of $ n $.
    \end{parts}
  \questioA Find the area underneath the given curves between the given bounds:
    \begin{parts}
      \part $ y = 6x^2 + 4x + 9 $ between $ x = 0 $ and $ x = 4 $
      \part $ y = \sin x $ between $ x = 0 $ and $ x = \pi $
      \part $ y = \sin x $ between $ x = -\pi $ and $ x = \pi $
      \part $ y = \cos x $ between $ x = -\pi $ and $ x = \pi $
      \part $ y = \frac{1}{x} $ between $ x = 1 $ and $ x = 2 $
    \end{parts}
  \questioA Find all the problems in the following working.
            \begin{align*}
              \rint^{-1}_1 \frac{\dif{x}}{x} = \ln \abs{-1} - \ln \abs{1} = 0
            \end{align*}
  \questioA Show that $ \rint \ln x \dif{x} = x\ln x - x + C $.
  \questioM Let $ f $ be a function such that for all $ x $, $ f(-x) = -f(x) $. Such a function is called \textit{odd}. Show that for all $ a $,
            \begin{displaymath}
              \rint^a_{-a} f(x) \dif{x} = 0.
            \end{displaymath}
            What does this mean geometrically?
  \questioM Let $ f $ be an odd function with period 2 such that $ \rint^{1}_0 f(x) \dif{x} = k $. Compute:
    \begin{parts}
      \part $ \rint^{1}_{-1} f(x) \dif{x} $
      \part $ \rint^{-1}_0 f(x) \dif{x} $
    \end{parts}
  \questioA Let $ f $ be a function such that for all $ x $, $ f(-x) = f(x) $. Such a function is called \textit{even}. Show that for all $ a $,
            \begin{displaymath}
              \rint^a_{-a} f(x) \dif{x} = 2\rint^a_0 f(x) \dif{x}.
            \end{displaymath}
            What does this mean geometrically?
  \questioM If $ \rint^1_{-2} f(x) \dif{x} = 2 $ and $ \rint^3_1 f(x) \dif{x} = -6 $, what is the value of $ \rint_{-2}^3 f(x) \dif{x} $?
  \clearpage
  \questioM Find the area between the curves $ y = x^2 + x $ and $ y = -x^2 - x $ shaded here.
            \begin{center}
              \includegraphics[width=0.3\textwidth]{int1}
            \end{center}
  \questioM Find the area between the two curves $ y = 1 + x^2 $ and $ y = 3 + x $.
  \questioM Find the area of the region bounded by $ f(x) = 4 $, $ g(x) = \frac{e^x}{5} $, and $ x = 0 $.
  \questioM What is the area of the region between the graphs of $ f(x) = 2x^2  + 5x $ and $ g(x) = -x^2 - 6x + 4 $ from $ x = -4 $ to $ x = 0 $?
  \questioE Find the area bounded by the curves $ y = \sin x $ and $ y = \cos x $ and the $ x-$axis graphed here.
            \begin{center}
              \includegraphics[width=0.2\textwidth]{int2}
            \end{center}
  \questioA Consider the function $ f $ graphed below; the total \textbf{unsigned} area between the curve and the $ x-$axis is 10 square units.
            Find $ \rint^D_A f(x) \dif{x} $.
            \begin{center}
              \includegraphics[width=0.3\textwidth]{int3}
            \end{center}
  \questioM
    \begin{parts}
      \part Sketch the graph of $ y = \abs{\sin x} $.
      \part Compute $ \rint^{\pi/2}_0 y \dif{x} $ using the FTC.
      \part Hence, without doing any anti-differentiation, compute $ \rint^{2\pi}_0 y \dif{x} $.
    \end{parts}
  \questioM Define $ F(x) $ by
            \begin{displaymath}
              F(x) = \rint^x_{\frac{\pi}{4}} \cos(2t) \dif{t}.
            \end{displaymath}
    \begin{parts}
      \part Use the Second Fundamental Theorem of Calculus to find $ F'(x) $.
      \part Verify part (a) by integration and differentiation.
    \end{parts}
  \questioM Compute $ \od{}{x} \rint^{x}_{2} t^t \dif{t} $.
\end{questions}
\end{document}
