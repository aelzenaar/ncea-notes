\documentclass{exam}
\usepackage[utf8]{inputenc}
\usepackage{lmodern}
\usepackage{microtype}

% \usepackage[parfill]{parskip}
\usepackage[dvipsnames]{xcolor}
\usepackage{amsmath}
\usepackage{amsfonts}
\usepackage{amsthm}
\usepackage{siunitx}
\DeclareSIUnit\year{yr}
\DeclareSIUnit\foot{ft}
\DeclareSIUnit\litre{\liter}

\usepackage{skull}

\usepackage{pgfplots}
\usepgfplotslibrary{polar}
\pgfplotsset{compat=1.11}
\usepackage{graphicx}
\usepackage{sidecap}
\sidecaptionvpos{figure}{c}
\usepackage{float}
\usepackage{gensymb}
\usepackage{tkz-euclide}
\usetkzobj{all}
\usepackage{commath}
\usepackage{hyperref}
\usepackage{enumitem}
\usepackage{wasysym}
\usepackage{multicol}
\usepackage{mathtools}
\usepackage{tcolorbox}
\usepackage{tabularx}
\usepackage[version=4]{mhchem}
\usepackage{changepage}
\usepackage{listings}
\lstset{basicstyle=\ttfamily\linespread{0.8}\small}

\renewcommand*{\thefootnote}{\fnsymbol{footnote}}

\newtheorem*{thm}{Theorem}
\newtheorem*{iden}{Identity}
\newtheorem*{lemma}{Lemma}
\newtheorem{obs}{Observation}
\theoremstyle{definition}
\newtheorem*{defn}{Definition}
\newtheorem*{ex}{Example}
\newtheorem{con}{Construction}
\newtheorem*{alg}{Algorithm}

\newtheoremstyle{break}
  {\topsep}{\topsep}%
  {\itshape}{}%
  {\bfseries}{}%
  {\newline}{}%
\theoremstyle{break}
\newtheorem*{bthm}{Theorem}

% russian integral
\usepackage{scalerel}
\DeclareMathOperator*{\rint}{\scalerel*{\rotatebox{17}{$\!\int\!$}}{\int}}

% \DeclareMathOperator*{\rint}{\int}

\pgfplotsset{vasymptote/.style={
    before end axis/.append code={
        \draw[densely dashed] ({rel axis cs:0,0} -| {axis cs:#1,0})
        -- ({rel axis cs:0,1} -| {axis cs:#1,0});
    }
}}

% \pointsinrightmargin
\boxedpoints
\pointname{}

\newcommand{\questioA}{\question[\texttt{\textbf{\color{Cerulean} A}}]}
\newcommand{\questioM}{\question[\texttt{\textbf{\color{PineGreen} M}}]}
\newcommand{\questioE}{\question[\texttt{\textbf{\color{WildStrawberry} E}}]}
\newcommand{\questioS}{\question[\texttt{\textbf{\color{Goldenrod} S}}]}
\newcommand{\questioO}{\question[\texttt{\textbf{\color{BurntOrange} O}}]}

\newcommand{\parA}{\part[\texttt{\textbf{\color{Cerulean} A}}]}
\newcommand{\parM}{\part[\texttt{\textbf{\color{PineGreen} M}}]}
\newcommand{\parE}{\part[\texttt{\textbf{\color{WildStrawberry} E}}]}
\newcommand{\parS}{\part[\texttt{\textbf{\color{Goldenrod} S}}]}
\newcommand{\parO}{\part[\texttt{\textbf{\color{BurntOrange} O}}]}

\newcommand{\subparA}{\subpart[\texttt{\textbf{\color{Cerulean} A}}]}
\newcommand{\subparM}{\subpart[\texttt{\textbf{\color{PineGreen} M}}]}
\newcommand{\subparE}{\subpart[\texttt{\textbf{\color{WildStrawberry} E}}]}
\newcommand{\subparS}{\subpart[\texttt{\textbf{\color{Goldenrod} S}}]}
\newcommand{\subparO}{\subpart[\texttt{\textbf{\color{BurntOrange} O}}]}

\newcommand{\mainHeader}[2]{\section*{NCEA Level 2 Mathematics\\#1. #2}}
\newcommand{\mainHeaderHw}[2]{\section*{NCEA Level 2 Mathematics (Homework)\\#1. #2}}

\begin{document}

\mainHeaderDiffHw{2}{Limits}
\subsection*{Reading}
You may be wondering why we bother introducing the concept of limits: after all, we are simply replacing one handwavy picture-based definition (that
of the derivative) with another! I will give the answer in two parts:
\begin{enumerate}
  \item Limits are a more general and hence more useful concept; and
  \item It is much easier to formally define a limit than a derivative.
\end{enumerate}

\subsubsection*{Limits are more general}
The obvious use of limit notation this year is to `plug gaps' in functions; however, we can also (as you have seen) take limits of things towards
infinity. This allows us to formalise things like infinite sums: we define the value of an infinite sum to be a special kind of limit.

Limiting situations come up surprisingly often in physics and chemistry as well, if we want to look at the behaviour of a system in the long term:
say the concentration of a particular compound in solution can be modelled by $ C(t) = \frac{k}{t^2} $; then, if we wait a long time (i.e. let $ t \to \infty $),
we predict that the concentration becomes negligible.

\subsubsection*{It is easier to formally define a limit}
Suppose that we have some function $ f $ such that
\begin{displaymath}
  \lim_{x \to a} f(x) = L.
\end{displaymath}

All we are saying here is (intuitively) that we can make the value of $ f $ \emph{as close as we like} to being $ L $, by taking $ x $ to be sufficiently close
to $ a $. I will not state the formal definition here (it is easy enough to find), except to state that it is a little stricter than this intuitive statement
suggests (i.e. for all $ x \neq a $ within the interval $ (a - \delta, a + \delta) $ we must have $ f(x) $ be within $ (L - \varepsilon, L + \varepsilon) $).

\begin{center}
  \includegraphics[width=0.5\textwidth]{limit}
\end{center}

\clearpage
\subsection*{Questions}
Derivatives and limits allow us to classify functions and their behaviour. Consider the following:
\subsubsection*{Properties of Functions}
\begin{itemize}
  \item A function is \textbf{increasing} if its derivative is positive.
  \item A function is \textbf{decreasing} if its derivative is negative.
  \item A function is \textbf{concave down} if its derivative is decreasing.
  \item A function is \textbf{concave up} if its derivative is increasing.
  \item A function $ f $ is \textbf{continuous} at a point $ a $ if $ \lim_{x \to a} f(x) = f(a) $.
\end{itemize}

\begin{questions}
  \question Describe all the function properties given above geometrically, and give an example of each.
  \question Consider the function graphed below.

            \begin{center}
              \includegraphics[width=0.5\textwidth]{limits3}
            \end{center}
    \begin{parts}
      \item Find $ \lim_{x \to -2} f(x) $ and $ \lim_{x \to 2} f(x) $.
      \item Does $ \lim_{x \to -3} f(x) $ exist? Why/why not?
      \item Does $ \lim_{x \to 0} f(x) $ exist? Why/why not?
      \item On what intervals is $ f(x) $ continuous?
      \item At what points is $ f(x) $ not differentiable?
    \end{parts}
  \question On an axis, sketch a graph of some function $ f $ that has the following features:
            \begin{itemize}[noitemsep]
              \item Is continuous for $ 0 < x < 5 $ and $ 5 < x < 9 $ and is discontinuous when $ x = 5 $
              \item Is concave down ($f''(x) < 0 $) for $ 0 < x < 5 $
              \item Has $ f'(x) = 0 $ at $ (3, 8) $
              \item Has $ \lim_{x \to 5} f(x) = 6 $.
              \item Is not differentiable at $ (7, 3) $.
            \end{itemize}
\end{questions}
\end{document}
