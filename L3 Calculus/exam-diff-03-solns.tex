\documentclass[a4paper]{report}

\usepackage{commath}
\usepackage{siunitx}

% russian integral
\usepackage{scalerel}
\DeclareMathOperator*{\rint}{\scalerel*{\rotatebox{17}{$\!\int\!$}}{\int}}

\title{Solutions to L3 Calculus Differentiation Exam 3}
\author{Alexander Elzenaar}
\date{16 November 2017}

\begin{document}

\maketitle

\section*{Question One}
\subsection*{Part (a)}
\paragraph{i.}
$ f'(x) = -\frac{1}{2\sqrt{x}} e^{-\sqrt{x}} $ (or accept in form with indices) (1 mark)

\paragraph{ii.}
$ g'(x) = \dfrac{\frac{2}{3} \ln x (x + 1)^{-2/3} - \frac{2}{x} (x+1)^{1/3}}{\ln^2 x} $ (or simplified) (1 mark)

\subsection*{Part (b)}
\paragraph{i.}
$ v_x(t) = 2\cos 2t $, $ v_y(t) = (1 + 2\cos 2t)\cos(t + \sin 2t) $ so
\begin{displaymath}
  v(t) = \left(2\cos 2t, (1 + 2\cos 2t)\cos(t + \sin 2t)\right)
\end{displaymath}
(3 marks)

\paragraph{ii.}
\begin{displaymath}
  \dod{y}{x} = \frac{(1 + 2\cos 2t)\cos(t + \sin 2t)}{2\cos 2t}
\end{displaymath}

This derivative represents the instantaneous change in the $ y$-ordinate of the particle as its $ x$-ordinate varies. (3 marks)

\section*{Question Two}
\subsection*{Part (a)}
First, we have $ \od{y}{x} 5y^4 + 8xy + 4x^2 \od{y}{x} - 3x^2 + 6x^2 y + 2x^3 \od{y}{x} = 0 $,
so it follows that
\begin{displaymath}
  \dod{y}{x} = -\frac{8xy + 3x^2 + 6 x^2 y}{5y^4 + 4x^2 + 2x^3}
\end{displaymath}
and at the given point the tangent line has slope $ m = \frac{37}{29} $. Hence the tangent line is described by
\begin{displaymath}
  (y - 1) = \frac{37}{29}(x + 5) \Rightarrow y = \frac{37}{29}x + \frac{214}{19}.
\end{displaymath} (3 marks)

\subsection*{Part (b)}
\paragraph{i.}
We have $ \od{V}{t} = -0.2 $. Then $ \frac{R}{r} = \frac{H}{h} $ so $ r = \frac{Rh}{H} $ and $ V(h) = \frac{\pi}{3} r^2 h = \frac{R^2\pi}{3H^2} h^3 $.
Hence $ \od{V}{h} = \frac{R^2\pi}{H^2} h^2 $, $ \od{h}{t} = - \frac{0.2 H^2}{R^2 h^2 \pi} $, so at $ h = 3 $ the depth of water is decreasing
at a rate of \SI{0.0442}{\metre\per\second}. (3 marks)

\paragraph{ii.}
Let the rate of pumping be $ k $. We now have $ \od{V}{t} = k - 0.2 $, so $ \od{h}{t} = \frac{H^2 (k - 0.2)}{R^2 h^2 \pi} $
and at $ h = 2 $, $ 0.1 = 0.4974(k - 0.2) $ and $ k = \SI{0.40}{\metre\cubed\per\second} $. (2 marks)

\section*{Question Three}
\subsection*{Part (a)}
Since $ \varphi(x) $ is undefined for all $ x < 0 $, we cannot take the limit $ \lim_{x \to 0} \varphi(x) $ as the
function cannot tend to a single value from both sides. (2 marks)

\subsection*{Part (b)}
Let the three sides of the triangle be $ x $, $ \frac{1}{2}(P - x) $, and $ \frac{1}{2}(P - x) $. Then the height of the triangle is given by
\begin{displaymath}
  h = \sqrt{\left(\frac{P - x}{2}\right)^2 - \left(\frac{x}{2}\right)^2} = \frac{1}{2}\sqrt{P^2 - 2Px}
\end{displaymath}
and the area is
\begin{displaymath}
  A = \frac{1}{2} xh = \frac{1}{4} \sqrt{P^2 x^2 - 2Px^3}.
\end{displaymath}
We wish to maximise this, so we take the derivative:
\begin{displaymath}
  \od{A}{x} = \frac{P^2 x - 3P x^2}{4\sqrt{P^2 x^2 - 2Px^3}}.
\end{displaymath}
Setting to zero, we have $ P^2 - 3Px = 0 $ and so $ x = \frac{P}{3} $; so the triangle needs to be equilateral.



\subsection*{Part (c)}
Profit $ P $ is income minus costs. Total income is $ Dc $, total cost is $ 5 + \frac{5D}{2} $, so:
\begin{displaymath}
  P = D\left(c - \frac{5}{2}\right) - 5 = 30\left(c - \frac{5}{2}\right)e^{-c/2} - 5
\end{displaymath}
Taking the derivative, $ \od{P}{c} = 30e^{-c/2} - 15\left(c - \frac{5}{2}\right)e^{-c/2} $.
Hence $ 30 = 15(c - \frac{5}{2}) $ and $ c = \$4.50 $. (3 marks)


\end{document}
