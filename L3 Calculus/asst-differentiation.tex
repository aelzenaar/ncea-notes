\documentclass[answers]{exam}
\usepackage[utf8]{inputenc}

\usepackage[parfill]{parskip}
\usepackage[dvipsnames]{xcolor}
\usepackage{amsmath}
\usepackage{amsfonts}
\usepackage{amsthm}
\usepackage{microtype}
\usepackage{siunitx}
\DeclareSIUnit\year{yr}
\usepackage{pgfplots}
\usepackage{graphicx}
\usepackage{sidecap}
\sidecaptionvpos{figure}{c}
\usepackage{float}
\usepackage{gensymb}
\usepackage{tkz-euclide}
\usetkzobj{all}
\usepackage{commath}
\usepackage{hyperref}
\usepackage{enumitem}
\usepackage{wasysym}
\usepackage[version=4]{mhchem}

\renewcommand*{\thefootnote}{\fnsymbol{footnote}}

\newtheorem*{thm}{Theorem}
\newtheorem*{iden}{Identity}
\newtheorem*{lemma}{Lemma}
\theoremstyle{definition}
\newtheorem*{defn}{Definition}
\newtheorem*{ex}{Example}

\newtheoremstyle{break}
  {\topsep}{\topsep}%
  {\itshape}{}%
  {\bfseries}{}%
  {\newline}{}%
\theoremstyle{break}
\newtheorem*{bthm}{Theorem}

% russian integral
\usepackage{scalerel}
\DeclareMathOperator*{\rint}{\scalerel*{\rotatebox{17}{$\!\int\!$}}{\int}}

\pgfplotsset{vasymptote/.style={
    before end axis/.append code={
        \draw[densely dashed] ({rel axis cs:0,0} -| {axis cs:#1,0})
        -- ({rel axis cs:0,1} -| {axis cs:#1,0});
    }
}}

% \qformat{Question \thequestion: \thequestiontitle\hfill}

\begin{document}

\section*{NCEA Level 3 Calculus\\Differentiation Assignment}
\begin{questions}
  \question
    \begin{parts}
      \part[2] If $ y = 3x^2 + 2x - \frac{1}{\sqrt{x + 1}} + \frac{e^x}{\sin x} $, find $ \od{y}{x} $.
      \part An important mathematical skill is the ability to write down examples of objects satisfying certain properties.
        \begin{subparts}
          \subpart[2] Draw the graph of a function $ f $ passing through $ (0,1) $ such that $ f'(x) < 0 $ for all $ x $, but $ f''(x) > 0 $ for all $ x $.
          \subpart[1] Give an explicit, simple example of such a function.
        \end{subparts}
      \part[3] Show that a solution to the differential equation
            \begin{displaymath}
              \od{x}{t} = rx(1-x)
            \end{displaymath}
            is given by
            \begin{displaymath}
              x(t) = \frac{1}{1 + \left(\frac{1}{x_0} - 1\right) e^{-rt}}.
            \end{displaymath}
    \end{parts}
  \question Let $ f $ be the function defined by
            \begin{displaymath}
              f(x) = \sin(\tan(x + \pi/6) + \cos(4\pi))\ln(1/x).
            \end{displaymath}
    \begin{parts}
      \part[2] Compute the derivative of $ \tan(x+\pi/6) + \cos(4\pi) $ with respect to $ x $.
      \part[3] Write down explicitly $ f'(x) $.
      \part[3] Give the equation of the best linear approximation to $ f(x) $ at $ \left(\pi, -\ln (\pi) \sin\left(1 + \frac{1}{\sqrt{3}}\right) \right) $, giving
            constants to three decimal places.
    \end{parts}
  \question
    \begin{parts}
      \part[2] Compute the derivative of $ (x + 1)^3 $ using the definition of the derivative.
      \part[3] Show that $ \lim_{x \to 0} \ln x $ does not exist.
      \part[3] Show that $ \lim_{x \to \infty} \frac{\sin x}{x} = 0 $. [Hint: use the fact that $ -1 \leq \sin x \leq 1 $, and
            write $ A(x) \leq \frac{\sin x}{x} \leq B(x) $ for two functions $ A $ and $ B $ that both go to zero as $ x $ grows].
    \end{parts}
\end{questions}
\end{document}
