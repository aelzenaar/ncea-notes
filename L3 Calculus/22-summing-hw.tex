\documentclass{exam}
\usepackage[utf8]{inputenc}
\usepackage{lmodern}
\usepackage{microtype}

% \usepackage[parfill]{parskip}
\usepackage[dvipsnames]{xcolor}
\usepackage{amsmath}
\usepackage{amsfonts}
\usepackage{amsthm}
\usepackage{siunitx}
\DeclareSIUnit\year{yr}
\DeclareSIUnit\foot{ft}
\DeclareSIUnit\litre{\liter}

\usepackage{skull}

\usepackage{pgfplots}
\usepgfplotslibrary{polar}
\pgfplotsset{compat=1.11}
\usepgfplotslibrary{statistics}
\usepackage{graphicx}
\usepackage{sidecap}
\sidecaptionvpos{figure}{c}
\usepackage{float}
\usepackage{gensymb}
\usepackage{tkz-euclide}
\usetkzobj{all}
\usepackage{commath}
\usepackage{hyperref}
\usepackage{enumitem}
\usepackage{wasysym}
\usepackage{multicol}
\usepackage{mathtools}
\usepackage{tcolorbox}
\usepackage{tabularx}
\usepackage[version=4]{mhchem}
\usepackage{changepage}
\usepackage{listings}
\lstset{basicstyle=\ttfamily\linespread{0.8}\small}

\renewcommand*{\thefootnote}{\fnsymbol{footnote}}

\newtheorem*{thm}{Theorem}
\newtheorem*{iden}{Identity}
\newtheorem*{lemma}{Lemma}
\newtheorem{obs}{Observation}
\theoremstyle{definition}
\newtheorem*{defn}{Definition}
\newtheorem*{ex}{Example}
\newtheorem{con}{Construction}
\newtheorem*{alg}{Algorithm}

\newtheoremstyle{break}
  {\topsep}{\topsep}%
  {\itshape}{}%
  {\bfseries}{}%
  {\newline}{}%
\theoremstyle{break}
\newtheorem*{bthm}{Theorem}

% russian integral
\usepackage{scalerel}
\DeclareMathOperator*{\rint}{\scalerel*{\rotatebox{17}{$\!\int\!$}}{\int}}

% \DeclareMathOperator*{\rint}{\int}

\pgfplotsset{vasymptote/.style={
    before end axis/.append code={
        \draw[densely dashed] ({rel axis cs:0,0} -| {axis cs:#1,0})
        -- ({rel axis cs:0,1} -| {axis cs:#1,0});
    }
}}

% \pointsinrightmargin
\boxedpoints
\pointname{}

\newcommand{\questioA}{\question[\texttt{\textbf{\color{Cerulean} A}}]}
\newcommand{\questioM}{\question[\texttt{\textbf{\color{PineGreen} M}}]}
\newcommand{\questioE}{\question[\texttt{\textbf{\color{WildStrawberry} E}}]}
\newcommand{\questioS}{\question[\texttt{\textbf{\color{Goldenrod} S}}]}
\newcommand{\questioO}{\question[\texttt{\textbf{\color{BurntOrange} O}}]}

\newcommand{\parA}{\part[\texttt{\textbf{\color{Cerulean} A}}]}
\newcommand{\parM}{\part[\texttt{\textbf{\color{PineGreen} M}}]}
\newcommand{\parE}{\part[\texttt{\textbf{\color{WildStrawberry} E}}]}
\newcommand{\parS}{\part[\texttt{\textbf{\color{Goldenrod} S}}]}
\newcommand{\parO}{\part[\texttt{\textbf{\color{BurntOrange} O}}]}

\newcommand{\subparA}{\subpart[\texttt{\textbf{\color{Cerulean} A}}]}
\newcommand{\subparM}{\subpart[\texttt{\textbf{\color{PineGreen} M}}]}
\newcommand{\subparE}{\subpart[\texttt{\textbf{\color{WildStrawberry} E}}]}
\newcommand{\subparS}{\subpart[\texttt{\textbf{\color{Goldenrod} S}}]}
\newcommand{\subparO}{\subpart[\texttt{\textbf{\color{BurntOrange} O}}]}

\newcommand{\mainHeader}[2]{\section*{NCEA Level 2 Mathematics\\#1. #2}}
\newcommand{\mainHeaderHw}[2]{\section*{NCEA Level 2 Mathematics (Homework)\\#1. #2}}
\newcommand{\seealso}[1]{\begin{center}\emph{See also #1.}\end{center}}
\newcommand{\drills}[1]{\begin{center}\emph{Drill problems: #1.}\end{center}}
\newcommand{\basedon}[1]{\begin{center}\emph{Notes largely based on #1.}\end{center}}

\begin{document}

\mainHeaderIntgHw{22}{Lengths, Volumes, and Areas}
\subsection*{Reading}
Absolute continuity of motion is not comprehensible to the human mind. Laws of motion of any kind only become comprehensible to man when he examines arbitrarily selected elements of that motion; but at the same time, a large proportion of human error comes from the arbitrary division of continuous motion into discontinuous elements.

There is a well-known so-called sophism of the ancients consisting in this, that Achilles could never catch up with a tortoise he was following in spite of the fact that he travelled ten times as fast as the tortoise. By the time Achilles has covered the distance that separated him from the tortoise, the tortoise has covered one-tenth of tat distance ahead of him: when Achiles has covered that tenth, the tortoise has covered another one-hundredth, and so on for ever. This problem seemed to the ancients insoluble. The absurd answer (that Achilles could never overtake the tortoise) resulted from this: that motion was arbitrarily divided into discontinuous elements, whereas the motion both of Achilles and of the tortoise was continuous.

By adopting smaller and smaller elements of motion we only approach a solution of the problem, but never reach it. Only when we have admitted the conception of the infinitely small, and the resulting geometrical progression with a common ratio of one-tenth, and have found the sum of this progression to infinity, do we reach a solution to this problem. A modern branch of mathematics, having achieved the art of dealing with the infinitely small, can now yield solutions in other more complex problems of motion, which used to appear insoluble.

This modern branch of mathematics, unknown to the ancients, when dealing with problems of motion, admits the conception of the infinitely small, and so conforms to the chief condition of motion (absolute continuity) and thereby corrects the inevitable error which the human mind cannot avoid when dealing with separate elements of motion instead of examining continuous motion.

In seeking the laws of historical movement just the same thing happens.

The movement of humanity, arising as it does from innumerable arbitrary human wills, is continuous.

To understand the laws of this continuous human movement is the aim of history. But to arrive at these laws, resulting from the sum of all those human wills, man's mind postulates arbitrary and disconnected units. The first method of history is to take an arbitrarily selected series of continuous events and examine it apart from others, though there is and can be no \emph{beginning} to any event, for one event always flows uninterruptedly from another. The second method is to consider the actions of some one man --- a king or a commander --- as equivalent to the sum of many individual wills; whereas the sum of individual wills is never expressed by the activity of a single historical personage.

Historical science in its endeavour to draw nearer to truth continually takes smaller and smaller units for examination. But however small the unit it takes, we begin to feel that to take any unit disconnected from others, or to assume a \emph{beginning} to any phenomenon, or to say that the will of many men is expressed by the actions of any one historic personage, is in itself false.

It needs no critical exertion to reduce utterly to dust any deductions drawn from history. It is merely necessary to select some larger or smaller unit as the subject of observation --- as criticism has every right to do, seeing that whatever unit history observes must always be arbitrarily selected.

Only by taking an infinitesimally small unit for observation (the differential of history, that is, the individual tendencies of men) and attaining to the art of integrating them (that is, finding the sum of those infinitesimals) can we hope to arrive at the laws of history.

\begin{flushright}
  From book 3, part 3, chapter 1 of \textit{War and Peace}, by Leo Tolstoy.

  Extra reading: \small{\url{http://teachers.dadeschools.net/akoski/downloads/Reading_Journals/Essays/tolstoy_integration.pdf}}
  (\textit{Tolstoy's Integration Metaphor}, S. Ahearn, 2004.)
\end{flushright}

\clearpage
\subsection*{Questions}
\begin{questions}
  \question Find the volume of revolution around the $ x$-axis of the hyperbola $ y = \frac{1}{x} $ from $ x = 1 $ to $ x = 2 $.
  \question Find the area of the surface created by rotating the curve $ y = \sin x $ around the $ x$-axis ($ 0 \leq x \leq \pi $).
            What is the radius of the circle with the same area?
  \question Write down a formula for the volume of a square-based pyramid with base side length $ L $ and height $ H $.
  \question Scholarship 2017: The length $ S $ of a curve expressed in polar coordinates is given by
            \begin{displaymath}
              S = \rint_{\theta_1}^{\theta_2} \sqrt{r^2 + \left(\od{r}{\theta}\right)^2} \dif{\theta}.
            \end{displaymath}
            Find the length of the entire curve $ r = a (1 - \cos \theta) $ in terms of the constant $ a $.

            [\textit{Note: you may wish to graph the curve first.}]
\end{questions}
\end{document}
