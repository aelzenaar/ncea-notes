\documentclass{exam}
\usepackage[utf8]{inputenc}
\usepackage{lmodern}
\usepackage{microtype}

% \usepackage[parfill]{parskip}
\usepackage[dvipsnames]{xcolor}
\usepackage{amsmath}
\usepackage{amsfonts}
\usepackage{amsthm}
\usepackage{siunitx}
\DeclareSIUnit\year{yr}
\DeclareSIUnit\foot{ft}
\DeclareSIUnit\litre{\liter}

\usepackage{skull}

\usepackage{pgfplots}
\usepgfplotslibrary{polar}
\pgfplotsset{compat=1.11}
\usepgfplotslibrary{statistics}
\usepackage{graphicx}
\usepackage{sidecap}
\sidecaptionvpos{figure}{c}
\usepackage{float}
\usepackage{gensymb}
\usepackage{tkz-euclide}
\usetkzobj{all}
\usepackage{commath}
\usepackage{hyperref}
\usepackage{enumitem}
\usepackage{wasysym}
\usepackage{multicol}
\usepackage{mathtools}
\usepackage{tcolorbox}
\usepackage{tabularx}
\usepackage[version=4]{mhchem}
\usepackage{changepage}
\usepackage{listings}
\lstset{basicstyle=\ttfamily\linespread{0.8}\small}

\renewcommand*{\thefootnote}{\fnsymbol{footnote}}

\newtheorem*{thm}{Theorem}
\newtheorem*{iden}{Identity}
\newtheorem*{lemma}{Lemma}
\newtheorem{obs}{Observation}
\theoremstyle{definition}
\newtheorem*{defn}{Definition}
\newtheorem*{ex}{Example}
\newtheorem{con}{Construction}
\newtheorem*{alg}{Algorithm}

\newtheoremstyle{break}
  {\topsep}{\topsep}%
  {\itshape}{}%
  {\bfseries}{}%
  {\newline}{}%
\theoremstyle{break}
\newtheorem*{bthm}{Theorem}

% russian integral
\usepackage{scalerel}
\DeclareMathOperator*{\rint}{\scalerel*{\rotatebox{17}{$\!\int\!$}}{\int}}

% \DeclareMathOperator*{\rint}{\int}

\pgfplotsset{vasymptote/.style={
    before end axis/.append code={
        \draw[densely dashed] ({rel axis cs:0,0} -| {axis cs:#1,0})
        -- ({rel axis cs:0,1} -| {axis cs:#1,0});
    }
}}

% \pointsinrightmargin
\boxedpoints
\pointname{}

\newcommand{\questioA}{\question[\texttt{\textbf{\color{Cerulean} A}}]}
\newcommand{\questioM}{\question[\texttt{\textbf{\color{PineGreen} M}}]}
\newcommand{\questioE}{\question[\texttt{\textbf{\color{WildStrawberry} E}}]}
\newcommand{\questioS}{\question[\texttt{\textbf{\color{Goldenrod} S}}]}
\newcommand{\questioO}{\question[\texttt{\textbf{\color{BurntOrange} O}}]}

\newcommand{\parA}{\part[\texttt{\textbf{\color{Cerulean} A}}]}
\newcommand{\parM}{\part[\texttt{\textbf{\color{PineGreen} M}}]}
\newcommand{\parE}{\part[\texttt{\textbf{\color{WildStrawberry} E}}]}
\newcommand{\parS}{\part[\texttt{\textbf{\color{Goldenrod} S}}]}
\newcommand{\parO}{\part[\texttt{\textbf{\color{BurntOrange} O}}]}

\newcommand{\subparA}{\subpart[\texttt{\textbf{\color{Cerulean} A}}]}
\newcommand{\subparM}{\subpart[\texttt{\textbf{\color{PineGreen} M}}]}
\newcommand{\subparE}{\subpart[\texttt{\textbf{\color{WildStrawberry} E}}]}
\newcommand{\subparS}{\subpart[\texttt{\textbf{\color{Goldenrod} S}}]}
\newcommand{\subparO}{\subpart[\texttt{\textbf{\color{BurntOrange} O}}]}

\newcommand{\mainHeader}[2]{\section*{NCEA Level 2 Mathematics\\#1. #2}}
\newcommand{\mainHeaderHw}[2]{\section*{NCEA Level 2 Mathematics (Homework)\\#1. #2}}
\newcommand{\seealso}[1]{\begin{center}\emph{See also #1.}\end{center}}
\newcommand{\drills}[1]{\begin{center}\emph{Drill problems: #1.}\end{center}}
\newcommand{\basedon}[1]{\begin{center}\emph{Notes largely based on #1.}\end{center}}

\begin{document}

\mainHeaderIntgHw{20}{Partial Fractions}
\subsection*{Reading}
Perhaps I can best describe my journey of doing mathematics in terms of a journey through a dark unexplored
mansion. You enter the first room of the mansion and it's completely dark. You stumble around bumping into
the furniture, but eventually you learn where each piece of furniture is. Finally after six months or so,
you find the light switch, you turn it on, and suddenly it's all illuminated. You can see exactly where you
were. Then you move into the next room and spend another six months in the dark. So each of these breakthroughs,
while sometimes they're momentary, sometimes over a period of a day or two, they are the culmination of --- and
couldn't exist without --- the many months of stumbling around in the dark that precede them.

\textit{From an interview by Nova of Andrew Wiles}

\textbf{Note:} For the interested, a proof that one can always expand a rational function into partial fractions
is outlined as exercise 11.1.13 in Artin (p. 441).

\subsection*{Questions}
\begin{questions}
  \question The \textit{logistic equation} is used when modelling populations:
            \begin{displaymath}
              \od{P}{t} = rP(1-P)
            \end{displaymath}
    \begin{parts}
      \part Find $ P(t) $ explicitly.
      \part Examine the behaviour of the population as $ t \to \infty $. Graph the function.
      \part Examine the behaviour of the population over time if you vary $ r $ (check $ r = 0 $, and $ r < 0 $ for example).
      \part Do you think the logistic equation is a good model? Why/why not?
    \end{parts}
  \question In the following, let $ t = \tan \frac{x}{2} $ (where $ \abs{x} < \pi $).
    \begin{parts}
      \part Show that:
        \begin{displaymath}
          \cos\left( \frac{x}{2} \right) = \frac{1}{\sqrt{1 + t^2}} \quad\text{and}\quad \sin\left(\frac{x}{2}\right) = \frac{t}{\sqrt{1 + t^2}}
        \end{displaymath}
      \part Show that:
        \begin{displaymath}
          \cos x = \frac{1 - t^2}{1 + t^2} \quad\text{and}\quad \sin x = \frac{2t}{1 + t^2}
        \end{displaymath}
      \part Show that:
        \begin{displaymath}
          \od{x}{t} = \frac{2}{1 + t^2}
        \end{displaymath}
      \part Use the substitution $ t $ to evaluate:
        \begin{subparts}
          \subpart $ \rint (1 - \cos x)^{-1} \dif{x} $
          \subpart $ \rint (3\sin x - 4\cos x)^{-1} \dif{x} $
        \end{subparts}
    \end{parts}
\end{questions}
\end{document}
