\documentclass{exam}
\usepackage[utf8]{inputenc}
\usepackage{lmodern}
\usepackage{microtype}

% \usepackage[parfill]{parskip}
\usepackage[dvipsnames]{xcolor}
\usepackage{amsmath}
\usepackage{amsfonts}
\usepackage{amsthm}
\usepackage{siunitx}
\DeclareSIUnit\year{yr}
\DeclareSIUnit\foot{ft}
\DeclareSIUnit\litre{\liter}

\usepackage{skull}

\usepackage{pgfplots}
\usepgfplotslibrary{polar}
\pgfplotsset{compat=1.11}
\usepgfplotslibrary{statistics}
\usepackage{graphicx}
\usepackage{sidecap}
\sidecaptionvpos{figure}{c}
\usepackage{float}
\usepackage{gensymb}
\usepackage{tkz-euclide}
\usetkzobj{all}
\usepackage{commath}
\usepackage{hyperref}
\usepackage{enumitem}
\usepackage{wasysym}
\usepackage{multicol}
\usepackage{mathtools}
\usepackage{tcolorbox}
\usepackage{tabularx}
\usepackage[version=4]{mhchem}
\usepackage{changepage}
\usepackage{listings}
\lstset{basicstyle=\ttfamily\linespread{0.8}\small}

\renewcommand*{\thefootnote}{\fnsymbol{footnote}}

\newtheorem*{thm}{Theorem}
\newtheorem*{iden}{Identity}
\newtheorem*{lemma}{Lemma}
\newtheorem{obs}{Observation}
\theoremstyle{definition}
\newtheorem*{defn}{Definition}
\newtheorem*{ex}{Example}
\newtheorem{con}{Construction}
\newtheorem*{alg}{Algorithm}

\newtheoremstyle{break}
  {\topsep}{\topsep}%
  {\itshape}{}%
  {\bfseries}{}%
  {\newline}{}%
\theoremstyle{break}
\newtheorem*{bthm}{Theorem}

% russian integral
\usepackage{scalerel}
\DeclareMathOperator*{\rint}{\scalerel*{\rotatebox{17}{$\!\int\!$}}{\int}}

% \DeclareMathOperator*{\rint}{\int}

\pgfplotsset{vasymptote/.style={
    before end axis/.append code={
        \draw[densely dashed] ({rel axis cs:0,0} -| {axis cs:#1,0})
        -- ({rel axis cs:0,1} -| {axis cs:#1,0});
    }
}}

% \pointsinrightmargin
\boxedpoints
\pointname{}

\newcommand{\questioA}{\question[\texttt{\textbf{\color{Cerulean} A}}]}
\newcommand{\questioM}{\question[\texttt{\textbf{\color{PineGreen} M}}]}
\newcommand{\questioE}{\question[\texttt{\textbf{\color{WildStrawberry} E}}]}
\newcommand{\questioS}{\question[\texttt{\textbf{\color{Goldenrod} S}}]}
\newcommand{\questioO}{\question[\texttt{\textbf{\color{BurntOrange} O}}]}

\newcommand{\parA}{\part[\texttt{\textbf{\color{Cerulean} A}}]}
\newcommand{\parM}{\part[\texttt{\textbf{\color{PineGreen} M}}]}
\newcommand{\parE}{\part[\texttt{\textbf{\color{WildStrawberry} E}}]}
\newcommand{\parS}{\part[\texttt{\textbf{\color{Goldenrod} S}}]}
\newcommand{\parO}{\part[\texttt{\textbf{\color{BurntOrange} O}}]}

\newcommand{\subparA}{\subpart[\texttt{\textbf{\color{Cerulean} A}}]}
\newcommand{\subparM}{\subpart[\texttt{\textbf{\color{PineGreen} M}}]}
\newcommand{\subparE}{\subpart[\texttt{\textbf{\color{WildStrawberry} E}}]}
\newcommand{\subparS}{\subpart[\texttt{\textbf{\color{Goldenrod} S}}]}
\newcommand{\subparO}{\subpart[\texttt{\textbf{\color{BurntOrange} O}}]}

\newcommand{\mainHeader}[2]{\section*{NCEA Level 2 Mathematics\\#1. #2}}
\newcommand{\mainHeaderHw}[2]{\section*{NCEA Level 2 Mathematics (Homework)\\#1. #2}}
\newcommand{\seealso}[1]{\begin{center}\emph{See also #1.}\end{center}}
\newcommand{\drills}[1]{\begin{center}\emph{Drill problems: #1.}\end{center}}
\newcommand{\basedon}[1]{\begin{center}\emph{Notes largely based on #1.}\end{center}}

\begin{document}

\mainHeaderDiff{3}{Derivatives of Common Functions}
Now that we understand the purpose and form of the derivative, we can begin to calculate
the derivatives of some common functions. We list them here without proof, but it is easy
to find the proofs yourself in a textbook or on the internet if you are interested.

\begin{center}
  \def\arraystretch{1.5}
  \begin{tabular}{|c|c|l|}\hline
    \textbf{Function} & \textbf{Derivative} & \textbf{Notes}\\\hline
    $ x^n $ & $ nx^{n - 1} $&\\\hline
    $ e^x $ & $ e^x $ & Here, $ e \approx 2.71828 $ is Euler's number. \\\hline
    $ \ln \abs{x} $ & $ \frac{1}{x} $ & Here, $ \ln = \log_e $. \\\hline
    $ \sin x $ & $ \cos x $&\\\hline
    $ \cos x $ & $ -\sin x $&\\\hline
    $ \tan x $ & $ \sec^2 x $&\\\hline
    $ \csc x $ & $ -\csc x \cot x $&\\\hline
    $ \sec x $ & $ \sec x \tan x $&\\\hline
    $ \cot x $ & $ -\csc^2 x $&\\\hline
  \end{tabular}
\end{center}

We also have several rules for finding the derivative of a more complicated function in terms of the derivatives of component
functions.

\begin{thm}
  Let $ f $ and $ g $ be functions, and let $ \lambda $ be a real constant. Then:
  \begin{enumerate}
    \item $ (\lambda)' = 0 $: The slope of a constant function is zero.
    \item $ (\lambda f)' = \lambda f' $: The slope of a scaled function is the scaled slope of the function.
    \item $ (f + g)' = f' + g' $: The slope of the sum of two functions is the sum of the individual slopes.
  \end{enumerate}
\end{thm}

These formulae can all be proved using the limit definition of the derivative, and it is a simple exercise to check them all. For example,
\begin{align*}
  (f + g)'(x) = \lim_{h \to 0} \frac{(f + g)(x + h) - (f + g)(x)}{h} &= \lim_{h \to 0} \frac{f(x + h) + g(x + h) - f(x) - g(x)}{h} \\
                                                                     &= \lim_{h \to 0} \frac{f(x + h) - f(x)}{h} + \lim_{h \to 0} \frac{g(x + h) - g(x)}{h} \\
                                                                     &= f'(x) + g'(x).
\end{align*}

Note that the obvious product rule, $ (fg)' = f'g' $ does \textbf{not} hold. We will discuss this further soon,
although it should be noted that Leibniz initially believed this rule to be true! A counterexample is outlined in the
exercises for you to work out.

\begin{exs}\leavevmode
  \begin{enumerate}
    \item $ \od{}{x} [\sin x + \cos x] = \cos x - \sin x $.
    \item $ \od{}{x} \frac{3x^2 + 2x + 1}{x} = \od{}{x} \left[ 3x + 2 + x^{-1} \right] = 3 + 0 + (-1)x^{-2} = 3 - \frac{1}{x^2} $.
    \item $ \od{}{x} \sqrt{x} = \od{}{x} x^{1/2} = \frac{1}{2}x^{-1/2} = \frac{1}{2\sqrt{x}} $.
  \end{enumerate}
\end{exs}

\begin{app}
  Many phenomena in physics can be modelled with sine waves; for example, if a particle on the end of a spring
  is moving with simple harmonic motion, then it has position $ x = A \sin (\omega t + \phi) $; taking derivatives,
  we find that it has velocity $ v = \od{x}{t} = A\omega \cos (\omega t + \phi) $ and acceleration $ a = \od[2]{x}{t} = -A \omega^2 \sin (\omega t + \phi) $.
  In other words, it is always accelerating in the opposite direction to its movement!
\end{app}

\clearpage
\subsection*{Questions}
\begin{questions}
  \questioA Find the derivatives of $ 3x^3 $, $ 2x^2 $, and $ 6x^5 $. Conclude that $ (fg)' \neq f' g' $ in general.
  \questioA Find the derivatives of the following functions with respect to $ t $:
    \begin{parts}
      \part $ y = 2t^3 + 3t^2 $
      \part $ y = \sqrt{t} $
      \part $ y = (2t + 1)(t - 4) $
      \part $ g(t) = 4 \sec t + 9 \tan t $
      \part $ h(t) = \sqrt[5]{t} + 2\csc t - \ln t^3 $
      \part $ \phi'(t) = \csc x + 12x^{1273} + 9 $
      \part $ y = 2017t^{2016} + (t + 2)^2 $
      \part $ y = 940\sin t + \frac{1}{2}e^{t + 2} $
    \end{parts}
  \questioA Where is the function $ x \mapsto x^3 - 2x^2 - x + 1 $ increasing?
  \questioA Find the velocity $ v $ of a particle at time $ t = 2\pi $ if its position function for $ t > 0 $ is $ x = e^t - \sin t $.
  \questioA Find the slope of the tangent line to $ y = x + \tan x $ at $ (\pi, \pi) $.
  \questioA It is \textbf{not} true that the derivative of $ f(g(x)) $ is $ f'(g'(x)) $.
    \begin{parts}
      \part For a counterexample, consider $ f(x) = x^2 $ and $ g(x) = x $; show that $ f'(g'(x)) = 2 $, but $ \od{}{x} f(g(x)) = 2x $.
      \part Compute the derivative of $ \ln x^2 $.
    \end{parts}
  \questioA Suppose the derivative of a function is $ \od{y}{x} = 3x^2 - x - 4 $. What could the original function be?
  \questioM Find the 64th derivative of $ \sin x $.
  \questioM Find the $ n$th derivative of $ x^n $.
  \questioM If $ y = 2\sin 3x \cos 2x $, find $ \od{y}{x} $. (Hint: use an identity to rewrite this as a sum of functions.)
  \questioM For which values of $ x $ does the graph of $ f(x) = x + 2\sin x $ have a horizontal tangent?
  \questioE Show that $ y = 6x^3 + 5x - 3 $ has no tangent line with a slope of 4.
  \questioE Find real values of $ \alpha $ and $ \beta $ such that, if $ y = \alpha \sin x + \beta \cos x $,
            then $ y'' + y' -2y = \sin x $.
  \questioE Consider a \SI{12}{\metre} long ladder leaning against a wall such that the top of the ladder makes an
            angle $ \theta $ with the wall. If this angle $ \theta $ is varied, the distance $ D $ between the bottom
            of the ladder and the wall also changes. If $ \theta = \pi/3 $, what is the rate of change of $ D $ with
            respect to $ \theta $?
  \questioE Prove that the function $ \varphi $ given by $ \varphi(x) = \frac{x^{101}}{101} + \frac{x^{51}}{51} + x + 1 $
            has no extreme values.
  \questioM The derivative is primarily a geometric concept, not an algebraic one.
    \begin{parts}
      \part The area of a circle of radius $ r $ is $ A = \pi r^2 $. Find $ \od{A}{r} $. What do you notice?
      \part Explain part (a) geometrically.
      \part The volume of a sphere is given by $ V = \frac{4}{3} \pi r^3 $. Find an expression for the surface area.
    \end{parts}
  \questioE We have the derivative of $ \log_e $, but not for any other log base. Calculate $ \od{y}{x} $ if $ y = \log_{10} x $.
  \questioE Prove the remaining two differentiation rules using the limit definition of the derivative.
  \questioS In this exercise, we will calculate the value of $ e $ based on solving the differential
            equation $ f(x) = f'(x) $. Clearly one solution is $ f(x) = 0 $, but by drawing some pictures
            you should see that a reasonable guess for a more interesting solution would be an exponential function, $ f(x) = a^x $.
            We just need to pick the right base.
    \begin{parts}
      \part Show that the derivative of $ a^x $ is given by
            \begin{displaymath}
              \lim_{h \to 0} a^x \left(\frac{a^h - 1}{h}\right).
            \end{displaymath}
      \part For $ a^x $ to be a solution to our differential equation, we need $ \lim_{h \to 0} \frac{a^h - 1}{h} $ to be 1. Show that
            \begin{displaymath}
              a = \lim_{h \to 0}(1 + h)^{1/h}
            \end{displaymath}
            works for this purpose.
      \part The expression in (b) is not the standard way to write this limit. Show that if $ n = 1/h $, then
            \begin{displaymath}
              a = \lim_{n \to \infty} \left(1 + \frac{1}{n}\right)^n.
            \end{displaymath}
      \part By calculating the value of $ \left(1 + \frac{1}{n}\right)^n $ for a suitably large value of $ n $, obtain a numerical
            approximation for Euler's number, $ e $.
    \end{parts}
\end{questions}
\end{document}
