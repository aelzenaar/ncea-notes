\documentclass{exam}
\usepackage[utf8]{inputenc}
\usepackage{lmodern}
\usepackage{microtype}

% \usepackage[parfill]{parskip}
\usepackage[dvipsnames]{xcolor}
\usepackage{amsmath}
\usepackage{amsfonts}
\usepackage{amsthm}
\usepackage{siunitx}
\DeclareSIUnit\year{yr}
\DeclareSIUnit\foot{ft}
\DeclareSIUnit\litre{\liter}

\usepackage{skull}

\usepackage{pgfplots}
\usepgfplotslibrary{polar}
\pgfplotsset{compat=1.11}
\usepgfplotslibrary{statistics}
\usepackage{graphicx}
\usepackage{sidecap}
\sidecaptionvpos{figure}{c}
\usepackage{float}
\usepackage{gensymb}
\usepackage{tkz-euclide}
\usetkzobj{all}
\usepackage{commath}
\usepackage{hyperref}
\usepackage{enumitem}
\usepackage{wasysym}
\usepackage{multicol}
\usepackage{mathtools}
\usepackage{tcolorbox}
\usepackage{tabularx}
\usepackage[version=4]{mhchem}
\usepackage{changepage}
\usepackage{listings}
\lstset{basicstyle=\ttfamily\linespread{0.8}\small}

\renewcommand*{\thefootnote}{\fnsymbol{footnote}}

\newtheorem*{thm}{Theorem}
\newtheorem*{iden}{Identity}
\newtheorem*{lemma}{Lemma}
\newtheorem{obs}{Observation}
\theoremstyle{definition}
\newtheorem*{defn}{Definition}
\newtheorem*{ex}{Example}
\newtheorem{con}{Construction}
\newtheorem*{alg}{Algorithm}

\newtheoremstyle{break}
  {\topsep}{\topsep}%
  {\itshape}{}%
  {\bfseries}{}%
  {\newline}{}%
\theoremstyle{break}
\newtheorem*{bthm}{Theorem}

% russian integral
\usepackage{scalerel}
\DeclareMathOperator*{\rint}{\scalerel*{\rotatebox{17}{$\!\int\!$}}{\int}}

% \DeclareMathOperator*{\rint}{\int}

\pgfplotsset{vasymptote/.style={
    before end axis/.append code={
        \draw[densely dashed] ({rel axis cs:0,0} -| {axis cs:#1,0})
        -- ({rel axis cs:0,1} -| {axis cs:#1,0});
    }
}}

% \pointsinrightmargin
\boxedpoints
\pointname{}

\newcommand{\questioA}{\question[\texttt{\textbf{\color{Cerulean} A}}]}
\newcommand{\questioM}{\question[\texttt{\textbf{\color{PineGreen} M}}]}
\newcommand{\questioE}{\question[\texttt{\textbf{\color{WildStrawberry} E}}]}
\newcommand{\questioS}{\question[\texttt{\textbf{\color{Goldenrod} S}}]}
\newcommand{\questioO}{\question[\texttt{\textbf{\color{BurntOrange} O}}]}

\newcommand{\parA}{\part[\texttt{\textbf{\color{Cerulean} A}}]}
\newcommand{\parM}{\part[\texttt{\textbf{\color{PineGreen} M}}]}
\newcommand{\parE}{\part[\texttt{\textbf{\color{WildStrawberry} E}}]}
\newcommand{\parS}{\part[\texttt{\textbf{\color{Goldenrod} S}}]}
\newcommand{\parO}{\part[\texttt{\textbf{\color{BurntOrange} O}}]}

\newcommand{\subparA}{\subpart[\texttt{\textbf{\color{Cerulean} A}}]}
\newcommand{\subparM}{\subpart[\texttt{\textbf{\color{PineGreen} M}}]}
\newcommand{\subparE}{\subpart[\texttt{\textbf{\color{WildStrawberry} E}}]}
\newcommand{\subparS}{\subpart[\texttt{\textbf{\color{Goldenrod} S}}]}
\newcommand{\subparO}{\subpart[\texttt{\textbf{\color{BurntOrange} O}}]}

\newcommand{\mainHeader}[2]{\section*{NCEA Level 2 Mathematics\\#1. #2}}
\newcommand{\mainHeaderHw}[2]{\section*{NCEA Level 2 Mathematics (Homework)\\#1. #2}}
\newcommand{\seealso}[1]{\begin{center}\emph{See also #1.}\end{center}}
\newcommand{\drills}[1]{\begin{center}\emph{Drill problems: #1.}\end{center}}
\newcommand{\basedon}[1]{\begin{center}\emph{Notes largely based on #1.}\end{center}}

\begin{document}

\mainHeaderDiffHw{12}{Sequences and Series}
\subsection*{Reading}
Leonardo of Pisa, Italian born, grew up in North Aftica, where his father Guilielmo was working as a diplomat on behalf
of merchants trading at Bugia (modern Algeria). He accompanied his father on his numerous travels, encountered the Arabic
system for writing numbers and understood its importance. In his \textit{Liber Abbaci} of 1202 he writes: `When my father,
who had been appointed by his country as public notary in the customs at Bugia acting for the Pisan merchants going there,
was in charge, he summoned me to him while I was still a child, and having an eye to usefulness and future convenience,
desired me to stay there and receive instruction in the school of accounting. There, when I had been introduced to the art
of the Indian's nine symbols through remarkable teaching, knowledge of the art very soon pleased me above all else.'

The book introduced the Hindu-Arabic notation to Europe, and formed a comprehensive arithmetic text, containing a wealth of
material related to trade and currency conversion. Although it took several centuries for Hindu-Arabic notation to replace
the traditional abacus, the advantages of a purely written system of calculation soon became apparent.

Leonardo is often known by his nickname `Fibonacci', which means `son of Bonaccio', but this name is not recorded befre the
18th century and was probably invented then by Guillaume Libri.

The third section of the \textit{Liber Abbaci} contains a problem that seems to have originated with Leonardo: `A certain man
put a pair of rabbits in a place surrounded on all sides by a wall. How many pairs of rabbits can be produced from that pair
in a year if in every month, each pair begets a new pair, which from the second month onwards becomes productive?'

This rather quirky problem leads to a curious, and famous, system of numbrs: 1, 2, 3, 5, 8, 13, 21, 34, 55, and so on. Each
number is the sum of the preceeding two numbers. This is known as the Fibonacci sequence, and it turns up repeatedly in
mathematics and in the natural world. In particular, many flowers have a Fibonacci number of petals. This is not a coincidence,
but a consequence of the growth pattern of the plant and the geometry of the `primordia' --- tiny clumps of cells at the tip of
the growing shoot that give rise to important structures, including petals.

Although Fibonacci's growth rule for rabbit populations is unrealistic, more general rules of a similar kind (called Leslie models)
are used today for certain problems in population dynamics, the study of how animal populations change in size as the animals breed
and die.
\begin{flushright}
  From \textit{Taming the Infinite}, by Ian Stewart.
\end{flushright}


\subsection*{Questions}
\begin{questions}
  \item Determine whether the following sequences converge. If so, give the limit.
    \begin{parts}
      \item $ (a_n) = \left(\frac{2 + n^3}{1 + 2n^3}\right) $
      \item $ (b_n) = \left(9^{n + 1}/10^n\right) $
    \end{parts}
  \item An alternating series is a series whose terms are alternately positive and negative. For the following alternating series, write
        out the first few terms of the series and the first few terms of the series of partial sums. Does the series seem to converge? You
        need not find the limit.
    \begin{parts}
      \item $ \sum_{n = 1}^\infty (-1)^n \frac{2}{2n + 1} $.
      \item $ \sum_{n = 1}^\infty (-1)^{n + 1} \frac{2n}{n + 4} $.
    \end{parts}
\end{questions}
\end{document}
