\documentclass{exam}
\usepackage[utf8]{inputenc}
\usepackage{lmodern}
\usepackage{microtype}

% \usepackage[parfill]{parskip}
\usepackage[dvipsnames]{xcolor}
\usepackage{amsmath}
\usepackage{amsfonts}
\usepackage{amsthm}
\usepackage{siunitx}
\DeclareSIUnit\year{yr}
\DeclareSIUnit\foot{ft}
\DeclareSIUnit\litre{\liter}

\usepackage{skull}

\usepackage{pgfplots}
\usepgfplotslibrary{polar}
\pgfplotsset{compat=1.11}
\usepackage{graphicx}
\usepackage{sidecap}
\sidecaptionvpos{figure}{c}
\usepackage{float}
\usepackage{gensymb}
\usepackage{tkz-euclide}
\usetkzobj{all}
\usepackage{commath}
\usepackage{hyperref}
\usepackage{enumitem}
\usepackage{wasysym}
\usepackage{multicol}
\usepackage{mathtools}
\usepackage{tcolorbox}
\usepackage{tabularx}
\usepackage[version=4]{mhchem}
\usepackage{changepage}
\usepackage{listings}
\lstset{basicstyle=\ttfamily\linespread{0.8}\small}

\renewcommand*{\thefootnote}{\fnsymbol{footnote}}

\newtheorem*{thm}{Theorem}
\newtheorem*{iden}{Identity}
\newtheorem*{lemma}{Lemma}
\newtheorem{obs}{Observation}
\theoremstyle{definition}
\newtheorem*{defn}{Definition}
\newtheorem*{ex}{Example}
\newtheorem{con}{Construction}
\newtheorem*{alg}{Algorithm}

\newtheoremstyle{break}
  {\topsep}{\topsep}%
  {\itshape}{}%
  {\bfseries}{}%
  {\newline}{}%
\theoremstyle{break}
\newtheorem*{bthm}{Theorem}

% russian integral
\usepackage{scalerel}
\DeclareMathOperator*{\rint}{\scalerel*{\rotatebox{17}{$\!\int\!$}}{\int}}

% \DeclareMathOperator*{\rint}{\int}

\pgfplotsset{vasymptote/.style={
    before end axis/.append code={
        \draw[densely dashed] ({rel axis cs:0,0} -| {axis cs:#1,0})
        -- ({rel axis cs:0,1} -| {axis cs:#1,0});
    }
}}

% \pointsinrightmargin
\boxedpoints
\pointname{}

\newcommand{\questioA}{\question[\texttt{\textbf{\color{Cerulean} A}}]}
\newcommand{\questioM}{\question[\texttt{\textbf{\color{PineGreen} M}}]}
\newcommand{\questioE}{\question[\texttt{\textbf{\color{WildStrawberry} E}}]}
\newcommand{\questioS}{\question[\texttt{\textbf{\color{Goldenrod} S}}]}
\newcommand{\questioO}{\question[\texttt{\textbf{\color{BurntOrange} O}}]}

\newcommand{\parA}{\part[\texttt{\textbf{\color{Cerulean} A}}]}
\newcommand{\parM}{\part[\texttt{\textbf{\color{PineGreen} M}}]}
\newcommand{\parE}{\part[\texttt{\textbf{\color{WildStrawberry} E}}]}
\newcommand{\parS}{\part[\texttt{\textbf{\color{Goldenrod} S}}]}
\newcommand{\parO}{\part[\texttt{\textbf{\color{BurntOrange} O}}]}

\newcommand{\subparA}{\subpart[\texttt{\textbf{\color{Cerulean} A}}]}
\newcommand{\subparM}{\subpart[\texttt{\textbf{\color{PineGreen} M}}]}
\newcommand{\subparE}{\subpart[\texttt{\textbf{\color{WildStrawberry} E}}]}
\newcommand{\subparS}{\subpart[\texttt{\textbf{\color{Goldenrod} S}}]}
\newcommand{\subparO}{\subpart[\texttt{\textbf{\color{BurntOrange} O}}]}

\newcommand{\mainHeader}[2]{\section*{NCEA Level 2 Mathematics\\#1. #2}}
\newcommand{\mainHeaderHw}[2]{\section*{NCEA Level 2 Mathematics (Homework)\\#1. #2}}

\begin{document}

\section*{NCEA Level 3 Calculus\\Prior Revision: Functions}
\begin{center}
  \includegraphics[width=0.8\textwidth]{hobbes}
\end{center}
\subsection* {What is Calculus?}
Calculus is the study of:
\begin{itemize}
  \item Continuous change.
  \item Slope, area, and volume.
  \item Functions and relationships.
\end{itemize}

It has applications in:
\begin{itemize}
  \item Physics and chemistry.
  \item Probability theory.
  \item Population theory.
  \item Economics (I am assured).
\end{itemize}

In pure mathematics, calculus can be seen as the computational side of a pretty
subject called \textbf{real analysis}.

\subsection*{Revision of Functions}
Arguably, the most fundamental concept in calculus is that of a \textit{function}.

\begin{defn}[Function]
  A function is a something which takes a set of things (for example, the real numbers $ \mathbb{R} $) and assigns
  to each one exactly one thing (which could be the same or different).
\end{defn}

\begin{ex}
  The map which takes a number $ x $ and spits out $ x^2 $ is a function --- for every input, there is exactly one output. If
  we \textit{graph} this function, we plot its input on the $ x-$axis and its output on the $ y-$axis and obtain a parabola.
\end{ex}

\begin{ex}
  The curve graphed below is \textit{not} a function, since for some inputs (like 1) it has more than one output. We can check this
  by drawing vertical lines along the function, like that pictured: if a graph is a function, no vertical line can ever cross the curve
  more than once (this is the \textit{vertical-line test}).

  \begin{center}
    \includegraphics[width=0.3\textwidth]{linetest}
  \end{center}
\end{ex}

\begin{ex}
  The map $ f : x \mapsto \sin x $ is a function. We could also define it by `the function $ f $ such that $ f(x) = \sin x $'.
  This function $ f $ can only produce numbers between $1$ and $-1$; we say that its \textit{range} is the interval from $ -1 $ to 1.
\end{ex}

\begin{ex}
  The map $ \iota : x \mapsto x $ is a function, called the \textit{identity function}.
\end{ex}

\begin{ex}
  The map $ \ln x $ is a function, but it is only defined when $ x > 0 $: we say that its \textit{domain} is the positive real numbers.
\end{ex}

\begin{ex}
  Some more non-examples:

  \begin{center}
    \includegraphics[width=0.5\textwidth]{notfunction}
  \end{center}
\end{ex}

\subsection*{Questions}
\begin{questions}
  \question Which of the following are functions?
    \begin{parts}
      \part $ E(x) = 2^x $
      \part $ \phi : x \mapsto \frac{2}{x} $
      \part The thing which maps every person to their youngest sibling.
      \part The thing which sends every person to their youngest sibling that isn't themself.
      \part $ x \mapsto \lfloor x \rfloor $ (the floor map).
      \part The relation that sends every person to their age.
    \end{parts}
  \question I will define two functions, $ \varphi $ and $ \vartheta $, as follows:
            \begin{displaymath}
              \varphi(x) = 2x - 7, \qquad \vartheta(\zeta) = \frac{1}{7}(14\zeta - 49).
            \end{displaymath}
            Explain why these functions are equal.
  \question If $ f(x) = x^2 + x $, find:
    \begin{parts}
      \item $ f(1) $
      \item $ f(y) $
      \item $ f(x + h) $
    \end{parts}
  \question Find the distance between $ (-3, 4) $ and $ (2, 1) $.
  \question Three sides of a triangle are have lengths 8, 15, and 17.
    \begin{parts}
      \part Show that the triangle is right-angled.
      \part Find the other two angles.
    \end{parts}
  \question Factorise and solve $ x^2 - 3x + 2 = 0 $.
  \question How many \textbf{lines} are there through the point $ (2,3) $ and the origin? Give the equations of all such lines.
  \question Find the slope of the line $ 4x + 3y + 2 = 0 $.
  \question Find the solution to the following linear system:
            \begin{align*}
              2x + y &= 7\\
              3x - y &= 8
            \end{align*}
  \question How many (real) solutions does $ x^2 + 4x + 1 $ have?
  \question Draw $ \sin(x) $, $ \cos(x) $, $ \tan(x) $, $ \exp(x) $, and $ \ln(x) $.
  \question How many solutions does $ \cos (3\pi x + 1) = 2 $ have?
  \question How many solutions does $ \sin (3x) = \frac{1}{3} $ have?
\end{questions}

\end{document}
