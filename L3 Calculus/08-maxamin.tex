\documentclass{exam}
\usepackage[utf8]{inputenc}
\usepackage{lmodern}
\usepackage{microtype}

% \usepackage[parfill]{parskip}
\usepackage[dvipsnames]{xcolor}
\usepackage{amsmath}
\usepackage{amsfonts}
\usepackage{amsthm}
\usepackage{siunitx}
\DeclareSIUnit\year{yr}
\DeclareSIUnit\foot{ft}
\DeclareSIUnit\litre{\liter}

\usepackage{skull}

\usepackage{pgfplots}
\usepgfplotslibrary{polar}
\pgfplotsset{compat=1.11}
\usepackage{graphicx}
\usepackage{sidecap}
\sidecaptionvpos{figure}{c}
\usepackage{float}
\usepackage{gensymb}
\usepackage{tkz-euclide}
\usetkzobj{all}
\usepackage{commath}
\usepackage{hyperref}
\usepackage{enumitem}
\usepackage{wasysym}
\usepackage{multicol}
\usepackage{mathtools}
\usepackage{tcolorbox}
\usepackage{tabularx}
\usepackage[version=4]{mhchem}
\usepackage{changepage}
\usepackage{listings}
\lstset{basicstyle=\ttfamily\linespread{0.8}\small}

\renewcommand*{\thefootnote}{\fnsymbol{footnote}}

\newtheorem*{thm}{Theorem}
\newtheorem*{iden}{Identity}
\newtheorem*{lemma}{Lemma}
\newtheorem{obs}{Observation}
\theoremstyle{definition}
\newtheorem*{defn}{Definition}
\newtheorem*{ex}{Example}
\newtheorem{con}{Construction}
\newtheorem*{alg}{Algorithm}

\newtheoremstyle{break}
  {\topsep}{\topsep}%
  {\itshape}{}%
  {\bfseries}{}%
  {\newline}{}%
\theoremstyle{break}
\newtheorem*{bthm}{Theorem}

% russian integral
\usepackage{scalerel}
\DeclareMathOperator*{\rint}{\scalerel*{\rotatebox{17}{$\!\int\!$}}{\int}}

% \DeclareMathOperator*{\rint}{\int}

\pgfplotsset{vasymptote/.style={
    before end axis/.append code={
        \draw[densely dashed] ({rel axis cs:0,0} -| {axis cs:#1,0})
        -- ({rel axis cs:0,1} -| {axis cs:#1,0});
    }
}}

% \pointsinrightmargin
\boxedpoints
\pointname{}

\newcommand{\questioA}{\question[\texttt{\textbf{\color{Cerulean} A}}]}
\newcommand{\questioM}{\question[\texttt{\textbf{\color{PineGreen} M}}]}
\newcommand{\questioE}{\question[\texttt{\textbf{\color{WildStrawberry} E}}]}
\newcommand{\questioS}{\question[\texttt{\textbf{\color{Goldenrod} S}}]}
\newcommand{\questioO}{\question[\texttt{\textbf{\color{BurntOrange} O}}]}

\newcommand{\parA}{\part[\texttt{\textbf{\color{Cerulean} A}}]}
\newcommand{\parM}{\part[\texttt{\textbf{\color{PineGreen} M}}]}
\newcommand{\parE}{\part[\texttt{\textbf{\color{WildStrawberry} E}}]}
\newcommand{\parS}{\part[\texttt{\textbf{\color{Goldenrod} S}}]}
\newcommand{\parO}{\part[\texttt{\textbf{\color{BurntOrange} O}}]}

\newcommand{\subparA}{\subpart[\texttt{\textbf{\color{Cerulean} A}}]}
\newcommand{\subparM}{\subpart[\texttt{\textbf{\color{PineGreen} M}}]}
\newcommand{\subparE}{\subpart[\texttt{\textbf{\color{WildStrawberry} E}}]}
\newcommand{\subparS}{\subpart[\texttt{\textbf{\color{Goldenrod} S}}]}
\newcommand{\subparO}{\subpart[\texttt{\textbf{\color{BurntOrange} O}}]}

\newcommand{\mainHeader}[2]{\section*{NCEA Level 2 Mathematics\\#1. #2}}
\newcommand{\mainHeaderHw}[2]{\section*{NCEA Level 2 Mathematics (Homework)\\#1. #2}}

\begin{document}

\mainHeaderDiff{8}{Optimisation}
Recall from Level 2 that a \textbf{local maximum} of a function $ f $ is some point $ (x, f(x)) $ such that, for a sufficiently
small interval around $ x $, whenever $ y $ is in the interval then $ f(y) \leq f(x) $. A \textbf{local minimum} is defined in
a similar way. Local extrema are also sometimes called \textbf{relative extrema}.

Many optimisation problems in applied mathematics can be reduced to finding relative extrema.

\begin{exs}\leavevmode
  \begin{enumerate}
    \item The function $ x \mapsto x^2 $ has a local minimum at $ (0, 0) $.
    \item The function $ x \mapsto 2x^3 + 15x^2 + 36x + 2 $ has a local maximum at $ (-3, -25) $ and a local minimum at $ (-2, -26) $.
    \item The function $ x \mapsto \sin x $ has a local maximum at $ (2n\pi + \frac{\pi}{2}, 1) $ for every integer $ n $, and
          a local minimum at $ (2n\pi - \frac{\pi}{2}, 1) $ for every integer $ n $.
  \end{enumerate}
\end{exs}

For classification, we have the following theorem which links the location of relative extrema to the value of the derivative. Rather than
memorising the proof, you should remember the geometric idea:- the derivative is changing from a positive value to a negative value (or vice
versa), and so must pass through zero.

\begin{thm}[Fermat's theorem]
  Let $ f $ be a function; suppose $ x_0 $ is a point in the interior of the domain of $ f $, and that $ f $ has a relative extremum
  at $ (x_0, f(x_0)) $. Then $ f'(x_0) = 0 $.
\end{thm}
The proof of this is relatively straightforward; we just need the concept of left- and right-handed limits. Recall that, roughly speaking,
a function has a limit at a point if it approaches the same value from both the left and the right. For left- and right-handed limits, we
only require the function to approach a value from one side or the other.
\begin{proof}[Proof of Fermat's theorem (optional).]
  Suppose $ f $ attains a relative maximum at $ x_0 $. Then for all $ h $ sufficiently close to zero, we have $ f(x_0 + h) - f(x_0) \leq 0 $.
  Hence, if $ h < 0 $, we have $ \frac{f(x_0 + h) - f(x_0)}{h} \geq 0 $ (i.e. the derivative to the left is positive) and if $ h > 0 $, we
  have $ \frac{f(x_0 + h) - f(x_0)}{h} \leq 0 $ (i.e. the derivative to the right is negative). Taking left- and right-hand limits around
  zero, we have the following chain of inequalities:
  \begin{displaymath}
    f'(x_0) = \lim_{h \to 0-} \frac{f(x_0 + h) - f(x_0)}{h} \geq 0 \geq \lim_{h \to 0+} \frac{f(x_0 + h) - f(x_0)}{h} = f'(x_0).
  \end{displaymath}
  Note that we use the fact that the derivative exists at $ x_0 $ and so the left- and right-hand limits both tend to the same value. Then the
  desired result, $ f'(x) = 0 $, follows directly.

  For a relative minimum, the proof is essentially the same but with some inequalities swapped.
\end{proof}

Motivated by this theorem, we define a \textbf{critical point} of a function $ f $ to be some value $ x $ in the domain of $ f $ such
that either $ f'(x) = 0 $, or $ f'(x) $ is undefined. In the first case, we also call the value a \textbf{stationary point}. All local
extrema occur at critical points, but not all critical points occur at extrema.

\begin{exs}\leavevmode
  \begin{enumerate}
    \item The function $ x \mapsto 2x^3 + 15x^2 + 36x + 2 $ above has critical points $ x = -2 $ and $ x = -3 $. Both of
          these are local extrema.
    \item The function $ x \mapsto x^3 $ above has a critical point at $ x = 0 $, but does not have a local extrema there.
    \item The function $ x \mapsto \frac{1}{x} $ \textit{does not} have a critical point at $ x = 0 $, \textbf{because it is not defined there}.
  \end{enumerate}
\end{exs}

\subsection*{Classifying Critical Points}
We can use the first derivative to classify extrema as either maxima or minima.
\begin{enumerate}
  \item Determine all critical points of $ f $.
  \item Determine the sign of $ f'(x) $ to the left and right of each critical point $ x_0 $:
    \begin{itemize}
      \item If $ f'(x) $ changes from positive to negative as we move from left to right across $ x_0 $, then $ f(x) $ has a local maximum at $ x_0 $.
      \item If $ f'(x) $ changes from negative to positive as we move from left to right across $ x_0 $, then $ f(x) $ has a local minimum at $ x_0 $.
      \item If $ f'(x) $ does not change sign across $ x_0 $, then $ f(x) $ does not have a relative extremum at $ x_0 $ (e.g. $ y = x^3 $).
    \end{itemize}
\end{enumerate}

On the other hand, using the second derivative, we can come up with a second test:
\begin{enumerate}
  \item Compute $ f'(x) $ and $ f''(x) $.
  \item Find all the stationary points of $ f $ by finding all the points $ x_0 $ such that $ f'(x_0) = 0 $.
  \item Determine the sign of $ f''(x) $ for each stationary point $ x_0 $:
    \begin{itemize}
      \item If $ f''(x_0) < 0 $, then $ f(x) $ has a relative maximum at $ x_0 $.
      \item If $ f''(x_0) > 0 $, then $ f(x) $ has a relative minimum at $ x_0 $.
      \item If $ f''(x_0) = 0 $, then $ f(x) $ could have a relative maximum, a relative minimum, or neither.
    \end{itemize}
\end{enumerate}

\begin{ex}
  Find and classify the critical points of $ y = x^3 - 3x^2 + 6 $.

  \textit{Solution.} We have $ \od{y}{x} = 3x^2 - 6x $ and $ \od[2]{y}{x} = 6x - 6 $. Hence
                     the critical points are $ x = 0 $ and $ x = 2 $. At the former point, $ \od[2]{y}{x} < 0 $,
                     and so the point is a maximum; at the latter point, $ \od[2]{y}{x} > 0 $ and so the point is
                     a minimum.
\end{ex}

\begin{ex}
  Find two numbers whose difference is 100 and whose product is a minimum.

  \textit{Solution.} Let the two numbers be $ x $ and $ x + 100 $. We wish to minimise $ y = x(x + 100) $;
  clearly $ y' = 2x + 100 $, and so $ x = -50 $ is a critical point. To the left of $ x = -50 $, the derivative
  is negative; to the right, the derivative is positive. Hence $ x = -50 $ is indeed a minimum. The two required
  numbers are therefore -50 and 50.
\end{ex}

\begin{ex}
  Find and classify the critical points of $ y = (x - 1)^2 + \ln x $.

  \textit{Solution.} The derivative is $ y' = 2x - 2 + \frac{1}{x} $. We therefore have one critical
  point at $ x = 0 $ (where $ y' $ is undefined); this is an asymptote.
  Setting $ y' = 0 $, we have $ 0 = 2x - 2 + \frac{1}{x} = 2x^2 - 2x + 1 $ which has no real roots. Hence $ x = 0 $ is
  the only critical point, and the curve has no local extrema.
\end{ex}

\begin{ex}
  A rectangular plot of land is to be fenced using two varieties of fence. Two opposite sides will
  use fences selling for \$3 per metre, while the other two sides will use cheaper fence selling for \$2 per metre.
  Given that the total budget is \$1200, what is the greatest area of land which can be fenced?

  \textit{Solution.} Let $ x $ be the length of one of the expensive sides; then the length of one of the cheaper
                    sides is $ \frac{1}{2}(1200 - 3x) $, and the total area is $ A = \frac{1}{2} x (1200 - 3x) = \frac{1}{2}(1200x - 3x^2) $.
                    Hence $ \od{A}{x} = 600 - 3x $. We wish to find the maximum area, so set $ \od{A}{x} = 0 $; hence $ 3x = 600 $ and $ x = 200 $.
                    Note that the second derivative is always negative, so this stationary point must be a maximum as required. The length
                    of the other side will be $ \frac{1}{2}(1200 - 600) = 300 $, and so the maximum area is $ 300 \times 200 = 60000 $ square metres.
\end{ex}

\subsection*{Questions}
\begin{questions}
  \questioA Write down a definition of a local minimum similar to that given for a maximum.
  \questioA Show that $ f(x) = x^4 $ has $ f''(0) = 0 $ but not a point of inflection at $ x = 0 $ (in fact, it has a minimum at that point).
  \questioA Describe the advantages and disadvantages of the first and second derivative tests for local extrema.
  \questioM Describe the local extrema, concavity, and points of inflection of the function $ f(x) = x^4 - 4x^3 $.
  \questioA Consider the following graph:
            \begin{center}
              \includegraphics[width=0.3\textwidth]{curvature}
            \end{center}
            Find the signs of $ \od{y}{x} $ and $ \od[2]{y}{x} $ at the three points $ A $, $ B $, and $ C $.
  \questioM Find all the local extrema of the following curves in the given intervals, and classify them as maxima, minima, or neither.
    \begin{parts}
      \part $ f(x) = \sin x - \cos x $ on the interval $ 0 < x < \pi $
      \part $ g(x) = x^3 - x^2 + x - 1 $ on the interval $ -\infty < x < \infty $
    \end{parts}
  \questioM The sum of two positive numbers $ x $ and $ y $ is 16. Find the smallest possible value for $ S = x^2 + y^2 $.
  \questioM A box with an open top is to be constructed from a square piece of cardboard with a side length of \SI{3}{\metre}
            by cutting out a square from each of the four corners and bending up the sides. Find the dimensions of the resultant
            box of maximum volume.
  \questioM Find the dimensions of a rectangle with area \SI{1000}{\metre\squared} such that the perimeter is minimised.
  \questioM A window consisting of a rectangle topped with a semicircle is to have a fixed perimeter $ p $. Find the radius
            of the semicircle in terms of $ p $ if the total area is to be maximised.
  \questioE A thin wire of length $ L $ is cut in two and the resulting lengths are bent to make a square and an equilateral triangle. Where
            should the wire be cut to make the total area of the shapes (a) a maximum and (b) a minimum?
  \questioE Find the point on the line $ y = 2x + 3 $ closest to the origin.
  \questioE Find the point on the curve $ y = \sqrt{x} $ closest to $ (3, 0) $.
  \questioE By finding the $ x$- and $ y$- intercepts, the asymptotes, the critical points, the
            intervals of increase and decrease, the intervals of concavity, and any other important
            points, sketch the following functions (199):
    \begin{parts}
      \part $ f(x) = \frac{x^2}{4 - x^2} $
      \part $ f(x) = \frac{4x}{x^2 + 1} $ [\textit{Hint: consider what happens to $ f(x) $ as $ x \to \pm\infty $.}]
      \part $ f(x) = \frac{x^2 - 4x + 5}{x - 2} = x - 2 + \frac{1}{x - 2} $ [\textit{Hint: consider what happens to $ f(x) - (x - 2) $ as $ x \to \pm\infty $.}]
    \end{parts}
  \questioE A cone with height $ h $ is inscribed in a larger cone of height $ H $ such that the vertex of the small cone
            is at the centre of the base of the larger cone. Show that the maximum volume of the smaller cone occurs when $ h = \frac{1}{3} H $.
  \questioE Show that the polynomial $ p(x) = 10x^3 + x^2 + x - 34 $ has exactly one real zero.
  \questioE A rain gutter is to be constructed from a metal sheet of width \SI{30}{\centi\metre} by bending up
            one third of the sheet on each side by an angle $ \theta $. What angle should be chosen in order to
            obtain the maximum possible volume?
  \questioE A steel pipe is carried around a right-angled corner from a hallway \SI{3}{\metre} wide into a hallway \SI{2}{\metre}
            wide. What is the length of the longest pipe that can be carried horizontally around the corner? [\textit{Hint: this is actually
            a minimisation problem, despite the wording.}]
  \questioE A large orange rectangle is to be drawn with one corner sitting on the origin and the opposite corner lying on
            the curve $ y = 0.2(x - 10)^2 $. What is the maximum possible area of the rectangle?
            \begin{center}
              \includegraphics[width=0.3\textwidth]{paramax}
            \end{center}
  \questioS Show that $ \frac{x^2 + 1}{x} \geq 2 $; hence (or otherwise) show that $ \frac{(x^2 + 1)(y^2 + 1)(z^2 + 1)}{xyz} \geq 8 $.
  \questioS Scholarship 2013: Prince Ruperts drops are made by dropping molten glass into cold water. A mathematical model for a drop
            as a volume of revolution uses $ y = \sqrt{\phi (e^{-x} - e^{-2x})} $ for $ x \geq 0 $, where $ \phi $ is the golden
            ratio $ \phi = \frac{1 + \sqrt{5}}{2} $.
    \begin{parts}
      \part Where is the modelled drop widest, and how wide is it there?
      \part The drop changes shape at a point $ B $, where the concavity of the function is zero. Use
            \begin{displaymath}
              \od[2]{y}{x} = \sqrt{\phi} \frac{e^{2x} - 6e^x + 4}{y^2 e^{4x}}
            \end{displaymath}
            to find the exact $ x$-ordinate of $ B $.
    \end{parts}
  \questioS Scholarship 2014: A family of functions is built from two functions $ f $ and $ g $, with a new function $ h_p $ defined
            for each value of $ p $, $ 0 \leq p \leq 1 $:
            \begin{gather*}
              f(x) = 2 + \sin x\\
              g(x) = 26 + \sin x\\
              h_p(x) = [f(x)]^{1 - p} [g(x)]^p.
            \end{gather*}
            Define a fourth function $ S $, where $ S(p) $ is the difference between the maximum and the minimum values of $ h_p $. Find
            the exact value of $ p $ that maximises $ S $.

            Note that if $ a $ is constant, $ \od{}{x} a^x = (\ln a) a^x $.
\end{questions}
\end{document}
