\documentclass{exam}
\usepackage[utf8]{inputenc}
\usepackage{lmodern}
\usepackage{microtype}

% \usepackage[parfill]{parskip}
\usepackage[dvipsnames]{xcolor}
\usepackage{amsmath}
\usepackage{amsfonts}
\usepackage{amsthm}
\usepackage{siunitx}
\DeclareSIUnit\year{yr}
\DeclareSIUnit\foot{ft}
\DeclareSIUnit\litre{\liter}

\usepackage{skull}

\usepackage{pgfplots}
\usepgfplotslibrary{polar}
\pgfplotsset{compat=1.11}
\usepgfplotslibrary{statistics}
\usepackage{graphicx}
\usepackage{sidecap}
\sidecaptionvpos{figure}{c}
\usepackage{float}
\usepackage{gensymb}
\usepackage{tkz-euclide}
\usetkzobj{all}
\usepackage{commath}
\usepackage{hyperref}
\usepackage{enumitem}
\usepackage{wasysym}
\usepackage{multicol}
\usepackage{mathtools}
\usepackage{tcolorbox}
\usepackage{tabularx}
\usepackage[version=4]{mhchem}
\usepackage{changepage}
\usepackage{listings}
\lstset{basicstyle=\ttfamily\linespread{0.8}\small}

\renewcommand*{\thefootnote}{\fnsymbol{footnote}}

\newtheorem*{thm}{Theorem}
\newtheorem*{iden}{Identity}
\newtheorem*{lemma}{Lemma}
\newtheorem{obs}{Observation}
\theoremstyle{definition}
\newtheorem*{defn}{Definition}
\newtheorem*{ex}{Example}
\newtheorem{con}{Construction}
\newtheorem*{alg}{Algorithm}

\newtheoremstyle{break}
  {\topsep}{\topsep}%
  {\itshape}{}%
  {\bfseries}{}%
  {\newline}{}%
\theoremstyle{break}
\newtheorem*{bthm}{Theorem}

% russian integral
\usepackage{scalerel}
\DeclareMathOperator*{\rint}{\scalerel*{\rotatebox{17}{$\!\int\!$}}{\int}}

% \DeclareMathOperator*{\rint}{\int}

\pgfplotsset{vasymptote/.style={
    before end axis/.append code={
        \draw[densely dashed] ({rel axis cs:0,0} -| {axis cs:#1,0})
        -- ({rel axis cs:0,1} -| {axis cs:#1,0});
    }
}}

% \pointsinrightmargin
\boxedpoints
\pointname{}

\newcommand{\questioA}{\question[\texttt{\textbf{\color{Cerulean} A}}]}
\newcommand{\questioM}{\question[\texttt{\textbf{\color{PineGreen} M}}]}
\newcommand{\questioE}{\question[\texttt{\textbf{\color{WildStrawberry} E}}]}
\newcommand{\questioS}{\question[\texttt{\textbf{\color{Goldenrod} S}}]}
\newcommand{\questioO}{\question[\texttt{\textbf{\color{BurntOrange} O}}]}

\newcommand{\parA}{\part[\texttt{\textbf{\color{Cerulean} A}}]}
\newcommand{\parM}{\part[\texttt{\textbf{\color{PineGreen} M}}]}
\newcommand{\parE}{\part[\texttt{\textbf{\color{WildStrawberry} E}}]}
\newcommand{\parS}{\part[\texttt{\textbf{\color{Goldenrod} S}}]}
\newcommand{\parO}{\part[\texttt{\textbf{\color{BurntOrange} O}}]}

\newcommand{\subparA}{\subpart[\texttt{\textbf{\color{Cerulean} A}}]}
\newcommand{\subparM}{\subpart[\texttt{\textbf{\color{PineGreen} M}}]}
\newcommand{\subparE}{\subpart[\texttt{\textbf{\color{WildStrawberry} E}}]}
\newcommand{\subparS}{\subpart[\texttt{\textbf{\color{Goldenrod} S}}]}
\newcommand{\subparO}{\subpart[\texttt{\textbf{\color{BurntOrange} O}}]}

\newcommand{\mainHeader}[2]{\section*{NCEA Level 2 Mathematics\\#1. #2}}
\newcommand{\mainHeaderHw}[2]{\section*{NCEA Level 2 Mathematics (Homework)\\#1. #2}}
\newcommand{\seealso}[1]{\begin{center}\emph{See also #1.}\end{center}}
\newcommand{\drills}[1]{\begin{center}\emph{Drill problems: #1.}\end{center}}
\newcommand{\basedon}[1]{\begin{center}\emph{Notes largely based on #1.}\end{center}}

\begin{document}

\mainHeaderIntgHw{19}{Differential Equations}
\subsection*{Reading}
So far, we have looked at equations where the unknown is either a number or a point in $ n$-dimensional space (that is, a sequence of $ n $ numbers).
In order to generate these equations, we took various combinations of the basic arithmetical operations and applied them to our unknowns.

Here, for comparison, are two well-known differential equations, the first ``ordinary'' and the second ``partial'':
\begin{gather*}
  \dod[2]{x}{t} + k^2 x = 0,\\
  \dpd{T}{t} = \kappa\left( \dpd[2]{T}{x} + \dpd[2]{T}{y} + \dpd[2]{T}{z} \right).
\end{gather*}
The first is the equation for simple harmonic motion, which has the general solution $ x(t) = A\sin kt + B\cos kt $;
the second is the heat equation which describes the way that the distribution of heat in a physical medium changes with
time.

For many reasons, differential equations represent a jump in sophistication. One is that the unknowns are \emph{functions},
which are much more complicated objects than numbers or $ n$-dimensional points. (For example, the first equation above
asks what function $ x $ of $ t $ has the property that if you differentiate it twice then you get $ -k^2 $ times the original
function.) A second is that the basic operations one performs on functions include differentiation and integration, which are
considerably less ``basic'' than addition and multiplication. A third is that differential equations which can be solved
in ``closed form,'' that is, by means of a formula for the unknown function $ f $, are the exception rather than the rule, even
when the equations are natural and important.

\textit{From `The Princeton Companion to Mathematics', I.4 \S 1.5}

\subsection*{Questions}
\begin{questions}
  \question Solve the following equations for $ y(t) $:
    \begin{parts}
      \part $ e^{y - t} \od{y}{t} = 1 $
      \part $ \od{y}{t} = ty^2 $
      \part $ \od{y}{t} = \frac{1}{\sec^2 y} $
      \part $ \od{y}{t} = -\frac{t}{\sec t \sin y} $ (\textit{Hint: first show that} $ \od{}{x} [\cos x - x\sin x] = -x\cos x $)
    \end{parts}
  \question A copper ball with temperature \SI{100}{\celsius} is dropped into a basin of water with constant temperature
            \SI{30}{\celsius}. After 3 minutes the temperature of the ball has decreased to \SI{70}{\celsius}. When will
            it reach a temperature of \SI{31}{\celsius}?
  \question Consider a tank of water. The rate of flow of water into the tank is a constant \SI{3}{\litre\per\second};
            the flow out is directly proportional to the volume of water in the tank. Initially, the volume of water
            in the tank is \SI{100}{\litre}; if the volume were to increase to  \SI{120}{\litre\per\second}, the rate
            of water flowing out would exactly balance the rate of water flowing in.
            \begin{parts}
              \part Form a differential equation and find the volume of water after ten minutes.
              \part Does the outward rate of flow ever become greater than the incoming rate of flow?
            \end{parts}
\end{questions}
\end{document}
