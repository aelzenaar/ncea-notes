\documentclass{exam}
\usepackage[utf8]{inputenc}
\usepackage{lmodern}
\usepackage{microtype}

% \usepackage[parfill]{parskip}
\usepackage[dvipsnames]{xcolor}
\usepackage{amsmath}
\usepackage{amsfonts}
\usepackage{amsthm}
\usepackage{siunitx}
\DeclareSIUnit\year{yr}
\DeclareSIUnit\foot{ft}
\DeclareSIUnit\litre{\liter}

\usepackage{skull}

\usepackage{pgfplots}
\usepgfplotslibrary{polar}
\pgfplotsset{compat=1.11}
\usepgfplotslibrary{statistics}
\usepackage{graphicx}
\usepackage{sidecap}
\sidecaptionvpos{figure}{c}
\usepackage{float}
\usepackage{gensymb}
\usepackage{tkz-euclide}
\usetkzobj{all}
\usepackage{commath}
\usepackage{hyperref}
\usepackage{enumitem}
\usepackage{wasysym}
\usepackage{multicol}
\usepackage{mathtools}
\usepackage{tcolorbox}
\usepackage{tabularx}
\usepackage[version=4]{mhchem}
\usepackage{changepage}
\usepackage{listings}
\lstset{basicstyle=\ttfamily\linespread{0.8}\small}

\renewcommand*{\thefootnote}{\fnsymbol{footnote}}

\newtheorem*{thm}{Theorem}
\newtheorem*{iden}{Identity}
\newtheorem*{lemma}{Lemma}
\newtheorem{obs}{Observation}
\theoremstyle{definition}
\newtheorem*{defn}{Definition}
\newtheorem*{ex}{Example}
\newtheorem{con}{Construction}
\newtheorem*{alg}{Algorithm}

\newtheoremstyle{break}
  {\topsep}{\topsep}%
  {\itshape}{}%
  {\bfseries}{}%
  {\newline}{}%
\theoremstyle{break}
\newtheorem*{bthm}{Theorem}

% russian integral
\usepackage{scalerel}
\DeclareMathOperator*{\rint}{\scalerel*{\rotatebox{17}{$\!\int\!$}}{\int}}

% \DeclareMathOperator*{\rint}{\int}

\pgfplotsset{vasymptote/.style={
    before end axis/.append code={
        \draw[densely dashed] ({rel axis cs:0,0} -| {axis cs:#1,0})
        -- ({rel axis cs:0,1} -| {axis cs:#1,0});
    }
}}

% \pointsinrightmargin
\boxedpoints
\pointname{}

\newcommand{\questioA}{\question[\texttt{\textbf{\color{Cerulean} A}}]}
\newcommand{\questioM}{\question[\texttt{\textbf{\color{PineGreen} M}}]}
\newcommand{\questioE}{\question[\texttt{\textbf{\color{WildStrawberry} E}}]}
\newcommand{\questioS}{\question[\texttt{\textbf{\color{Goldenrod} S}}]}
\newcommand{\questioO}{\question[\texttt{\textbf{\color{BurntOrange} O}}]}

\newcommand{\parA}{\part[\texttt{\textbf{\color{Cerulean} A}}]}
\newcommand{\parM}{\part[\texttt{\textbf{\color{PineGreen} M}}]}
\newcommand{\parE}{\part[\texttt{\textbf{\color{WildStrawberry} E}}]}
\newcommand{\parS}{\part[\texttt{\textbf{\color{Goldenrod} S}}]}
\newcommand{\parO}{\part[\texttt{\textbf{\color{BurntOrange} O}}]}

\newcommand{\subparA}{\subpart[\texttt{\textbf{\color{Cerulean} A}}]}
\newcommand{\subparM}{\subpart[\texttt{\textbf{\color{PineGreen} M}}]}
\newcommand{\subparE}{\subpart[\texttt{\textbf{\color{WildStrawberry} E}}]}
\newcommand{\subparS}{\subpart[\texttt{\textbf{\color{Goldenrod} S}}]}
\newcommand{\subparO}{\subpart[\texttt{\textbf{\color{BurntOrange} O}}]}

\newcommand{\mainHeader}[2]{\section*{NCEA Level 2 Mathematics\\#1. #2}}
\newcommand{\mainHeaderHw}[2]{\section*{NCEA Level 2 Mathematics (Homework)\\#1. #2}}
\newcommand{\seealso}[1]{\begin{center}\emph{See also #1.}\end{center}}
\newcommand{\drills}[1]{\begin{center}\emph{Drill problems: #1.}\end{center}}
\newcommand{\basedon}[1]{\begin{center}\emph{Notes largely based on #1.}\end{center}}

\begin{document}

\mainHeaderDiff{5}{The Product and Quotient Rules}
Two weeks ago we saw that the derivative of a product is not simply the product of the derivatives;
for example, take $ (x)(x) $. If we differentiate each term and multiply, we obtain 1; however,
the derivative of $ x^2 $ is (of course) $ 2x $.

Suppose $ f $ and $ g $ are functions; then the \textit{real} product rule is
\begin{displaymath}
  (fg)' = gf' + fg'.
\end{displaymath}

\begin{ex}
  Consider $ y = 2t \sin t $. Then $ \od{y}{t} = 2 \sin t + 2t \cos t $.
\end{ex}

We can also write a rule for the derivative of a quotient of functions. You will be asked to
prove it as an exercise, using the product rule.
\begin{displaymath}
  \left( \frac{f}{g} \right)' = \frac{gf' - fg'}{g^2}
\end{displaymath}

\begin{ex}
  Consider $ f(s) = \sqrt{\frac{s^2 + 1}{s^2 + 4}} $. Then
  \begin{align*}
    f'(s) &= \od{}{s} \left[ \frac{s^1 + 1}{s^2 + 4} \right] \cdot \frac{1}{2\sqrt{\frac{s^2 + 1}{s^2 + 4}}}\\
          &= \frac{(s^2 + 4)(2s) - (s^2 + 1)(2s)}{(s^2 + 4)^2} \cdot \frac{\sqrt{s^2 + 4}}{2\sqrt{s^2 + 1}}\\
          &= \frac{6s}{(s^2 + 4)^{\frac{3}{2}} (s^2 + 1)^{\frac{1}{2}}}
  \end{align*}
\end{ex}

\clearpage
\subsection*{Questions}
\begin{questions}
  \questioA In each case, find $ \od{y}{t} $.
    \begin{multicols}{2}
    \begin{parts}
      \part $ y = \left(3 + 2t^2\right)^4 $
      \part $ y = \frac{t^3}{\ln t} $
      \part $ y = t\sqrt{t} $
      \part $ y = 2t \sin t - (t^2 - 2) \cos t $
      \part $ y = \frac{t}{\sqrt{a^2 - t^2}} $ ($ a $ constant)
      \part $ y = \frac{1}{8} t^8 \left(1 - t^2\right)^{-4} $
      \part $ y = e^t \ln t $
      \part $ y = \log \left[1 + \frac{t^2 + 3t + 17}{t^{17}}\right] $
      \part $ y = \sin \left[e^{\tan t} \ln \tan t\right] $
      \part $ y = \frac{3t - 2}{\sqrt{2t + 1}} $
      \part $ y = \frac{\sec 2t}{1 + \tan 2t} $
      \part $ y = \frac{(t - 1)(t - 4)}{(t - 2)(t - 3)} $
      \part $ y = t \sin^2(\cos \sqrt{\sin \pi t}) $
      \part $ y = \sqrt[5]{t \tan t} $
      \part $ y = \dfrac{(t + \lambda)^4}{t^4 + \lambda^4} $
    \end{parts}
    \end{multicols}
  \questioA If $ f(x) = e^{-x} $, find $ f(0) + xf'(0) $.
  \questioM Suppose $ f $ and $ g $ are functions ($ g $ not the zero function). Write $ \frac{f}{g} = fg^{-1} $ and
            prove the quotient rule from the product rule.
  \questioM Show that if $ f $, $ g $, and $ h $ are functions then $ (fgh)' = f'gh + fg'h + fgh' $.
  \questioE Suppose $ f(x) = f(-x) $ for all $ x $ in the domain of $ f $. Prove that $ f'(x) = f'(-x) $ for all $ x $
            in the domain of $ f'(x) $.
  \questioE Consider the function defined by $ f(x) = x^x $.
    \begin{parts}
      \part Rewrite $ f $ in the form $ f(x) = e^{x \ln x} $, and hence find $ f'(x) $.
      \part Find $ \od{y}{t} $ if $ y = (t^2 + 3)^{(t^2 + 3)} $.
    \end{parts}
  \questioE A circle that closely fits points on a local section of a curve can be drawn for any continuous curve. The
            radius of curvature of the curve is defined as the radius of the approximating circle, which changes as
            we move around the curve.
            \begin{displaymath}
              \textrm{radius of curvature} = \frac{\left[1 +  \left(\od{y}{x}\right)^2 \right]^{\frac{3}{2}}}{\abs{\od[2]{y}{x}}}
            \end{displaymath}
            Find the radius of curvature of the curve $ y = e^{-x} \sin x $ at the point $ (0, 0) $.
  \questioM Show that $ y = xe^{-x} $ satisfies the differential equation $ xy' = (1-x)y $.
  \questioM If $ y = \ln \frac{1 + \sqrt{\sin x}}{1 - \sqrt{\sin x}} $, find $ y'' $.
  \questioM Find the equation of the tangent line to the graph of $ y = \ln \cos \frac{x - 1}{x} $ at
            the point $(1, 0)$.
  \questioM Show that $ y = (1 + x + \ln x)^{-1} $ satisfies the differential equation $ xy' = y(y \ln x - 1) $.
  \questioE Find the angle at which $ y = x^2 \ln [(x - 2)^2] $ cuts the $ x$-axis at the point $ (0,0) $.
  \questioM When $ x = 0 $, is the curve $ y = (x + 20)^2 (2x^2 - 3)^6 - \ln \sin (x - \frac{\pi}{2}) $ concave up or concave down?
  \questioM If $ y = \frac{e^x}{\sin x} $, show that $ \od{y}{x} = y(1 - \cot x) $.
\end{questions}
\end{document}
