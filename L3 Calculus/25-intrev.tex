\documentclass{exam}
\usepackage[utf8]{inputenc}
\usepackage{lmodern}
\usepackage{microtype}

% \usepackage[parfill]{parskip}
\usepackage[dvipsnames]{xcolor}
\usepackage{amsmath}
\usepackage{amsfonts}
\usepackage{amsthm}
\usepackage{siunitx}
\DeclareSIUnit\year{yr}
\DeclareSIUnit\foot{ft}
\DeclareSIUnit\litre{\liter}

\usepackage{skull}

\usepackage{pgfplots}
\usepgfplotslibrary{polar}
\pgfplotsset{compat=1.11}
\usepgfplotslibrary{statistics}
\usepackage{graphicx}
\usepackage{sidecap}
\sidecaptionvpos{figure}{c}
\usepackage{float}
\usepackage{gensymb}
\usepackage{tkz-euclide}
\usetkzobj{all}
\usepackage{commath}
\usepackage{hyperref}
\usepackage{enumitem}
\usepackage{wasysym}
\usepackage{multicol}
\usepackage{mathtools}
\usepackage{tcolorbox}
\usepackage{tabularx}
\usepackage[version=4]{mhchem}
\usepackage{changepage}
\usepackage{listings}
\lstset{basicstyle=\ttfamily\linespread{0.8}\small}

\renewcommand*{\thefootnote}{\fnsymbol{footnote}}

\newtheorem*{thm}{Theorem}
\newtheorem*{iden}{Identity}
\newtheorem*{lemma}{Lemma}
\newtheorem{obs}{Observation}
\theoremstyle{definition}
\newtheorem*{defn}{Definition}
\newtheorem*{ex}{Example}
\newtheorem{con}{Construction}
\newtheorem*{alg}{Algorithm}

\newtheoremstyle{break}
  {\topsep}{\topsep}%
  {\itshape}{}%
  {\bfseries}{}%
  {\newline}{}%
\theoremstyle{break}
\newtheorem*{bthm}{Theorem}

% russian integral
\usepackage{scalerel}
\DeclareMathOperator*{\rint}{\scalerel*{\rotatebox{17}{$\!\int\!$}}{\int}}

% \DeclareMathOperator*{\rint}{\int}

\pgfplotsset{vasymptote/.style={
    before end axis/.append code={
        \draw[densely dashed] ({rel axis cs:0,0} -| {axis cs:#1,0})
        -- ({rel axis cs:0,1} -| {axis cs:#1,0});
    }
}}

% \pointsinrightmargin
\boxedpoints
\pointname{}

\newcommand{\questioA}{\question[\texttt{\textbf{\color{Cerulean} A}}]}
\newcommand{\questioM}{\question[\texttt{\textbf{\color{PineGreen} M}}]}
\newcommand{\questioE}{\question[\texttt{\textbf{\color{WildStrawberry} E}}]}
\newcommand{\questioS}{\question[\texttt{\textbf{\color{Goldenrod} S}}]}
\newcommand{\questioO}{\question[\texttt{\textbf{\color{BurntOrange} O}}]}

\newcommand{\parA}{\part[\texttt{\textbf{\color{Cerulean} A}}]}
\newcommand{\parM}{\part[\texttt{\textbf{\color{PineGreen} M}}]}
\newcommand{\parE}{\part[\texttt{\textbf{\color{WildStrawberry} E}}]}
\newcommand{\parS}{\part[\texttt{\textbf{\color{Goldenrod} S}}]}
\newcommand{\parO}{\part[\texttt{\textbf{\color{BurntOrange} O}}]}

\newcommand{\subparA}{\subpart[\texttt{\textbf{\color{Cerulean} A}}]}
\newcommand{\subparM}{\subpart[\texttt{\textbf{\color{PineGreen} M}}]}
\newcommand{\subparE}{\subpart[\texttt{\textbf{\color{WildStrawberry} E}}]}
\newcommand{\subparS}{\subpart[\texttt{\textbf{\color{Goldenrod} S}}]}
\newcommand{\subparO}{\subpart[\texttt{\textbf{\color{BurntOrange} O}}]}

\newcommand{\mainHeader}[2]{\section*{NCEA Level 2 Mathematics\\#1. #2}}
\newcommand{\mainHeaderHw}[2]{\section*{NCEA Level 2 Mathematics (Homework)\\#1. #2}}
\newcommand{\seealso}[1]{\begin{center}\emph{See also #1.}\end{center}}
\newcommand{\drills}[1]{\begin{center}\emph{Drill problems: #1.}\end{center}}
\newcommand{\basedon}[1]{\begin{center}\emph{Notes largely based on #1.}\end{center}}

\begin{document}

\mainHeaderIntg{25}{Integration Revision}
\subsection*{Questions}
\begin{questions}
  \questioA True or false:
    \begin{parts}
      \part $ \rint^b_a \od{y}{x} \dif{x} = y(b) - y(a) $
      \part $ \rint^1_0 f(x) \dif{x} + \rint^1_0 g(x) \dif{x} = \rint^1_0 f(x) + g(x) \dif{x} $
      \part $ \rint^2_0 f(x) \dif{x} + \rint^1_0 f(x) \dif{x} = \rint^2_1 f(x) \dif{x} $
      \part $ \rint \sin(x) \dif{x} \rint \cos(x) \dif{x} = \rint \sin(x) \cos(x) \dif{x} $
      \part $ \rint \frac{1}{u} \dif{u} = \rint \frac{1}{x} \dif{x} $
      \part A definite integral always represents the area under a curve.
      \part If $ u = 2x $ then $ \rint \sqrt{2x} \dif{x} = \rint \sqrt{u} \dif{u} $.
      \part The indefinite integral of $ \ln x $ is just $ \ln x $.
    \end{parts}
  \questioA Compute the following indefinite integrals:
    \begin{parts}
      \part $ \rint \sin x \dif{x} $
      \part $ \rint \sec 3x \tan 3x \dif{x}$
      \part $ \rint \frac{2x^4 - x^2}{x^3} \dif{x} $
      \part $ \rint \sin x \cos x \dif{x} $
      \part $ \rint \sin^2 x \dif{x} $
      \part $ \rint \frac{1}{2u} \dif{u} $
      \part $ \rint \frac{2x + \sec^2 x}{2\sqrt{x^2 + \tan x}} \dif{x} $
      \part $ \rint \frac{t^4 - 1}{t - 1} \dif{t} $
      \part $ \rint \frac{t^{2017} + \sqrt{t^{2017}} + \sqrt[3]{t^{2017}}}{\sqrt[2017]{t^2}} \dif{t} $
      \part $ \rint \sec \theta \dif{\theta} $ (hint: multiply through by $ \frac{\sec \theta + \tan \theta}{\sec \theta + \tan \theta} $)
      \part $ \rint \sec x \tan x + \sec^2 x \tan x \dif{x} $
    \end{parts}
  \questioA Compute the definite integrals:
    \begin{parts}
      \part $ \rint_0^1 x^2 \dif{x} $
      \part $ \rint_0^{\pi/2} \tan(x/2) \dif{x} $
      \part $ \rint^4_0 x^3 + x^2 + \sqrt{x} \dif{x} $
      \part $ \rint_1^e \frac{1}{t} \dif{t} $
    \end{parts}
  \questioM Suppose $ f'(x) + \frac{f(x)}{x} = 0 $, and $ f(1) = 1 $. Find $ f(x) $ explicitly.
  \questioM If $ \rint^B_A f(x) \dif{x} = 3 $ and $ \rint^C_A f(x) \dif{x} = 4 $, find $ \rint^C_B f(x) \dif{x} $.
  \questioM A function $ \phi $, whose graph passes through the origin, is such that the slope of $ \phi $ at any given
            point $ x $ is exactly $ 2\phi(x) $. Find $ \phi $ exactly.
  \questioE Find $ A $ such that $ \rint^A_0 \sin x \dif{x} $ is maximised.
  \questioE Find $ B $ such that $ \rint_B^{B^2} x^2 - 4x - 4 \dif{x} $ is maximised.
  \questioM Find the area of the region bounded by $ y = 1 + x^2 $, $ y = - 1 - x^2 $, $ x = 1 $, and $ x = -1 $.
  \questioM Use integration to find the area enclosed between the curve $ y = e^{2x} - \frac{1}{e^{3x}} $ and the
            lines $ y = 0 $, $ x = 0 $, and $ x = 1.2 $.
  \questioM A function $ f $ is \textit{even} if $ f(-x) = f(x) $ for all $ x $ in its domain.
    \begin{parts}
      \part What geometric property does the graph of an even function have?
      \part Suppose $ f $ is an even function with $ \rint^7_0 f(x) \dif{x} = 20 $. Find:
        \begin{subparts}
          \subpart $ \rint^7_{-7} f(x) \dif{x} $
          \subpart $ \rint^7_0 3f(x) + 2 \dif{x} $
        \end{subparts}
    \end{parts}
  \questioM A function $ y $ is implicitly defined in terms of $ x $ in each case; find $ y(x) $ explicitly.
    \begin{parts}
      \part $ \od{y}{x} = xy^2 $, $ y(1) = 1 $.
      \part $ \od{y}{x} = \frac{\cos x}{3y} $, $ y(\pi/6) = 1 $.
    \end{parts}
  \questioM Use integration to find the area enclosed between $ y = 1-0.2x^4 $ and $ y = 0.4x^4 $.
  \questioE Mr Leibniz has a continer of oil and places it in the garage. Unfortunately, he puts the container on
            top of a sharp nail and it begins to leak. The rate of decrease of the volume of oil in the container is
            given by the differential equation $ \od{V}{t} = -kVt $, where $ V $ is the remaining volume of oil remaining
            after $ t $ hours have passed. The volume of oil in the container when it was placed in the garage
            was \SI{3000}{\milli\litre}; after twenty hours have passed, the volume remaining is \SI{2400}{\milli\litre}.
            How much (if any) oil will remain in the container after 96 hours have passed?
  \questioM An object has acceleration (in one dimension) $ a(t) = 0.2t + 0.3\sqrt{t} $ for $ 0 \leq t \leq 10 $. At $ t = 4 $,
            the velocity of the object is \SI{5}{\metre\per\second}. How far has the object travelled after nine seconds?
  \questioS The formula for integration by parts is $ \rint f(x) g'(x) \dif{x} = f(x) g(x) - \rint f'(x) g(x) \dif{x} $.
    \begin{parts}
      \part Find $ \rint \ln x \dif{x} $ explicitly.
      \part Hence, or otherwise, find $ \rint^2_1 4x \ln x \dif{x} $.
    \end{parts}
  \question
    \begin{parts}
      \part
        \begin{subparts}
          \subparE Show that $ \frac{3}{x + 1} - \frac{4}{x - 2} = \frac{-x - 10}{x^2 - x - 2} $.
          \subparE Hence, or otherwise, compute $ \rint^1_0 \frac{-x - 10}{x^2 - x - 2} \dif{x} $.
        \end{subparts}
      \parS Find $ \beta $ such that $ \rint^{2\beta}_{\beta} \frac{1}{x^2 + 5x + 6} \dif{x} = 1 $.
    \end{parts}
  \questioA Use the trapezium rule to approximate the value of $ \rint^{4}_{1} x^x \dif{x} $. This function cannot
            be integrated in terms of elementary functions.
  \questioM The indefinite integral $ \rint e^{-x^2} \dif{x} $ cannot be integrated in terms of elementary
            functions, but it can be shown that the definite integral $ \rint^{\infty}_{-\infty} e^{-x^2} \dif{x} $
            has the value $ \sqrt{\pi} $. Most of the curve lies within the bounds $ -2 \leq x \leq 2 $. Use Simpson's
            rule to approximate the value of $ \rint^{2}_{-2} e^{-x^2} $, and compare this with the expected value.
  \questioM Consider the function implicitly defined by $ y(x) = x + \rint^{\sqrt{2}}_0 y(x) \dif{x} $. If the graph
            of the function includes the point $ (0,1) $, find $ y(x) $ explicitly.
  \questioE An object is at \SI{4}{\degree} in a refrigerator. It is removed and placed on a shelf with an ambient
            temperature of \SI{20}{\degree}. After two minutes, the object has warmed to \SI{5}{\degree}. How long will
            it take for the object to reach \SI{10}{\degree}? [\textit{Use Newton's law of cooling.}]
  \questioE A property owner assumes that the rate of increase of the value of his property at any time is proportional
            to the value, \$$V$, of the property at that time.
    \begin{parts}
      \part Write the differential equation that expresses this statement.
      \part The property was valued at \$365 000 in May 2012, and at \$382 000 in November 2013. Solve the differential
            equation from (a) to find the price that the owner paid for the property in May 2007 when he purchased the property,
            given his assumption is accurate.
    \end{parts}
  \questioM Find $ x $ such that $ \rint_1^x \frac{\ln u}{u} \dif{u} = 1 $.
  \questioE The formula for surface area of the volume of revolution of $ y = f(x) $ is $ 2\pi \rint^b_a f(x) \sqrt{f'(x) + 1} \dif{x} $.
            Find the area of the surface obtained by rotating about the $ x$-axis the part of the curve $ x = \frac{1}{4}y^2 - \frac{1}{2}\ln y $
            that lies between $ y = 1 $ and $ y = e $.
  \questioE The base of a solid is the region bounded by the parabolae $ y = x^2 $ and $ y = 2 - x^2 $. Find the volume of the solid,
            if cross-sections perpendicular to the $ x$-axis are squares with one side lying along the base.
  \questioS Find the volume of the ellipsoid $ \frac{x^2}{a^2} + \frac{y^2}{b^2} + \frac{z^2}{c^2} = 1 $.
  \questioO Scholarship 2016: Consider $ I_n = \rint^{\pi/2}_0 \frac{\sin 2nx}{\sin x} \dif{x} $, where $ n \geq 0 $. Show
            that $ I_n - I_{n-1} = \frac{2(-1)^{n-1}}{2n - 1} $.
  \questioS Scholarship 2010:
            A flower pattern is constructed by using a sinusoidal function $ r(\theta) $ to define the distance from the origin to the
            curve at a radial angle $ \theta $. The $ x $ and $ y $ coordinates of a point on the curve are given by the following equations,
            where $ 0 \leq b \leq a $ and $ n $ is a positive integer (the number of petals).
            \begin{align*}
              r(\theta) = a + b \sin (n\theta) && x(\theta) = r(\theta) \cos (\theta) && y(\theta) = r(\theta) \sin(\theta)
            \end{align*}
    \begin{parts}
      \part The area inside such a function is given by
            \begin{displaymath}
              A = \frac{1}{2} \rint^{2\pi}_0 (r(\theta))^2 \dif{\theta}
            \end{displaymath}
            when $ r(\theta) \geq 0 $.

            \textbf{Show} that the area of a flower pattern is $ \pi \left( a^2 + \frac{1}{2}b^2 \right) $.
      \part A lemon squeezer with base radius $ a_0 $ and height $ H $ is made to the following specifications.

            At a height $ h $ (where $ 0 \leq h \leq H $) the cross-section is a flower pattern with
            \begin{displaymath}
              a(h) = \frac{H - h}{H} a_0 \text{ and } b(h) = \frac{h}{H} a(h).
            \end{displaymath}

            Use integration with respect to $ h $ to show that the volume of a lemon squeezer is \textbf{exactly $5\%$ greater than} the
            volume of a cone with the same base and height.
    \end{parts}
  \questioO Evaluate the following:
            \begin{displaymath}
              \od[2]{}{x} \rint^x_0 \left( \rint^{\sin t}_0 \sqrt{1 + u^4} \dif{u} \right) \dif{t}
            \end{displaymath}
            [\textit{Hint: no integration is required. Use the FTC and the (differentiation) chain rule.}]
\end{questions}

\end{document}
