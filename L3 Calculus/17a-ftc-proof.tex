\documentclass{amsart}
\usepackage[utf8]{inputenc}

\usepackage[parfill]{parskip}
\usepackage[dvipsnames]{xcolor}
\usepackage{microtype}
\usepackage{siunitx}
\DeclareSIUnit\year{yr}
\usepackage{pgfplots}
\usepackage{graphicx}
\usepackage{sidecap}
\sidecaptionvpos{figure}{c}
\usepackage{float}
\usepackage{gensymb}
\usepackage{tkz-euclide}
\usetkzobj{all}
\usepackage{commath}
\usepackage{hyperref}
\usepackage{enumitem}
\usepackage{wasysym}

\renewcommand*{\thefootnote}{\fnsymbol{footnote}}

\newtheorem*{thm}{Theorem}
\newtheorem*{iden}{Identity}
\newtheorem*{lemma}{Lemma}
\theoremstyle{definition}
\newtheorem*{defn}{Definition}
\newtheorem*{ex}{Example}
\newtheorem*{exercise}{Exercise}

% russian integral
\usepackage{scalerel}
\DeclareMathOperator*{\rint}{\scalerel*{\rotatebox{17}{$\!\int\!$}}{\int}}

% \qformat{Question \thequestion: \thequestiontitle\hfill}

\begin{document}

Here we prove the FTC for `nice' functions. We actually prove the second FTC first as it is easier.

\begin{thm}[Second Fundamental Theorem]
  Suppose that $ f $ is a continuous function on the closed interval $ [a,b] $.\footnote{i.e. $ f $ is continuous at every $ x $ such that $ a \leq x \leq b $.} Then:
  \begin{displaymath}
    \od{}{x} \rint^x_a f(t) \dif{t} = f(x)
  \end{displaymath}
\end{thm}

\begin{proof}
  Let us take the derivative in a straightforward manner.
  \begin{displaymath}
    \od{}{x} \rint^x_a f(t) \dif{t} = \lim_{h \to 0} \frac{\rint^{x + h}_a f(t) \dif{t} - \rint^{x}_a f(t) \dif{t}}{h}
                                    = \lim_{h \to 0} \frac{\rint^{x + h}_x f(t) \dif{t}}{h}.
  \end{displaymath}
  Now, let $ f(M) $ be the maximum value obtained by $ f $ on the closed interval $ [x,x+h] $; let $ f(m) $ be the minimum value. Interpreting the integral as
  an area, we have
  \begin{displaymath}
    hf(m) \leq \rint^{x + h}_x f(t) \dif{t} \leq hf(M) \implies f(m) \leq \frac{1}{h} \rint^{x + h}_x f(t) \dif{t} \leq f(M).
  \end{displaymath}
  Now, as $ h \to 0 $ we must have $ f(m) \to f(x) $ and $ f(M) \to f(x) $ (because as we make the interval smaller, $ m $ and $ M $ move towards $ x $). Hence
  \begin{displaymath}
    f(x) \leq \frac{1}{h} \rint^{x + h}_x f(t) \dif{t} \leq f(x)
  \end{displaymath}
  and so $ \od{}{x} \rint^x_a f(t) \dif{t} = f(x) $.
\end{proof}

\begin{thm}[First Fundamental Theorem]
  Suppose $ f $ is continuous on the closed interval $ [a,b] $, and suppose $ F $ is any antiderivative of $ f $ (so $ F' = f $). Then:
  \begin{displaymath}
    \rint^b_a f(x) \dif{x} = F(b) - F(a) = \eval{F(x)}_a^b
  \end{displaymath}
\end{thm}

\begin{proof}
  Consider $ \od{}{x} \rint^x_a f(t) \dif{t} = f(x) $. In particular, $ \rint^x_a f(t) \dif{t} $ is an antiderivative of $ f $ and we can antidifferentiate both
  sides, obtaining
  \begin{equation}\label{eqn:ftc1}
    \rint^x_a f(t) \dif{t} = F(x) + C \tag{*}
  \end{equation}
  (where $ C $ is some constant). Now substitute $ a $ for $ x $ in (*): we find that $ 0 = \rint^a_a f(t) \dif{t} = F(a) + C $, and in particular $ -C = F(a) $.
  Substituting $ b $ for $ x $ in (*), we find that $ \rint^b_a f(t) \dif{t} = F(b) + C = F(b) - F(a) $; and we are done.
\end{proof}

\begin{exercise}
  Explain why we cannot calculate $ \rint^1_0 1/x^2 \dif{x} $ via the version of the fundamental theorem proved here. Then
  calculate $ \lim_{\alpha \to 0} \rint^1_\alpha 1/x^2 \dif{x} $; why is this allowed?
\end{exercise}

\end{document}
