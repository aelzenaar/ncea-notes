\documentclass{exam}
\usepackage[utf8]{inputenc}
\usepackage{lmodern}
\usepackage{microtype}

% \usepackage[parfill]{parskip}
\usepackage[dvipsnames]{xcolor}
\usepackage{amsmath}
\usepackage{amsfonts}
\usepackage{amsthm}
\usepackage{siunitx}
\DeclareSIUnit\year{yr}
\DeclareSIUnit\foot{ft}
\DeclareSIUnit\litre{\liter}

\usepackage{skull}

\usepackage{pgfplots}
\usepgfplotslibrary{polar}
\pgfplotsset{compat=1.11}
\usepackage{graphicx}
\usepackage{sidecap}
\sidecaptionvpos{figure}{c}
\usepackage{float}
\usepackage{gensymb}
\usepackage{tkz-euclide}
\usetkzobj{all}
\usepackage{commath}
\usepackage{hyperref}
\usepackage{enumitem}
\usepackage{wasysym}
\usepackage{multicol}
\usepackage{mathtools}
\usepackage{tcolorbox}
\usepackage{tabularx}
\usepackage[version=4]{mhchem}
\usepackage{changepage}
\usepackage{listings}
\lstset{basicstyle=\ttfamily\linespread{0.8}\small}

\renewcommand*{\thefootnote}{\fnsymbol{footnote}}

\newtheorem*{thm}{Theorem}
\newtheorem*{iden}{Identity}
\newtheorem*{lemma}{Lemma}
\newtheorem{obs}{Observation}
\theoremstyle{definition}
\newtheorem*{defn}{Definition}
\newtheorem*{ex}{Example}
\newtheorem{con}{Construction}
\newtheorem*{alg}{Algorithm}

\newtheoremstyle{break}
  {\topsep}{\topsep}%
  {\itshape}{}%
  {\bfseries}{}%
  {\newline}{}%
\theoremstyle{break}
\newtheorem*{bthm}{Theorem}

% russian integral
\usepackage{scalerel}
\DeclareMathOperator*{\rint}{\scalerel*{\rotatebox{17}{$\!\int\!$}}{\int}}

% \DeclareMathOperator*{\rint}{\int}

\pgfplotsset{vasymptote/.style={
    before end axis/.append code={
        \draw[densely dashed] ({rel axis cs:0,0} -| {axis cs:#1,0})
        -- ({rel axis cs:0,1} -| {axis cs:#1,0});
    }
}}

% \pointsinrightmargin
\boxedpoints
\pointname{}

\newcommand{\questioA}{\question[\texttt{\textbf{\color{Cerulean} A}}]}
\newcommand{\questioM}{\question[\texttt{\textbf{\color{PineGreen} M}}]}
\newcommand{\questioE}{\question[\texttt{\textbf{\color{WildStrawberry} E}}]}
\newcommand{\questioS}{\question[\texttt{\textbf{\color{Goldenrod} S}}]}
\newcommand{\questioO}{\question[\texttt{\textbf{\color{BurntOrange} O}}]}

\newcommand{\parA}{\part[\texttt{\textbf{\color{Cerulean} A}}]}
\newcommand{\parM}{\part[\texttt{\textbf{\color{PineGreen} M}}]}
\newcommand{\parE}{\part[\texttt{\textbf{\color{WildStrawberry} E}}]}
\newcommand{\parS}{\part[\texttt{\textbf{\color{Goldenrod} S}}]}
\newcommand{\parO}{\part[\texttt{\textbf{\color{BurntOrange} O}}]}

\newcommand{\subparA}{\subpart[\texttt{\textbf{\color{Cerulean} A}}]}
\newcommand{\subparM}{\subpart[\texttt{\textbf{\color{PineGreen} M}}]}
\newcommand{\subparE}{\subpart[\texttt{\textbf{\color{WildStrawberry} E}}]}
\newcommand{\subparS}{\subpart[\texttt{\textbf{\color{Goldenrod} S}}]}
\newcommand{\subparO}{\subpart[\texttt{\textbf{\color{BurntOrange} O}}]}

\newcommand{\mainHeader}[2]{\section*{NCEA Level 2 Mathematics\\#1. #2}}
\newcommand{\mainHeaderHw}[2]{\section*{NCEA Level 2 Mathematics (Homework)\\#1. #2}}

\begin{document}

\mainHeaderIntg{15}{Approximating Areas}
We now move from finding rates of change of curves to finding areas under curves. Just as
for differentiation, we begin by finding finite approximations to area.

\subsection*{Approximating area using rectangles}
\begin{center}
  \includegraphics[width=0.4\linewidth]{approx-rectangles}
\end{center}
We start with the simplest and most naive approximation: using a bunch of rectangles under the curve. Suppose
we wish to find the area under the curve $ y = f(x) $ between the two lines $ x = a $ and $ x = b $ by splitting
it up into $ n $ rectangles of width $ \Delta x $. If we pick a value $ x_i^\ast $ inside each rectangle, as shown
in the diagram, the approximate area is
\begin{displaymath}
  f(x_1^\ast) \Delta x + \cdots + f(x_n^\ast) \Delta x = \sum_{i = 1}^n f(x_i^\ast) \Delta x.
\end{displaymath}

Usually, we pick each $ x_i^\ast $ to be either the left-hand or right-hand edge of each rectangle. Note that
we also have $ \Delta x = \frac{b - a}{n} $ and so our left-hand approximation becomes
\begin{displaymath}
  L_n = \frac{b - a}{n} \left[ f(x_0) + \cdots + f(x_{n - 1}) \right]
\end{displaymath}
and the right-hand approximation is
\begin{displaymath}
  R_n = \frac{b - a}{n} \left[ f(x_1) + \cdots + f(x_{n}) \right].
\end{displaymath}

\subsection*{Approximating area using trapezoids}
\begin{center}
  \includegraphics[width=0.4\linewidth]{approx-trapezoids}
\end{center}
A more useful approximation is found when we inscribe trapezoids into the curve. This can be done by taking
the average of the left-hand and right-hand rectangle approximations (recall that the area of a trapezoid is
the average of the area of two rectangles). When we do this, we obtain
\begin{displaymath}
  T_n = \frac{b - a}{2n} \left( f(x_0) + f(x_n) + 2\left[f(x_2) + \cdots + f(x_{n - 1})\right] \right).
\end{displaymath}

\subsection*{Approximating area using parabolae}
\begin{center}
  \includegraphics[width=0.4\linewidth]{approx-parabolae}
\end{center}
An even better approximation (for smooth functions) is formed when we use parabolae to estimate the area. The
resulting formula is called \textit{Simpson's rule}. Note that $ n $ must be even to use Simpson's rule.
\begin{displaymath}
  S_n = \frac{b - a}{3n} \left( f(x_0) + f(x_n) + 4\left[f(x_1) + f(x_3) + \cdots + f(x_{n - 1})\right] + 2\left[ f(x_2) + \cdots + f(x_{n - 2}) \right] \right).
\end{displaymath}

\subsection*{The Definite Integral}
It is obvious that all three methods of approximating area above will approach the `real' area of the curve as $ n \to \infty $. We
call the true area under a curve $ y = f(x) $ from $ x = a $ to $ x = b $ the \textit{definite integral} of the curve from $ a $ to $ b $,
and we notate it as
\begin{displaymath}
  \lim_{n \to \infty} R_n = \lim_{n \to \infty} L_n = \lim_{n \to \infty} T_n = \lim_{n \to \infty} S_n = \rint^b_a f(x) \dif{x}.
\end{displaymath}

The large $ \rint $ is an elongated `S', which stands for \textbf{sum} --- we are summing all of the little areas from $ a $ to $ b $.
Note the similarity to the notation above: we replace $ \sum $ with $ \rint $, $ x_i^\ast $ with $ x $, and $ \Delta x $ with $ \dif{x} $.

The $ \dif{x} $ is \textbf{not a number}; it is merely a piece of notation which tells us which variable we are taking area with respect to.
For example, $ \rint^b_a x^2 + y^3 \dif{x} \neq \rint^b_a x^2 + y^3 \dif{y} $. At this stage in time, you can think of the notation as a
pair of fancy brackets: the $ \rint^{z_1}_{z_0} $ corresponds to $ \big( $, and the $ \dif{z} $ corresponds to $ \big) $. It makes no
sense to have one without the other (or, for that matter, to have $ \rint $ without the bounds).

\clearpage
\subsection*{Questions}
\begin{questions}
  \questioA Estimate the area under the graph of $ f(x) = \cos x $ from $ x = 0 $ to $ x = \frac{\pi}{2} $ using four approximating
            rectangles and right endpoints. Sketch the graph and the rectangles. Is your estimate an overestimate or an underestimate?
  \questioA Estimate the area under the graph of $ f(x) = \sqrt x $ from $ x = 0 $ to $ x = 4 $ using four approximating
            rectangles and left endpoints. Sketch the graph and the rectangles. Is your estimate an overestimate or an underestimate?
  \questioA Using the trapezoidal rule with $ n = 4 $, estimate the area under the graph of $ f(x) = 1 + x^2 $ from $ x = -1 $ to $ x = 2 $.
            Repeat with $ n = 8 $. Compare the two results.
  \questioA Use Simpson's rule with $ n = 10 $ to estimate the area under the graph of $ y = e^{x^2} $ from $ x = 0 $ to $ x = 1 $.
  \questioA Use both the trapezoidal rule and Simpson's rule (both with $ n = 10 $) to compute the area under the graph of $ y = \sqrt{z} e^{-z} $
            from $ z = 0 $ to $ z = 1 $.
  \questioA Using four rectangles, find the approximate area under the the graph of $ f(x) $ from $ x = 1 $ to $ x = 5 $ given the following table.
            \begin{center}
              \begin{tabular}{|c|c||c|c|}\hline
                $ x $ & $ f(x) $ & $ x $ & $ f(x) $\\\hline
                1.0 & 2.4 & 3.5 & 4.0\\
                1.5 & 2.9 & 4.0 & 4.1\\
                2.0 & 3.3 & 4.5 & 3.9\\
                2.5 & 3.6 & 5.0 & 3.5\\
                3.0 & 3.8 &&\\\hline
              \end{tabular}
            \end{center}
  \questioM Find the area of a circle of radius 4 using Simpson's rule with $ n = 4 $. What is the percentage error of the estimate?
  \questioA Approximate the area under $ y = f(x) $ from $ x = 0 $ to $ x = 6 $ using Simpson's rule.
            \begin{center}
              \includegraphics[width=0.35\linewidth]{aa1}
            \end{center}
  \clearpage
  \questioA Approximate the area under $ y = f(x) $ from $ x = 0 $ to $ x = 10 $ using the trapezoidal rule.
            \begin{center}
              \includegraphics[width=0.35\linewidth]{aa2}
            \end{center}
  \questioM Approximate the shaded area using the trapezoidal rule.
            \begin{center}
              \includegraphics[width=0.35\linewidth]{aa3}
            \end{center}
  \questioM Show that $ T_n = \frac{1}{2} (L_n + R_n) $.
  \questioA (Lifted straight from a Level 2 worksheet.)
    \begin{parts}
      \part Draw the line $ y = 2t + 1 $ and use geometry to find the area under
            this line, above the $ t$-axis, and between the vertical lines $ t = 1 $
            and $ t = 3 $ (i.e. find $ \rint_1^3 2t + 1 \dif{t} $ using geometry).
      \part If $ x > 1 $, let $ A(x) $ be the area of the region that lies under the line $ y = 2t + 1 $
            between $ t = 1 $ and $ t = x $. Sketch this region and use geometry to find an expression
            for $ A(x) $ (i.e. find $ A(x) = \rint_1^x 2t + 1 \dif{t} $ using geometry).
      \part Find $ A'(x) $. What do you notice?
    \end{parts}
\end{questions}

\end{document}
