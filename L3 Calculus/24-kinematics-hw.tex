\documentclass{exam}
\usepackage[utf8]{inputenc}
\usepackage{lmodern}
\usepackage{microtype}

% \usepackage[parfill]{parskip}
\usepackage[dvipsnames]{xcolor}
\usepackage{amsmath}
\usepackage{amsfonts}
\usepackage{amsthm}
\usepackage{siunitx}
\DeclareSIUnit\year{yr}
\DeclareSIUnit\foot{ft}
\DeclareSIUnit\litre{\liter}

\usepackage{skull}

\usepackage{pgfplots}
\usepgfplotslibrary{polar}
\pgfplotsset{compat=1.11}
\usepackage{graphicx}
\usepackage{sidecap}
\sidecaptionvpos{figure}{c}
\usepackage{float}
\usepackage{gensymb}
\usepackage{tkz-euclide}
\usetkzobj{all}
\usepackage{commath}
\usepackage{hyperref}
\usepackage{enumitem}
\usepackage{wasysym}
\usepackage{multicol}
\usepackage{mathtools}
\usepackage{tcolorbox}
\usepackage{tabularx}
\usepackage[version=4]{mhchem}
\usepackage{changepage}
\usepackage{listings}
\lstset{basicstyle=\ttfamily\linespread{0.8}\small}

\renewcommand*{\thefootnote}{\fnsymbol{footnote}}

\newtheorem*{thm}{Theorem}
\newtheorem*{iden}{Identity}
\newtheorem*{lemma}{Lemma}
\newtheorem{obs}{Observation}
\theoremstyle{definition}
\newtheorem*{defn}{Definition}
\newtheorem*{ex}{Example}
\newtheorem{con}{Construction}
\newtheorem*{alg}{Algorithm}

\newtheoremstyle{break}
  {\topsep}{\topsep}%
  {\itshape}{}%
  {\bfseries}{}%
  {\newline}{}%
\theoremstyle{break}
\newtheorem*{bthm}{Theorem}

% russian integral
\usepackage{scalerel}
\DeclareMathOperator*{\rint}{\scalerel*{\rotatebox{17}{$\!\int\!$}}{\int}}

% \DeclareMathOperator*{\rint}{\int}

\pgfplotsset{vasymptote/.style={
    before end axis/.append code={
        \draw[densely dashed] ({rel axis cs:0,0} -| {axis cs:#1,0})
        -- ({rel axis cs:0,1} -| {axis cs:#1,0});
    }
}}

% \pointsinrightmargin
\boxedpoints
\pointname{}

\newcommand{\questioA}{\question[\texttt{\textbf{\color{Cerulean} A}}]}
\newcommand{\questioM}{\question[\texttt{\textbf{\color{PineGreen} M}}]}
\newcommand{\questioE}{\question[\texttt{\textbf{\color{WildStrawberry} E}}]}
\newcommand{\questioS}{\question[\texttt{\textbf{\color{Goldenrod} S}}]}
\newcommand{\questioO}{\question[\texttt{\textbf{\color{BurntOrange} O}}]}

\newcommand{\parA}{\part[\texttt{\textbf{\color{Cerulean} A}}]}
\newcommand{\parM}{\part[\texttt{\textbf{\color{PineGreen} M}}]}
\newcommand{\parE}{\part[\texttt{\textbf{\color{WildStrawberry} E}}]}
\newcommand{\parS}{\part[\texttt{\textbf{\color{Goldenrod} S}}]}
\newcommand{\parO}{\part[\texttt{\textbf{\color{BurntOrange} O}}]}

\newcommand{\subparA}{\subpart[\texttt{\textbf{\color{Cerulean} A}}]}
\newcommand{\subparM}{\subpart[\texttt{\textbf{\color{PineGreen} M}}]}
\newcommand{\subparE}{\subpart[\texttt{\textbf{\color{WildStrawberry} E}}]}
\newcommand{\subparS}{\subpart[\texttt{\textbf{\color{Goldenrod} S}}]}
\newcommand{\subparO}{\subpart[\texttt{\textbf{\color{BurntOrange} O}}]}

\newcommand{\mainHeader}[2]{\section*{NCEA Level 2 Mathematics\\#1. #2}}
\newcommand{\mainHeaderHw}[2]{\section*{NCEA Level 2 Mathematics (Homework)\\#1. #2}}

\begin{document}

\mainHeaderIntgHw{24}{Kinematics}
\subsection*{Reading}
Kinematics may seem like an odd topic to end with, especially as it is more of a physics topic than a mathematics topic. The
reason for its inclusion in these notes is by means of revision from level 2; and the reason it is in level 2, is so that you
can apply calculus to L3 physics. If you are not studying physics, you may wonder why you should bother learning this particular
application of calculus; the answer is that, historically, calculus began as an attempt to formalise mechanics and so a physical
intuition can often be useful when solving problems that are not at first glance physical.

Beyond this, there is only one fundamental concept in this topic that you must remember: the derivative is just a rate of
change. Velocity is rate of change of position, and acceleration is rate of change of velocity. If you slow down faster,
your acceleration is more negative.

In terms of integration, if you have a certain velocity at a given point then that implies that over a given period of
time, you travel a certain distance. It follows that if you add up all these instantaneous velocities, multiplying each
by the infinitesimal time that you are travelling for each one, then you obtain the total distance you travel; that is,
you see that $ x = \rint \od{x}{t} \dif{t} $.

\clearpage
\subsection*{Questions}
All distances are given in \si{\metre}, and all times in \si{\second}, unless otherwise stated.
\begin{questions}
  \question A distress flare is fired vertically into the air from a boat at sea. The height in metres of the flare $ t $ seconds
            after firing is given by
            \begin{displaymath}
              h = 122.5t - 4.9t^2.
            \end{displaymath}
    \begin{parts}
      \part What is the initial velocity of the flare?
      \part At the peak of its flight, what is the vertical velocity of the flare?
      \part What is the maximum height reached by the flare?
    \end{parts}
  \question Part of the course for an ocean swim runs from bouy $ A $ to bouy $ B $. Swimmers must come ashore
            on the at some point $ P $ along a long straight beach on the way. Bouy $ A $ is \SI{800}{\metre} away
            from the beach, and bouy $ B $ is \SI{600}{\metre} away from the beach. What is the least distance that
            a swimmer must swim? (Hint: minimise $ PA + PB $.)
            \begin{center}
              \includegraphics[width=0.5\linewidth]{bouys}
            \end{center}
  \question The following graph shows the acceleration of a rocket from launch until it reaches orbit. Given that the
            initial velocity of the rocket was \SI{0}{\foot\per\second}, find the final velocity of the rocket.
            \begin{center}
              \includegraphics[width=0.7\linewidth]{acceleration-time}
            \end{center}
\end{questions}
\end{document}
