\documentclass{exam}
\usepackage[utf8]{inputenc}
\usepackage{lmodern}
\usepackage{microtype}

% \usepackage[parfill]{parskip}
\usepackage[dvipsnames]{xcolor}
\usepackage{amsmath}
\usepackage{amsfonts}
\usepackage{amsthm}
\usepackage{siunitx}
\DeclareSIUnit\year{yr}
\DeclareSIUnit\foot{ft}
\DeclareSIUnit\litre{\liter}

\usepackage{skull}

\usepackage{pgfplots}
\usepgfplotslibrary{polar}
\pgfplotsset{compat=1.11}
\usepackage{graphicx}
\usepackage{sidecap}
\sidecaptionvpos{figure}{c}
\usepackage{float}
\usepackage{gensymb}
\usepackage{tkz-euclide}
\usetkzobj{all}
\usepackage{commath}
\usepackage{hyperref}
\usepackage{enumitem}
\usepackage{wasysym}
\usepackage{multicol}
\usepackage{mathtools}
\usepackage{tcolorbox}
\usepackage{tabularx}
\usepackage[version=4]{mhchem}
\usepackage{changepage}
\usepackage{listings}
\lstset{basicstyle=\ttfamily\linespread{0.8}\small}

\renewcommand*{\thefootnote}{\fnsymbol{footnote}}

\newtheorem*{thm}{Theorem}
\newtheorem*{iden}{Identity}
\newtheorem*{lemma}{Lemma}
\newtheorem{obs}{Observation}
\theoremstyle{definition}
\newtheorem*{defn}{Definition}
\newtheorem*{ex}{Example}
\newtheorem{con}{Construction}
\newtheorem*{alg}{Algorithm}

\newtheoremstyle{break}
  {\topsep}{\topsep}%
  {\itshape}{}%
  {\bfseries}{}%
  {\newline}{}%
\theoremstyle{break}
\newtheorem*{bthm}{Theorem}

% russian integral
\usepackage{scalerel}
\DeclareMathOperator*{\rint}{\scalerel*{\rotatebox{17}{$\!\int\!$}}{\int}}

% \DeclareMathOperator*{\rint}{\int}

\pgfplotsset{vasymptote/.style={
    before end axis/.append code={
        \draw[densely dashed] ({rel axis cs:0,0} -| {axis cs:#1,0})
        -- ({rel axis cs:0,1} -| {axis cs:#1,0});
    }
}}

% \pointsinrightmargin
\boxedpoints
\pointname{}

\newcommand{\questioA}{\question[\texttt{\textbf{\color{Cerulean} A}}]}
\newcommand{\questioM}{\question[\texttt{\textbf{\color{PineGreen} M}}]}
\newcommand{\questioE}{\question[\texttt{\textbf{\color{WildStrawberry} E}}]}
\newcommand{\questioS}{\question[\texttt{\textbf{\color{Goldenrod} S}}]}
\newcommand{\questioO}{\question[\texttt{\textbf{\color{BurntOrange} O}}]}

\newcommand{\parA}{\part[\texttt{\textbf{\color{Cerulean} A}}]}
\newcommand{\parM}{\part[\texttt{\textbf{\color{PineGreen} M}}]}
\newcommand{\parE}{\part[\texttt{\textbf{\color{WildStrawberry} E}}]}
\newcommand{\parS}{\part[\texttt{\textbf{\color{Goldenrod} S}}]}
\newcommand{\parO}{\part[\texttt{\textbf{\color{BurntOrange} O}}]}

\newcommand{\subparA}{\subpart[\texttt{\textbf{\color{Cerulean} A}}]}
\newcommand{\subparM}{\subpart[\texttt{\textbf{\color{PineGreen} M}}]}
\newcommand{\subparE}{\subpart[\texttt{\textbf{\color{WildStrawberry} E}}]}
\newcommand{\subparS}{\subpart[\texttt{\textbf{\color{Goldenrod} S}}]}
\newcommand{\subparO}{\subpart[\texttt{\textbf{\color{BurntOrange} O}}]}

\newcommand{\mainHeader}[2]{\section*{NCEA Level 2 Mathematics\\#1. #2}}
\newcommand{\mainHeaderHw}[2]{\section*{NCEA Level 2 Mathematics (Homework)\\#1. #2}}

\begin{document}

\mainHeaderDiff{12}{Sequences and Series}
\subsection*{Sequences}
So far, we have looked at functions that are continuous. This week, we take a short break from this and look at sequences and series. A sequence
is, intuitively, a list of numbers written in a definite order:
\begin{displaymath}
  (a_n) = (a_1, a_2, \dots, a_i, \dots).
\end{displaymath}
Note that we index sequences starting at 1, unless stated otherwise. Theoretical computer scientists will often start at 0 instead.

Some examples of sequences include
\begin{gather}
  \left(n\right) = (1,2,3,\dots) \text{}\\
  \left(\frac{n-1}{n}\right) = \left(0,\frac{1}{2},\frac{2}{3},\frac{3}{4},\dots\right)\\
  \left(\cos \frac{n\pi}{6} \right) = \left(1, \frac{\sqrt{3}}{2}, \frac{1}{2}, 0, \dots\right)\\
  \left(1 - 0.2^n \right) = \left(0.8, 0.96,0.992,0.9984,\dots \right)\\
  \left(\frac{n^5}{n!}\right) = \left(0, 1, 16,40.5,42.6,26.04,10.8,\dots\right)\\
  \left( a_n \right) = \left(1,0,1,0,0,1,0,0,0,1,\dots\right)
\end{gather}
Clearly some of these sequences grow forever, while others settle down and become closer and closer to a particular value.
\begin{defn}
  A sequence $ (a_n) $ has a limit $ L $, written $ \lim_{n\to\infty} a_n = L $, if we can make the terms $ a_i $ stay as close as we like
  to $ L $ by taking $ i $ sufficiently large. If $ (a_n) $ has a limit, then it is called \textbf{convergent}; otherwise, it is
  called \textbf{divergent}.
\end{defn}

Examples (2) and (4) above are clearly convergent (they both converge to 1); example (5) is convergent to 0, even though it is initially
increasing (because factorials grow faster than powers eventually). The others are all divergent:
\begin{enumerate}
  \item[(1)] is clearly divergent because as $ n \to \infty $, $ a_n \to \infty $.
  \item[(3)] is divergent because there is no number $ L $ such that $ a_i $ stays close to $ L $ as we increase $ i $ (it always jumps around).
  \item[(6)] is divergent for the same reason as (3).
\end{enumerate}

Some sequences can be written recursively.
\begin{gather*}
  a_1 = 1, a_{n + 1} = 5a_n - 3 \quad (1, 2, 7, \dots)\\
  a_1 = 2, a_{n + 1} = \frac{a_n}{1 + a_n} \quad \left(2, \frac{2}{3}, \frac{2}{5}, \frac{2}{7}, \dots\right)
\end{gather*}

\begin{ex}
  The \textbf{Fibonacci} sequence is defined by
  \begin{displaymath}
    (F_i) = \begin{cases}
      1 & \text{if $ i = 1 $ or $ i = 2 $,}\\
      F_{i - 1} + F_{i - 2} & \text{otherwise.}
    \end{cases}
  \end{displaymath}
  The first few elements of the Fibonacci sequence are 1, 1, 2, 3, 5, 8, 13, 21.

  There is actually a closed-form formula for the Fibonacci sequence:
  \begin{displaymath}
    F_i = \frac{\left(1 + \sqrt{5}\right)^n - \left(1 - \sqrt{5}\right)^n}{2^n \sqrt{5}}.
  \end{displaymath}
  More information about this sequence is given in the reading for the homework.
\end{ex}

\subsection*{Series}
From sequences, we move on to infinite sums (also known as series). Given some sequence $ (a_n) $, we have a series
\begin{displaymath}
  \sum_{n = 1}^\infty a_n = a_1 + a_2 + a_3 + \cdots.
\end{displaymath}

Often it does not make sense to talk about the sum of infinitely many terms; obviously $ 1 + 2 + 3 + \cdots $ will increase
forever. On the other hand, if we add the first $ n $ terms of
\begin{displaymath}
  \left(\frac{1}{2}, \frac{1}{4}, \dots, \frac{1}{2^n}, \dots\right)
\end{displaymath}
we obtain $ 1 - 1/2^n $, which converges to 1.

This suggests that we can consider the partial sums of a sequence $ (a_n) $
\begin{align*}
  s_1 &= a_1\\
  s_2 &= a_1 + a_2\\
  s_3 &= a_1 + a_2 + a_3\\
      &\vdots\\
  s_n &= a_1 + \cdots + a_n = \sum_{i = 1}^n a_n
\end{align*}
and then say that the sum $ \sum a_n $ exists if the sequence $ (s_n) $ converges.

An important class of series is the geometric series
\begin{displaymath}
  \sum_{n = 1}^\infty ar^{n - 1} = a + ar + ar^2 + \cdots
\end{displaymath}
where $ a \neq 0 $ and $ r $ are constants. Clearly if $ \abs{r} \geq 1 $ then the sum becomes $ a + a + \cdots $ which converges;
if $ \abs{r} < 1 $, then we have
\begin{align*}
  s_n  &= a + ar + ar^2 + \cdots + ar^{n - 1}\\
  rs_n &= \phantom{a + {}} ar + ar^2 + ar^3 + \cdots + ar^n\\
  s_n - rs_n &= a - ar^n\\
  s_n = \frac{a - ar^n}{1 - r}.
\end{align*}
which converges to $ a/(1-r) $ as $ n \to \infty $; so the sum exists.

\subsection*{Questions}
\begin{questions}
  \questioM Which of the following sequences are convergent (write the limit), and which are divergent (give a reason)?
    \begin{parts}
      \part $ ((-1)^n) $
      \part $ (e^{1/n}) $
      \part $ (n^2 e^{-n}) $
      \part $ (3^n/(1+2^n)) $
      \part $ a_1 = 6 $, $ a_{n + 1} = a_n/n $
      \part $ (\cos 2\pi/n) $
      \part $ (\sin n!/n^n) $
      \part $ (\sin n^n/n!) $
    \end{parts}
  \questioE Find the limit of the sequence $ (\sqrt{2}, \sqrt{2\sqrt{2}}, \sqrt{2\sqrt{2\sqrt{2}}}, \dots) $.
  \questioA Find the sum of the geometric series $ 1 + 1/2 + 1/4 + \cdots $.
  \questioM Find the sum of the geometric series $ 5 - \frac{10}{3} + \frac{20}{9} - \frac{40}{27} + \cdots $.
  \questioE Show that the series $ \sum_{n = 1}^{\infty} \frac{1}{n(n + 1)} $ is convergent, and find its sum. [Hint: calculate $ \frac{1}{i} - \frac{1}{1 + i} $.]
  \questioE Determine whether $ \sum_{n =1}^\infty \ln \frac{n}{n + 1} $ is convergent or divergent. If it is convergent, find its sum.
  \questioE Consider the harmonic series
            \begin{displaymath}
              \sum_{n = 1}^\infty \frac{1}{n} = 1 + \frac{1}{2} + \frac{1}{3} + \cdots.
            \end{displaymath}
    \begin{parts}
      \part Let $ s_n $ be the sequence of partial sums. Show that $ s_4 = 1 + \frac{1}{2} + \frac{1}{3} + \frac{1}{4} > 1 + \frac{2}{2} $ [Hint: $1/3 > 1/4$.]
      \part Show that $ s_8 > 1 + \frac{3}{2} $.
      \part Prove that $ s_{2^n} \geq 1 + \frac{n}{2} $ in general. Conclude that the harmonic series diverges.
    \end{parts}
  \questioE Find $ c $ if $ \sum_{n = 1}^\infty (1 + c)^{-n} = 2 $.
  \questioS A power series is a series of the form
            \begin{displaymath}
              \sum_{n = 0}^{\infty} c_n x^n = c_0 + c_1 x + c_2 x^2 + \cdots.
            \end{displaymath}
            You can think of a power series as a polynomial with infinitely many terms. It can be shown that the normal sum rule for differentiation holds,
            so that $ \od{}{x} \sum_{n = 0}^{\infty} c_n x^n = \sum_{n = 0}^{\infty} \od{}{x} c_n x^n $.
    \begin{parts}
      \item Find the derivative of the Bessel function, $ J_0(x) = \sum^{\infty}_{n = 0} \dfrac{(-1)^n x^{2n}}{2^{2n} (n!)^2} $.
      \item If $ \abs{x} < 1 $, show that $ \sum_{n = 0}^\infty x^n = \frac{1}{1 - x} $.
      \item Find a power series representation for $ f(x) = 1/(1-x)^2 $. [Hint: differentiate the equation in (b).]
      \item Find a power series representation for $ g(x) = \ln(1 + x) $. [Hint: differentiate $ g $, use (c), and antidifferentiate.]
      \item When is the power series representation of $ g $ valid?
    \end{parts}
  \questioS The Cantor set is constructed as follows. Start with the closed interval $ [0,1] $ and remove the open interval $ (\frac{1}{3}, \frac{2}{3}) $. This leaves
            the two closed intervals $ [0, \frac{1}{3}] $ and $ [\frac{2}{3},1] $; remove the open middle third of each of these. Four intervals remain; remove the
            open middle third of each. Continue this process indefinitely, at each step removing the open middle third of each interval remaining from the previous
            step. The Cantor set is the set of numbers remaining from the interval $ [0,1] $ after this process has been carried out.
            \begin{center}
              \includegraphics[width=0.8\textwidth]{cantor}
            \end{center}
    \begin{parts}
      \part Show that the total length of all the removed intervals is 1.
      \part Despite this, there are still infinitely many numbers remaining. Exhibit some members of the Cantor set.
      \part The Sierpinski carpet is a two-dimensional version of the Cantor set; remove the centre one-ninth of a square of side 1, then the centres
            of the eight remaining squares, and so on. Show that the sum of the areas of the removed squares approaches 1. Hence the Sierpinski carpet
            has area zero.
            \begin{center}
              \includegraphics[width=0.4\textwidth]{sierpinski}
            \end{center}
    \end{parts}
\end{questions}
\end{document}
