\documentclass{exam}
\usepackage[utf8]{inputenc}
\usepackage{lmodern}
\usepackage{microtype}

% \usepackage[parfill]{parskip}
\usepackage[dvipsnames]{xcolor}
\usepackage{amsmath}
\usepackage{amsfonts}
\usepackage{amsthm}
\usepackage{siunitx}
\DeclareSIUnit\year{yr}
\DeclareSIUnit\foot{ft}
\DeclareSIUnit\litre{\liter}

\usepackage{skull}

\usepackage{pgfplots}
\usepgfplotslibrary{polar}
\pgfplotsset{compat=1.11}
\usepackage{graphicx}
\usepackage{sidecap}
\sidecaptionvpos{figure}{c}
\usepackage{float}
\usepackage{gensymb}
\usepackage{tkz-euclide}
\usetkzobj{all}
\usepackage{commath}
\usepackage{hyperref}
\usepackage{enumitem}
\usepackage{wasysym}
\usepackage{multicol}
\usepackage{mathtools}
\usepackage{tcolorbox}
\usepackage{tabularx}
\usepackage[version=4]{mhchem}
\usepackage{changepage}
\usepackage{listings}
\lstset{basicstyle=\ttfamily\linespread{0.8}\small}

\renewcommand*{\thefootnote}{\fnsymbol{footnote}}

\newtheorem*{thm}{Theorem}
\newtheorem*{iden}{Identity}
\newtheorem*{lemma}{Lemma}
\newtheorem{obs}{Observation}
\theoremstyle{definition}
\newtheorem*{defn}{Definition}
\newtheorem*{ex}{Example}
\newtheorem{con}{Construction}
\newtheorem*{alg}{Algorithm}

\newtheoremstyle{break}
  {\topsep}{\topsep}%
  {\itshape}{}%
  {\bfseries}{}%
  {\newline}{}%
\theoremstyle{break}
\newtheorem*{bthm}{Theorem}

% russian integral
\usepackage{scalerel}
\DeclareMathOperator*{\rint}{\scalerel*{\rotatebox{17}{$\!\int\!$}}{\int}}

% \DeclareMathOperator*{\rint}{\int}

\pgfplotsset{vasymptote/.style={
    before end axis/.append code={
        \draw[densely dashed] ({rel axis cs:0,0} -| {axis cs:#1,0})
        -- ({rel axis cs:0,1} -| {axis cs:#1,0});
    }
}}

% \pointsinrightmargin
\boxedpoints
\pointname{}

\newcommand{\questioA}{\question[\texttt{\textbf{\color{Cerulean} A}}]}
\newcommand{\questioM}{\question[\texttt{\textbf{\color{PineGreen} M}}]}
\newcommand{\questioE}{\question[\texttt{\textbf{\color{WildStrawberry} E}}]}
\newcommand{\questioS}{\question[\texttt{\textbf{\color{Goldenrod} S}}]}
\newcommand{\questioO}{\question[\texttt{\textbf{\color{BurntOrange} O}}]}

\newcommand{\parA}{\part[\texttt{\textbf{\color{Cerulean} A}}]}
\newcommand{\parM}{\part[\texttt{\textbf{\color{PineGreen} M}}]}
\newcommand{\parE}{\part[\texttt{\textbf{\color{WildStrawberry} E}}]}
\newcommand{\parS}{\part[\texttt{\textbf{\color{Goldenrod} S}}]}
\newcommand{\parO}{\part[\texttt{\textbf{\color{BurntOrange} O}}]}

\newcommand{\subparA}{\subpart[\texttt{\textbf{\color{Cerulean} A}}]}
\newcommand{\subparM}{\subpart[\texttt{\textbf{\color{PineGreen} M}}]}
\newcommand{\subparE}{\subpart[\texttt{\textbf{\color{WildStrawberry} E}}]}
\newcommand{\subparS}{\subpart[\texttt{\textbf{\color{Goldenrod} S}}]}
\newcommand{\subparO}{\subpart[\texttt{\textbf{\color{BurntOrange} O}}]}

\newcommand{\mainHeader}[2]{\section*{NCEA Level 2 Mathematics\\#1. #2}}
\newcommand{\mainHeaderHw}[2]{\section*{NCEA Level 2 Mathematics (Homework)\\#1. #2}}

\begin{document}

\mainHeaderDiffHw{3}{Derivatives of Common Functions}
\subsection*{Reading}

Mathematics is fantastic. It is a subject where we do not have to take anyone’s word or
opinion. The truth is not determined by a higher authority who says ‘because I say so’,
or because they saw it in a dream, the pixies at the bottom of their garden told them, or
it came from some ancient mystical tradition. The truth is determined and justified with a
mathematical proof.

A proof is an explanation of why a statement is true. More properly it is a convincing
explanation of why the statement is true. By convincing I mean that it is convincing to
a mathematician. (What that means is an important philosophical point which I am not
going to get into; my interest is more in practical matters.)

Statements are usually proved by starting with some obvious statements, and proceeding
by using small logical steps and applying definitions, axioms and previously established
statements until the required statement results.

The mathematician’s concept of proof is different to everyday usage. In everyday usage
or in court for instance, proof is evidence that something is likely to be true. Mathematicians
require more than this. We like to be 100\% confident that a statement has been proved.
We do not like to be ‘almost certain’.

Having said that, how confident can we be that a theorem has been proved? Millions
have seen a proof of Pythagoras’ Theorem; we can be certain it is true. Proofs of newer
results, however, may contain mistakes. I know from my own experience that some proofs
given in books and research journals are in fact wrong.

\begin{flushright}
  From \textit{How to Think Like a Mathematician}, by Kevin Houston.
\end{flushright}

\begin{thm}
  Suppose $ f $ and $ g $ are functions which are differentiable at some point $ x $, and
  suppose that $ \lambda $ is a real constant. Then:
  \begin{enumerate}
    \item $ (\lambda)'(x) = 0 $,
    \item $ (f + g)'(x) = f'(x) + g'(x) $, and
    \item $ (\lambda f)'(x) = \lambda f'(x) $.
  \end{enumerate}
\end{thm}

\begin{proof}
  We prove these using the properties of the limits.
  \begin{align*}
      (\lambda)'(x) &= \lim_{h \to 0} \frac{\lambda - \lambda}{h} = 0.\\
      (f + g)'(x) &= \lim_{h \to 0} \frac{(f + g)(x + h) - (f + g)(x)}{h}\\
                  &= \lim_{h \to 0} \frac{f(x + h) - f(x) + g(x + h) - g(x)}{h}\\
                  &= \lim_{h \to 0} \frac{f(x + h) - f(x)}{h} + \lim_{h \to 0} \frac{g(x + h) - g(x)}{h}\\
                  &= f'(x) + g'(x).\\
      (\lambda f)'(x) &= \lim_{h \to 0} \frac{(\lambda f)(x + h) - (\lambda f)(x)}{h}\\
                      &= \lim_{h \to 0} \frac{\lambda (f(x + h) - f(x))}{h}\\
                      &= \lambda \lim_{h \to 0} \frac{f(x + h) - f(x)}{h}\\
                      &= \lambda f'(x).\qedhere
  \end{align*}
\end{proof}

\subsection*{Questions}
\begin{questions}
  \question Differentiate with respect to $ x $:
    \begin{parts}
      \part $ x^2 + \ln x $
      \part $ tx^t $
      \part $ \sin x - \cos x $
      \part $ \sqrt[5]{x^4} $
    \end{parts}
  \question Explain why you cannot use the power rule to find the derivative of $ x^x $.
  \question Find the $ n$th derivative of $ \frac{1}{x^n} $.
  \question (More difficult!) Suppose a population grows exponentially with time, such that after $ t $ years the population $ P = P_0 + 10^t $.
    \begin{parts}
      \part Given that the derivative of $ e^{cx} $ is $ ce^{cx} $ when $ c $ is constant,
            find the rate of change of the population at $ t = 100 $.
      \part Explain why this population model is unrealistic.
    \end{parts}
\end{questions}
\end{document}
