\documentclass{exam}
\usepackage[utf8]{inputenc}

\usepackage[dvipsnames]{xcolor}
\usepackage{microtype}
\usepackage{siunitx}
\DeclareSIUnit\year{yr}
\usepackage{pgfplots}
\usepackage{graphicx}
\usepackage{float}

\renewcommand*{\thefootnote}{\fnsymbol{footnote}}


\begin{document}

\section*{NCEA Level 1 Science (Genetics \#1)}

This worksheet is on DNA and genetic information.

\subsection*{Questions}
\begin{questions}
  \question What is DNA, where is it found, and why is it important?
  \question If one `leg' of a portion of DNA has base sequence AGTCGCCAATGGGCTGAGTACGC, what is the base sequence on the other leg?
  \question Consider the DNA of a chimpanzee versus the DNA of a human. How will the DNA be different? How will it be the same?
  \question What is the difference between an \textit{allele} and a \textit{gene}?
  \question Define \textit{mutation} and \textit{carcinogen}. Explain why mutations are generally harmful.
  \question Compare and contrast sexual and asexual reproduction of cells. Which allows for genetic variation to occur?
  \question Draw a Punnett square to calculate the chance of formation of a male human zygote vs a female human zygote.
  \question A man and a woman have six children, all girls. Calculate the chance that, if they have a seventh child, it is a boy.
  \question {[NZQA 2017]} Some leopards or jaguars have a mutation causing them to have a black coat. These are known as ``black panthers''.
    \begin{parts}
      \part How can this mutation cause the coat colour to be different? In your answer you should use the terms DNA, gene, allele,
            phenotype, and mutation to explain how this colour change occurs.
      \part Leopards in the wild commonly have scars, especially around their faces. Explain why the leopard cubs can be born with black coats 
            but not with scars. 
    \end{parts} 
\end{questions}

\subsection*{Homework}
\begin{questions}
  \question Research the Human Genome Project. What did they do? Why was what they did important?
  \question Explain why sexual reproduction is more advantageous for a species in the long term than asexual reproduction.
\end{questions}

\end{document}
