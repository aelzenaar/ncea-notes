\documentclass{exam}
\usepackage[utf8]{inputenc}
\usepackage{lmodern}
\usepackage{microtype}

% \usepackage[parfill]{parskip}
\usepackage[dvipsnames]{xcolor}
\usepackage{amsmath}
\usepackage{amsfonts}
\usepackage{amsthm}
\usepackage{siunitx}
\DeclareSIUnit\year{yr}
\DeclareSIUnit\foot{ft}
\DeclareSIUnit\litre{\liter}

\usepackage{skull}

\usepackage{pgfplots}
\usepgfplotslibrary{polar}
\pgfplotsset{compat=1.11}
\usepgfplotslibrary{statistics}
\usepackage{graphicx}
\usepackage{sidecap}
\sidecaptionvpos{figure}{c}
\usepackage{float}
\usepackage{gensymb}
\usepackage{tkz-euclide}
\usetkzobj{all}
\usepackage{commath}
\usepackage{hyperref}
\usepackage{enumitem}
\usepackage{wasysym}
\usepackage{multicol}
\usepackage{mathtools}
\usepackage{tcolorbox}
\usepackage{tabularx}
\usepackage[version=4]{mhchem}
\usepackage{changepage}
\usepackage{listings}
\lstset{basicstyle=\ttfamily\linespread{0.8}\small}

\renewcommand*{\thefootnote}{\fnsymbol{footnote}}

\newtheorem*{thm}{Theorem}
\newtheorem*{iden}{Identity}
\newtheorem*{lemma}{Lemma}
\newtheorem{obs}{Observation}
\theoremstyle{definition}
\newtheorem*{defn}{Definition}
\newtheorem*{ex}{Example}
\newtheorem{con}{Construction}
\newtheorem*{alg}{Algorithm}

\newtheoremstyle{break}
  {\topsep}{\topsep}%
  {\itshape}{}%
  {\bfseries}{}%
  {\newline}{}%
\theoremstyle{break}
\newtheorem*{bthm}{Theorem}

% russian integral
\usepackage{scalerel}
\DeclareMathOperator*{\rint}{\scalerel*{\rotatebox{17}{$\!\int\!$}}{\int}}

% \DeclareMathOperator*{\rint}{\int}

\pgfplotsset{vasymptote/.style={
    before end axis/.append code={
        \draw[densely dashed] ({rel axis cs:0,0} -| {axis cs:#1,0})
        -- ({rel axis cs:0,1} -| {axis cs:#1,0});
    }
}}

% \pointsinrightmargin
\boxedpoints
\pointname{}

\newcommand{\questioA}{\question[\texttt{\textbf{\color{Cerulean} A}}]}
\newcommand{\questioM}{\question[\texttt{\textbf{\color{PineGreen} M}}]}
\newcommand{\questioE}{\question[\texttt{\textbf{\color{WildStrawberry} E}}]}
\newcommand{\questioS}{\question[\texttt{\textbf{\color{Goldenrod} S}}]}
\newcommand{\questioO}{\question[\texttt{\textbf{\color{BurntOrange} O}}]}

\newcommand{\parA}{\part[\texttt{\textbf{\color{Cerulean} A}}]}
\newcommand{\parM}{\part[\texttt{\textbf{\color{PineGreen} M}}]}
\newcommand{\parE}{\part[\texttt{\textbf{\color{WildStrawberry} E}}]}
\newcommand{\parS}{\part[\texttt{\textbf{\color{Goldenrod} S}}]}
\newcommand{\parO}{\part[\texttt{\textbf{\color{BurntOrange} O}}]}

\newcommand{\subparA}{\subpart[\texttt{\textbf{\color{Cerulean} A}}]}
\newcommand{\subparM}{\subpart[\texttt{\textbf{\color{PineGreen} M}}]}
\newcommand{\subparE}{\subpart[\texttt{\textbf{\color{WildStrawberry} E}}]}
\newcommand{\subparS}{\subpart[\texttt{\textbf{\color{Goldenrod} S}}]}
\newcommand{\subparO}{\subpart[\texttt{\textbf{\color{BurntOrange} O}}]}

\newcommand{\mainHeader}[2]{\section*{NCEA Level 2 Mathematics\\#1. #2}}
\newcommand{\mainHeaderHw}[2]{\section*{NCEA Level 2 Mathematics (Homework)\\#1. #2}}
\newcommand{\seealso}[1]{\begin{center}\emph{See also #1.}\end{center}}
\newcommand{\drills}[1]{\begin{center}\emph{Drill problems: #1.}\end{center}}
\newcommand{\basedon}[1]{\begin{center}\emph{Notes largely based on #1.}\end{center}}

\begin{document}

\mainHeaderDiff{11}{Related Rates of Change}
Moving on from optimisation and shape, we will now study the use of the derivative in modelling rates of change.

We can use the chain rule to relate rates of change together --- for example, the area of a circle is given by $ A = \pi r^2 $
and so the rate of change of area with respect to radius $ \od{A}{r} = 2\pi r $; but if $ r $ varies with respect to time then
we can find the rate of change of the area with respect to time using the chain rule.

A useful mnemonic is (if $ x $ is a function of $ y $ which is itself a function of $ z $)
\begin{equation}
  \od{x}{y} \cdot \od{y}{z} = \od{x}{z}. \tag{chain rule}
\end{equation}
We can also apply the inverse function rule for differentiation, which tells us that
\begin{equation}
  \od{x}{y} = \frac{1}{\od{y}{x}}. \tag{inverse function rule}
\end{equation}

Although these two operations allow us to rearrange equations as if $ \od{y}{x} $ were a fraction, it is
fairly important to remember that it is not a fraction and that whenever we treat it like one then it should
be possible to justify that step more rigorously.

\begin{ex}
  A ladder \SI{5}{\metre} long rests against a vertical wall. If the bottom of the ladder slides away
  from the wall at a rate of \SI{1}{\metre\per\second}, how fast is the top of the ladder sliding down the
  wall when the bottom of the ladder is \SI{3}{\metre} from the wall?

  \textit{Solution.} Let $ x $ be the distance of the bottom of the ladder from the wall, and let $ y $
  be the height of the top of the ladder up the wall. We have $ \od{x}{t} = 1 $ and $ x = 3 $; we also
  know that $ y = \sqrt{25 - x^2} $, so:
  \begin{align*}
    \dod{y}{t} &= \dod{y}{x} \cdot \dod{x}{t} = -\frac{x}{\sqrt{25 - x^2}} \cdot 1\\
    \eval{\dod{y}{t}}_{x = 3} &= -\frac{3}{\sqrt{25 - 9}} = -\frac{3}{4}.
  \end{align*}
  Hence the ladder is sliding down the wall at a rate of \SI{-0.75}{\metre\per\second}.
\end{ex}

\begin{ex}
  The radius of a sphere is increasing at a rate of $ \od{r}{t} = -\ln(t - 1) $ metres per second. At what rate will the surface area of
  the sphere be growing at $ t = 2 $?

  \textit{Solution.} We have $ \mathrm{SA} = 4\pi r^2 $, so $ \od{\mathrm{SA}}{r} = 8\pi r $ and
    \begin{displaymath}
      \od{\mathrm{SA}}{t} = \od{\mathrm{SA}}{r} \od{r}{t} = -\ln(t - 1) \times 8\pi r = 0.
    \end{displaymath}

    The surface area of the sphere will be momentarily constant at $ t = 2 $.
\end{ex}

\subsection*{Questions}
\begin{questions}
  \fullwidth{For each of the following situations: (a) identify the quantities given in the problem; (b) identify the unknown;
            (c) draw a diagram; (d) write an equation relating the given quantities to an equation; (e) solve the problem.}
  \questioM Each side of a square is increasing at a rate of \SI{6}{\centi\metre\per\second}. At what rate is the
            area of the square increasing when the area of the square is \SI{16}{\centi\metre\squared}?
  \questioM Gas is being forced into a spherical balloon at a rate of \SI{400}{\centi\metre\cubed\per\minute}. How fast
            is the radius of the balloon increasing when the radius is \SI{5}{\centi\metre}?
  \questioM If a snowball melts so that its surface area decreases at a rate of \SI{1}{\centi\metre\squared\per\minute}.
            find the rate at which the diameter decreases when the diameter is \SI{10}{\centi\metre}.
  \questioM If $ W = 4H^3 S $ where $ H = 3S^2 - 5 $, find $ \od{W}{S} $ in terms of $ H $ and $ S $.
  \questioM If $ x^2 + y^2 + z^2 = 9 $, $ \od{x}{t} = 5 $, and $ \od{y}{t} = 4 $, find $ \od{z}{t} $ when $ (x,y,z) = (2,2,1) $.
  \questioE The demand for an article varies inversely as the $ \frac{5}{2} $ power of the selling price (i.e. if $ x $ is
            the number of articles sold and $ p $ is the price then $ x = p^{-5/2} $). If the manufacture cost of one
            article is \$1, determine the selling price which will produce the greatest profit.
  \questioE A particle moves along the curve $ y = 2\sin(\pi x/2) $. As the particle moves through the point $ (1/3, 1) $,
            its $ x$-ordinate increases at a rate of $ \sqrt{10} $ \si{\centi\metre\per\second}. How fast is the distance
            from the particle to the origin changing at this instant?
  \questioM Gravel is dumped from a conveyor belt at a rate of \SI{3}{\metre\cubed\per\minute}, and forms a pile in the shape
            of a cone with equal height and base diameter. How fast is the height of the cone increasing when the pile is \SI{3}{\metre}
            tall?
  \questioM The top of a ladder slides down a vertical wall at a rate of \SI{0.15}{\metre\per\second}. A the moment when the bottom
            of the ladder is \SI{3}{\metre} from the bottom of the wall, it slides away from the wall at a rate of \SI{0.2}{\metre\per\second}.
            Find the length of the ladder.
  \questioM Two sides of a triangle have lengths \SI{2}{\metre} and \SI{3}{\metre}. The angle between these sides is increasing
            at a rate of \SI{4}{\degree\per\second}. How fast is the length of the third side changing when it is of length \SI{4}{\metre}?
  \questioE A particle is moving along a hyperbola $ xy = 8 $. As it reaches the point $ (4, 2) $, the $ y$-ordinate is decreasing
            at a rate of 3 units per second. How fast is the $ x$-ordinate of the particle changing at that instant?
  \questioE The minute hand on a watch is \SI{8}{\milli\metre} long and the second hand is \SI{4}{\milli\metre} long. How fast is
            the distance between the tips of the hands changing at 1 o'clock?
  \questioS Scholarship 2015: A tank contains 200 litres of brine (solution of salt in water). Initially, the concentration is \SI{0.5}{\kilo\gram}
            of salt per litre. Brine containing \SI{0.8}{\kilo\gram} of salt per litre runs into the tank at a rate of 6 litres per
            minute. The mixture is kept thoroughly mixed and is running out at the same rate.

            Find out how long it takes for the amount of salt in the tank to be \SI{130}{\kilo\gram}.
\end{questions}
\end{document}
