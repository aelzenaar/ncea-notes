\documentclass{exam}
\usepackage[utf8]{inputenc}
\usepackage{lmodern}
\usepackage{microtype}

% \usepackage[parfill]{parskip}
\usepackage[dvipsnames]{xcolor}
\usepackage{amsmath}
\usepackage{amsfonts}
\usepackage{amsthm}
\usepackage{siunitx}
\DeclareSIUnit\year{yr}
\DeclareSIUnit\foot{ft}
\DeclareSIUnit\litre{\liter}

\usepackage{skull}

\usepackage{pgfplots}
\usepgfplotslibrary{polar}
\pgfplotsset{compat=1.11}
\usepgfplotslibrary{statistics}
\usepackage{graphicx}
\usepackage{sidecap}
\sidecaptionvpos{figure}{c}
\usepackage{float}
\usepackage{gensymb}
\usepackage{tkz-euclide}
\usetkzobj{all}
\usepackage{commath}
\usepackage{hyperref}
\usepackage{enumitem}
\usepackage{wasysym}
\usepackage{multicol}
\usepackage{mathtools}
\usepackage{tcolorbox}
\usepackage{tabularx}
\usepackage[version=4]{mhchem}
\usepackage{changepage}
\usepackage{listings}
\lstset{basicstyle=\ttfamily\linespread{0.8}\small}

\renewcommand*{\thefootnote}{\fnsymbol{footnote}}

\newtheorem*{thm}{Theorem}
\newtheorem*{iden}{Identity}
\newtheorem*{lemma}{Lemma}
\newtheorem{obs}{Observation}
\theoremstyle{definition}
\newtheorem*{defn}{Definition}
\newtheorem*{ex}{Example}
\newtheorem{con}{Construction}
\newtheorem*{alg}{Algorithm}

\newtheoremstyle{break}
  {\topsep}{\topsep}%
  {\itshape}{}%
  {\bfseries}{}%
  {\newline}{}%
\theoremstyle{break}
\newtheorem*{bthm}{Theorem}

% russian integral
\usepackage{scalerel}
\DeclareMathOperator*{\rint}{\scalerel*{\rotatebox{17}{$\!\int\!$}}{\int}}

% \DeclareMathOperator*{\rint}{\int}

\pgfplotsset{vasymptote/.style={
    before end axis/.append code={
        \draw[densely dashed] ({rel axis cs:0,0} -| {axis cs:#1,0})
        -- ({rel axis cs:0,1} -| {axis cs:#1,0});
    }
}}

% \pointsinrightmargin
\boxedpoints
\pointname{}

\newcommand{\questioA}{\question[\texttt{\textbf{\color{Cerulean} A}}]}
\newcommand{\questioM}{\question[\texttt{\textbf{\color{PineGreen} M}}]}
\newcommand{\questioE}{\question[\texttt{\textbf{\color{WildStrawberry} E}}]}
\newcommand{\questioS}{\question[\texttt{\textbf{\color{Goldenrod} S}}]}
\newcommand{\questioO}{\question[\texttt{\textbf{\color{BurntOrange} O}}]}

\newcommand{\parA}{\part[\texttt{\textbf{\color{Cerulean} A}}]}
\newcommand{\parM}{\part[\texttt{\textbf{\color{PineGreen} M}}]}
\newcommand{\parE}{\part[\texttt{\textbf{\color{WildStrawberry} E}}]}
\newcommand{\parS}{\part[\texttt{\textbf{\color{Goldenrod} S}}]}
\newcommand{\parO}{\part[\texttt{\textbf{\color{BurntOrange} O}}]}

\newcommand{\subparA}{\subpart[\texttt{\textbf{\color{Cerulean} A}}]}
\newcommand{\subparM}{\subpart[\texttt{\textbf{\color{PineGreen} M}}]}
\newcommand{\subparE}{\subpart[\texttt{\textbf{\color{WildStrawberry} E}}]}
\newcommand{\subparS}{\subpart[\texttt{\textbf{\color{Goldenrod} S}}]}
\newcommand{\subparO}{\subpart[\texttt{\textbf{\color{BurntOrange} O}}]}

\newcommand{\mainHeader}[2]{\section*{NCEA Level 2 Mathematics\\#1. #2}}
\newcommand{\mainHeaderHw}[2]{\section*{NCEA Level 2 Mathematics (Homework)\\#1. #2}}
\newcommand{\seealso}[1]{\begin{center}\emph{See also #1.}\end{center}}
\newcommand{\drills}[1]{\begin{center}\emph{Drill problems: #1.}\end{center}}
\newcommand{\basedon}[1]{\begin{center}\emph{Notes largely based on #1.}\end{center}}

\begin{document}

\mainHeaderDiffHw{4}{The Chain Rule}
\subsection*{Reading}
Suppose we have the function $ y = \sin(x^2) $ which we saw in the tutorial. Consider
what happens if we let $ x $ change by a small amount, to $ x + \dif{x} $. Then $ y $
will change to $ y + \dif{y} $, and we have
\begin{displaymath}
  y + \dif{y} = \sin((x + \dif{x})^2) = \sin(x^2 + 2 x\dif{x} + \dif{x}^2).
\end{displaymath}
Let's let $ \dif{x} $ become really small; so $ \dif{x}^2 $ becomes even smaller. In fact, we
will take $ \dif{x} $ to be so small that $ \dif{x}^2 \to 0 $. Then we can basically ignore
it, and continue computing:
\begin{displaymath}
  \sin(x^2 + 2 x\dif{x} + \dif{x}^2) = \sin(x^2 + 2 x\dif{x}) = \sin(x^2) \cos(2x\dif{x}) + \cos(x^2) \sin(2x \dif{x}).
\end{displaymath}
Now, when $ t $ is small, $ \sin(t) \approx t $. Similarly, $ \cos(t) \approx 1 $. Since $ \dif{x} $ is small,
we apply these approximations:
\begin{displaymath}
  \sin(x^2) \cos(2x\dif{x}) + \cos(x^2) \sin(2x \dif{x}) = \sin(x^2) + 2x\dif{x} \cos(x^2).
\end{displaymath}
But recall that this is equal to $ y + \dif{y} $; so
\begin{align*}
  y + \dif{y} &= \sin(x^2) + 2x\dif{x} \cos(x^2)\\
  \dif{y} &= 2x \dif{x} \cos(x^2)\\
  \od{y}{x} &= 2x \cos(x^2).
\end{align*}

Note that this computation is most definitely \textbf{not rigorous} --- we don't justify why we can
ignore $ \dif{x}^2 $ but not $ \dif{x} $, and we don't define what it even means to be "small enough
to ignore"! However, it does at least suggest that, intuitively, the chain rule does what we expect it
to do.

This calculation can be made rigorous if we define infinitesimals and rules for calculating with them,
and this branch of mathematics is known as non-standard analysis. However, for the remainder of this
year we will continue to base our discussions of calculus on limits and inequalities because they are
easier to make rigorous, despite the initial barriers to intuition.

\subsection*{Questions}
\begin{questions}
  \question If $ y = \sqrt{\cot x} - \sqrt{\cot a} $ (where $ a $ is constant), find $ \od{y}{x} $.
  \question
    \begin{parts}
      \part Show that if $ y = f(g(h(x))) $ then $ \od{y}{x} = h'(x) \cdot g'(h(x)) \cdot f'(g(h(x))) $.
      \part Calculate the derivative of $ y = \sin \cos \sin \cos \sin x^5 $.
    \end{parts}
  \question We will prove the double angle formula for cosine from the double angle formula for sine.
            Suppose $ f(\theta) = \cos 2\theta $, and $ g(\theta) =  1 - 2\sin^2 \theta $.
    \begin{parts}
      \part Show that $ f' = g' $. (You may assume that $ \sin 2\theta = 2\sin \theta \cos \theta $.)
      \part Verify that $ f $ and $ g $ agree at $ \theta = 0 $, and conclude that $ f = g $.
    \end{parts}
\end{questions}
\end{document}
