\documentclass[a4paper]{report}

\usepackage{commath}
\usepackage{siunitx}

% russian integral
\usepackage{scalerel}
\DeclareMathOperator*{\rint}{\scalerel*{\rotatebox{17}{$\!\int\!$}}{\int}}

\title{Solutions to L3 Calculus Differentiation Exam 1}
\author{Alexander Elzenaar}
\date{3 September 2017}

\begin{document}

\maketitle

\section*{Question One}
\subsection*{Part (a)}
\begin{displaymath}
  \od{y}{x} = 2x + 1 - \frac{1}{x^2} - \frac{2}{x^3}
\end{displaymath}
(1 mark)

\subsection*{Part (b)}
We have $ \od{r}{t} = 0.5 $. Since $ A = \pi r^2 $, we can write $ \od{A}{t} = \od{r}{t} \times 2\pi r = \pi r $.
After 10 seconds, the radius will be 5 metres and so $ \eval{\od{A}{t}}_{t = 10} = 5\pi $.
(3 marks)

\subsection*{Part (c)}
Noting that an equilateral triangle has angles of $ \frac{\pi}{3} $, the height of the rectangle must
be $ \frac{L - x}{2} \tan \frac{\pi}{3} = \frac{\sqrt{3}}{2}(L - x) $. Hence the area of the rectangle is
\begin{displaymath}
  A = \frac{\sqrt{3}}{2}(L - x)x = \frac{\sqrt{3}}{2}(Lx - x^2)
\end{displaymath}
and, taking the derivative,
\begin{displaymath}
  A' = \frac{L\sqrt{3}}{2} - x\sqrt{3}
\end{displaymath}
The area will be maximised when $ A' = 0 $, which is exactly when $ x = L/2 $.
(4 marks)

\section*{Question Two}
\subsection*{Part (a)}
\paragraph{(i)}
\begin{align*}
  f'(x) &= -2\sin(2x) + \frac{1}{2} e^{x/2}\\
  f''(x) &= -4\cos(2x) + \frac{1}{4} e^{x/2}
\end{align*}
(2 marks)

\paragraph{(ii)}
We have $ f''(0) = -4 + \frac{1}{4} < 0 $, so the function is concave down and the point is a minimum.
(1 mark)

\subsection*{Part (b)}
\paragraph{(i)}
$ g(3) = 9 $ (1 mark)

\paragraph{(ii)}
The function approaches a value near $ y = 5.5 $ on the left, but a value near $ y = 9 $ from
the right. Hence the left and right limits are different, and the limit of the function at $ x = 3 $
does not exist.
(2 marks)

\paragraph{(iii)}
$ \lim\limits_{x \to 5} = 7 $ (1 mark)

\subsection*{Part (c)}
\begin{align*}
  \od{}{t} (x^2 - 5x) &= \lim_{h \to 0} \frac{[(x + h)^2 -  5(x + h)] - (x^2 - 5x) }{h}\\
                      &= \lim_{h \to 0} \frac{x^2 + 2xh + h^2 - 5x - 5h - x^2 + 5x}{h}\\
                      &= \lim_{h \to 0} \frac{2xh + h^2 - 5h}{h}\\
                      &= \lim_{h \to 0} 2x - 5 + h\\
                      &= 2x - 5.
\end{align*}
(3 marks)

\section*{Question Three}
\subsection*{Part (a)}
\begin{align*}
  \od{x}{t} &= \frac{(1 + \tan t)(\sec t)' - (\sec t)(1 + \tan t)'}{(1 + \tan t)^2}\\
            &= \frac{(1 + \tan t)(\sec t \tan t) - (\sec t)(\sec^2 t)}{1 + 2\tan t + \tan^2 t}\\
            &= \frac{\sec t \tan t + \sec t \tan^2 t - \sec^3 t}{1 + 2\tan t + \tan^2 t}
\end{align*}
(2 marks)

\subsection*{Part (b)}
\paragraph{(i)}
We have $ y'(t) = 4t $ and $ x'(t) = 4 $. Hence $ \od{y}{x} = t = \frac{x}{4} $. (2 marks)

\paragraph{(ii)}
Suppose that $ B = (x,y) $. Then the distance we wish to minimise is
\begin{displaymath}
  D = \sqrt{(x + 8)^2 + (y - 2)^2} = \sqrt{(4t + 8)^2 + (2t^2 - 2)^2} = \sqrt{4t^4 + 8t^2 + 64t + 68}
\end{displaymath}
Then we can take the derivative and set it to zero:
\begin{align*}
  \od{D}{t} &= \frac{1}{2(4t^4 + 8t^2 + 64t + 68)} \times 16t^3 + 16t + 64\\
  0 &= 16t^3 + 16t + 64\\
    &= t^3 + t + 4\\
  t &\approx -1.3788.
\end{align*}
(4 marks)

\subsection*{Part (c)}
Taking the derivative:
\begin{align*}
  y &= \frac{1}{m} [\sec(m \ln \theta)]^2\\
  \od{y}{\theta} &= \frac{1}{m} \cdot 2 \cdot \sec (m \ln \theta) \cdot \sec(m \ln \theta) \tan(m \ln \theta) \cdot \frac{m}{\theta}\\
                 &= \frac{2}{\theta} \cdot \sec^2(m \ln \theta) \cdot \tan(m \ln \theta).
\end{align*}
Plugging in $ \theta = 1 $, we have $ \frac{2}{1} \cdot \sec^2(m \ln 1) \cdot \tan(m \ln 1) = 2 \cdot 1 \cdot 0 = 0 $.
(2 marks)

\end{document}
