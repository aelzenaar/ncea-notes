\documentclass{exam}
\usepackage[utf8]{inputenc}
\usepackage{lmodern}
\usepackage{microtype}

% \usepackage[parfill]{parskip}
\usepackage[dvipsnames]{xcolor}
\usepackage{amsmath}
\usepackage{amsfonts}
\usepackage{amsthm}
\usepackage{siunitx}
\DeclareSIUnit\year{yr}
\DeclareSIUnit\foot{ft}
\DeclareSIUnit\litre{\liter}

\usepackage{skull}

\usepackage{pgfplots}
\usepgfplotslibrary{polar}
\pgfplotsset{compat=1.11}
\usepackage{graphicx}
\usepackage{sidecap}
\sidecaptionvpos{figure}{c}
\usepackage{float}
\usepackage{gensymb}
\usepackage{tkz-euclide}
\usetkzobj{all}
\usepackage{commath}
\usepackage{hyperref}
\usepackage{enumitem}
\usepackage{wasysym}
\usepackage{multicol}
\usepackage{mathtools}
\usepackage{tcolorbox}
\usepackage{tabularx}
\usepackage[version=4]{mhchem}
\usepackage{changepage}
\usepackage{listings}
\lstset{basicstyle=\ttfamily\linespread{0.8}\small}

\renewcommand*{\thefootnote}{\fnsymbol{footnote}}

\newtheorem*{thm}{Theorem}
\newtheorem*{iden}{Identity}
\newtheorem*{lemma}{Lemma}
\newtheorem{obs}{Observation}
\theoremstyle{definition}
\newtheorem*{defn}{Definition}
\newtheorem*{ex}{Example}
\newtheorem{con}{Construction}
\newtheorem*{alg}{Algorithm}

\newtheoremstyle{break}
  {\topsep}{\topsep}%
  {\itshape}{}%
  {\bfseries}{}%
  {\newline}{}%
\theoremstyle{break}
\newtheorem*{bthm}{Theorem}

% russian integral
\usepackage{scalerel}
\DeclareMathOperator*{\rint}{\scalerel*{\rotatebox{17}{$\!\int\!$}}{\int}}

% \DeclareMathOperator*{\rint}{\int}

\pgfplotsset{vasymptote/.style={
    before end axis/.append code={
        \draw[densely dashed] ({rel axis cs:0,0} -| {axis cs:#1,0})
        -- ({rel axis cs:0,1} -| {axis cs:#1,0});
    }
}}

% \pointsinrightmargin
\boxedpoints
\pointname{}

\newcommand{\questioA}{\question[\texttt{\textbf{\color{Cerulean} A}}]}
\newcommand{\questioM}{\question[\texttt{\textbf{\color{PineGreen} M}}]}
\newcommand{\questioE}{\question[\texttt{\textbf{\color{WildStrawberry} E}}]}
\newcommand{\questioS}{\question[\texttt{\textbf{\color{Goldenrod} S}}]}
\newcommand{\questioO}{\question[\texttt{\textbf{\color{BurntOrange} O}}]}

\newcommand{\parA}{\part[\texttt{\textbf{\color{Cerulean} A}}]}
\newcommand{\parM}{\part[\texttt{\textbf{\color{PineGreen} M}}]}
\newcommand{\parE}{\part[\texttt{\textbf{\color{WildStrawberry} E}}]}
\newcommand{\parS}{\part[\texttt{\textbf{\color{Goldenrod} S}}]}
\newcommand{\parO}{\part[\texttt{\textbf{\color{BurntOrange} O}}]}

\newcommand{\subparA}{\subpart[\texttt{\textbf{\color{Cerulean} A}}]}
\newcommand{\subparM}{\subpart[\texttt{\textbf{\color{PineGreen} M}}]}
\newcommand{\subparE}{\subpart[\texttt{\textbf{\color{WildStrawberry} E}}]}
\newcommand{\subparS}{\subpart[\texttt{\textbf{\color{Goldenrod} S}}]}
\newcommand{\subparO}{\subpart[\texttt{\textbf{\color{BurntOrange} O}}]}

\newcommand{\mainHeader}[2]{\section*{NCEA Level 2 Mathematics\\#1. #2}}
\newcommand{\mainHeaderHw}[2]{\section*{NCEA Level 2 Mathematics (Homework)\\#1. #2}}

\begin{document}

\section*{NCEA Level 3 Calculus\\Introduction to the Notes}

\begin{center}
  \includegraphics[width=0.8\textwidth]{hobbes}
\end{center}

\subsection* {What is Calculus?}
I will refrain from trying to advertise the subject to you, and will simply try to explain what calculus is and what kind of
person uses it. Calculus is the broad study of:
\begin{itemize}
  \item Continuous change;
  \item Slope, area, and volume; and
  \item Functions and relationships.
\end{itemize}

It has applications in physics, where calculus is the most natural language for Newtonian mechanics and classical electromagnetism; in chemistry
and biology, where calculus can be used to model anything which changes over time (like rates of reaction, concentrations, and populations); in
statistics (the study of probability distributions is just calculus); and in economics (I am assured). My own view, which I try to sprinkle throughout
these notes, is mainly a mixture of geometry and physical intuition.

Within mathematics itself, calculus is the computational side of \textbf{real analysis}, the study of the properties of the real number system.

\subsection*{Mathematical Prerequisites}
There are a number of things from Level 2 Mathematics that students should be
comfortable with; generally, I assume in these notes a vague merit-level understanding
of the core level 2 standards (by which I mean, the reader should be comfortable solving
achieved level problems without guidance and have some idea how to approach more difficult
problems):
\begin{itemize}
  \item L2 Algebra: All material on quadratics (factorising, solving, discriminants), logs and exponents.
  \item L2 Calculus: Basic differentiation, geometric meaning of derivative (in particular, integration is \textit{not} assumed).
  \item L2 Graphing: Recognising $ x-$/$ y-$shift of general functions, slope-intercept and point-slope form of linear equations,
                          recognising period/frequency/amplitude/$ x-$/$ y-$shift from a trig function.
  \item L2 Trigonometry: Trig ratios, the Pythagorean theorem.
  \item L2 Simultaneous Equations: Solving linear and quadratic simultaneous equations.
  \item L2 Co-ordinate Geometry: Distances and linear equations.
\end{itemize}

I would also recommend covering the L3 trigonometry topics before beginning calculus; but beyond this, no special knowledge
from other L3 standards is needed.

I do expect the reader to be able to draw graphs of various simple functions (e.g. parabolae, the three basic trig functions, logs and
exponents, and so forth) `by eye' --- I realise that this is a little unrealistic, but at some point one must learn to become comfortable
with the shapes and sizes of the objects we study. In order to help build this intuition, it is (very highly) recommended that the reader
works through every example in detail, drawing pictures and so forth.

\subsection*{A Note on Problem Difficulty}
One of the main goals for these notes is that they should be useful for students at all levels, from achieved to outstanding scholarship. Accordingly, the
problems each week range from simple (most students should be able to just write down an answer without thinking too hard) to
very difficult (it took \textbf{me} a while to work the problem, and I know this material quite well). If you can't do
a problem, the best thing to do is to move on and come back to it --- the problems don't always increase in difficulty. Of course,
it is important to do a good number of problems \textbf{including some difficult ones}; you're not under exam conditions here
and you're going to get an awful lot more out of a hard problem than an easy one!

I have marked the problems in the weekly worksheets (\textbf{not} the homework) with symbols relating vaguely to difficulty:

\begin{center}
\texttt{\textbf{\color{Cerulean} A}}
\texttt{\textbf{\color{PineGreen} M}}
\texttt{\textbf{\color{WildStrawberry} E}}
\texttt{\textbf{\color{Goldenrod} S}}
\texttt{\textbf{\color{BurntOrange} O}}
\end{center}

However, these are for my own reference and should not be taken to be accurate with respect to actual examinations.

\subsection*{Required content for Level 3}
Some of the material goes beyond that required for NCEA Level 3; the following list gives some idea of the level of each sheet.

\begin{multicols}{2}
\subsubsection*{Differentiation}
\begin{enumerate}
  \item[01.] The Derivative
  \item[02.] Limits
  \item[03.] Derivatives of Common Functions
  \item[04.] The Chain Rule
  \item[05.] The Product and Quotient Rules
  \item[06.] Tangent and Normal Lines
  \item[07.] The Geometry of Functions
  \item[08.] Optimisation
  \item[09.] Implicit Differentiation
  \item[10.] $^*$Inverse Functions
  \item[11.] Related Rates of Change
  \item[12.] Parametric Functions
  \item[13.] $^{*\perp}$Sequences and Series
  \item[14.] $^\dagger$Differentiation Revision
\end{enumerate}

\subsubsection*{Integration}
\begin{enumerate}
  \item[15.] Approximating Areas
  \item[16.] Anti-differentiation
  \item[17.] The Fundamental Theorem of Calculus
  \item[18.] Substitution
  \item[19.] Differential Equations
  \item[20.] $^*$Partial Fractions
  \item[21.] $^*$Integration by Parts
  \item[22.] $^{*\perp}$Lengths, Volumes, and Areas
  \item[23.] $^*$Trigonometric Substitution
  \item[24.] $^\perp$Kinematics
  \item[25.] $^\dagger$Integration Revision
  \item[26.] $^{*\perp\dagger\skull}$More Interesting Problems
\end{enumerate}
\end{multicols}

($^*$scholarship topic, $^\perp$interest topic, $^\dagger$revision, $^\skull$reader discretion advised)

In particular, the author tends to follow the following rough guidelines:

\paragraph{Standard L3 student}
1--9, 11, 12, 14, 15--19, 24, 25. No proofs; one week per sheet (so around two school terms).

\paragraph{Scholarship student}
All but 20, 23 and 26 (unless time available). Easy proofs. Leave some sheets as homework (e.g. cover the material of 3 -- 5 in one week and
leave a lot of problems for self study). Implicit differentiation and differential equations are revised in the conics notes, so some time
can be saved there if needed. Sequences and series (13) and kinematics (24) are just revision from Y12, so can be left entirely as reading.

\subsection*{Homework}
Every week has an associated homework sheet with a page of reading (five minutes or so) and a few questions (generally all
pretty straightforward, but there is often a challenge question on there to keep you occupied).

I cannot emphasise enough how important it is to \textbf{do the homework}.

\subsection*{How to Read Mathematics}
\begin{center}
  \textit{So we shall now explain how to read the book. The right way is to put it on your desk in the day, below your pillow at night, devoting yourself to the reading, and solving the exercises till you know it by heart. Unfortunately, I suspect the reader is looking for advice on how not to read, i.e. what to skip, and even better, how to read only some isolated highlights.}\\ - Saharon Shelah, `Classification Theory and the Number of Non-Isomorphic Models'
\end{center}

A major part of Level 3 Mathematics is preparation for university-level study in pure mathematics or the hard or soft sciences. As such, there are
a lot of new concepts that may be conceptually difficult. Everything we study in this topic is very geometric, and so drawing a lot of pictures
should be very useful.

There are a lot of definitions and results, but they allow us to capture our intuition of mathematical objects, and make sure that they behave in
the way we expect. It is important to read slowly and understand the statements as you go; you have a whole year to understand the subject. I do
not tend to repeat myself, so often you will find statements from earlier used later without comment.

I have also included proofs of a few of the theorems that we use, but rigorous proofs of many seemingly obvious ideas (like the fact that every function
reaches a minimum and a maximum value on any closed interval) require some subtle properties of the real numbers and so it is more important to gain a
geometric and intuitive feeling for many of the results rather than worrying too much.

\end{document}
