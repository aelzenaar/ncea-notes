\documentclass{exam}
\usepackage[utf8]{inputenc}
\usepackage{lmodern}
\usepackage{microtype}

% \usepackage[parfill]{parskip}
\usepackage[dvipsnames]{xcolor}
\usepackage{amsmath}
\usepackage{amsfonts}
\usepackage{amsthm}
\usepackage{siunitx}
\DeclareSIUnit\year{yr}
\DeclareSIUnit\foot{ft}
\DeclareSIUnit\litre{\liter}

\usepackage{skull}

\usepackage{pgfplots}
\usepgfplotslibrary{polar}
\pgfplotsset{compat=1.11}
\usepgfplotslibrary{statistics}
\usepackage{graphicx}
\usepackage{sidecap}
\sidecaptionvpos{figure}{c}
\usepackage{float}
\usepackage{gensymb}
\usepackage{tkz-euclide}
\usetkzobj{all}
\usepackage{commath}
\usepackage{hyperref}
\usepackage{enumitem}
\usepackage{wasysym}
\usepackage{multicol}
\usepackage{mathtools}
\usepackage{tcolorbox}
\usepackage{tabularx}
\usepackage[version=4]{mhchem}
\usepackage{changepage}
\usepackage{listings}
\lstset{basicstyle=\ttfamily\linespread{0.8}\small}

\renewcommand*{\thefootnote}{\fnsymbol{footnote}}

\newtheorem*{thm}{Theorem}
\newtheorem*{iden}{Identity}
\newtheorem*{lemma}{Lemma}
\newtheorem{obs}{Observation}
\theoremstyle{definition}
\newtheorem*{defn}{Definition}
\newtheorem*{ex}{Example}
\newtheorem{con}{Construction}
\newtheorem*{alg}{Algorithm}

\newtheoremstyle{break}
  {\topsep}{\topsep}%
  {\itshape}{}%
  {\bfseries}{}%
  {\newline}{}%
\theoremstyle{break}
\newtheorem*{bthm}{Theorem}

% russian integral
\usepackage{scalerel}
\DeclareMathOperator*{\rint}{\scalerel*{\rotatebox{17}{$\!\int\!$}}{\int}}

% \DeclareMathOperator*{\rint}{\int}

\pgfplotsset{vasymptote/.style={
    before end axis/.append code={
        \draw[densely dashed] ({rel axis cs:0,0} -| {axis cs:#1,0})
        -- ({rel axis cs:0,1} -| {axis cs:#1,0});
    }
}}

% \pointsinrightmargin
\boxedpoints
\pointname{}

\newcommand{\questioA}{\question[\texttt{\textbf{\color{Cerulean} A}}]}
\newcommand{\questioM}{\question[\texttt{\textbf{\color{PineGreen} M}}]}
\newcommand{\questioE}{\question[\texttt{\textbf{\color{WildStrawberry} E}}]}
\newcommand{\questioS}{\question[\texttt{\textbf{\color{Goldenrod} S}}]}
\newcommand{\questioO}{\question[\texttt{\textbf{\color{BurntOrange} O}}]}

\newcommand{\parA}{\part[\texttt{\textbf{\color{Cerulean} A}}]}
\newcommand{\parM}{\part[\texttt{\textbf{\color{PineGreen} M}}]}
\newcommand{\parE}{\part[\texttt{\textbf{\color{WildStrawberry} E}}]}
\newcommand{\parS}{\part[\texttt{\textbf{\color{Goldenrod} S}}]}
\newcommand{\parO}{\part[\texttt{\textbf{\color{BurntOrange} O}}]}

\newcommand{\subparA}{\subpart[\texttt{\textbf{\color{Cerulean} A}}]}
\newcommand{\subparM}{\subpart[\texttt{\textbf{\color{PineGreen} M}}]}
\newcommand{\subparE}{\subpart[\texttt{\textbf{\color{WildStrawberry} E}}]}
\newcommand{\subparS}{\subpart[\texttt{\textbf{\color{Goldenrod} S}}]}
\newcommand{\subparO}{\subpart[\texttt{\textbf{\color{BurntOrange} O}}]}

\newcommand{\mainHeader}[2]{\section*{NCEA Level 2 Mathematics\\#1. #2}}
\newcommand{\mainHeaderHw}[2]{\section*{NCEA Level 2 Mathematics (Homework)\\#1. #2}}
\newcommand{\seealso}[1]{\begin{center}\emph{See also #1.}\end{center}}
\newcommand{\drills}[1]{\begin{center}\emph{Drill problems: #1.}\end{center}}
\newcommand{\basedon}[1]{\begin{center}\emph{Notes largely based on #1.}\end{center}}

\begin{document}

\mainHeaderIntg{20}{Partial Fractions}
\textbf{\color{red} This is a Scholarship topic! The algebraic computations required can get quite messy.}

\begin{defn}[Rational Function]
  A \textbf{rational function} is a function $ f $ which can be written in the form
  \begin{displaymath}
    f(x) = \dfrac{p(x)}{q(x)}
  \end{displaymath}
  for suitable polynomials $ p $ and $ q $ (where $ q \neq 0 $).
\end{defn}

We can already integrate some rational functions; in particular, those of the form $ f(x) = p'(x)/p(x) $:
\begin{displaymath}
  \rint \dfrac{p'(x)}{p(x)} \dif{x} = \ln\abs{p(x)}.
\end{displaymath}

This week we will learn a technique that, in theory, allows us to integrate \textit{all} rational functions. To understand
the idea, note that we can easily integrate all functions of the form
\begin{displaymath}
  f(x) = \dfrac{A}{(ax + b)^n}
\end{displaymath}
for real constants $ A $, $ a $, $ b $, and $ n $.

Our task is simply to `deconstruct' arbitrary fractions into this form.
\begin{ex}
  \begin{displaymath}
    \dfrac{2}{x-4} + \dfrac{3}{x + 1} = \dfrac{5x - 10}{x^2 - 3x - 4}
  \end{displaymath}
  so
  \begin{displaymath}
    \rint \dfrac{5x - 10}{x^2 - 3x - 4} \dif{x} = \rint \dfrac{2}{x-4} + \dfrac{3}{x + 1} \dif{x} = 2\ln\abs{x-4} + 2\ln\abs{x-4}.
  \end{displaymath}
\end{ex}

In effect, the technique of partial fractions is the reverse of this: we \emph{decompose} the more
complex rational function into two or more functions which are easier to integrate.

Let $ f(x) = \dfrac{p(x)}{q(x)} $. Then we have four cases
\begin{enumerate}
  \item $ q(x) $ is the product of distinct linear factors.
  \item $ q(x) $ is the product of linear factors, some of which are repeated.
  \item $ q(x) $ is the product of distinct factors, some of which are irreducible quadratics.
  \item $ q(x) $ contains a repeated irreducible quadratic factor.
\end{enumerate}

Note: the degree of $ p $ must be less than the degree of $ q $, so you may need to use long division
before applying the technique of partial fractions.

We consider only the first two cases here. See Stewart \S 7.4 for the others.

\clearpage
\subsection*{Type I: Distinct linear factors}
Suppose that $ q(x) = (\alpha_1 x + \beta_1) + \cdots + (\alpha_n x + \beta_n) $. Then the partial
fraction decomposition is of the form
\begin{displaymath}
  \sum_{i = 1}^n \dfrac{A_i}{\alpha_i x + \beta_i}.
\end{displaymath}

\begin{ex}
  \begin{displaymath}
    \dfrac{11x - 2}{6x^2 + x - 1} = \dfrac{11x - 2}{(2x + 1)(3x - 1)} = \dfrac{A}{2x + 1} + \dfrac{B}{3x - 1}.
  \end{displaymath}
  So $ 11x - 2 = A(3x - 1) + B(2x + 1) $. Let $ x = 1/3 $, so $ B = \frac{11/3 - 2}{2/3 + 1} = 1 $;
  then let $ x = -1/2 $, so $ A = \frac{11/2 + 2}{3/2 + 1} = 3 $.

  Hence
  \begin{displaymath}
    \rint \dfrac{11x - 2}{6x^2 + x - 1} \dif{x} = \rint \dfrac{3}{2x + 1} + \dfrac{1}{3x - 1} \dif{x}
      = \dfrac{3}{2} \ln\abs{2x + 1} + \dfrac{1}{2} \ln\abs{3x - 1} + C.
  \end{displaymath}
\end{ex}

\subsection*{Type II: Repeated linear factors}
Suppose some factor $ (\alpha_i x + \beta_i)^r $ appears in the factorisation of $ Q(x) $. Then
the partial fraction decomposition will include
\begin{displaymath}
  \sum_{j = 1}^{r} \frac{A_{i_j}}{(\alpha_i x + \beta_i)^j}.
\end{displaymath}

\begin{ex}
  Consider $ \rint \frac{2x + 4}{x^3 - 2x^2} \dif{x} $. We wish to find a partial fraction expansion:
  \begin{align*}
    \frac{A}{x} + \frac{B}{x^2} + \frac{C}{x-2} = \frac{2x + 4}{x^3 - 2x^2} &\iff 2x + 4 = Ax(x-2) + B(x-2) + Cx^2\\
                                                                            &\iff 2x + 4 = (A + C)x^2 + (B - 2A)x -2B
  \end{align*}
  Matching coefficients, we find $ B = -2 $, $ A = -2 $, and $ C = 2 $. Then:
  \begin{align*}
    \rint \frac{2x + 4}{x^3 - 2x^2} \dif{x} &= \rint \frac{-2}{x} + \frac{-2}{x^2} + \frac{2}{x-2} \dif{x}\\
                                            &= -2\ln \abs{x} + \frac{2}{x} + 2\ln\abs{x-2} + C\\
                                            &= 2\ln\abs{\frac{x-2}{x}} + \frac{2}{x} + C.
  \end{align*}
\end{ex}

\clearpage
\subsection*{Questions}
\begin{questions}
  \questioS Find $ \displaystyle\rint \dfrac{A}{(ax + b)^n} \dif{x} $ if $ A $, $ a $, $ b $, and $ n $ are real constants.
  \questioS Evaluate, using partial fractions:
    \begin{parts}
      \part $ \displaystyle\rint \dfrac{3x - 1}{(x - 3)(x + 4)} $
      \part $ \displaystyle\rint \dfrac{1}{x^2 - 3x - 4} $
      \part $ \displaystyle\rint \dfrac{1}{x^2 - 6x - 7} $
      \part $ \displaystyle\rint \dfrac{11x + 17}{2x^2 +7x - 4} $
      \part $ \displaystyle\rint \dfrac{5x - 10}{x^2 - 3x - 4} $
      \part $ \displaystyle\rint \dfrac{x + 7}{x^2 - x - 6} $
      \part $ \displaystyle\rint \dfrac{1}{x^2 + 5x + 6} $
      \part $ \displaystyle\rint \dfrac{2x^2 + 3}{x(x-1)^2} $
    \end{parts}
  \questioS Some more interesting problems:
      \begin{parts}
        \part Rewrite in the form $ \dfrac{A}{x-1} + \dfrac{B}{(x-1)^2} + \dfrac{C}{x + 1} $ and integrate:
          \begin{displaymath}
            \rint \dfrac{4x}{x^3 - x^2 - x + 1} \dif{x}.
          \end{displaymath}
        \part Use the obvious substitution and divide through:
          \begin{displaymath}
            \rint \dfrac{\sqrt{x+1}}{x} \dif{x}.
          \end{displaymath}
      \end{parts}
  \questioS Use appropriate substitutions to evaluate:
    \begin{parts}
      \part $ \displaystyle\rint \dfrac{\cos \theta}{\sin^2 \theta + 4 \sin \theta - 5} \dif{\theta} $
      \part $ \displaystyle\rint \dfrac{e^{3x}}{e^{2x} + 4} \dif{t} $
      \part $ \displaystyle\rint \dfrac{5 + 2\ln x}{x(1 + \ln x)^2} $
    \end{parts}
  \clearpage
  \questioE Don't use a sledgehammer to kill a fly, and compute the following:
    \begin{parts}
      \part $ \displaystyle\rint \dfrac{x^2(5x^2 + 4x - 3)}{x^5 + x^4 - x^3 + 1} \dif{x} $`
      \part $ \displaystyle\rint \dfrac{x^2 + 1}{x(x^2 + 3)} \dif{x} $
    \end{parts}
  \questioO We have already computed $ \rint \sec x \dif{x} $ via a bit of a trick, but we could also use partial fractions.
    \begin{parts}
      \part Show that $ \sec x = \frac{\cos x}{1 - \sin^2 x} $.
      \part Hence, or otherwise, compute $ \rint \sec x \dif{x} $.
    \end{parts}
  \questioS Solve the following differential equation for $ y(x) $:
    \begin{displaymath}
      \od{y}{x} = \frac{y^2 - a^4}{x^2 - a^2}
    \end{displaymath}
  \questioS Scholarship 2008:
    \begin{parts}
      \part $ \dfrac{A}{x} + \dfrac{B}{P-x} = \dfrac{1}{x(P-x)} $ where $ x $ is a variable and $ P $ is a constant.
            Find $ A $ and $ B $ in terms of $ P $.
      \part When a rumour about a teacher is started at a school of size $ P $ students, it spreads at a rate (in students
            per day) that is proportional to the product of the number of students who know the rumour, $ N $, and those who
            do not. Find an expression for the number of students $ N $ who know the rumour after $ t $ days.
      \part For a particular rumour about a teacher, 0.5\% of students know the rumour initially. The principal will need to act
            to stop the rumour once more than half the school's students know it. When $ \dfrac{1}{5} $ of the students know the
            rumour, the number who know the rumour is increasing at a rate of $ 0.08P $ students per day.
            How long will it be before the principal must act?
    \end{parts}
  \questioO Scholarship 2015: The rate of spread of a rumour at a particular school is proportional to both
            the number of students who know a rumour, $ S $, and the number of students who do not.
            If $ N $ is the total number of students in the school, then $ \od{S}{t} = kS(N-S) $.
            Initially, two students knew the rumour. Show that the number of students who know the
            rumour at time $ t $ is $ S(t) = \dfrac{N}{1 + \dfrac{1}{2} e^{-kNt}(N-2)} $.
  \questioO Recall that $ \od{}{x} \tan^{-1} x = \frac{1}{x^2 + 1} $.
    \begin{parts}
      \part Find a partial expansion of the given rational function as follows:
        \begin{displaymath}
          \frac{x^2 + x - 2}{3x^3 - x^2 + 3x - 1} = \frac{A}{3x - 1} + \frac{Bx + C}{x^2 + 1}
        \end{displaymath}
      \part Hence (or otherwise) compute:
        \begin{displaymath}
          \rint^{\pi/3}_{\pi/4} \frac{x^2 + x - 2}{3x^3 - x^2 + 3x - 1} \dif{x}.
        \end{displaymath}
    \end{parts}
  \questioS (Revenge of the limits.) Compute the following series. [\textit{Hint: this sheet is on partial fractions.}]
            \begin{displaymath}
              \lim_{N \to \infty} \sum^N_{n = 1} \left( \frac{1}{n(n + 1)} \right)
            \end{displaymath}

\end{questions}
\end{document}
