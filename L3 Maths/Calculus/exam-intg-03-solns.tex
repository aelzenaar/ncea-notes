\documentclass[a4paper]{report}

\usepackage{commath}
\usepackage{siunitx}
\usepackage{hhline}

% russian integral
\usepackage{scalerel}
\DeclareMathOperator*{\rint}{\scalerel*{\rotatebox{17}{$\!\int\!$}}{\int}}

\title{Solutions to L3 Calculus Integration Exam 3}
\author{Alexander Elzenaar}
\date{19 November 2017}

\begin{document}

\maketitle

\section*{Question One}
\subsection*{Part (a)}
\paragraph{i.}
Note that $ \od{}{x} \sqrt{x} = \frac{1}{2\sqrt{x}} $, so we must have
\begin{displaymath}
  \rint \frac{1}{\sqrt{2x + 2}} \dif{x} = \rint \frac{2}{2\sqrt{2x + 2}} \dif{x} = \sqrt{2x + 2} + C.
\end{displaymath}

\paragraph{ii.}
Let $ u = \tan x $ so $ \dif{u} = \sec^2 x \dif{x} $ and our integral becomes
\begin{displaymath}
  \rint \sec(u) \tan(u) \dif{u} = \sec u + C = \sec \tan x + C.
\end{displaymath}

\subsection*{Part (b)}
First, we find the left-hand integration limit: $ 1/x = 1/x^2 $ so $ 1 = 1/x $ and $ x = 1 $.
\begin{align*}
  \rint_1^2 \frac{1}{x} - \frac{1}{x^2} \dif{x} &= \eval{\left[\ln x + \frac{1}{x}\right]}_1^2\\
                                                &= (\ln 2 - \ln 1) + \left(\frac{1}{2} - 1\right)\\
                                                &= \ln \frac{1}{2} - \frac{1}{2}.
\end{align*}

\subsection*{Part (c)}
Separation of variables.
\begin{align*}
  \rint \frac{\cos y}{\sin y} \dif{y} &= \rint \frac{\sin x}{\cos x} \dif{x}\\
  \ln\sin y &= -\ln\cos x + C\\
  \sin y &= Ke^{\ln\sec x} = K\sec x\\
  y &= \sin^{-1} (K\sec x).
\end{align*}
(where $ K = e^C $.)

\section*{Question Two}
\subsection*{Part (a)}
\begin{center}
  \def\arraystretch{1.5}%  1 is the default, change whatever you need
  \begin{tabular}{|c|c|}          \hline
    $ x $ & \mathstrut$ (x^2 e^{-x})^2 $\\\hhline{|=|=|}
    1.0   & \textbf{0.1353}   \\\hline
    1.5   & 0.2520            \\\hline
    2.0   & \textbf{0.2931}   \\\hline
    2.5   & 0.2632            \\\hline
    3.0   & \textbf{0.2008}   \\\hline
    3.5   & 0.1368            \\\hline
    4.0   & 0.0859            \\\hline
    4.5   & 0.0506            \\\hline
    5.0   & 0.0284            \\\hline
  \end{tabular}
\end{center}

Note that $ n = 8 $ (number of intervals is one less than numbers of points), $ a = 1.0 $, $ b = 5.0 $. Hence:
\begin{align*}
  \rint^{5}_1 (x^2 e^{-x})^2 \dif{x} &\approx \frac{5 - 1}{8} [0.1353 + 0.0284 + 4(0.2520 + 0.2632 + 0.1368 + 0.0506) +\\
                                     &\qquad+ 2(0.2931 + 0.2008 + 0.0859)]\\
                                     &=2.0669
\end{align*}

and the required value is 6.4932.

\subsection*{Part (b)}
We have $ 5 = \eval{[\frac{1}{2}(\ln x)^2]}_1^k = \frac{1}{2} (\ln k)^2 $, so $ k = \exp(\sqrt{10}) $.

\subsection*{Part (c)}
Let $ B(t) $ be the bank balance after $ t $ years. Then:
\begin{displaymath}
  \od{B}{t} = 0.04B,
\end{displaymath}
so $ \ln B = 0.04t + C \Rightarrow B = B_0 e^{0.04t} $ ($ B_0 = e^C $). We have $ t = 4 $ and $ B_0 = 2500 $,
so $ B(4) = 2500 e^{0.16} = \$2933.78 $.

\section*{Question Three}
\subsection*{Part (a)}
Computing:
\begin{align*}
  \frac{\rint^{12}_0 20 + 8\sin \left(\frac{\pi t}{12} \right) \dif{t}}{12}
     &= \frac{1}{12} \eval{\left[20t - \frac{96}{\pi} \cos\left(\frac{\pi t}{12}\right)\right]}_0^{12}\\
     &= \frac{1}{12} \left[\left(240 + \frac{96}{\pi}\right) + \frac{96}{\pi}\right]\\
     &= \SI{25}{\celsius}.
\end{align*}

\subsection*{Part (b)}
The area of the cross-section at height $ x $ is $ A = (3 - 0.1x)(4 - 0.2x) = 0.02x^2 - x + 12 $; hence the total volume is:
\begin{align*}
  \rint_0^{18} 0.02x^2 - x + 12 \dif{x} &= \frac{0.02 \times 18^3}{3} - \frac{18^2}{2} + 12 \times 18 = \SI{92.88}{\metre\cubed}.
\end{align*}

\subsection*{Part (c)}
\begin{align*}
  1 &= \rint^{2m}_m x \cos(mx^2) \dif{x}\\
    &= \eval{\left[\frac{1}{2m} \sin(mx^2)\right]}_m^{2m}\\
    &= \frac{1}{2m} \left( \sin 4m^3 - \sin m^3 \right)\\
    &= \frac{1}{2m} 2 \cos \frac{5m^3}{2} \sin \frac{3m^3}{2}\\
  m &= \cos \frac{5m^3}{2} \sin \frac{3m^3}{2}.
\end{align*}
Since $ \forall x $ we have $ \cos x \leq 1 $ and $ \sin x \leq 1 $, it follows that $ \cos \frac{5m^3}{2} \sin \frac{3m^3}{2} \leq 1 $ and
so $ m \leq 1 $.

\end{document}
