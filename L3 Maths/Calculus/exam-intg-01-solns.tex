\documentclass[a4paper]{report}

\usepackage{commath}
\usepackage{siunitx}

% russian integral
\usepackage{scalerel}
\DeclareMathOperator*{\rint}{\scalerel*{\rotatebox{17}{$\!\int\!$}}{\int}}

\title{Solutions to L3 Calculus Integration Exam 1}
\author{Alexander Elzenaar}
\date{4 September 2017}

\begin{document}

\maketitle

\section*{Question One}
\subsection*{Part (a)}
\paragraph{(i)}
\begin{displaymath}
  \rint \frac{3t^2 + 2t}{\sqrt{t}} \dif{t} = \rint 3t^{3/2} + 2t^{1/2} \dif{t} = \frac{6}{5} t^{5/2} + \frac{4}{3} t^{3/2} + C.
\end{displaymath}
(1 mark)

\paragraph{(ii)}
\begin{displaymath}
  \rint 2 \sin 2x \sin (\cos 2x) \dif{x} = \sin \cos 2x + C.
\end{displaymath}
(1 mark)

\subsection*{Part (b)}
\begin{displaymath}
  y = \frac{1}{\ln 2} \rint \frac{1}{x + 2} = \frac{\ln(x + 2)}{\ln 2} + C
\end{displaymath}
Since $ 3 = \frac{\ln{0 + 2}}{\ln 2} + C = 1 + C $, $ C = 2 $. Hence $ y = \frac{\ln(x + 2)}{\ln 2} + 2 $,
and when $ x = -1, y = \frac{\ln(-1 + 2)}{\ln 2} + 2 = 2 $.
(3 marks)

\subsection*{Part (c)}
\begin{displaymath}
  \rint^{10}_0 f(t) \dif{t} = \rint^y_0 \dif{t} + \rint^{10}_x f(t) \dif{t} - \rint^y_x f(t) \dif{t} = 4 + 3 - 2 = 5
\end{displaymath}
(2 marks)

\subsection*{Part (d)}
We integrate along the $ x$-axis from $ x = -1 $ to $ x = 0 $ and then from $ x = 0 $ to $ x = 1 $. Over the first half,
the radius of each semicircle is $ r = x + 1 $, and over the second half the radius of each semicircle is $ r = -x + 1 $.
Hence our volume will be
\begin{displaymath}
  \rint^0_{-1} \frac{1}{2} \pi (x + 1)^2 \dif{x} + \rint^1_0 \frac{1}{2} \pi (-x + 1)^2 \dif{x}
    = \eval{\frac{\pi}{6} (x + 1)^3}^0_{-1} - \eval{\frac{\pi}{6} (-x + 1)^3}^1_{0} = \frac{\pi}{3}
\end{displaymath}
(5 marks)

\section*{Question Two}
\subsection*{Part (a)}
\begin{displaymath}
  \rint^{\pi/3}_{\pi/4} \csc^2 \theta \dif{\theta} = \eval{-\cot \theta}^{\pi/3}_{\pi/4} = 1 - \frac{1}{\sqrt{3}} \approx 0.4226
\end{displaymath}
(2 marks)

\subsection*{Part (b)}
\begin{multline*}
  \rint^{\frac{3\pi}{2}}_{\frac{\pi}{2}} \sin x - x^2 + \frac{2(1 + \pi^2)}{\pi} x + \left( \frac{3\pi^2}{4} + 2 \right) \dif{x}\\
    = \eval{-\cos x - \frac{1}{3}x^3 + \frac{(1 + \pi^2)}{\pi} x^2 + \left( \frac{3\pi^2}{4} + 2 \right)x}^{\frac{3\pi}{2}}_{\frac{\pi}{2}}\\
    \approx 86.25748 - 22.01398 \approx 64.24350
\end{multline*}
(3 marks)

\subsection*{Part (c)}
\paragraph{(i)}
\begin{align*}
  \rint \frac{\dif{P}}{1 - \frac{P}{M}} &= \rint \dif{t}\\
  -M\ln\left(1 - \frac{P}{M}\right) &= t + C\\
  \ln \left(1 - \frac{P}{M}\right) &= -Mt + C\\
  1 - \frac{P}{M} &= Ke^{-Mt}\\
  P &= M\left(1 - Ke^{-Mt}\right)
\end{align*}

At $ t = 0 $, $ P = 100 $. So $ 100 = M(1 - Ke) $ and $ K = \frac{M - 100}{Me} $.

Hence the explicit formula for $ P $ is
\begin{displaymath}
  P = M\left(1 - \frac{M - 100}{M}e^{-Mt - 1} \right) = M + \frac{(100 - M)}{e^{Mt + 1}}.
\end{displaymath}
(4 marks)

\paragraph{(ii)}
At $ t = 0$, $ \od{P}{t} = 1 - \frac{P}{M} = 1 $. Hence:
\begin{align*}
  1 = 1 - \frac{M + \frac{(100 - M)}{e^{Mt + 1}}}{M} &= \frac{(100 - M)}{Me}\\
                                                   M &= \frac{100}{e + 1}
\end{align*}
and therefore
\begin{align*}
  P = M + \frac{(100 - M)}{e^{Mt + 1}} = \frac{100}{e + 1} + \frac{\left(100 - \frac{100}{e + 1}\right)}{e^{\frac{100t}{e + 1} + 1}}.
\end{align*}

At $ t = 100 $, $ P \approx 26.89 $ --- or around 27 animals.
(3 marks)

\section*{Question Three}
\subsection*{Part (a)}
Applying Simpson's rule:
\begin{displaymath}
  \frac{1}{3} \cdot \frac{6 - 0}{6} \cdot \left[ 3.2 + 1.1 + 4(2.7 + 1.7 + 1.0) + 2(1.9 + 1.3) \right] \approx 10.76.
\end{displaymath}
(2 mark)

\subsection*{Part (b)}
\begin{displaymath}
  \rint e^x (15 + e^x)^{2017} + 3 \dif{x} = \frac{(15 + e^x)^{2018}}{2018} + 3x + C.
\end{displaymath}
(2 marks)

\subsection*{Part (c)}
\begin{align*}
  \rint \frac{\dif{y}}{y + 2} &= -\rint n\pi \sin(xn \pi) \dif{x}\\
  \ln(y + 2) &= \cos(xn \pi) + C\\
  y &= Ke^{\cos(xn \pi)} - 2
\end{align*}

Since $ y = 0 $ when $ x = 0 $, we have $ 2 = Ke $ and so $ K = \frac{2}{e} $. Hence $ y = 2e^{\cos(xn \pi) - 1} - 2 $,
and when $ x = \frac{1}{n} $, $ y = 2e^{\cos(\pi) - 1} - 2 = 2e^{-2} - 2 = \frac{2}{e^2} - 2 \approx -1.729 $.
(4 marks)

\subsection*{Part (d)}
Let $ x = \sin \theta $; then $ \dif{x} = \cos \theta \dif{\theta} $, and:
\begin{align*}
  \rint \frac{1}{\sqrt{1 - x^2}} \dif{x} &= \rint \frac{\cos \theta}{\sqrt{1 - (\sin \theta)^2}} \dif{\theta}\\
                                         &= \rint \frac{\cos \theta}{\cos \theta} \dif{\theta}\\
                                         &= \rint \dif{\theta}\\
                                         &= \theta = \sin^{-1} x.
\end{align*}
(4 marks)

\end{document}
