\documentclass[a4paper]{report}

\usepackage{commath}
\usepackage{siunitx}

% russian integral
\usepackage{scalerel}
\DeclareMathOperator*{\rint}{\scalerel*{\rotatebox{17}{$\!\int\!$}}{\int}}

\title{Solutions to L3 Calculus Differentiation Exam 2}
\author{Alexander Elzenaar}
\date{4 September 2017}

\begin{document}

\maketitle

\section*{Question One}
\subsection*{Part (a)}
\begin{align*}
  \od{D}{t} &= -\omega A \cos(kx  - \omega t + \phi_0)\\
  \od{D}{x} &= kA \cos(kx  - \omega t + \phi_0)
\end{align*}
(2 marks)

\subsection*{Part (b)}
\begin{align*}
  \tan y &= e^x\\
  \od{y}{x} \sec^2 y &= e^x\\
  \od{y}{x} &= e^x \cos^2 y
\end{align*}
So at $ (0, \frac{\pi}{4}) $ we have $ \od{y}{x} = e^0 \cos^2 \frac{\pi}{4} = \frac{1}{2} $.
(3 marks)

\subsection*{Part (c)}
We have that the angle of the slope of the cone with the base is $ \tan^{-1} \frac{H}{R} $ and
so the height of the cylinder at radius $ r $ is $ h = (R - r)\frac{H}{R} = H(1 - \frac{r}{R}) $. The
volume of the cylinder is $ V = \pi r^2 h = \pi r^2 H(1 - \frac{r}{R}) = H\pi r^2 - \frac{H\pi}{R} r^3 $.
Taking the derivative, $ \od{V}{r} = 2H\pi r - \frac{3H\pi}{R} r^2 $; setting to zero, $ 2H\pi r = \frac{3H\pi}{R} r^2 $
and so (since $ r \neq 0 $) $ r = \frac{2R}{3} $.

We also have $ \od[2]{V}{r} = 2H\pi - \frac{6H\pi}{R} r $, so at our found radius $ \od[2]{V}{r} = 2H\pi - 4H\pi < 0 $ and
so the function is concave down at that point --- we have indeed found a maximum.

(5 marks)

\section*{Question Two}
\subsection*{Part (a)}
\paragraph{(i)}
\begin{displaymath}
  \od{y}{x} = -\frac{e^{-x}}{x^2} - \frac{e^{-x}}{x} - \frac{\sin x}{x^2} + \frac{\cos x}{x}.
\end{displaymath}
(2 mark)

\paragraph{(ii)}
\begin{displaymath}
  \od{y}{x} = \frac{5}{2} x^{3/2} + \frac{15}{2} x^{1/2} - \frac{1}{2} x^{-3/2}.
\end{displaymath}
(1 mark)

\subsection*{Part (b)}
\paragraph{(i)}
\begin{displaymath}
  \od{}{x} x^3 = \lim_{h \to 0} \frac{(x + h)^3 - x^3}{h}
\end{displaymath}
(1 mark)

\paragraph{(ii)}
Noticing that we are just finding the  value of the derivative of $ x^3 $ at $ x = 2 $, we evaluate $ 3x^2 $ at 2 and obtain 12 as
the value of the limit. Alternatively, one could expand the brackets and evaluate the limit algebraically.
(2 marks)

\subsection*{Part (c)}
\begin{align*}
  f'(x) &= \frac{x^4}{4} - 2x^2 + 16\\
  f''(x) &= x^3 - 4x = x(x^2 - 4) = (x - 2)x(x + 2)
\end{align*}
Since $ f''(x) $ is a positive conic, it is negative on the intervals $ x < -2 $ and $ 0 < x < 2 $ and positive on
the intervals $ -2 < x < 0 $ and $ x > 2 $. Hence the function $ f $ is concave up on the latter two intervals.
(4 marks)

\section*{Question Three}
\subsection*{Part (a)}
\begin{displaymath}
  g'(t) = 6t \cdot \frac{\pi}{2\sqrt{3t^2 + 4}} \cdot \cos(\pi\sqrt{3t^2 + 4}).
\end{displaymath}
We therefore have $ g'(2) = 12 \cdot \frac{\pi}{2\sqrt{16}} \cdot \cos(\pi\sqrt{16}) = 12 \cdot \frac{\pi}{8} \cdot 1 = \frac{3\pi}{2} $.
(2 marks)

\subsection*{Part (b)}
\begin{displaymath}
  \od{y}{x} = 4x^3 - 9x^2 - 4x + 2
\end{displaymath}
Therefore at the point $ (0, -1) $ the gradient is $ m = 2 $ and the normal line will have a slope of $ -\frac{1}{2} $.
Hence the equation of the normal is $ (y + 1) = -\frac{1}{2}x $, or $ y = -\frac{1}{2}x - 1 $.
(3 marks)

\subsection*{Part (c)}
We have $ \od{h}{t} = \od{S}{t} \cdot (-\frac{1}{\sqrt{S}}) = -\frac{3t + 4}{\sqrt{S}} $. This is zero exactly when $ 3t + 4 = 0 $,
which is when $ t = -\frac{4}{3} < 0 $.
(2 marks)

\subsection*{Part (d)}
\begin{displaymath}
  \od{y}{x} = 6x^2 - 36x + 90
\end{displaymath}
Suppose $ \od{y}{x} =  4 $. Then $ 6x^2 - 36x + 86 = 0 $. But $ 36^2 - 4\times 6 \times 86 < 0 $, so the
gradient of the curve is never 4.
(3 marks)

\end{document}
