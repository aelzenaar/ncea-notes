\documentclass{exam}
\usepackage[utf8]{inputenc}
\usepackage{lmodern}
\usepackage{microtype}

% \usepackage[parfill]{parskip}
\usepackage[dvipsnames]{xcolor}
\usepackage{amsmath}
\usepackage{amsfonts}
\usepackage{amsthm}
\usepackage{siunitx}
\DeclareSIUnit\year{yr}
\DeclareSIUnit\foot{ft}
\DeclareSIUnit\litre{\liter}

\usepackage{skull}

\usepackage{pgfplots}
\usepgfplotslibrary{polar}
\pgfplotsset{compat=1.11}
\usepgfplotslibrary{statistics}
\usepackage{graphicx}
\usepackage{sidecap}
\sidecaptionvpos{figure}{c}
\usepackage{float}
\usepackage{gensymb}
\usepackage{tkz-euclide}
\usetkzobj{all}
\usepackage{commath}
\usepackage{hyperref}
\usepackage{enumitem}
\usepackage{wasysym}
\usepackage{multicol}
\usepackage{mathtools}
\usepackage{tcolorbox}
\usepackage{tabularx}
\usepackage[version=4]{mhchem}
\usepackage{changepage}
\usepackage{listings}
\lstset{basicstyle=\ttfamily\linespread{0.8}\small}

\renewcommand*{\thefootnote}{\fnsymbol{footnote}}

\newtheorem*{thm}{Theorem}
\newtheorem*{iden}{Identity}
\newtheorem*{lemma}{Lemma}
\newtheorem{obs}{Observation}
\theoremstyle{definition}
\newtheorem*{defn}{Definition}
\newtheorem*{ex}{Example}
\newtheorem{con}{Construction}
\newtheorem*{alg}{Algorithm}

\newtheoremstyle{break}
  {\topsep}{\topsep}%
  {\itshape}{}%
  {\bfseries}{}%
  {\newline}{}%
\theoremstyle{break}
\newtheorem*{bthm}{Theorem}

% russian integral
\usepackage{scalerel}
\DeclareMathOperator*{\rint}{\scalerel*{\rotatebox{17}{$\!\int\!$}}{\int}}

% \DeclareMathOperator*{\rint}{\int}

\pgfplotsset{vasymptote/.style={
    before end axis/.append code={
        \draw[densely dashed] ({rel axis cs:0,0} -| {axis cs:#1,0})
        -- ({rel axis cs:0,1} -| {axis cs:#1,0});
    }
}}

% \pointsinrightmargin
\boxedpoints
\pointname{}

\newcommand{\questioA}{\question[\texttt{\textbf{\color{Cerulean} A}}]}
\newcommand{\questioM}{\question[\texttt{\textbf{\color{PineGreen} M}}]}
\newcommand{\questioE}{\question[\texttt{\textbf{\color{WildStrawberry} E}}]}
\newcommand{\questioS}{\question[\texttt{\textbf{\color{Goldenrod} S}}]}
\newcommand{\questioO}{\question[\texttt{\textbf{\color{BurntOrange} O}}]}

\newcommand{\parA}{\part[\texttt{\textbf{\color{Cerulean} A}}]}
\newcommand{\parM}{\part[\texttt{\textbf{\color{PineGreen} M}}]}
\newcommand{\parE}{\part[\texttt{\textbf{\color{WildStrawberry} E}}]}
\newcommand{\parS}{\part[\texttt{\textbf{\color{Goldenrod} S}}]}
\newcommand{\parO}{\part[\texttt{\textbf{\color{BurntOrange} O}}]}

\newcommand{\subparA}{\subpart[\texttt{\textbf{\color{Cerulean} A}}]}
\newcommand{\subparM}{\subpart[\texttt{\textbf{\color{PineGreen} M}}]}
\newcommand{\subparE}{\subpart[\texttt{\textbf{\color{WildStrawberry} E}}]}
\newcommand{\subparS}{\subpart[\texttt{\textbf{\color{Goldenrod} S}}]}
\newcommand{\subparO}{\subpart[\texttt{\textbf{\color{BurntOrange} O}}]}

\newcommand{\mainHeader}[2]{\section*{NCEA Level 2 Mathematics\\#1. #2}}
\newcommand{\mainHeaderHw}[2]{\section*{NCEA Level 2 Mathematics (Homework)\\#1. #2}}
\newcommand{\seealso}[1]{\begin{center}\emph{See also #1.}\end{center}}
\newcommand{\drills}[1]{\begin{center}\emph{Drill problems: #1.}\end{center}}
\newcommand{\basedon}[1]{\begin{center}\emph{Notes largely based on #1.}\end{center}}

\begin{document}

\mainHeaderDiff{14}{Differentiation Revision}
Halfway there! Let's do a bit of revision.
\subsection*{Questions}
\begin{questions}
  \questioA True or False:
    \begin{parts}
      \part If a function $ f $ is continuous around a point, then it is differentiable at that point.
      \part If a function $ f $ is differentiable around a point, then it is continuous at that point.
      \part If $ f $ and $ g $ are differentiable, then $ \od{}{x} [f(x) + g(x)] = f'(x) + g'(x) $.
      \part If $ f $ and $ g $ are differentiable, then $ \od{}{x} [f(x)g(x)] = f'(x) g'(x) $.
      \part $ \od{}{x} (x + 3)^2 = 2(x + 3) $.
      \part $ \od{}{x} (x^2 + 3)^2 = 2(x^2 + 3) $.
      \part $ \od{}{x} \tan^2 x = \od{}{x} \sec^2 x $.
      \part $ \od[2]{y}{x} = \left( \od{y}{x} \right)^2 $.
      \part If $ A = \pi r^2 $, then $ \dif{A} \pi = 0 $.
    \end{parts}
  \questioA Find $ \od{y}{t} $ in each case.
    \begin{multicols}{2}
    \begin{parts}
      \part $ y = t^2 + 3t $
      \part $ y = \frac{4 - t}{3 + t} $
      \part $ y = (t^4 - 3t^2 + 5)^2 $
      \part $ y = (t + 1)^{2017} $
      \part $ y = \frac{3t + 4t^2}{\sqrt{t}} $
      \part $ y = \sin 2t $
      \part $ y = \sin^2 t $
      \part $ y = \sin t^2 $
      \part $ y = \cos \tan t $
      \part $ y = (27t + 3)^{2017} (t^2 - \sqrt{t})^{2020} $
      \part $ y = \left( t + \frac{1}{t^2} \right)^{\sqrt{7}} $
    \end{parts}
    \end{multicols}
  \questioM Find the equation of the tangent line to the curve $ \sqrt{1 + 4\sin x} $ at the point $ (0, 1) $.
  \questioA Find $ y'' $ in each case:
    \begin{parts}
      \part $ y = 3x^3 + 2x + \sqrt{2x} + \frac{1}{x^2} $
      \part $ y = e^{2x} $
      \part $ y = \sqrt{4t + 1} $
      \part $ y = 4 \sin^2 x $
    \end{parts}
  \questioM Find the $ n$th derivative of $ e^{2x} $ (where $ n $ is a natural (counting) number).
  \questioE The height of a projectile after $ t $ seconds can be modelled by $ h = 3t(t - 10) $. At what time
            is the height of the projectile at a maximum? Use the second-derivative test to prove that you have
            found a maximum.
  \clearpage
  \questioM Find $ f'(x) $ if:
    \begin{parts}
      \part $ f(x) = \frac{1}{3} \ln \abs{\frac{x - 1}{x + 2}} + 3 $
      \part $ f(x) = 2\ln \abs{\frac{x - 2}{x}} + \frac{2}{x} + 7 $
    \end{parts}
  \questioM In each case, find $ y' $ in terms of $ x $ and $ y $:
    \begin{parts}
      \part $ x^2 + y^2 = 4 $
      \part $ x^2 + 4xy + y^2 = 13 $
      \part $ xy^4 + x^2y = x + 3y $
      \part $ x^2 \cos y + \sin 2y = xy $
    \end{parts}
  \questioM By differentiating the double-angle formula for cosine,
            \begin{displaymath}
              \cos 2x = \cos^2 x - \sin^2 x,
            \end{displaymath}
            obtain the double-angle formula for the sine function.
  \questioM Find $ f' $ in terms of $ g' $ if $ f(x) = x^2 g(x) $.
  \questioM The volume of a cube is increasing at a rate of \SI{10}{\centi\metre\cubed\per\minute}. How fast is the surface
            area increasing when the length of an edge is \SI{30}{\centi\metre}?
  \questioM The volume of a right circular cone is $ V = \frac{1}{3} \pi r^2 h $.
    \begin{parts}
      \part Find the rate of change of volume with respect to height if the radius is constant.
      \part Find the rate of change of volume with respect to radius if the height is constant.
    \end{parts}
  \questioM A particle moves along a horizontal line such that its coordinate at time $ t $ is $ x = \sqrt{b^2 + c^2 t^2} $ ($ t \geq 0 $),
            where $ b $ and $ c $ are positive constants. Find its velocity and acceleration functions.
  \questioM From first principles, show that:
    \begin{parts}
      \part $ \od{}{x} x^2 = 2x $
      \part $ \od{}{x} [2x^3 + 2x] = 6x^2 + 2 $
    \end{parts}
  \questioE Find the derivative of $ f(x) = \frac{4}{\sqrt{1 - x}} $ from first principles.
  \questioE Each limit represents the derivative of a function $ f $ at a point $ a $. Identify each function and point.
    \begin{parts}
      \part $ \lim_{h \to 0} \frac{\cos(\pi + h) + 1}{h} $
      \part $ \lim_{t \to 1} \frac{t^4 + t - 2}{t - 1} $
    \end{parts}
  \questioM Find the best linear approximation to $ f(x) = \sqrt{25 - x^2} $ near $ x = 3 $.
  \questioM A balloon is rising at a constant speed of \SI{2}{\metre\per\second}. A girl is cycling along a straight
            road at a speed of \SI{5}{\metre\per\second}. When she passes under the balloon, it is \SI{15}{\metre} above
            her. How fast is the distance between the person and the balloon increasing 3 seconds later?
  \questioE On a straight shoreline there is a tree and exactly opposite it, \SI{100}{\metre} away in the sea, stands a lighthouse.
            A strong and thin spotlight on its top revolves at the rate of one revolution per 4 seconds, its light
            creating a running light spot on the shore. You stand on the shore \SI{100}{\metre} from the tree. How fast does this
            spot move when it goes past you?
  \questioM Find $ \od{y}{x} $ in terms of $ t $ for the following parametrically-defined curves.
    \begin{parts}
      \part $ t \mapsto (1 + e^{2t}, e^t) $
      \part $ x = \tan t $, $ y = \sec 2t $
      \part $ x = \frac{t^2 - 10}{t^2 + 1} $, $ y = tx $
    \end{parts}
  \questioE Find the lowest point on the curve $ \gamma : t \mapsto (t^3 - 3t, t^2 + t + 1) $. Prove you have found a minimum.
  \questioM Find the acceleration of a particle at time $ t $ if its displacement from the origin at time $ t $ is $ -t^6 + 5t^4 + \sin t $.
  \questioE A piece of wire \SI{10}{\metre} long is cut into two pieces; one is bent into a square and the other into a circle. Where
            should the wire be cut to ensure the total area of the enclosed shapes is (a) a minimum and (b) a maximum?
  \questioM A cone is made by cutting a sector out of a circle of paper of radius $ R $ and gluing together the edges of the cut. Find
            the maximum possible volume of the cone.
  \questioE Find the dimensions of the rectangle of largest area that can be inscribed in a circle of radius $ r $.
  \questioE At which points does the `bouncing wagon' curve $ 2y^3 + y^2 - y^5 = x^4 - 2x^3 + x^2 $ have horizontal tangents?
  \questioE Salt forms a cone as it falls from a conveyor belt. The slant of the cone forms an angle of \SI{30}{\degree} with
            the horizontal. The belt delivers salt at a rate of \SI{2}{\metre\per\cubed} per minute. When the radius of the cone
            is ten metres, what is the rate of increase of the slant height (measured along the surface of the cone)?
  \questioS Suppose we take a circle of radius $ r $, and inscribe within it a triangle such that each corner of the triangle is
            located on the circle. What is the maximum possible area of the triangle?
  \questioE Take derivatives of the following functions with respect to $ x $:
    \begin{parts}
      \part $ f(x) = 2^x $
      \part $ g(x) = \log_{\log x} x $
    \end{parts}
  \questioS Scholarship 2017: For $ y = x^{\left(x^x\right)} $, find $ \od{y}{x} $ when $ x = 2 $.
  \questioS Scholarship 2015 (adapted): A car is driving along a road shaped like a parabola at night. The parabola has a vertex at the origin,
            and the car starts at a point 100 m west and 100 m north of the origin.
    \begin{parts}
      \part Write an equation modelling the road as a parabola.
      \part Find the general equation for the tangent line to the parabola at some point $ (x_0, y_0) $, and substitute into it
            the parabola equation to obtain an equation only in $ x $, $ x_0 $, and $ y_0 $.
      \part Suppose there is a statue of the Roman emperor Augustus located 100 m east and 50 m north of the origin. Write the
            equation for the tangent line of the parabola passing through the statue (so that it only depends on a value $ x $
            on the parabola).
      \part Hence find the single point $ (x,y) $ on the road where the headlights of the car illuminate the statue.
    \end{parts}

  \questioO (Difficult) Find the two points on the curve $ y = x^4 - 2x^2 - x $ that have a common tangent line.
  \questioS (Difficult) A cone of radius $ r $ centimetres and height $ h $ centimetres is lowered point first
            at a rate of \SI{1}{\centi\metre\squared} into a cylinder of radius $ R $ centimetres that is partially filled with
            water. How fast is the water level rising at the instant that the cone is fully submerged?
\end{questions}

\end{document}
