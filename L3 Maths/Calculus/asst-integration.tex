\documentclass[answers]{exam}
\usepackage[utf8]{inputenc}

\usepackage[parfill]{parskip}
\usepackage[dvipsnames]{xcolor}
\usepackage{amsmath}
\usepackage{amsfonts}
\usepackage{amsthm}
\usepackage{microtype}
\usepackage{siunitx}
\DeclareSIUnit\year{yr}
\usepackage{pgfplots}
\usepackage{graphicx}
\usepackage{sidecap}
\sidecaptionvpos{figure}{c}
\usepackage{float}
\usepackage{gensymb}
\usepackage{tkz-euclide}
\usetkzobj{all}
\usepackage{commath}
\usepackage{hyperref}
\usepackage{enumitem}
\usepackage{wasysym}
\usepackage[version=4]{mhchem}

\renewcommand*{\thefootnote}{\fnsymbol{footnote}}

\newtheorem*{thm}{Theorem}
\newtheorem*{iden}{Identity}
\newtheorem*{lemma}{Lemma}
\theoremstyle{definition}
\newtheorem*{defn}{Definition}
\newtheorem*{ex}{Example}

\newtheoremstyle{break}
  {\topsep}{\topsep}%
  {\itshape}{}%
  {\bfseries}{}%
  {\newline}{}%
\theoremstyle{break}
\newtheorem*{bthm}{Theorem}

% russian integral
\usepackage{scalerel}
\DeclareMathOperator*{\rint}{\scalerel*{\rotatebox{17}{$\!\int\!$}}{\int}}

\pgfplotsset{vasymptote/.style={
    before end axis/.append code={
        \draw[densely dashed] ({rel axis cs:0,0} -| {axis cs:#1,0})
        -- ({rel axis cs:0,1} -| {axis cs:#1,0});
    }
}}

% \qformat{Question \thequestion: \thequestiontitle\hfill}

\begin{document}

\section*{NCEA Level 3 Calculus\\Integration Assignment}
\begin{questions}
  \question Compute the indefinite integrals.
    \begin{parts}
      \part[2] $ \displaystyle\rint x \cdot \cos (x^2) \cdot \sin(\sin x^2) \dif{x} $
      \part[2] $ \displaystyle\rint \pi t \csc^2 (2t^2) \dif{t} $
      \part[2] $ \displaystyle\rint \frac{\sqrt{j} + 3j^5 + 3j^6 + 3j^7 + 2}{2j^7} \dif{j} $
      \part[2] $ \displaystyle\rint \frac{\ln t^2}{t} \dif{t} $
    \end{parts}
  \begin{solution}
    \paragraph{(a)} Let $ u = \sin x^2 $. Then $ \dif{u} = 2x \cos x^2 \dif{x} $ and our integral
    becomes $ \rint \frac{1}{2} \sin u \dif{u} = -\frac{1}{2} \cos u + C = -\frac{1}{2} \cos \sin x^2 + C $. This
    could also be done by two substitutions, $ u = x^2 $ and then $ v = \sin u $.
    \paragraph{(b)} Let $ u = 2t^2 $. Then $ \dif{u} = 4t \dif{t} $ and our integral becomes $ \rint \frac{\pi}{4} \csc^2(u) \dif{u}
    = -\frac{\pi}{4} \cot u + C = -\frac{\pi}{4} \cot (2x^2) + C $.
    \paragraph{(c)} We simplify to find that our integral becomes $ \frac{1}{2} \rint j^{-6.5} + 3j^{-2} + 3j^{-1} + 3 + 2j^{-7} \dif{j}
    = \frac{1}{2} \left( -\frac{j^{-5.5}}{5.5} - 3j^{-1} + 3\ln\abs{j} + 3j - \frac{1j^{-6}}{3} \right) + C
    = \frac{1}{2} \left( -\frac{1}{5.5\sqrt{j^{11}}} - \frac{3}{j} + 3\ln\abs{j} + 3j - \frac{1}{3j^6} \right) + C $.
    \paragraph{(d)} First note that $ \ln t^2 = 2\ln t $. Then let $ u = \ln t $ so $ \dif{u} = \frac{1}{t} \dif{t} $ and the
    integral becomes $ \rint 2u \dif{u} = u^2 + C = (\ln t)^2 + C $.
  \end{solution}

  \question We will prove the identity $ 1 + \tan^2 x = \sec^2 $.
    \begin{parts}
      \part[1] Calculate $ \od{}{x} \sec^2 x $.
      \part[4] Using the substitution $ u = \tan x $, integrate your answer to part (a). Conclude that $ \sec^2 x = \tan^2 x + C $ for some constant $ C $.
      \part[2] Find the value of $ C $ and conclude the identity above.
    \end{parts}
  \begin{solution}
    \paragraph{(a)} Using the chain rule the required derivative is $ 2\sec x \cdot \sec x \tan x = 2 \sec^2 x \tan x $.
    \paragraph{(b)} We find $ \rint 2 \sec^2 x \tan x \dif{x} $. Let $ u = \tan x $. Then $ \dif{u} = \sec^2 x $,
    and our integral becomes $ \rint 2u \dif{u} = u^2 + C = \tan^2 x + C $. But from (a) we have $ \rint 2 \sec^2 x \tan x = \sec^2 x + C' $.
    Hence $ \tan^2 x + C = \sec^2 x + C' $ and the two differ only by a constant.
    \paragraph{(c)} The identity must hold for all $ x $, and so we set $ x = 0 $. Then $ \sec^2 0 = \tan^2 0 + C $ and $ C = 1 $. Hence
    we have $ \sec^2 x = \tan^2 x + 1 $ as expected.
  \end{solution}

  \question[2] Find the area bounded by the curve $ y = 3x^2 + x - 2 $ and the $ x-$axis.
  \begin{solution}
    The curve can be factored as $ y = (3x - 2)(x + 1) $ and so the $ x-$intercepts are $ x = -1 $ and $ x = \frac{2}{3} $.
    We must therefore find $ \rint^{2/3}_{-1} 3x^2 + x - 2 \dif{x} = \eval{x^3 + 0.5x^2 - 2x}^{2/3}_{-1} = -\frac{125}{54} \approx -2.315 $.
  \end{solution}

  \question[2] If $ \rint^2_{-1} 3f(x) \dif{x} = 9 $ and $ \rint_{-1}^{3} f(x) \dif{x} = 1 $, find $ \rint_2^3 f(x) \dif{x} $.
  \begin{solution}
    $ \rint_2^3 f(x) \dif{x} = \rint_{-1}^{3} f(x) \dif{x} - \frac{1}{3} \rint^2_{-1} 3f(x) \dif{x} = 1 - 3 = -2 $.
  \end{solution}

  \question
    \begin{parts}
      \part[3] Compute $ \rint_0^R 2\pi r \dif{r} $. Interpret your answer (you may wish to draw a diagram).
      \part[2] Find the volume of a sphere of radius $ R $ by integration; the surface area of a sphere of radius $ r $ is given by $ SA = 4\pi r^2 $.
    \end{parts}
  \begin{solution}
    \paragraph{(a)}
    $ \rint_0^R 2\pi r \dif{r} = \eval{\pi r^2}_0^R = \pi R^2 $, which is the area of a circle of radius $ R $. This makes sense as we
    are summing up all the circumferi of circles radiating out from the centre of our larger circle: we expect to get the full area.
    \paragraph{(b)} Same reasoning: $ \rint^R_0 4\pi r^2 \dif{r} = \eval{4/3 \pi r^3}_0^R = 4/3 \pi R^3 $.
  \end{solution}

  \question[5] Find $ y(\sqrt{\pi/2}) $ if $ y(0) = 0 $ and
    \begin{displaymath}
      \od{y}{x} = x\sin(x^2) \cot y.
    \end{displaymath}
  \begin{solution}
    Separating variables, we have $ \rint \tan y \dif{y} = \rint x\sin x^2 \dif{x} $. The RHS is simply $ -\frac{1}{2} \cos x^2 + C$ (substitute $ x^2 $ out),
    and we can rewrite the LHS as $ \tan y = \frac{\sin y}{\cos y} $ so (using the substitution $ \cos y $); hence overall we have got
    $ -\ln \abs{\cos y} = -\frac{1}{2} \cos x^2 + C $. But $ y(0) = 0 $ so $ -\ln 1 = -\frac{1}{2} + C $ and $ \ln{1} = 0 $ so $ C = 0.5 $
    and $ \abs{\cos y} = e^{\frac{1}{2} \cos x^2 - \frac{1}{2}} $. At $ x = \sqrt{\pi/2} $, the RHS becomes $ e^{-0.5} $ and so $ \cos y = \pm e^{-0.5} $;
    therefore we have two solutions for $ y $: $ y = \cos^{-1} (\pm e^{-0.5}) $.
  \end{solution}
\end{questions}
\end{document}
