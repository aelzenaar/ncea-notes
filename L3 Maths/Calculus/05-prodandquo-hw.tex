\documentclass{exam}
\usepackage[utf8]{inputenc}
\usepackage{lmodern}
\usepackage{microtype}

% \usepackage[parfill]{parskip}
\usepackage[dvipsnames]{xcolor}
\usepackage{amsmath}
\usepackage{amsfonts}
\usepackage{amsthm}
\usepackage{siunitx}
\DeclareSIUnit\year{yr}
\DeclareSIUnit\foot{ft}
\DeclareSIUnit\litre{\liter}

\usepackage{skull}

\usepackage{pgfplots}
\usepgfplotslibrary{polar}
\pgfplotsset{compat=1.11}
\usepackage{graphicx}
\usepackage{sidecap}
\sidecaptionvpos{figure}{c}
\usepackage{float}
\usepackage{gensymb}
\usepackage{tkz-euclide}
\usetkzobj{all}
\usepackage{commath}
\usepackage{hyperref}
\usepackage{enumitem}
\usepackage{wasysym}
\usepackage{multicol}
\usepackage{mathtools}
\usepackage{tcolorbox}
\usepackage{tabularx}
\usepackage[version=4]{mhchem}
\usepackage{changepage}
\usepackage{listings}
\lstset{basicstyle=\ttfamily\linespread{0.8}\small}

\renewcommand*{\thefootnote}{\fnsymbol{footnote}}

\newtheorem*{thm}{Theorem}
\newtheorem*{iden}{Identity}
\newtheorem*{lemma}{Lemma}
\newtheorem{obs}{Observation}
\theoremstyle{definition}
\newtheorem*{defn}{Definition}
\newtheorem*{ex}{Example}
\newtheorem{con}{Construction}
\newtheorem*{alg}{Algorithm}

\newtheoremstyle{break}
  {\topsep}{\topsep}%
  {\itshape}{}%
  {\bfseries}{}%
  {\newline}{}%
\theoremstyle{break}
\newtheorem*{bthm}{Theorem}

% russian integral
\usepackage{scalerel}
\DeclareMathOperator*{\rint}{\scalerel*{\rotatebox{17}{$\!\int\!$}}{\int}}

% \DeclareMathOperator*{\rint}{\int}

\pgfplotsset{vasymptote/.style={
    before end axis/.append code={
        \draw[densely dashed] ({rel axis cs:0,0} -| {axis cs:#1,0})
        -- ({rel axis cs:0,1} -| {axis cs:#1,0});
    }
}}

% \pointsinrightmargin
\boxedpoints
\pointname{}

\newcommand{\questioA}{\question[\texttt{\textbf{\color{Cerulean} A}}]}
\newcommand{\questioM}{\question[\texttt{\textbf{\color{PineGreen} M}}]}
\newcommand{\questioE}{\question[\texttt{\textbf{\color{WildStrawberry} E}}]}
\newcommand{\questioS}{\question[\texttt{\textbf{\color{Goldenrod} S}}]}
\newcommand{\questioO}{\question[\texttt{\textbf{\color{BurntOrange} O}}]}

\newcommand{\parA}{\part[\texttt{\textbf{\color{Cerulean} A}}]}
\newcommand{\parM}{\part[\texttt{\textbf{\color{PineGreen} M}}]}
\newcommand{\parE}{\part[\texttt{\textbf{\color{WildStrawberry} E}}]}
\newcommand{\parS}{\part[\texttt{\textbf{\color{Goldenrod} S}}]}
\newcommand{\parO}{\part[\texttt{\textbf{\color{BurntOrange} O}}]}

\newcommand{\subparA}{\subpart[\texttt{\textbf{\color{Cerulean} A}}]}
\newcommand{\subparM}{\subpart[\texttt{\textbf{\color{PineGreen} M}}]}
\newcommand{\subparE}{\subpart[\texttt{\textbf{\color{WildStrawberry} E}}]}
\newcommand{\subparS}{\subpart[\texttt{\textbf{\color{Goldenrod} S}}]}
\newcommand{\subparO}{\subpart[\texttt{\textbf{\color{BurntOrange} O}}]}

\newcommand{\mainHeader}[2]{\section*{NCEA Level 2 Mathematics\\#1. #2}}
\newcommand{\mainHeaderHw}[2]{\section*{NCEA Level 2 Mathematics (Homework)\\#1. #2}}

\begin{document}

\mainHeaderDiffHw{5}{The Product and Quotient Rules}
\subsection*{Reading}
Even next year, when you will be expected to know the derivative rules, it is not necessary to remember the quotient rule because
it is just a special case of the product rule. However, here is a little rhyme:

\begin{center}\itshape
  If it's the quotient rule you wish to know,\\
  It's low-de-high less high-de-low.\\
  Then draw the line and down below,\\
  Denominator squared will go.\footnote{Quoted in \textit{Mathematical Apocrypha} by Steven G. Krantz (p.36).}
\end{center}

\subsection*{Questions}
\begin{questions}
  \question Find the derivatives:
    \begin{parts}
      \part $ \od{y}{x} $ if $ y = \sin x \ln x $.
      \part $ \od{y}{x} $ if $ y = x \sec kx $ ($ k $ constant).
      \part $ \od{f}{\theta} $ if $ f(\theta) = \frac{\cos \pi \theta}{\sin \pi \theta + \cos \pi \theta} $.
      \part $ \od{y}{t} $ if $ y = \cos^4 (\sin^3 t) $.
    \end{parts}
  \question The force $ F $ acting on a body with mass $ m $ and velocity $ v $ is the rate
            of change of momentum, $ F = \od{}{t} [mv] $. If $ m $ is constant, this becomes $ F = ma $,
            where $ a = \od{v}{t} $ is the acceleration of the body. However, due to relativistic
            effects, the mass of a particle varies with $ v $ as
            \begin{displaymath}
              m = \frac{m_0}{\sqrt{1 - \dfrac{v^2}{c^2}}},
            \end{displaymath}
            where $ m_0 $ is the rest mass of the body and $ c $ is the speed of light. Show that
            \begin{displaymath}
              F = \frac{m_0 a}{\left(1 - \dfrac{v^2}{c^2}\right)^{\frac{3}{2}}}.
            \end{displaymath}
  \question Recall that if $ \theta $ is given in degrees, then $ \frac{\pi \theta}{180} $ is the equivalent angle in radians.
            Find the derivative of $ \sin \theta $ if $ \theta $ is given in degrees.
\end{questions}
\end{document}
