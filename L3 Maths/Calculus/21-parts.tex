\documentclass{exam}
\usepackage[utf8]{inputenc}
\usepackage{lmodern}
\usepackage{microtype}

% \usepackage[parfill]{parskip}
\usepackage[dvipsnames]{xcolor}
\usepackage{amsmath}
\usepackage{amsfonts}
\usepackage{amsthm}
\usepackage{siunitx}
\DeclareSIUnit\year{yr}
\DeclareSIUnit\foot{ft}
\DeclareSIUnit\litre{\liter}

\usepackage{skull}

\usepackage{pgfplots}
\usepgfplotslibrary{polar}
\pgfplotsset{compat=1.11}
\usepgfplotslibrary{statistics}
\usepackage{graphicx}
\usepackage{sidecap}
\sidecaptionvpos{figure}{c}
\usepackage{float}
\usepackage{gensymb}
\usepackage{tkz-euclide}
\usetkzobj{all}
\usepackage{commath}
\usepackage{hyperref}
\usepackage{enumitem}
\usepackage{wasysym}
\usepackage{multicol}
\usepackage{mathtools}
\usepackage{tcolorbox}
\usepackage{tabularx}
\usepackage[version=4]{mhchem}
\usepackage{changepage}
\usepackage{listings}
\lstset{basicstyle=\ttfamily\linespread{0.8}\small}

\renewcommand*{\thefootnote}{\fnsymbol{footnote}}

\newtheorem*{thm}{Theorem}
\newtheorem*{iden}{Identity}
\newtheorem*{lemma}{Lemma}
\newtheorem{obs}{Observation}
\theoremstyle{definition}
\newtheorem*{defn}{Definition}
\newtheorem*{ex}{Example}
\newtheorem{con}{Construction}
\newtheorem*{alg}{Algorithm}

\newtheoremstyle{break}
  {\topsep}{\topsep}%
  {\itshape}{}%
  {\bfseries}{}%
  {\newline}{}%
\theoremstyle{break}
\newtheorem*{bthm}{Theorem}

% russian integral
\usepackage{scalerel}
\DeclareMathOperator*{\rint}{\scalerel*{\rotatebox{17}{$\!\int\!$}}{\int}}

% \DeclareMathOperator*{\rint}{\int}

\pgfplotsset{vasymptote/.style={
    before end axis/.append code={
        \draw[densely dashed] ({rel axis cs:0,0} -| {axis cs:#1,0})
        -- ({rel axis cs:0,1} -| {axis cs:#1,0});
    }
}}

% \pointsinrightmargin
\boxedpoints
\pointname{}

\newcommand{\questioA}{\question[\texttt{\textbf{\color{Cerulean} A}}]}
\newcommand{\questioM}{\question[\texttt{\textbf{\color{PineGreen} M}}]}
\newcommand{\questioE}{\question[\texttt{\textbf{\color{WildStrawberry} E}}]}
\newcommand{\questioS}{\question[\texttt{\textbf{\color{Goldenrod} S}}]}
\newcommand{\questioO}{\question[\texttt{\textbf{\color{BurntOrange} O}}]}

\newcommand{\parA}{\part[\texttt{\textbf{\color{Cerulean} A}}]}
\newcommand{\parM}{\part[\texttt{\textbf{\color{PineGreen} M}}]}
\newcommand{\parE}{\part[\texttt{\textbf{\color{WildStrawberry} E}}]}
\newcommand{\parS}{\part[\texttt{\textbf{\color{Goldenrod} S}}]}
\newcommand{\parO}{\part[\texttt{\textbf{\color{BurntOrange} O}}]}

\newcommand{\subparA}{\subpart[\texttt{\textbf{\color{Cerulean} A}}]}
\newcommand{\subparM}{\subpart[\texttt{\textbf{\color{PineGreen} M}}]}
\newcommand{\subparE}{\subpart[\texttt{\textbf{\color{WildStrawberry} E}}]}
\newcommand{\subparS}{\subpart[\texttt{\textbf{\color{Goldenrod} S}}]}
\newcommand{\subparO}{\subpart[\texttt{\textbf{\color{BurntOrange} O}}]}

\newcommand{\mainHeader}[2]{\section*{NCEA Level 2 Mathematics\\#1. #2}}
\newcommand{\mainHeaderHw}[2]{\section*{NCEA Level 2 Mathematics (Homework)\\#1. #2}}
\newcommand{\seealso}[1]{\begin{center}\emph{See also #1.}\end{center}}
\newcommand{\drills}[1]{\begin{center}\emph{Drill problems: #1.}\end{center}}
\newcommand{\basedon}[1]{\begin{center}\emph{Notes largely based on #1.}\end{center}}

\begin{document}

\mainHeaderIntg{21}{Integration by Parts}
The substitution rule is the inverse of the chain rule; similarly, there is an inverse of the product rule.
\begin{align*}
       &\dod{}{x} [f(x) g(x)] = f'(x) g(x) + f(x) g'(x)\\
  \iff &\rint f'(x) g(x) + f(x) g'(x) \dif{x} = f(x) g(x)\\
  \iff &\rint f(x) g'(x) \dif{x} = f(x)g(x) - \rint f'(x) g(x) \dif{x}
\end{align*}

Mnemonically,
\begin{displaymath}
  \rint u \dif{v} = uv - \rint v \dif{u}.
\end{displaymath}

\begin{ex}
Consider $ \rint x \sin x \dif{x} $, which does not yield to any obvious change of variable. Let $ u = x $, and
let $ \dif{v} = \sin x \dif{x} $. So $ \dif{u} = \dif{x} $, and $ v = -\cos x $. Hence:
\begin{displaymath}
  \rint x \sin x \dif{x} = -x\cos x + \rint \cos x \dif{x} = -x\cos x + \sin x + C,
\end{displaymath}
where $ C $ is an arbitrary constant. Check that $ (-x\cos x + \sin x)' = x\sin x $.
\end{ex}

The aim is to end up with an easier integral than the one that was started with. A good choice for $ u $ is usually (in descending
order of usefulness):
\begin{enumerate}
  \item Logarithms
  \item Powers of $ x $
  \item Exponentials
  \item Trig functions
\end{enumerate}

\subsection*{Questions}
\begin{questions}
  \questioS Compute the following indefinite integrals.
    \begin{parts}
      \part $ \rint x e^x \dif{x} $
      \part $ \rint x^2 e^{2x} \dif{x} $
      \part $ \rint \ln x \dif{x} $
      \part $ \rint p^5 \ln p \dif{p} $
      \part $ \rint t^3 e^{-t^2} \dif{t} $
      \part $ \rint \sin \ln y \dif{y} $
      \part $ \rint x \tan^2 x \dif{x} $
    \end{parts}
  \questioS Prove that
    \begin{displaymath}
      \rint \cos^n x \dif{x} = \frac{1}{n} \sin x \cos^{(n - 1)} x + \frac{n-1}{n} \rint \cos^{(n - 2)} x \dif{x}
    \end{displaymath}
  \questioS If $ I_n = \rint^{n}_0 x^n e^x \dif{x} $, write down an explicit general formula for $ I_n $.
  \questioS Evaluate $ \rint (\ln x)^2 \dif{x} $.
  \questioS Compute $ \rint^\lambda_0 te^{-\lambda t} \dif{t} $.
  \questioS Suppose that $ f(1) = 2 $, $ f(4) = 7 $, $ f'(1) = 5 $, and $ f'(4) = 3 $. Evaluate $ \rint^4_1 xf''(x) \dif{x} $.
  \questioS A particle moving in one dimension has a velocity function $ v(t) = t^2 e^{-t} $ (where $ t $ is in seconds). What is its displacement from its
            starting position after three minutes?
  \questioS Find the area bounded by $ y = x^2 \ln x $ and $ y = 4\ln x $
  \question Scholarship 2012:
    \begin{parts}
      \parE Find $ \od{}{x}[x \cos x] $ and use this result to find $ \rint x \sin x \dif{x} $.
      \parS Hence find the value of $ \rint^{n\pi}_0 x \sin x \dif{x} $ for integer values of $ n $.
    \end{parts}
  \questioO Scholarship 2016:
    \begin{parts}
      \part A function $ f(x) $, where $ x $ is a real number, is defined implicitly by the formula
            \begin{displaymath}
              f(x) = x - \int^{\pi/2}_0 f(x) \sin(x) \dif{x}.
            \end{displaymath}
            Find the explicit expression for $ f(x) $ in simplest form.
      \part A curve passing through the point $ (1,1) $ has the property that at each point $ (x,y) $ on the curve,
            the gradient of the curve is $ x - 2y $; that is, $ \od{y}{x} = x - 2y $.
        \begin{subparts}
          \subpart Show that $ \od{}{x} e^{2x} y = xe^{2x} $.
          \subpart Hence, or otherwise, find the equation of the curve.
        \end{subparts}
    \end{parts}
  \questioS It is well known that
       \begin{displaymath}
         \rint^{\infty}_{-\infty} e^{-x^2} \dif{x} = \sqrt{\pi}.
       \end{displaymath}
       Using this result, show that
       \begin{displaymath}
         \rint^{\infty}_{-\infty} x^2 e^{-x^2} \dif{x} = \frac{\sqrt{\pi}}{2}.
       \end{displaymath}
  \questioO Find $ I = \int e^x \cos x \dif{x} $.
  \questioS Recall that $ \od{}{x} \tan^{-1} x = \frac{1}{1 + x^2} $. Find $ \rint \tan^{-1} x \dif{x} $.
  \questioS We integrate $ \rint 1/x \dif{x} $ by parts:
            \begin{align*}
              \rint \frac{1}{x} \dif{x} = \frac{1}{x} \cdot x - \rint -\frac{1}{x^2} \cdot x \dif{x} = 1 + \rint \frac{1}{x} \dif{x}
            \end{align*}
            Cancelling the indefinite integral from both sides, we have $ 0 = 1 $. Explain.
\end{questions}
\end{document}
