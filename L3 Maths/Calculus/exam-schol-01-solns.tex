\documentclass[a4paper]{report}

\usepackage{commath}
\usepackage{siunitx}
\usepackage{hhline}

% russian integral
\usepackage{scalerel}
\DeclareMathOperator*{\rint}{\scalerel*{\rotatebox{17}{$\!\int\!$}}{\int}}

\title{Solutions to L3 Calculus Integration Exam 3}
\author{Alexander Elzenaar}
\date{19 November 2017}

\begin{document}

\maketitle

\section*{Question One}
\subsection*{Part (a)}
We calculate the left-hand integral:
\begin{align*}
  \lim_{n \to \infty} \rint_0^1 f_n(x) \dif{x} &= \lim_{n \to\infty} \rint^1_0 nx(1 - x^2)^n \dif{x}\\
  \intertext{Let $ u = 1 - x^2 $.}\\
                                               &= \lim_{n \to\infty} \rint^1_0 \frac{n}{2} u^n \dif{u}\\
                                               &= \lim_{n \to\infty} \eval{\left[\frac{n}{2(n+1)} u^{n + 1}\right]}_0^1\\
                                               &= \frac{1}{2}.
\end{align*}

Now, note that $ \lim_{n \to\infty} nx(1 - x^2)^n = 0 $ (since $ 1 - x^2 \leq 1 $ for the given bounds on $ x $, and $ \alpha^n $ when $ \alpha < 1 $
shrinks faster than $ n\alpha $). This can be proved using L'Hopital's Rule, or by using the binomial theorem and bounding the limigrand above.

Hence the two integrals are unequal.

\subsection*{Part (b)}
\begin{align*}
  \rint \sin^3 x \cos^4 x + \sin^4 x \cos^3 x \dif{x} &= \rint \sin^3 x \cos^4 x \dif{x} + \rint \sin^4 x \cos^3 x \dif{x}\\
                                                      &= \rint \sin x (1 - \cos^2 x) \cos^4 x \dif{x} + \rint \sin^4 x (1 - \sin^2 x)\cos x \dif{x}\\
                                                      &= \rint \sin x (\cos^4 x - \cos^6 x) \dif{x} + \rint \cos x (\sin^4 x - \sin^6 x) \dif{x}\\
                                                      &= \rint -(u^4 - u^6) \dif{u} + \rint (v^4 - v^6) \dif{v}\\
                                                      &= \frac{\sin^5 x}{5} - \frac{\sin^7}{7} - \frac{\cos^5 x}{5} + \frac{\cos 7}{7} + C.
\end{align*}
Hence, substituting the bounds, we obtain the value of the integral as approximately 0.0916.

\subsection*{Part (c)}
To simplify notation, let $ y = f(x) $ so $ \od{y}{x} = f'(x) $. We have therefore:
\begin{align*}
  \frac{y}{2} \od{x}{y} &= 3(x^3 - 2x^2 - x + 2)\\
  \rint \frac{\dif{x}}{x^3 - 2x^2 - x + 2} = \rint \frac{6}{y} \dif{y}
\end{align*}

On the left, we notice that $ x^3 - 2x^2 - x + 2 = (x-1)(x-2)(x+1) $ and so we have a partial fraction decomposition
\begin{displaymath}
  \frac{1}{x^3 - 2x^2 - x + 2} = \frac{A}{x - 1} + \frac{B}{x - 2} + \frac{C}{x + 1}.
\end{displaymath}
Hence $ 1 = (x - 2)(x + 1)A + (x - 1)(x + 1)B + (x - 1)(x - 2)C $; substituting $ x = 1 $, we find that $ A = 1/2 $; for $ x = 2 $,
we find $ B = 1/3 $; and for $ x = -1 $ we find $ C = 1/6 $. Hence the integral on the left is
\begin{displaymath}
  \rint \frac{1/2}{x - 1} + \frac{1/3}{x - 2} + \frac{1/6}{x + 1} \dif{x} = \frac{\ln(x - 1)}{2} + \frac{\ln(x - 2)}{3} + \frac{\ln(x + 1)}{6} + C.
\end{displaymath}

The integral on the right is easier, we have
\begin{displaymath}
  \rint \frac{6}{y} \dif{y} = 6\ln y + C',
\end{displaymath}
and overall
\begin{displaymath}
  \frac{\ln(x - 1)}{2} + \frac{\ln(x - 2)}{3} + \frac{\ln(x + 1)}{6} + C = 6\ln y.
\end{displaymath}

When $ x = 3 $, $ y = 1 $ so $ C = - \frac{\ln 2}{2} - \frac{\ln 1}{3} - \frac{\ln 4}{6} \approx 0.5776 $. We therefore have
\begin{displaymath}
  \frac{1}{6}\exp\left(\frac{\ln(x - 1)}{2} + \frac{\ln(x - 2)}{3} + \frac{\ln(x + 1)}{6} + 0.5776 \right) = f(x).
\end{displaymath}

\section*{Question Two}
\subsection*{Part (a)}
\paragraph{i.}
Note first that
\begin{displaymath}
  p(x) = x^3 + px + q = (x - \alpha)(x - \beta)(x - \gamma) = x^3 - (\alpha + \beta + \gamma)x^2 + (\alpha\beta + \beta\gamma + \gamma\alpha)x - \alpha\beta\gamma,
\end{displaymath}
so it immediately follows that
\begin{displaymath}
  \begin{pmatrix} 0 \\ p \\ q
  \end{pmatrix} =
  \begin{pmatrix}
    \alpha + \beta + \gamma\\
    \alpha\beta + \beta\gamma + \gamma\alpha\\
    - \alpha\beta\gamma
  \end{pmatrix}.
\end{displaymath}

From here, it is a simple exercise in expanding the two expressions in terms of $ \alpha $, $ \beta $, and $ \gamma $.

\paragraph{ii.}
Case I: $ \Delta_3 = 0 $. In this case, one of the roots is repeated.
Case II: $ \Delta_3 < 0 $. Obviously if all three roots are real, then the discriminant is positive. Suppose then, that there
is at least one root that is non-real; there must, in fact, be two (since the coefficients are real). Let the conjugate
pair of real roots be $ \alpha = a + bi $ and $ \beta = a - bi $; the third root, $ \gamma $, must be real. Hence:
\begin{align*}
  \Delta_3 &= (a + bi - a + bi)^2 (a - bi - \gamma)^2 (\gamma - a - bi)^2\\
           &= -
\end{align*}

\subsection*{Part (b)}

\end{document}
