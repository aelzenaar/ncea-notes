\documentclass{exam}
\usepackage[utf8]{inputenc}
\usepackage{lmodern}
\usepackage{microtype}

% \usepackage[parfill]{parskip}
\usepackage[dvipsnames]{xcolor}
\usepackage{amsmath}
\usepackage{amsfonts}
\usepackage{amsthm}
\usepackage{siunitx}
\DeclareSIUnit\year{yr}
\DeclareSIUnit\foot{ft}
\DeclareSIUnit\litre{\liter}

\usepackage{skull}

\usepackage{pgfplots}
\usepgfplotslibrary{polar}
\pgfplotsset{compat=1.11}
\usepgfplotslibrary{statistics}
\usepackage{graphicx}
\usepackage{sidecap}
\sidecaptionvpos{figure}{c}
\usepackage{float}
\usepackage{gensymb}
\usepackage{tkz-euclide}
\usetkzobj{all}
\usepackage{commath}
\usepackage{hyperref}
\usepackage{enumitem}
\usepackage{wasysym}
\usepackage{multicol}
\usepackage{mathtools}
\usepackage{tcolorbox}
\usepackage{tabularx}
\usepackage[version=4]{mhchem}
\usepackage{changepage}
\usepackage{listings}
\lstset{basicstyle=\ttfamily\linespread{0.8}\small}

\renewcommand*{\thefootnote}{\fnsymbol{footnote}}

\newtheorem*{thm}{Theorem}
\newtheorem*{iden}{Identity}
\newtheorem*{lemma}{Lemma}
\newtheorem{obs}{Observation}
\theoremstyle{definition}
\newtheorem*{defn}{Definition}
\newtheorem*{ex}{Example}
\newtheorem{con}{Construction}
\newtheorem*{alg}{Algorithm}

\newtheoremstyle{break}
  {\topsep}{\topsep}%
  {\itshape}{}%
  {\bfseries}{}%
  {\newline}{}%
\theoremstyle{break}
\newtheorem*{bthm}{Theorem}

% russian integral
\usepackage{scalerel}
\DeclareMathOperator*{\rint}{\scalerel*{\rotatebox{17}{$\!\int\!$}}{\int}}

% \DeclareMathOperator*{\rint}{\int}

\pgfplotsset{vasymptote/.style={
    before end axis/.append code={
        \draw[densely dashed] ({rel axis cs:0,0} -| {axis cs:#1,0})
        -- ({rel axis cs:0,1} -| {axis cs:#1,0});
    }
}}

% \pointsinrightmargin
\boxedpoints
\pointname{}

\newcommand{\questioA}{\question[\texttt{\textbf{\color{Cerulean} A}}]}
\newcommand{\questioM}{\question[\texttt{\textbf{\color{PineGreen} M}}]}
\newcommand{\questioE}{\question[\texttt{\textbf{\color{WildStrawberry} E}}]}
\newcommand{\questioS}{\question[\texttt{\textbf{\color{Goldenrod} S}}]}
\newcommand{\questioO}{\question[\texttt{\textbf{\color{BurntOrange} O}}]}

\newcommand{\parA}{\part[\texttt{\textbf{\color{Cerulean} A}}]}
\newcommand{\parM}{\part[\texttt{\textbf{\color{PineGreen} M}}]}
\newcommand{\parE}{\part[\texttt{\textbf{\color{WildStrawberry} E}}]}
\newcommand{\parS}{\part[\texttt{\textbf{\color{Goldenrod} S}}]}
\newcommand{\parO}{\part[\texttt{\textbf{\color{BurntOrange} O}}]}

\newcommand{\subparA}{\subpart[\texttt{\textbf{\color{Cerulean} A}}]}
\newcommand{\subparM}{\subpart[\texttt{\textbf{\color{PineGreen} M}}]}
\newcommand{\subparE}{\subpart[\texttt{\textbf{\color{WildStrawberry} E}}]}
\newcommand{\subparS}{\subpart[\texttt{\textbf{\color{Goldenrod} S}}]}
\newcommand{\subparO}{\subpart[\texttt{\textbf{\color{BurntOrange} O}}]}

\newcommand{\mainHeader}[2]{\section*{NCEA Level 2 Mathematics\\#1. #2}}
\newcommand{\mainHeaderHw}[2]{\section*{NCEA Level 2 Mathematics (Homework)\\#1. #2}}
\newcommand{\seealso}[1]{\begin{center}\emph{See also #1.}\end{center}}
\newcommand{\drills}[1]{\begin{center}\emph{Drill problems: #1.}\end{center}}
\newcommand{\basedon}[1]{\begin{center}\emph{Notes largely based on #1.}\end{center}}

\begin{document}

\mainHeaderDiffHw{8}{Optimisation}
\subsection*{Reading}
Our students are not fencing in farm fields, cutting wires and folding them, or designing windows, so they are often uninspired by the
optimization problems we give them. They seem like something that ``someone, somewhere'' might use, but the examples feel distant.

\textbf{What are good examples of constrained optimization problems (perhaps not simple!) that today's students might actually encounter in their lives?}

\subsubsection*{Bad Optimization Problems}
I thought that Jack M made an interesting comment about this question:

\begin{adjustwidth}{1cm}{}
    There aren't any. There may be situations where it's possible to apply optimization to solve a problem you've encountered, but in none of these cases is it honestly worth the effort of solving the problem analytically. I optimize path lengths every day when I walk across the grass on my way to classes, but I'm not going to get out a notebook and calculate an optimal route just to save myself twelve seconds of walking every morning. Mathematics beyond basic arithmetic is simply not useful in ordinary life. But I'm not sure if that's exactly what you mean. --- JackM
\end{adjustwidth}

To some extent, I agree with this comment. With few exceptions, mathematics beyond basic arithmetic is simply not useful in everyday life. Students know this, and you'll have trouble convincing them otherwise.

Because of this, I've always found ``everyday''-style calculus problems a little artificial. Consider the following problem from Stewart's Calculus:
\begin{adjustwidth}{1cm}{}
    A fence 8 ft tall runs parallel to a tall building at a distance of 4 ft from the building. What is the length of the shortest ladder that will reach from the ground over the fence to the wall of the building?
\end{adjustwidth}

The proper response to this question is: who cares? Is there any reason to calculate this length precisely? Why would anyone ever use calculus to compute this? If you have an actual building and an actual ladder, you could just try it and see if the ladder fits. If you don't have a specific ladder in mind (e.g. you are buying a ladder), the thing to do would be to draw the situation on paper and then use a ruler to estimate the minimum length. Of course, it's neat that you can use calculus to solve this problem precisely, but this is more of a curiosity than a legitimate application.

Chris specifically mentions the farmer fence problem, the wire-cutting problem, and the Norman window problem as not relevant to the students' lives. I agree—none of these problems are relevant. But it's not because the students aren't farmers, or wire-cutters, or architects. Even in a class full of future farmers, the fence problem would still be bad, because farmers don't use calculus to plan their fences.

\subsubsection*{Good Optimization Problems}
What calculus is useful for is science, economics, engineering, industrial operations, finance, and so forth. That is, it's useful for all the things that make our society run. Most students who take calculus at a university are planning to go into one of these fields, so calculus will be relevant in their lives --- specifically in their future studies and in their professions.

Here's something that's closer to a real-life optimization problem:

\begin{adjustwidth}{1cm}{}
  When a critically damped RLC circuit is connected to a voltage source, the current $ I $ in the circuit varies with time according to the equation
  \begin{displaymath}
    I = \left(\frac{V}{L}\right)te^{-Rt/(2L)}
  \end{displaymath}
  where $V$ is the applied voltage, $L$ is the inductance, and $R$ is the resistance (all of which are constant).

  Suppose an RLC circuit with a resistance of \SI{30.0}{\ohm} and an inductance of \SI{0.400}{\henry} is attached to a \SI{12.0}{\volt} voltage source. Find the maximum current that will occur in the circuit.
\end{adjustwidth}

This is at least close to something that a physics or engineering student might actually come across in their future studies. It's real in a way that the farmer fence problem isn't, and even students who don't plan to study physics can sense that this is a legitimate application. (By the way, if you have good students, you might even ask them to come up with a formula for the maximum current, without giving them specific numbers for $ V $, $ L $ and $ R $. This has the advantage that it can't simply be solved using a graphing calculator.)

Of course, this isn't actually a constrained optimization problem --- it's just an optimization problem. I'm not actually aware of any place in science that simple constrained optimization problems arise, although there are examples from economics (maximizing utility), finance (optimal portfolios), and industrial design (e.g. shape of a can type problems). When I cover constrained optimization in calculus, I usually stick to industrial-type problems (best cans, best shipping crates/boxes, best pipeline across a river, etc.), but that's probably just because I don't know enough about economics or finance to make up problems that involve them.

Finally, I should mention that I've never found the optimization portion of Calculus I particularly compelling. It's good to introduce the idea of optimization, but setting the derivative equal to zero isn't actually a very useful optimization technique by itself. It only really works for simple formulas—for anything complicated it just replaces one essentially numerical problem (finding the maximum of a function) with another (finding roots of a function). I agree that it should be covered, but it's far from the most important application of calculus.

\begin{flushright}
  From \url{https://matheducators.stackexchange.com/questions/1550}.
\end{flushright}

\subsection*{Questions}
\begin{questions}
  \question What is the minimum vertical distance between the parabolae $ y = x^2 + 1 $ and $ y = x - x^2 $?
  \question Show that $ 3x + 2\cos x + 5 = 0 $ has exactly one real root by showing that it
            is increasing everywhere and crosses the $ x$-axis somewhere between two values of $ x $.
  \question Find the area of the largest rectangle that can be inscribed in the ellipse $ \frac{x^2}{a^2} + \frac{y^2}{b^2} = 1 $.
  \question Let $ ABCD $ be a square piece of paper with sides of length \SI{1}{\metre}. A
            quarter-circle is drawn from $ B $ to $ D $ with centre $ A $. The piece of paper is folded along $ EF $, with $ E $ on $ AB $
            and $ F $ on $ AD $, so that $ A $ falls on the quarter-circle. Determine the maximum and minimum areas that the triangle $ AEF $
            can have. (Hint: you may want to introduce a coordinate system.)
\end{questions}
\end{document}
