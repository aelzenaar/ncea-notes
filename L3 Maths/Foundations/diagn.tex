\documentclass{exam}
\usepackage[utf8]{inputenc}

\usepackage[parfill]{parskip}
\usepackage[dvipsnames]{xcolor}
\usepackage{amsmath}
\usepackage{amsfonts}
\usepackage{amsthm}
\usepackage{microtype}
\usepackage{siunitx}
\DeclareSIUnit\year{yr}
\usepackage{pgfplots}
\usepackage{graphicx}
\usepackage{sidecap}
\sidecaptionvpos{figure}{c}
\usepackage{float}
\usepackage{gensymb}
\usepackage{tkz-euclide}
\usetkzobj{all}
\usepackage{commath}
\usepackage{hyperref}
\usepackage{enumitem}
\usepackage{wasysym}
\usepackage{tabularx}

\renewcommand*{\thefootnote}{\fnsymbol{footnote}}

\newtheorem*{thm}{Theorem}
\newtheorem*{iden}{Identity}
\newtheorem*{lemma}{Lemma}
\theoremstyle{definition}
\newtheorem*{defn}{Definition}
\newtheorem*{ex}{Example}

% russian integral
\usepackage{scalerel}
\DeclareMathOperator*{\rint}{\scalerel*{\rotatebox{17}{$\!\int\!$}}{\int}}

\pgfplotsset{vasymptote/.style={
    before end axis/.append code={
        \draw[densely dashed] ({rel axis cs:0,0} -| {axis cs:#1,0})
        -- ({rel axis cs:0,1} -| {axis cs:#1,0});
    }
}}

% \qformat{Question \thequestion: \thequestiontitle\hfill}

\begin{document}

\section*{NCEA Level 3 Calculus\\Prerequisite Knowledge}

I am often asked by Y12 students, `what should I know for Y13?'. This is a list of the \emph{absolute minimum}
knowledge required for L3 calculus: and by `minimum knowledge', I mean that you should be able to do the following
things (a) \emph{without} a calculator, (b) without too much thought (i.e. they should be second nature), and (c)
understanding what you are doing at each step.

If you cannot do the following things without a calculator, or if they take you more than about five minutes each,
you will struggle significantly with getting a bare pass in NCEA L3 calculus. That being said, I normally give this
sheet out at the end of term 2 of Y12, and so you have the time to get up to speed before next year.

\begin{enumerate}
  \item Write the following expression as a single fraction:
    \begin{displaymath}
      \frac{y + x}{3x + 2} + \frac{x^2 - xy}{y}
    \end{displaymath}
  \item Fully expand the following expression:
    \begin{displaymath}
      (x + 2)(x - 1)(x + y)(y + 3)
    \end{displaymath}
  \item Draw the graph of $ (x - 3)^2 + 4 = y $.
  \item Draw the graph of $ -x^2 - 7x - 12 = y $.
  \item Draw the graph of $ 10^{x + 1} = y + 2 $.
  \item Draw the graph of $ (x - 3)(x - 2)(x - 1) = y $.
  \item Find the vertex of $ (x - 3)^2 + 1 = y $.
  \item Find the vertex of $ x^2 + px + q = y $.
  \item If $ -1 < x < 1 $, what values can $ x^2 + 2x $ assume?
  \item Find all $ x $ such that $ \frac{x}{x - 4} < \frac{-1}{x + 2} $.
  \item Find $ \log_{10} (100000) $ and $ \log_{10} (0.00001) $.
  \item The following illustrate properties of logarithms; identify the property and the corresponding rule for
        exponents. As an example the first is done for you.
    \begin{parts}
      \part $ \log_{10} 710 = \log_{10} 7.1 + \log_{10} 100 $. \emph{Solution:} This illustrates the property $ \log_b xy = \log_b x + \log_b y $,
            which corresponds to the exponent property $ b^{x + y} = b^x b^y $.
      \part $ \log_3 2^{7.1} = 7.1 \log_3 2 $
      \part $ \log_7 1 = 0 $
      \part $ \log_{37} 37 = 1 $
      \part $ \log_{37} \frac{(37)(59)}{67} = \log_{37} 37 + \log_{37} 59 - \log_{37} 67 $
      \part $ \log_{18} \sqrt[18]{318} = \frac{1}{18} \log_{18} 318 $
    \end{parts}
  \item Solve for $ x $, if $ \log_x \frac{1}{x^4} = -4 $.
  \item Write $ \frac{\log_{10} x}{\log_{10} b} $ as a single logarithm.
  \item Solve for $ x $ if $ 3^{x^2 - 2x + 1} = 9 $.
  \item Find the equation of a line through $ (3,2) $ and $ (6,3) $. What is the slope of this line?
  \item Prove that there is no line passing through all three points $ (0,1) $, $ (1,0) $, and $ (1,1) $.
  \item If a right angled triangle has side lengths 5, 12, and 13:
        \begin{parts}
          \item State Pythagoras' theorem in general, and check that it `works' in this specific case.
          \item What are the sines of the three angles of the triangle?
        \end{parts}
  \item Draw $ y = \sin x $, $ y = \tan x $, and $ y = \cos x $ on the same pair of axes, labelling the important points (maxima, minima, roots, asymptotes).
  \item Prove that the distance between $ (x_0, y_0) $ and $ (x_1, y_1) $ is $ \left((x_0 - x_1)^2 + (y_0 - y_1)\right)^{1/2} $.
  \item On a number line indicate the values of $ x $ which satisfy $ (x + 2)^2 \geq 0 $.
  \item Find explicitly all values of $ x $ that satisfy $ 2x^2 - 13 \leq x^2 + 12 $.
  \item Find all points $ (x,y) $ that lie on both the following lines:
    \begin{gather*}
      y = x - 9\\
      2x + y = 3
    \end{gather*}
  \item Solve the following system of equations:
    \begin{gather*}
      x - 2y + 3z = 9\\
      -x + 3y - z = -6\\
      2x - 5y + 5z = 17.
    \end{gather*}
  \item If $ f $ is a function defined by $ f(x) = (x + 3)(x - 2)(10^{x - 3}) $, find (a) $ f(3) $; (b) $ f(z) $; (c) $ f(x + 3) $.
\end{enumerate}

\paragraph{References.}
\begin{itemize}
  \item David Crowdis and Brendon Wheeler, \textit{Precalculus Mathematics}. Benziger Bruce and Glencoe (1976). In particular chapters 2, 4, 5, 6, 7.
\end{itemize}

\section*{Where can I go with L3 calculus?}
Calculus is \emph{required} for the following:
\begin{itemize}
  \item Engineering
  \item Physics
  \item Pure mathematics
  \item Applied mathematics
  \item Economics
\end{itemize}

Calculus is \emph{strongly recommended} for the following (i.e. you should take it, and if you don't
then you will find the first year university calculus paper they make you take very difficult indeed):
\begin{itemize}
  \item Chemistry
  \item Medicine
  \item Computer science
  \item Biology
  \item Statistics
\end{itemize}

\end{document}
