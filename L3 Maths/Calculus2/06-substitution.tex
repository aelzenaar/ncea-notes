\section{Substitution}
Recall that the chain rule for differentiation is given by
\begin{displaymath}
  \od{}{x} f(g(x)) = f'(g(x)) g'(x).
\end{displaymath}

In other words, $ f(g(x)) $ is an anti-derivative of $ f'(g(x)) g'(x) $ and so we can write
\begin{displaymath}
  \rint f'(g(x)) g'(x) \dif{x} = f(g(x)) + C.
\end{displaymath}

To make this rule easier to apply in practice, we often perform what is known as a change of variables. We
let $ u = g(x) $, and then $ \od{u}{x} = g'(x) $. Substituting this in, we obtain
\begin{displaymath}
  \rint f'(g(x)) g'(x) \dif{x} = \rint f'(u) \od{u}{x} \dif{x}
\end{displaymath}
and then the rule is just the statement that we can `cancel' the $ \dif{x} $'s, producing
\begin{displaymath}
  \rint f'(g(x)) g'(x) \dif{x} = \rint f'(u) \frac{\dif{u}}{\cancel{\dif{x}}} \cancel{\dif{x}} = \rint f'(u) \dif{u} = f(u) + C = f(g(x)) + C.
\end{displaymath}

This rule, which gives us a kind of chain rule for integration, is called \emph{substitution}, or the \emph{inverse chain rule}. It
can be thought of as a change in coordinate system from an $ x$-based system to one based on $ u $, and we have to `resize' our curve based
on how much $ u $ stretches the coordinate system compared to $ x $ --- and this `stretch factor' is simply $ \od{u}{x} $.

\begin{exs}\leavevmode
  \begin{enumerate}
    \item Suppose we wish to find $ \rint \sin x \cos x \dif{x} $. Then let $ u = \sin x $, so $ \dif{u} = \cos x \dif{x} $
          and
          \begin{displaymath}
            \rint \sin x \cos x \dif{x} = \rint u \dif{u} = \frac{1}{2} u^2 + C = \frac{1}{2} \sin^2 x + C.
          \end{displaymath}
          In this case, we also could have used a trigonometric identity.
    \item Suppose we wish to find $ \rint xe^{x^2} \dif{x} $. We can let $ u = x^2 $, and then $ du = 2x \dif{x} \Rightarrow \dif{x} = \frac{du}{2x} $.
          Hence:
          \begin{displaymath}
            \rint xe^{x^2} \dif{x} = \rint \frac{1}{2} e^u \dif{u} = \frac{1}{2} e^u + C = \frac{1}{2}e^{x^2} + C.
          \end{displaymath}
    \item Suppose we wish to find $ \rint \frac{4}{x} (\ln x)^3 \dif{x} $. We let $ u = \ln x $, and then $ \dif{u} = \frac{\dif{x}}{x} $.
          Hence:
          \begin{displaymath}
            \rint \frac{4}{x} (\ln x)^3 \dif{x} = 4\rint u^3 \dif{u} = u^4 + C = (\ln u)^4 + C.
          \end{displaymath}
  \end{enumerate}
\end{exs}

\subsection{Exercises and Problems}
\begin{enumerate}
  \item Find the following indefinite integrals. (Remember, the indefinite integral of $ f $, $ \rint f(x) \dif{x} $, is
        the family of anti-derivatives of $ f $.)
    \begin{multicols}{2}
    \begin{enumerate}
      \item $\displaystyle \rint \sin 2x \dif{x} $
      \item $\displaystyle \rint \tan x \dif{x} $
      \item $\displaystyle \rint 3x\cos x \dif{x} $
      \item $\displaystyle \rint \frac{\cos x}{\sin x + 1} \dif{x} $
      \item $\displaystyle \rint (4x - 44)^{2019} \dif{x} $
      \item $\displaystyle \rint 4x \sqrt{x^2 + 3} \dif{x} $
      \item $\displaystyle \rint (3x - 4)^2 \dif{x} $
      \item $\displaystyle \rint \frac{x}{x^2 + 1} \dif{x} $
      \item $\displaystyle \rint \frac{2}{4x + 3} \dif{x} $
      \item $\displaystyle \rint e^{2x + 1} \dif{x} $
      \item $\displaystyle \rint \sec 4x \tan 4x \dif{x} $
      \item $\displaystyle \rint 2\cos x + \sin 2x \dif{x} $
      \item $\displaystyle \rint -2x\csc^2 (3x^2) \dif{x} $
      \item $\displaystyle \rint \frac{3}{x^3} - \frac{4}{x + 1} \dif{x} $
      \item $\displaystyle \rint e^{x/2} + \frac{2}{x} \dif{x} $
      \item $\displaystyle \rint x^2 \sec^2 x^3 + 9 \dif{x} $
      \item $\displaystyle \rint -\csc (\tan x) \cot (\tan x) \sec^2 x \dif{x} $
      \item $\displaystyle \rint \frac{\cos x - \sin x}{\cos x + \sin x} \dif{x} $
      \item $\displaystyle \rint \frac{2017}{x\ln x} \dif{x} $
      \item $\displaystyle \rint \tan x + \frac{1}{\tan x} \dif{x} $
      \item $\displaystyle \rint (\cos x) (\sin \sin x) (\cos \cos \sin x) \dif{x} $
    \end{enumerate}
    \end{multicols}
  \item By using the substitution $ x = \sin \theta $, find
        \begin{displaymath}
          \rint \frac{1}{\sqrt{1 - x^2}} \dif{x}.
        \end{displaymath}
  \item Evaluate $ \rint \cos^5 x \dif{x} $ using the substitution $ t = \sin x $.
  \item Find $ \rint \tan \theta \dif{\theta} $ and $ \rint \cot \theta \dif{\theta} $.
  \item Complete the following working:
        \begin{align*}
          \rint \sec x \dif{x} &= \rint \sec x \frac{\sec x + \tan x}{\sec x + \tan x} \dif{x}\\
                              &= \rint \frac{\dots}{\sec x + \tan x} \dif{x}\\
                              \text{Let $ u = \dots $}\\
                              &= \rint \frac{1}{\dots} \dif{u}\\
                              &= \dots
        \end{align*}
  \item Find an anti-derivative of $ \csc x $. (Hint: consider the previous problem.)
  \item The velocity of a particle at time $ t $ is given by $ v = \dfrac{\cos(\sqrt{2t + 1})}{\sqrt{2t + 1}} $.
        What is the position of the particle at time $ t = 5 $, given that $ x(0.5) = 0 $? (Recall that $ v = \od{x}{t} $.)
  \item Consider the following indefinite integral:
        \begin{displaymath}
          \rint \frac{1}{\sqrt{1 - x^2}} \dif{x}.
        \end{displaymath}
    \begin{enumerate}
      \item Show, using the inverse function rule for differentiation, that the anti-derivatives
            of $ \frac{1}{\sqrt{1 - x^2}} $ are $ \sin^{-1} x + C $.
      \item Compute the indefinite integral a different way, using the substitution $ x = \sin \theta $.
      \item Find the anti-derivatives of
            \begin{displaymath}
              f(x) = \frac{-1}{2\sqrt{x - x^2}}.
            \end{displaymath}
            (Hint: try to substitute $ u = \sqrt{1 - x} $.)
    \end{enumerate}
  \item Compute the following:
    \begin{enumerate}
      \item $ \displaystyle\rint \dfrac{x^2(5x^2 + 4x - 3)}{x^5 + x^4 - x^3 + 1} \dif{x} $`
       \item $ \displaystyle\rint \dfrac{x^2 + 1}{x(x^2 + 3)} \dif{x} $
    \end{enumerate}
\end{enumerate}
The previous problem involved finding anti-derivatives of \emph{rational functions}: those of the form $ \frac{P(x)}{Q(x)} $
for polynomials $ P $ and $ Q $. In general, it is possible to find anti-derivatives of all such functions by writing them
as sums of fractions with linear or quadratic denominators; this is known as \emph{expansion via partial fractions}.
\begin{enumerate}[resume]
  \item Some more interesting problems:
    \begin{enumerate}
      \item Rewrite in the form $ \dfrac{A}{x-1} + \dfrac{B}{(x-1)^2} + \dfrac{C}{x + 1} $ and integrate:
        \begin{displaymath}
          \rint \dfrac{4x}{x^3 - x^2 - x + 1} \dif{x}.
        \end{displaymath}
      \item Use the obvious substitution and divide through:
        \begin{displaymath}
          \rint \dfrac{\sqrt{x+1}}{x} \dif{x}.
        \end{displaymath}
    \end{enumerate}
  \item Recall that $ \od{}{x} \tan^{-1} x = \frac{1}{x^2 + 1} $.
    \begin{enumerate}
      \item Rewrite the given rational function as follows:
            \begin{displaymath}
              \frac{x^2 + x - 2}{3x^3 - x^2 + 3x - 1} = \frac{A}{3x - 1} + \frac{Bx + C}{x^2 + 1}
            \end{displaymath}
      \item Hence (or otherwise) compute:
            \begin{displaymath}
              \rint \frac{x^2 + x - 2}{3x^3 - x^2 + 3x - 1} \dif{x}.
            \end{displaymath}
    \end{enumerate}
  \item Use appropriate substitutions to evaluate:
    \begin{enumerate}
      \item $ \displaystyle\rint \dfrac{\cos \theta}{\sin^2 \theta + 4 \sin \theta - 5} \dif{\theta} $
      \item $ \displaystyle\rint \dfrac{e^{3x}}{e^{2x} + 4} \dif{t} $
      \item $ \displaystyle\rint \dfrac{5 + 2\ln x}{x(1 + \ln x)^2} $
    \end{enumerate}
  \item In the following, let $ t = \tan \frac{x}{2} $ (where $ \abs{x} < \pi $). We will apply the techniques
        from the last few problems in the notes, calculating some anti-derivatives of rational functions by
        expanding them as sums of fractions.
    \begin{enumerate}
      \item Show that:
            \begin{displaymath}
              \cos\left( \frac{x}{2} \right) = \frac{1}{\sqrt{1 + t^2}} \quad\text{and}\quad \sin\left(\frac{x}{2}\right) = \frac{t}{\sqrt{1 + t^2}}
            \end{displaymath}
      \item Show that:
            \begin{displaymath}
              \cos x = \frac{1 - t^2}{1 + t^2} \quad\text{and}\quad \sin x = \frac{2t}{1 + t^2}
            \end{displaymath}
      \item Show that:
            \begin{displaymath}
              \od{x}{t} = \frac{2}{1 + t^2}
            \end{displaymath}
      \item Use the substitution $ t $ to evaluate:
        \begin{enumerate}
          \item $ \rint (1 - \cos x)^{-1} \dif{x} $
          \item $ \rint (3\sin x - 4\cos x)^{-1} \dif{x} $
        \end{enumerate}
    \end{enumerate}
\end{enumerate}

\subsection{References}
For exercises and notes on substitution, see Thompson chapter XX (the section on substitution
is two or three pages in). For partial fractions, see chapter XIII.

For the interested, a proof that one can always expand a rational function into partial fractions
is outlined as exercise 11.1.13 in Artin (p. 441).

\subsection{Homework problems}
\begin{enumerate}
  \item Calculate the following indefinite integrals:
    \begin{enumerate}
      \item $ \rint -\csc 3x \cot 3x \dif{x} $
      \item $ \rint -x\sec^2 3x^2 \dif{x} $
      \item $ \rint \frac{\sqrt{x} + 3x - 2}{x} \dif{x} $
      \item $ \rint \sin^3 x \cos^2 x \dif{x} $ (Hint: use $ \sin^2 x = 1 - \cos^2 x $ to rewrite the integrand.)
    \end{enumerate}
  \item Recall that $ \od{}{x} \tan^{-1} x = \frac{1}{1 + x^2} $. Find $ \rint \frac{x}{1 + x^4} \dif{x} $.
  \item Let $ y $ be a function of $ x $, and let $ x $ in turn be a function of $ t $. If $ \od{y}{x} = 3 $ when $ x = 0 $,
        and if $ x(t) = 7t + e^t $, find an explicit expression for $ y(t) $.
\end{enumerate}
