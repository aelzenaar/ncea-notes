\section{Taking derivatives}
Taking derivatives using the definitions quickly becomes unmanagable. Because of this,
we want to produce a set of rules which will allow us to take derivatives of common
functions.

\subsection{Differentiation of polynomials}
We begin with a few easy observations that will allow us to take derivatives of
polynomials.

\begin{thm}[Arithmetic of derivatives]
  Let $ f $ and $ g $ be functions.
  \begin{enumerate}
    \item If $ f(x) = \lambda $ for all $ x $ (where $ \lambda $ is a constant), then $ f'(x) = 0 $ for all $ x $.
    \item The derivative of $ f + g $ is $ f' + g' $ (i.e. for all $ x $, $ (f + g)'(x) = f'(x) + g'(x) $).
    \item If $ \lambda $ is a constant, then $ (\lambda f)' = \lambda f' $.
  \end{enumerate}
\end{thm}
\begin{proof}
  \begin{enumerate}
    \item In this case, $ \frac{f(x + h) - f(x)}{h} = \frac{\lambda - \lambda}{h} = 0 $, so the difference quotient is always zero and so is the derivative.
    \item If $ h $ is small, $ (f + g)(x + h) - (f + g)(x) =  f(x + h) + g(x + h) - f(x) - g(x) \approx f'(x)h + g'(x)h $ and so the derivative at $ x $
          is $ f'(x) + g'(x) $.
    \item $ (\lambda f)(x + h) - (\lambda f)(x) = \lambda (f(x + h) - f(x)) \approx \lambda (f'(x) h) = (\lambda f')(x) h $.
  \end{enumerate}
\end{proof}

Now we consider $ f(x) = x^n $ for integers $ n $. Using the binomial theorem,
\begin{displaymath}
  \lim_{h \to 0} \frac{f(x + h) - f(x)}{h} = \lim_{h \to 0} \frac{(x + h)^n - x^n}{h} = \lim_{h \to 0} \frac{x^n + n x^{n - 1} h + \cdots - x^n}{h}
\end{displaymath}
where every term hidden in the $ \cdots $ includes an $ h^2 $ factor, so we obtain
\begin{equation}
  f'(x) = \lim_{h \to 0} n x^{n - 1} + h(\cdots) = nx^{n - 1}.
\end{equation}
In fact, although we proved this for integer $ n $, it holds in general:
\begin{thm}[Power law]
  If $ f(x) = x^\alpha $ then $ f'(x) = \alpha x^{\alpha - 1} $.
\end{thm}

\begin{ex}
  We can now differentiate every function of the form $ f(x) = a_1 x^{b_1} + \cdots + a_n x^{b_n} $: $ f'(x) = b_1 a_1 x^{b_1 - 1} + \cdots + b_n a_n x^{b_n - 1} $.
  In particular, if $ f(x) = \sqrt{x} $ then $ f'(x) = \frac{1}{2\sqrt{x}} $; if $ g(x) = 2x^2 + 3 $ then $ g'(x) = 4x $; and if $ h(x) = \frac{1}{x} + x^7 $
  then $ h'(x) = -\frac{1}{x^2} + 7x^6 $.
\end{ex}

\subsection{Trigonometric derivatives}
We have already seen that $ \sin' = \cos $; using similar reasoning, we can prove the following:

\begin{thm}[Trigonometric derivatives]
  \def\arraystretch{1.5}
  \begin{tabular}{|c|c|l|}\hline
    \textbf{Function} & \textbf{Derivative} \\\hline
    $ \sin x $ & $ \cos x $\\\hline
    $ \cos x $ & $ -\sin x $\\\hline
    $ \tan x $ & $ \sec^2 x $\\\hline
    $ \csc x $ & $ -\csc x \cot x $\\\hline
    $ \sec x $ & $ \sec x \tan x $\\\hline
    $ \cot x $ & $ -\csc^2 x $\\\hline
  \end{tabular}
\end{thm}

I will use the result $ \sin' = \cos $ to prove that $ \cos' = -\sin $, and we will prove
the rest at a later time. Indeed, $ \cos x = \sin (x + \pi/2) $; then $ \od{}{x} \cos x = \od{}{x} \sin(x + \pi/2) $.
But the graph of $ \sin(x + \pi/2) $ is just the graph of $ \sin x $, shifted to the left
by $ \pi/2 $. Hence the slope of $ \sin (x + \pi/2) $ is the same as the slope of $ \sin x $,
but shifted to the left by $ \pi/2 $; and the slope of $ \sin x $ is $ \cos x $. Hence:
\begin{equation}
  \od{}{x} \cos x = \od{}{x} \sin(x + \pi/2) = \cos(x + \pi/2) = -\sin x.
\end{equation}
(This little trick I used here is explored in more detail in the L2 notes; we don't need it
too often this year, because in a couple of sections we will learn a much more general way
of dealing with this kind of situation.)

\begin{app}
  Many phenomena in physics can be modelled with sine waves; for example, if a particle on the end of a spring
  is moving with simple harmonic motion, then it has position $ x = A \sin (\omega t + \phi) $; taking derivatives,
  we find that it has velocity $ v = \od{x}{t} = A\omega \cos (\omega t + \phi) $ and acceleration $ a = \od[2]{x}{t} = -A \omega^2 \sin (\omega t + \phi) $.
  In other words, it is always accelerating in the opposite direction to its movement!
\end{app}

\subsection{Exponential functions}
The next function we want to consider here is $ f(x) = a^x $, for constants $ a $. We can
compute that
\begin{displaymath}
  f'(x) = \lim{h \to 0} \frac{a^{x + h} - a^x}{h} =  a^x \lim_{h \to 0} \left( \frac{a^h - 1}{h} \right).
\end{displaymath}
So the exponential functions $ a^x $ have derivatives of the form $ Ka^x $, where $ K $ is some constant. This
begs the question, for which value of $ a $ (if any) does $ \od{}{x} a^x = a^x $ (i.e. $ K = 1 $)? Well, we
need to solve
\begin{displaymath}
  \lim_{h \to 0} \left( \frac{a^h - 1}{h} \right) = 1
\end{displaymath}
for $ a $. We will begin by setting $ u = 1/h $, so when $ h \to 0 $ we have $ u \to \infty $. Thus
\begin{displaymath}
  \lim_{u \to \infty} (a^{1/u} - 1)u = 1 = \lim_{u \to \infty} 1;
\end{displaymath}
and applying the limit laws,
\begin{align*}
  \lim_{u \to \infty} (a^{1/u} - 1) &= \lim_{u \to \infty} \frac{1}{u};\\
  \lim_{u \to \infty} a^{1/u} &= \lim_{u \to \infty} \frac{1}{u} + 1;\\
  a = \lim_{u \to \infty} (a^{1/u})^u &= \lim_{u \to \infty} \left(\frac{1}{u} + 1\right)^u.
\end{align*}
It can be shown fairly easily that $ \lim_{u \to \infty} \left(\frac{1}{u} + 1\right)^u $ does
indeed exist (it has a value of $ 2.71828... $), and we define its value to be $ e $. Thus $ e $
is the base for the exponential function that is its own derivative: $ \od{}{x} e^x = e^x $. Often,
we write $ \exp(x) := e^x $.

Finally, note that if $ K = \lim_{h \to 0} \left( \frac{a^h - 1}{h} \right) = \lim_{u \to \infty} u(a^{1/u} - 1) $ then
\begin{displaymath}
  a = \lim_{u \to \infty} \left(\frac{K}{u} + 1\right)^u = \lim_{u \to \infty} \left(\left(\frac{K}{u} + 1\right)^{u/K}\right)^K = \lim_{(u/K) \to \infty} \left(\left(\frac{K}{u} + 1\right)^{u/K}\right)^K = e^K;
\end{displaymath}
hence $ \od{}{x} a^x = a^x K = a^x \log_e a $. (We normally write $ \log_e = \ln $.)

\subsection{Logarithmic derivatives}
Finally, let us calculate $ \od{}{x} \ln x $. (This will allow us to find $ \od{}{x} \log_a x $ for all $ a $, using the
relationship $ \log_a x = \frac{1}{\ln a} \ln x $.)

\begin{displaymath}
  \ln'(x) = \lim_{h \to 0} \frac{\ln(x + h) - \ln(x)}{h} = \lim_{h \to 0} \frac{1}{h} \ln\left(1 - \frac{h}{x} \right) = \lim_{h \to 0} \ln\left(1 - \frac{h}{x} \right)^{1/h}
\end{displaymath}
Let $ u = 1/h $; so as $ h \to 0 $, $ u \to \infty $. Then, substituting, we obtain
\begin{displaymath}
  \ln'(x) = \lim_{u \to \infty} \ln \left( 1 - \frac{1}{ux} \right)^u.
\end{displaymath}
Now, we use the fact that $ \exp(\ln x) = x $:
\begin{displaymath}
  e^{\ln'(x)} = \lim_{u \to \infty} \exp(\ln \left( 1 - \frac{1}{ux} \right)^u) = \lim_{u \to \infty}\left( 1 - \frac{1}{ux} \right)^u
              = \lim_{u \to \infty} \left(\left( 1 - \frac{1}{ux} \right)^{ux}\right)^{(1/x)}
              = e^{1/x}
\end{displaymath}
and thus $ \ln'(x) = 1/x $.

\subsection{Exercises and Problems}
\begin{enumerate}
  \item Find the derivatives of $ 3x^3 $, $ 2x^2 $, and $ 6x^5 $. Conclude that $ (fg)' \neq f' g' $ in general.
  \item Find the derivatives of the following functions with respect to $ t $:
    \begin{enumerate}
      \item $ y = 2t^3 + 3t^2 $
      \item $ y = \sqrt{t} $
      \item $ y = (2t + 1)(t - 4) $
      \item $ g(t) = 4 \sec t + 9 \tan t $
      \item $ h(t) = \sqrt[5]{t} + 2\csc t - \ln t^3 $
      \item $ \phi'(t) = \csc x + 12x^{1273} + 9 $
      \item $ y = 2017t^{2016} + (t + 2)^2 $
      \item $ y = 940\sin t + \frac{1}{2}e^{t + 2} $
    \end{enumerate}
  \item Where is the function $ x \mapsto x^3 - 2x^2 - x + 1 $ increasing?
  \item Find the velocity $ v $ of a particle at time $ t = 2\pi $ if its position function for $ t > 0 $ is $ x = e^t - \sin t $.
  \item Find the slope of the tangent line to $ y = x + \tan x $ at $ (\pi, \pi) $.
  \item Find a linear approximation $ \tilde f $ to $ f(x) = x^2 + x + 1 $ at $ (0,1) $, and find some $ \delta $ such
        that for all $ x $ satisfying $ -\delta < x < \delta $, $ -0.1 < \tilde f(x) - f(x) < 0.1 $.
  \item It is \textbf{not} true that the derivative of $ f(g(x)) $ is $ f'(g'(x)) $.
    \begin{enumerate}
      \item For a counterexample, consider $ f(x) = x^2 $ and $ g(x) = x $; show that $ f'(g'(x)) = 2 $, but $ \od{}{x} f(g(x)) = 2x $.
      \item Compute the derivative of $ \ln x^2 $.
    \end{enumerate}
  \item Suppose the derivative of a function is $ \od{y}{x} = 3x^2 - x - 4 $. What could the original function be?
  \item Find the 64th derivative of $ \sin x $.
  \item Find the $ n$th derivative of $ x^n $.
  \item If $ y = 2\sin 3x \cos 2x $, find $ \od{y}{x} $. (Hint: use an identity to rewrite this as a sum of functions.)
  \item For which values of $ x $ does the graph of $ f(x) = x + 2\sin x $ have a horizontal tangent?
  \item Show that $ y = 6x^3 + 5x - 3 $ has no tangent line with a slope of 4.
  \item Find real values of $ \alpha $ and $ \beta $ such that, if $ y = \alpha \sin x + \beta \cos x $,
        then $ y'' + y' -2y = \sin x $.
  \item Consider a \SI{12}{\metre} long ladder leaning against a wall such that the top of the ladder makes an
        angle $ \theta $ with the wall. If this angle $ \theta $ is varied, the distance $ D $ between the bottom
        of the ladder and the wall also changes. If $ \theta = \pi/3 $, what is the rate of change of $ D $ with
        respect to $ \theta $?
  \item Prove that the function $ \varphi $ given by $ \varphi(x) = \frac{x^{101}}{101} + \frac{x^{51}}{51} + x + 1 $
        never has a horizontal tangent line.
  \item The derivative is primarily a geometric concept, not an algebraic one.
    \begin{enumerate}
      \item The area of a circle of radius $ r $ is $ A = \pi r^2 $. Find $ \od{A}{r} $. What do you notice?
      \item Explain item (a) geometrically.
      \item The volume of a sphere is given by $ V = \frac{4}{3} \pi r^3 $. Find an expression for the surface area.
    \end{enumerate}
\end{enumerate}

\subsection{References}
See sections 2.1 -- 2.4 of Stewart. For a discussion of the exponential function and its
relationship to compound interest and rates of growth, see Thompson chapter XIV.

\subsection{Homework problems}
\begin{enumerate}
  \item Differentiate with respect to $ x $:
    \begin{enumerate}
      \item $ x^2 + \frac{1}{x} $
      \item $ tx^t $
      \item $ \sin x - \cos x $
      \item $ \sqrt[5]{x^4} $
    \end{enumerate}
  \item Explain why you cannot use the power rule to find the derivative of $ x^x $.
  \item Find the $ n$th derivative of $ \frac{1}{x^n} $.
  \item Suppose a population grows exponentially with time, such that after $ t $ years the population is $ P(t) = P_0 + 10^t $.
    \begin{enumerate}
      \item Find the rate of change of the population at $ t = 100 $.
      \item Explain why this population model is unrealistic.
    \end{enumerate}
\end{enumerate}
