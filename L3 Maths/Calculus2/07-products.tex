\section{The product law}

So far, we can differentiate functions which are made up of sums and compositions
of polynomials, trig functions, and $ \ln $ and $ \exp $. However, the following
function will leave us lost and confused if we try to compute its derivative directly:
\begin{displaymath}
  f(x) = (\sin x) (\cos x) \qquad f'(x) = ?
\end{displaymath}

In this particular case, we can use the identity $ \sin 2x = 2\sin x \cos x $
to rewrite $ f(x) = \frac{1}{2} \sin 2x $ and then apply the chain rule to
find that $ f'(x) = \cos 2x $. However, in general we don't have nice things
like trig identities; thus, we need a rule to differentiate products of functions.

First of all, we notice that $ (fg)' \neq (f')(g') $.\footnote{To save ink, I will
write $ (fg) $ for the function defined by $ (fg)(x) = f(x) g(x) $.} Indeed, for
the function $ f $ defined above, $ (\sin x)' (\cos x)' = -\sin x \cos x $; this
is zero at $ x = 1 $, but we have already seen that the derivative of $ f $ is $ \cos 2x $,
which is equal to $ 1 $ when $ x = 1 $.

\begin{figure}
  \centering
  \begin{tikzpicture}[scale=.8]
    \draw (0,0) -- (0,10) -- (10,10) -- (10,0) -- (0,0);
    \draw (0,9) -- (10,9);
    \draw (9,0) -- (9,10);
    \draw (4.5,4.5) node{$ f(x) g(x) $};
    \draw[left] (0,4.5) node{$ g(x) $};
    \draw[left] (0,9.5) node{$ \approx g'(x) h $};
    \draw[below] (4.5,0) node{$ f(x) $};
    \draw[below] (9.5,0) node{$ \approx f'(x) h $};
    \draw (-1.95,0) -- (-2,0) -- (-2,10) -- (-1.95,10);
    \draw[left] (-2,5) node{$ g(t + h) $};
    \draw (0,-0.95) -- (0,-1) -- (10,-1) -- (10,-0.95);
    \draw[below] (5,-1) node{$ f(t + h) $};
    \fill[fill=gray!10] (9,9) -- (9,10) -- (10,10) -- (10,9) -- (9,9);
    \path[->] (11,10.5) edge[bend right] (9.5,9.5);
    \draw[right] (11,10.5) node[align=center]{negligible:\\$ \approx f'(t) g'(t) h^2 $};
  \end{tikzpicture}
  \caption{The approximate errors in the product rule estimation.}
\end{figure}

We will try to derive one by estimation; consider the following difference:
\begin{displaymath}
  (fg)(x + h) - fg(x) = f(x + h) g(x + h) - f(x) g(x)
\end{displaymath}
We may assume that $ f $ and $ g $ are differentiable at $ x $, and so we can
approximate them with their derivatives,
\begin{align*}
  f(x + h) g(x + h) - f(x) g(x) &\approx (f'(x)h + f(x))(g'(x)h + g(x)) - f(x) g(x)\\
                                &= f'(x) g'(x) h^2 + f'(x) g(x) h + f(x) g'(x) h.
\end{align*}
Applying the reasoning we developed a few sections ago, we note that as $ h \to 0 $, the
approximation becomes exact.\footnote{Technical note. Recall that we defined $ f(x) \approx g(x) $
if $ f(x) = g(x) + \vartheta(h) $ where $ \vartheta(h)/h \to 0 $ as $ h \to 0 $. One can
therefore make the reasoning here rigorous by carrying through the $ \vartheta$'s that
give us the approximations for $ f(x + h) $ and $ g(x + h) $, checking that we end up with
an estimation term that is also satisfies the $ \vartheta $ condition.} Taking the limit
of our difference quotient, we find that
\begin{align*}
  \lim_{h \to 0} \frac{(fg)(x + h) - (fg)(x)}{h} &= \lim_{h \to 0} \frac{f'(x) g'(x) h^2 + f'(x) g(x) h + f(x) g'(x) h}{h}\\
                                                 &= f'(x) g(x) + f(x) g'(x);
\end{align*}
and we have justified the following
\begin{thm}[Product law]
  If $ f $ and $ g $ are differentiable at $ x $, then
  \begin{displaymath}
    (fg)'(x) = f'(x) g(x) + f(x) g'(x).
  \end{displaymath}
\end{thm}

\begin{ex}
  Consider $ y = 2t \sin t $. Then $ \od{y}{t} = 2 \sin t + 2t \cos t $.
\end{ex}

With our rules (sum, chain, and product, together with our basic derivatives), we can now differentiate almost any
combination of functions that we are currently aware of. The process of differentiation is entirely mechanical, and
can be easily performed by a computer. As such, learning to differentiate more complicated combinations of functions
is very similar to learning how to add, multiply, and perform long division, and is only a matter of practice.

\begin{ex}
  Let $ f(x) = \sin x^2 + e^{10x^2 + 3x + e^x} + \frac{2x + 3}{\ln x} $. We can split
  this into three derivative-taking problems by applying the sum rule; so
  \begin{displaymath}
    f'(x) = \od{}{x} \sin x^2 + \od{}{x} e^{10x^2 + 3x + e^x} + \od{}{x} \frac{2x + 3}{\ln x}
  \end{displaymath}

  Taking these in turn, $ \sin x^2 \mapsto 2x \cos x^2 $ (applying the chain rule, since we have a
  function composition) and $ e^{10x^2 + 3x + e^x} \mapsto (20x + 3 + e^x) e^{10x^2 + 3x + e^x} $
  (applying the chain rule again). Finally, note that $ \frac{2x + 3}{\ln x} = (2x + 3)[(\ln x)^{-1}] $
  and so we need to apply the product rule:
  \begin{align*}
    \od{}{x} (2x + 3)[(\ln x)^{-1}] &= \left(\od{}{x} (2x + 3)\right)\left( (\ln x)^{-1} \right) + \left(2x + 3\right)\left( \od{}{x} (\ln x)^{-1} \right)\\
                                    &= 2(\ln x)^{-1} + (2x + 3)\left( -1 (\ln x)^{-2} \cdot \od{}{x} \ln x \right)\\
                                    &= 2(\ln x)^{-1} + (2x + 3)\left( -1 (\ln x)^{-2} \cdot \frac{1}{x} \right)\\
                                    &= \frac{2}{\ln x} - \frac{2x + 3}{x (\ln x)^2}\\
                                    &= \frac{2x\ln x - 2x - 3}{x (\ln x)^2}.
  \end{align*}
  Thus the derivative of $ f $ is
  \begin{displaymath}
    f'(x) = 2x \cos x^2 + (20x + 3 + e^x) e^{10x^2 + 3x + e^x} + \frac{2x\ln x - 2x - 3}{x (\ln x)^2}.
  \end{displaymath}
\end{ex}

\subsection{Exercises and Problems}
\begin{enumerate}
  \item In each case, find $ \od{y}{t} $.
    \begin{multicols}{2}
    \begin{enumerate}
      \item $ y = \left(3 + 2t^2\right)^4 $
      \item $ y = \dfrac{t^3}{\ln t} $
      \item $ y = t\sqrt{t} $
      \item $ y = 2t \sin t - (t^2 - 2) \cos t $
      \item $ y = \dfrac{t}{\sqrt{a^2 - t^2}} $ ($ a $ constant)
      \item $ y = \dfrac{1}{8} t^8 \left(1 - t^2\right)^{-4} $
      \item $ y = e^t \ln t $
      \item $ y = \log \left[1 + \dfrac{t^2 + 3t + 17}{t^{17}}\right] $
      \item $ y = \sin \left[e^{\tan t} \ln \tan t\right] $
      \item $ y = \dfrac{3t - 2}{\sqrt{2t + 1}} $
      \item $ y = \dfrac{\sec 2t}{1 + \tan 2t} $
      \item $ y = \dfrac{(t - 1)(t - 4)}{(t - 2)(t - 3)} $
      \item $ y = t \sin^2(\cos \sqrt{\sin \pi t}) $
      \item $ y = \sqrt[5]{t \tan t} $
      \item $ y = \dfrac{(t + \lambda)^4}{t^4 + \lambda^4} $ ($\lambda $ constant)
    \end{enumerate}
    \end{multicols}
  \item If $ f(x) = e^{-x} $, find $ f(0) + xf'(0) $.
  \item Show that $ \od{}{x} e^{\tan x} e^{-\cot x} = \left(\od{}{x}e^{\tan x}\right)\left(\od{}{x}e^{-\cot x}\right) $. Reconcile
        this with our statement above that the naive product rule does not work in general.
  \item The altitude $ h $ of a triangle is increasing at a constant rate of \SI{1}{\centi\metre\per\minute} while
        the area $ A $ increases at a constant rate of \SI{2}{\centi\metre\squared\per\minute}. At what rate
        is the length $ b $ of the base of the triangle increasing when $ h = \SI{10}{\centi\metre} $ and $ A = \SI{100}{\centi\metre\squared} $?
  \item Show that if $ f $ and $ g $ are differentiable at $ x $, such that $ g(x) \neq 0 $, we have
        \begin{displaymath}
          \left( \frac{f}{g} \right)'(x) = \frac{g(x)f'(x) - f(x)g'(x)}{[g(x)]^2}.
        \end{displaymath}
        This is often called the \emph{quotient law}.
  \item Show that $ y = xe^{-x} $ satisfies the differential equation $ xy' = (1-x)y $.
  \item If $ y = \ln \frac{1 + \sqrt{\sin x}}{1 - \sqrt{\sin x}} $, find $ y'' $.
  \item Find the equation of the tangent line to the graph of $ y = \ln \cos \frac{x - 1}{x} $ at
        the point $(1, 0)$.
  \item Show that $ y = (1 + x + \ln x)^{-1} $ satisfies the differential equation $ xy' = y(y \ln x - 1) $.
  \item Find the angle at which $ y = x^2 \ln [(x - 2)^2] $ cuts the $ x$-axis at the point $ (0,0) $.
  \item When $ x = 0 $, is the curve $ y = (x + 20)^2 (2x^2 - 3)^6 - \ln \sin (x - \frac{\pi}{2}) $ concave up or concave down?
  \item If $ y = \frac{e^x}{\sin x} $, show that $ \od{y}{x} = y(1 - \cot x) $.
  \item Show that if $ f $, $ g $, and $ h $ are functions then $ (fgh)' = f'gh + fg'h + fgh' $.
  \item Suppose $ f(x) = f(-x) $ for all $ x $ in the domain of $ f $. Prove that $ f'(x) = -f'(-x) $ for all $ x $
        in the domain of $ f'(x) $.
  \item Consider the function defined by $ f(x) = x^x $.
    \begin{enumerate}
      \item Rewrite $ f $ in the form $ f(x) = e^{x \ln x} $, and hence find $ f'(x) $.
      \item Find $ \od{y}{t} $ if $ y = (t^2 + 3)^{(t^2 + 3)} $.
    \end{enumerate}
  \item Prove the product rule a different way by writing $ f(x) g(x) = e^{\ln(f(x) g(x))} $.
  \item Find $ f'(x + 3) $ if $ f(x + 3) = (x + 5)^7 $.
  \item The number $ a $ is called a \textbf{double root} of some polynomial function $ f $ if $ f(x) = (x - a)^2g(x) $ for
        some polynomial $ g $. Prove that $ a $ is a double root of $ f $ if and only if $ a $ is a root of both $ f $ and $ f' $.
\end{enumerate}

\subsection{References}

\subsection{Homework}
\paragraph{Reading}
Even next year, when you will be expected to know the derivative rules, it is not necessary to remember the quotient rule because
it is just a special case of the product rule. However, here is a little rhyme:

\begin{center}\itshape
  If it's the quotient rule you wish to know,\\
  It's low-de-high less high-de-low.\\
  Then draw the line and down below,\\
  Denominator squared will go.\footnote{Quoted in \textit{Mathematical Apocrypha} by Steven G. Krantz (p.36).}
\end{center}

\paragraph{Problems}
\begin{enumerate}
  \item Find the derivatives:
    \begin{enumerate}
      \item $ \od{y}{x} $ if $ y = \sin x \ln x $.
      \item $ \od{y}{x} $ if $ y = x \sec kx $ ($ k $ constant).
      \item $ \od{f}{\theta} $ if $ f(\theta) = \frac{\cos \pi \theta}{\sin \pi \theta + \cos \pi \theta} $.
      \item $ \od{y}{t} $ if $ y = \cos^4 (\sin^3 t) $.
    \end{enumerate}
  \item Find an expression for $ (fg)''(x) $ in terms of $ f'(x) $ and $ g'(x) $.
  \item Suppose a liquid is oozing from a corner across a rectangular ridged surface that makes it
        easier to flow in one direction than the other; call the corner $ (0,0) $, and suppose that
        the ridges are in the $ y$--direction: so we would expect the flow to be faster towards increasing
        $ y $ compared to increasing $ x $. As the total volume $ V $ of liquid oozed increases, the flow rate increases
        due to the pressure. Suppose that the rate of ooze is constant, at $ \od{V}{t} = \SI{3}{\metre\cubed\per\hour} $.
        A measurement shows that the flow rates of the liquid at its edges in the two directions are $ \od{x}{t} = V^{-1/2} e^{k(V^{1/2})} $
        and $ \od{y}{t} = e^{kV} $, for some small constant $ k \approx 0.24 $. (Both rates are in metres per hour.)
    \begin{enumerate}
      \item Assuming that the liquid covers the surface uniformly, and that the area covered is roughly rectangular,
            what is the area covered after three hours (the initial volume being zero), and what is the rate of change
            of the area covered at that time? [Useless hint: you will need to take some anti-derivatives at some point, and you
            should need to use the product and chain rules for derivatives eventually as well.]
      \item (Even more funner question.) Suppose the room measures ten metres by ten metres; when the liquid reaches the wall
            in the $ y$--direction, suppose that the full `force' of the liquid is now pushing in the $ x$--direction and the
            flow rate in that direction is the sum of the original flow rates: so $ \od{y}{t} = 0 $ and $ \od{x}{t} = V^{-1/2} e^{k(V^{1/2})} + e^{kV} $.

            Calculate how long the liquid will take to reach the wall in the $ y$--direction, and then (taking into account
            the changed flow rates) work out how long the liquid takes to fill the entire floor area of the room. [Even more
            useless hint: the final answer should be large.]
    \end{enumerate}
\end{enumerate}
