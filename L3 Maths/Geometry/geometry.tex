\documentclass[a4paper,leqno]{article}
\usepackage[utf8]{inputenc}
\usepackage{lmodern}
\usepackage{microtype}
\usepackage[inline]{enumitem}

\usepackage{siunitx}
\usepackage{multirow}
\usepackage{subcaption}

\usepackage[english]{babel}
\usepackage[autostyle, english=british]{csquotes}
\MakeOuterQuote{"}

\usepackage{commath}
\usepackage{amsmath}
\usepackage{amsthm}
\usepackage{amssymb}
\usepackage{mathtools}

\usepackage{pgfplots}
\pgfplotsset{compat=1.11}
\usepgfplotslibrary{fillbetween}
\usetikzlibrary{patterns}

\usepackage{hyperref}

\usepackage[margin=1in]{geometry}
\usepackage{changepage}
\usepackage{titlesec}
\titleformat{\section}{\normalfont\Large\bfseries\centering}{Section~\thesection:}{1em}{}

\def\signed #1{{\leavevmode\unskip\nobreak\hfil\penalty50\hskip2em
  \hbox{}\nobreak\hfil(#1)%
  \parfillskip=0pt \finalhyphendemerits=0 \endgraf}}
\newsavebox\mybox
\newenvironment{aquote}[1]
  {\savebox\mybox{#1}\begin{quote}}
  {\signed{\usebox\mybox}\end{quote}}

% Augmented matrices.
\makeatletter
\renewcommand*\env@matrix[1][*\c@MaxMatrixCols c]{%
  \hskip -\arraycolsep
  \let\@ifnextchar\new@ifnextchar
  \array{#1}}
\makeatother

%--------grstep
% For denoting a Gauss' reduction step.
% Use as: \grstep{\rho_1+\rho_3} or \grstep[2\rho_5 \\ 3\rho_6]{\rho_1+\rho_3}
\newcommand{\grstep}[2][\relax]{%
   \ensuremath{\mathrel{
       {\mathop{\longrightarrow}\limits^{#2\mathstrut}_{
                                     \begin{subarray}{l} #1 \end{subarray}}}}}}
\newcommand{\swap}{\leftrightarrow}


\swapnumbers
\numberwithin{equation}{section}
\newtheorem{thm}[equation]{Theorem}
\newtheorem{lem}[equation]{Lemma}
\newtheorem{cor}[equation]{Corollary}
\newtheorem{prp}[equation]{Proposition}
\theoremstyle{definition}
\newtheorem{defn}[equation]{Definition}
\newtheorem{ex}[equation]{Example}
\newtheorem{exercise}[equation]{Exercise}
\newtheorem{alg}[equation]{Algorithm}
\newtheorem{axiom}[equation]{Axiom}
\theoremstyle{remark}
\newtheorem{rem}[equation]{Remark}

\newcommand{\df}[1]{\textbf{#1}}
\newcommand{\T}{\mathrm{T}}
\newcommand{\F}{\mathrm{F}}
\newcommand{\IndSet}{\mathbf{I}}
\DeclareMathOperator{\cis}{cis}
\DeclareMathOperator{\arcsec}{arcsec}

\title{Level Three Conic Sections}
\author{Alex Elzenaar}
\date{\today}

\begin{document}
\maketitle
\tableofcontents
\section*{Preface}
The conic sections are the simplest curves in the plane which are not simply straight lines. However, most secondary school treatments of the subject
tend to present a set of vaguely connected case-by-case results, rather than any kind of coherent story. These notes are my attempt to avoid this.

\subsection*{Prerequisites}
These notes have perhaps the most `formal' prerequisites out of all my Y13 notes. I will assume results from trigonometry, linear systems, calculus, and
even algebra. However, this does not mean that the full power of these subjects are used. The most important prerequisites are actually Y11 and Y12
geometry because there are a number of results from there about triangles, circles, lines, and so forth that will be used here. My Y13 notes on trigonometry
revise a number of these results, and so the reader is directed there initially.

\begin{center}
  \includegraphics[width=0.4\textwidth]{conics}
\end{center}

\titleformat{\section}{\clearpage\titlerule[0.8pt]\vspace{0.5ex}\normalfont\Large\bfseries\centering}{Section~\thesection:}{1em}{}[{\titlerule[0.8pt]}]
\let\oldsection\section
\renewcommand\section{\clearpage\oldsection}
\section{Basic results}
\begin{defn}
  Let $ O $ be a fixed point and $ \ell $ be some line not passing through $ O $. A \df{conic} $ \mathcal{C} $ is the locus of a point $ P $ such
  that, if $ K $ is the point on $ \ell $ so that $ \ell \perp PK $,
  \begin{displaymath}
    \varepsilon = \frac{\abs{OP}}{\abs{PK}}
  \end{displaymath}
  where $ \varepsilon $ is some fixed constant.

  The point $ O $ is called the \df{focus}, the line $ \ell $ the \df{directrix}, and the constant $\varepsilon$ the \df{eccentricity}.
  The \df{latus rectum} (Latin: straight side) is defined to be the chord of a conic through the focus and parallel to the
  directrix; let $ l $ be the length of the latus rectum.

  The conic is variously called:
  \begin{itemize}
    \item an \df{ellipse} if $ \varepsilon < 1 $;
    \item a \df{parabola} if $ \varepsilon = 1 $; and
    \item an \df{hyperbola} if $ \varepsilon > 1 $.
  \end{itemize}
\end{defn}

\begin{exercise}[Polar form]
  Show that, if $ X $ is on the directrix of a conic such that $ OX \perp \ell $, then the polar equation of the
  conic with respect to this axis and origin $ O $ is
  \begin{displaymath}
    \frac{l}{r} = 1 + \varepsilon \cos \theta.
  \end{displaymath}
  Conclude that:
  \begin{enumerate}
    \item Every conic is symmetric with respect to $ OX $.
    \item The ellipse is a closed and bounded curve (i.e. it does not extend towards infinity).
    \item The parabola is unbounded, but is connected.
    \item The hyperbola consists of two seperate branches, each extending to infinity, given by $ -\alpha < \theta < \alpha $
          and $ \alpha < \theta < 2\pi - \alpha $ (where $ \alpha = \arcsec(-\varepsilon) $).
  \end{enumerate}
\end{exercise}

\begin{exercise}[Rectangular form]
  By squaring the polar form equation, show that the Cartesian equation for a conic, taking suitable axes and origin, is
  \begin{displaymath}
    x^2 + y^2 = (l - \varepsilon x)^2.
  \end{displaymath}
  \begin{enumerate}
    \item If $ \varepsilon \neq 1 $, and a suitable origin is chosen, show that the conic equation can be written in the form
          \begin{displaymath}
            \frac{x^2}{a^2} \pm \frac{y^2}{b^2} = 1
          \end{displaymath}
          for some real numbers $ a $ and $ b $. The new location of the origin is called the \df{centre} of the conic. Conclude that:
          \begin{enumerate}
            \item Both the ellipse and hyperbola are symmetric across both Cartesian axes.
            \item For an ellipse, the values $ 2a $ and $ 2b $ are the lengths of the chords through the origin along the $ x$- and $ y$-axes respectively.
            \item The two branches of a hyperbola lie in opposite regions formed by the two lines (\df{asymptotes})
                    \begin{displaymath}
                      \left(\frac{x}{a} - \frac{y}{b} \right)\left(\frac{x}{a} + \frac{y}{b} \right) = 0.
                    \end{displaymath}
                    The value $ 2a $ is the length of the transverse axis of the hyperbola and the value $ 2b $ is the length of the conjugate
                    axis: the two dimensions of the rectangle whose diagonals are the asymptotes and which is bounded by the hyperbola.
          \end{enumerate}
    \item If $ \varepsilon = 1 $, show that the conic equation for the parabola can be written in the form
          \begin{displaymath}
            y^2 = 2l(\frac{1}{2}l - x).
          \end{displaymath}
          By reflecting in a suitable vertical line, derive the standard form equation
          \begin{displaymath}
            y^2 = 2lx.
          \end{displaymath}
          Given this latter equation, give the coordinates of the focus, and the Cartesian equation of the directrix.
  \end{enumerate}
\end{exercise}

\begin{exercise}[Parametric form]
  Show that:
  \begin{enumerate}
    \item The ellipse $ x^2/a^2 + y^2/b^2 = 1 $ is parameterised by $ (a \cos t, b \sin t) $ for $ 0 \leq t < 2\pi $.
    \item The parabola $ y^2 = 2lx $ is parameterised by $ (2lt^2, 2lt) $.
    \item The hyperbola $ x^2/a^2 - y^2/b^2 = 1 $ is parameterised by $ (a \sec t, b \tan t) $ for $ 0 \leq t < 2\pi $.
  \end{enumerate}
\end{exercise}

\begin{exercise}
  Given the equation
  \begin{displaymath}
    \frac{x^2}{a^2} \pm \frac{y^2}{b^2} = 1,
  \end{displaymath}
  compute the coordinates of the focus, equation of the directrix, and eccentricity of the conic. Hence show that
  ellipses and hyperbolae have two focii and two directrices each. Where is the directrix of a circle?

  Show that the following two properties can be used as alternative definitions for the ellipse and hyperbola:
  \begin{enumerate}[label={AP\arabic*}.]
    \item Show that the ellipse with focii $ O_1 $ and $ O_2 $ is the locus of all points $ P $
          such that $ d(O_1, P) + d(P, O_2) = R $ for some constant $ R $. Give $ R $ in terms of $ a $ and $ b $.
    \item Show that the hyperbola with focii $ O_1 $ and $ O_2 $ is the locus of all points $ P $
          such that $ d(O_1, P) - d(P, O_2) = R $ for some constant $ R $. Give $ R $ in terms of $ a $ and $ b $.
  \end{enumerate}

  As an application of this definition, we will derive the `reflection property' of the ellipse.
  \begin{enumerate}
    \item Suppose $ O_1 $ and $ O_2 $ are on the same side of a line $ \ell $. Show that the shortest path from $ O_1 $
          to $ O_2 $ which touches the line $ \ell $ is the broken line $ O_1XO_2 $, where $ X $ is on $ \ell $ and
          the angles between $ O_1 X $ and $ \ell $ and between $ X O_2 $ and $ \ell $ are equal.
    \item Show that, if $ O_1 $ and $ O_2 $ are two focii of an ellipse and $ X $ is any point on the ellipse, then the broken line $ O_1 X O_2 $
          makes equal angles with the tangent line of the ellipse at $ X $. (Hint: show that this broken line is the shortest path from $ O_1 $ to $ O_2 $
          that touches the tangent line at $ X $.)
    \item Thus, given the law of reflection for waves, show that if a light source is placed at $ O_1 $ then every ray from the source
          will arrive at $ O_2 $ at precisely the same time.
    \item Derive some result of this kind for parabolae by treating a parabola as an ellipse with one focus `at infinity'. Suggest an
          application of this property related to, say, torches.
  \end{enumerate}
\end{exercise}

\section{Isometries}
In the previous section, we saw geometrically that every ellipse or hyperbola is just a translated and rotated version of the graph of
an equation with the form
\begin{equation}
  \frac{x^2}{a^2} + \frac{y^2}{b^2} = 1.
\end{equation}
(Herein, in this section we will write $ p = 1/a^2 $ and $ q = 1/b^2 $ for convenience.)

Here, we will show this relation using coordinates --- the advantage of this approach is that it allows us to calculate precisely
from the equation of a conic its position and angle with respect to the standard axes. The subtle point here is basically that the
same curve will have different equations depending on its position in the coordinate system --- in the previous section we chose our
coordinate system in the `nicest way possible' by geometric means, and now we are given a conic already sitting in some coordinate
system which we want to understand.

In order to pursue this programme, we need to study the effect of rotations and translations on coordinate systems. We
begin with rotations because it turns out that rotating before translating is easier for our purposes.

Let $ \rho = \cis \theta $ be a complex number such that $ \abs{\rho} = 1 $. If $ z = x + yi $ is a point on the complex plane,
then $ \rho $ acts by multiplication on $ z $ to rotate it around the origin by an angle $ \theta $; we can then calculate
the resulting rectangular coordinates of $ \rho z $ to see how a rotation affects our normal coordinate system.
\begin{align*}
  z &= \abs{z} \cis(\tan^{-1} x/y)\\
  \rho z &= \abs{z} \cis(\tan^{-1} x/y + \theta)\\
         &= \abs{z} \left[\cos(\tan^{-1} x/y + \theta) + i\sin(\tan^{-1} x/y + \theta)\right].
\end{align*}

Using trig identities, we calculate
\begin{align*}
  \cos(\tan^{-1} x/y + \theta) = \cos(\tan^{-1} x/y) \cos \theta - \sin(\tan^{-1} x/y) \sin \theta\\
                               = \frac{x}{\sqrt{x^2 + y^2}} \cos \theta - \frac{y}{\sqrt{x^2 + y^2}} \sin \theta\\
  \sin(\tan^{-1} x/y + \theta) = \sin(\tan^{-1} x/y) \cos \theta + \cos(\tan^{-1} x/y) \sin \theta\\
                               = \frac{y}{\sqrt{x^2 + y^2}} \cos \theta + \frac{x}{\sqrt{x^2 + y^2}} \sin \theta
\end{align*}
and so
\begin{align*}
  \rho z &= \abs{z} \left[\cos(\tan^{-1} x/y + \theta) + i\sin(\tan^{-1} x/y + \theta)\right]\\
         &= (x \cos \theta - y \sin \theta) + i(y \cos \theta + x \sin \theta).
\end{align*}

In other words, if the point $ (x,y) $ is rotated about the origin by an angle $ \theta $, then
\begin{equation}
  (x,y) \xmapsto{\rho} (x \cos \theta - y \sin \theta,y \cos \theta + x \sin \theta).
\end{equation}

Considering $ ax^2 + bx + cxy + dy + ey^2 $, then, we want to rotate $ (x,y) $ by some $ \theta $ so that some terms vanish. Doing
a long computation, we find that
\begin{align*}
  ax^2 + bx + cxy + dx + ey^2 &\mapsto a(x \cos \theta - y \sin \theta)^2 + b(x \cos \theta - y \sin \theta)\\
                              &  \qquad + c(x \cos \theta - y \sin \theta)(y \cos \theta + x \sin \theta)\\
                              &  \qquad + d(y \cos \theta + x \sin \theta) + e(y \cos \theta + x \sin \theta)^2\\
                              &= x^2(a\cos^2 \theta + \frac{c}{2}\sin 2\theta + e\sin^2 \theta) + x(b \cos \theta + d\sin \theta)\\
                              &  \qquad + xy ((e - a)\sin 2\theta + c \cos 2\theta)\\
                              &  \qquad + y(-b \sin \theta + d \cos \theta) + y^2(a\sin^2 \theta - \frac{c}{2}\sin 2\theta + e\cos^2 \theta)
\end{align*}
and, looking at this, we see that an easy candidate we can try to get rid of is the $ xy $ term. In fact, if we want this term
to be zero we need only solve
\begin{equation}
  (e - a)\sin 2\theta + c \cos 2\theta = 0
\end{equation}
which is easy: $ \frac{-c}{e - a} = \frac{\sin 2\theta}{\cos 2\theta} = \tan 2\theta $, and so in order to remove
the $ xy $ term we need only rotate our coordinate system by
\begin{equation}
  \theta = \frac{1}{2}\tan^{-1} \frac{c}{a - e}.
\end{equation}

Making the coordinate system change $ (x,y) \mapsto (x \cos \theta - y \sin \theta,y \cos \theta + x \sin \theta) $ therefore leaves
us with something that looks like $ ax^2 + bx + dy + ey^2 = 1 $, where $ x $ and $ y $ are now coordinates in our rotated coordinate
system and where the constants $ a $ to $ e $ are not necessarily the same as before.

We already know that if $ y = f(x) $ is graphed, then we can shift the graph up by $ x_0 $ and to the right
by $ y_0 $ by suitable transformations of the coordinates: $ y - y_0 = f(x - x_0) $ has the shifted graph. Last
year, we performed similar transformations on parabolae by completing the square, and so this is the technique
which we will use now. We proceed as we did last year (but now completing the square in both $ x $ and $ y $):
\begin{displaymath}
  ax^2 + bx + dy + ey^2 = a\left(x + \frac{b}{2a}\right)^2 - \frac{b^2}{4a} + e\left(\frac{d}{2e} + y\right)^2 - \frac{d^2}{4e} = 1.
\end{displaymath}
If we now let $ (x, y) \mapsto \left(x - \frac{b}{2a}, y - \frac{d}{2e}\right) $, then we have got a coordinate
transformation that removes all the linear terms and leaves us with something like
\begin{displaymath}
  ax^2 + ey^2 = 1 + \frac{b^2}{4a} + \frac{d^2}{4e}
\end{displaymath}
and upon division of both sides by $ 1 + \frac{b^2}{4a} + \frac{d^2}{4e} $ we end up, as promised, with something of
the form $ px^2 + qy^2 = 1 $. Note that our coordinate system has completely changed: the new curve has the same
shape as our original curve, but sits inside our new coordinate system much more naturally than it sat in our old
coordinate system.

We have thus proved:
\begin{thm}
  The graph of any ellipse or hyperbola can be obtained by rotating and translating the graph of
  a quadratic equation in canonical form, $ px^2 + qy^2 = 1 $.
\end{thm}

\begin{exercise}
  Clearly $ y = x^2 $ cannot be written in the canonical form. Where does the proof above fall down for a parabola?
\end{exercise}

\begin{exercise}
  Classify the following conics:
  \begin{enumerate}
    \item $ \frac{10}{4}x^2 + 3xy + \frac{10}{4}y^2 = 1 $
    \item $ 4x^2 + 24xy + 11y^2 = 5 $
  \end{enumerate}
\end{exercise}

\begin{exercise}
  What happens to the equation for the hyperbola $ x^2 - y^2 = a^2 $ upon rotation by $ \pi/4 $?
\end{exercise}

\end{document}
