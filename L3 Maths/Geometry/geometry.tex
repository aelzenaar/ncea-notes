\documentclass[a4paper,leqno]{article}
\usepackage[utf8]{inputenc}
\usepackage{lmodern}
\usepackage{microtype}
\usepackage[inline]{enumitem}

\usepackage{siunitx}
\usepackage{multirow}
\usepackage{subcaption}

\usepackage[english]{babel}
\usepackage[autostyle, english=british]{csquotes}
\MakeOuterQuote{"}

\usepackage{commath}
\usepackage{amsmath}
\usepackage{amsthm}
\usepackage{amssymb}
\usepackage{mathtools}

\usepackage{pgfplots}
\pgfplotsset{compat=1.11}
\usepgfplotslibrary{fillbetween}
\usetikzlibrary{patterns}

\usepackage{hyperref}

\usepackage[margin=1in]{geometry}
\usepackage{changepage}
\usepackage{titlesec}
\titleformat{\section}{\normalfont\Large\bfseries\centering}{Section~\thesection:}{1em}{}

\def\signed #1{{\leavevmode\unskip\nobreak\hfil\penalty50\hskip2em
  \hbox{}\nobreak\hfil(#1)%
  \parfillskip=0pt \finalhyphendemerits=0 \endgraf}}
\newsavebox\mybox
\newenvironment{aquote}[1]
  {\savebox\mybox{#1}\begin{quote}}
  {\signed{\usebox\mybox}\end{quote}}

% Augmented matrices.
\makeatletter
\renewcommand*\env@matrix[1][*\c@MaxMatrixCols c]{%
  \hskip -\arraycolsep
  \let\@ifnextchar\new@ifnextchar
  \array{#1}}
\makeatother

%--------grstep
% For denoting a Gauss' reduction step.
% Use as: \grstep{\rho_1+\rho_3} or \grstep[2\rho_5 \\ 3\rho_6]{\rho_1+\rho_3}
\newcommand{\grstep}[2][\relax]{%
   \ensuremath{\mathrel{
       {\mathop{\longrightarrow}\limits^{#2\mathstrut}_{
                                     \begin{subarray}{l} #1 \end{subarray}}}}}}
\newcommand{\swap}{\leftrightarrow}


\swapnumbers
\numberwithin{equation}{section}
\newtheorem{thm}[equation]{Theorem}
\newtheorem{lem}[equation]{Lemma}
\newtheorem{cor}[equation]{Corollary}
\newtheorem{prp}[equation]{Proposition}
\theoremstyle{definition}
\newtheorem{defn}[equation]{Definition}
\newtheorem{ex}[equation]{Example}
\newtheorem{exercise}[equation]{Exercise}
\newtheorem{alg}[equation]{Algorithm}
\theoremstyle{remark}
\newtheorem{rem}[equation]{Remark}

\newcommand{\df}[1]{\textbf{#1}}
\newcommand{\T}{\mathrm{T}}
\newcommand{\F}{\mathrm{F}}
\newcommand{\IndSet}{\mathbf{I}}
\DeclareMathOperator{\cis}{cis}

\title{Level Three Conic Sections}
\author{Alex Elzenaar}
\date{\today}

\begin{document}
\maketitle
\tableofcontents
\section*{Preface}
The conic sections are the simplest curves in the plane which are not simply straight lines. However, most secondary school treatments of the subject
tend to present a set of vaguely connected case-by-case results, rather than any kind of coherent story. These notes are my attempt to avoid this.

\subsection*{Prerequisites}
These notes have perhaps the most `formal' prerequisites out of all my Y13 notes. I will assume results from trigonometry, linear systems, calculus, and
even algebra. However, this does not mean that the full power of these subjects are used. The most important prerequisites are actually Y11 and Y12
geometry because there are a number of results from there about triangles, circles, lines, and so forth that will be used here. My Y13 notes on trigonometry
revise a number of these results, and so the reader is directed there initially.

\begin{center}
  \includegraphics[width=0.4\textwidth]{conics}
\end{center}

\titleformat{\section}{\clearpage\titlerule[0.8pt]\vspace{0.5ex}\normalfont\Large\bfseries\centering}{Section~\thesection:}{1em}{}[{\titlerule[0.8pt]}]
\let\oldsection\section
\renewcommand\section{\clearpage\oldsection}
\section{Quadratic Equations (Again)}
We will study in this topic precisely equations of the form
\begin{equation}\label{eqn:prototype}
  ax^2 + 2hxy + by^2 + 2gx + 2fy + c = 0.
\end{equation}
This class of equations is the class of \df{quadratic equations} (in two variables). Note that, without loss
of generality, we can assume $ c = 1 $ (otherwise, we just divide through by $ c $).

The set of all points $ (x,y) $ satisfying a quadratic equation is called a \df{curve of the second degree}.

Our first theorem is a generalisation of the statement from algebra that every quadratic equation in one variable
(i.e. every equation $ x^2 + px + q = 0 $) has precisely two solutions (if you count correctly).
\begin{thm}
  Every line meets a curve of the second degree in precisely two points.
\end{thm}

This theorem is very nice, but (like the proof for quadratics in one variable) we do have to make a trade:
we have to take our coordinate system to be over the complex numbers. Thus, for the remainder of these notes,
we will assume implicitly that rather than living in the real plane $ \mathbb{R}^2 $ we are in fact living
in the complex plane $ \mathbb{C}^2 $. In some sense, then, we are doing geometry in four dimensions. On the
other hand, in many places we will only be interested in real coordinates. For example, the graph of a four-dimensional
object is a little tricky to draw if we only have three dimensions available!

Our first proof attempt looks all right, and indeed it is almost correct;

\begin{proof}[Proof attempt]
  Let $ ax + by = 1 $ be our line. At least one of $ a $ or $ b $ is non-zero; assume that $ a $ is
  non-zero (but essentially the same proof will work if $ a = 0 $ but $ b $ is non-zero). Then we
  can substitute $ x = \frac{1 - by}{a} $ into our curve of the second degree to obtain a quadratic
  equation in the variable $ y $; by the fundamental theorem of algebra, this equation has two solutions;
  and each of these corresponds with a point of intersection.
\end{proof}

The problem with this proof is illustrated by the following example:-
\begin{ex}
  Consider the quadratic $ xy + y + x^2 = 1 $. This describes a perfectly good curve in two dimensions, graphed here.
  \begin{center}
    \fbox{\begin{tikzpicture}
      \begin{axis}[
        axis lines = center,
        xlabel = $ x $,
        ylabel = $ y $
      ]
        \addplot[domain = -5:5, color = red] {(1 - x^2)/(1 + x)};
        \draw[color = red] ({axis cs:1,0}|-{rel axis cs:0,0}) -- ({axis cs:1,0}|-{rel axis cs:0,1});
      \end{axis}
    \end{tikzpicture}}
  \end{center}
  If we intersect this curve with the line $ x = 0 $, then we obtain a linear equation $ y = 1 $ instead of a quadratic
  equation --- and so even counting multiplicities and moving up to $ \mathbb{C} $, we only have one solution.
\end{ex}

The reader might be tempted to discount this example as uninteresting, because it's simply the intersection of a pair of two lines
with a third line (admittedly the first pair of lines has a hole, but the hole is far away from where we take the intersection so it
doesn't matter) and thus isn't really a problem that affects our programme of studying curves of the second degree.

The next example, though, is both more fundamental and more worrying.
\begin{ex}
  If we intersect the parabola $ x^2 = y $ with the line $ x = 0 $, we obtain a single point of intersection, even
  up to multiplicity.
\end{ex}

Does this mean that we have to abandon the beautiful intersection theorem?

It turns out that, like how we expanded our number system to make the fundamental theorem of algebra work
nicely, we can expand our plane $ \mathbb{C}^2 $ further to make our intersection theorem work.

We will not make this precise, but the method we will use is some kind of limiting process; the result, and the
system we will work in from now on, is the projective plane over $ \mathbb{C} $. Let us see how we climb to it.

\section{Isometries}
I claim that the graph of every quadratic equation like \ref{eqn:prototype} is just a translated and
rotated version of the graph of an equation with the form
\begin{equation}
  px^2 + qy^2 = 1.
\end{equation}

In order to show this, we need to study the effect of rotations and translations on coordinate systems. We
begin with rotations because it turns out that rotating before translating is easier for our purposes.

Let $ \rho = \cis \theta $ be a complex number such that $ \abs{\rho} = 1 $. If $ z = x + yi $ is a point on the complex plane,
then $ \rho $ acts by multiplication on $ z $ to rotate it around the origin by an angle $ \theta $; we can then calculate
the resulting rectangular coordinates of $ \rho z $ to see how a rotation affects our normal coordinate system.
\begin{align*}
  z &= \abs{z} \cis(\tan^{-1} x/y)\\
  \rho z &= \abs{z} \cis(\tan^{-1} x/y + \theta)\\
         &= \abs{z} \left[\cos(\tan^{-1} x/y + \theta) + i\sin(\tan^{-1} x/y + \theta)\right].
\end{align*}

Using trig identities, we calculate
\begin{align*}
  \cos(\tan^{-1} x/y + \theta) = \cos(\tan^{-1} x/y) \cos \theta - \sin(\tan^{-1} x/y) \sin \theta\\
                               = \frac{x}{\sqrt{x^2 + y^2}} \cos \theta - \frac{y}{\sqrt{x^2 + y^2}} \sin \theta\\
  \sin(\tan^{-1} x/y + \theta) = \sin(\tan^{-1} x/y) \cos \theta + \cos(\tan^{-1} x/y) \sin \theta\\
                               = \frac{y}{\sqrt{x^2 + y^2}} \cos \theta + \frac{x}{\sqrt{x^2 + y^2}} \sin \theta
\end{align*}
and so
\begin{align*}
  \rho z &= \abs{z} \left[\cos(\tan^{-1} x/y + \theta) + i\sin(\tan^{-1} x/y + \theta)\right]\\
         &= (x \cos \theta - y \sin \theta) + i(y \cos \theta + x \sin \theta).
\end{align*}

In other words, if the point $ (x,y) $ is rotated about the origin by an angle $ \theta $, then
\begin{equation}
  (x,y) \xmapsto{\rho} (x \cos \theta - y \sin \theta,y \cos \theta + x \sin \theta).
\end{equation}

Considering $ ax^2 + bx + cxy + dy + ey^2 $, then, we want to rotate $ (x,y) $ by some $ \theta $ so that some terms vanish. Doing
a long computation, we find that
\begin{align*}
  ax^2 + bx + cxy + dx + ey^2 &= a(x \cos \theta - y \sin \theta)^2 + b(x \cos \theta - y \sin \theta)\\
                              &  \qquad + c(x \cos \theta - y \sin \theta)(y \cos \theta + x \sin \theta)\\
                              &  \qquad + d(y \cos \theta + x \sin \theta) + e(y \cos \theta + x \sin \theta)^2\\
                              &= ax^2 \cos^2 \theta - 2axy \cos\theta\sin\theta + ay^2 \sin^2\theta + bx\cos\theta - by\sin\theta\\
                              &  \qquad + cxy \cos^2 \theta + cx^2 \sin \theta\cos\theta - cy^2 \sin\theta\cos\theta - cxy \sin^2\theta\\
                              &  \qquad + dy\cos\theta + dx\sin\theta + ey^2\cos^2\theta + 2exy\cos\theta\sin\theta + ex^2\sin^2\theta\\
                              &= x^2(a\cos^2 \theta + \frac{c}{2}\sin 2\theta + e\sin^2 \theta) + x(b \cos \theta + d\sin \theta)\\
                              &  \qquad + xy ((e - a)\sin 2\theta + c \cos 2\theta)\\
                              &  \qquad + y(-b \sin \theta + d \cos \theta) + y^2(a\sin^2 \theta - \frac{c}{2}\sin 2\theta + e\cos^2 \theta)
\end{align*}
and, looking at this, we see that an easy candidate we can try to get rid of is the $ xy $ term. In fact, if we want this term
to be zero we need only solve
\begin{equation}
  (e - a)\sin 2\theta + c \cos 2\theta = 0
\end{equation}
which is easy: $ \frac{-c}{e - a} = \frac{\sin 2\theta}{\cos 2\theta} = \tan 2\theta $, and so in order to remove
the $ xy $ term we need only rotate our coordinate system by
\begin{equation}
  \theta = \frac{1}{2}\tan^{-1} \frac{c}{a - e}.
\end{equation}

Making the coordinate system change $ (x,y) \mapsto (x \cos \theta - y \sin \theta,y \cos \theta + x \sin \theta) $ therefore leaves
us with something that looks like $ ax^2 + bx + dy + ey^2 = 1 $, where $ x $ and $ y $ are now coordinates in our rotated coordinate
system and where the constants $ a $ to $ e $ are not necessarily the same as before.

We already know that if $ y = f(x) $ is graphed, then we can shift the graph up by $ x_0 $ and to the right
by $ y_0 $ by suitable transformations of the coordinates: $ y - y_0 = f(x - x_0) $ has the shifted graph. Last
year, we performed similar transformations on parabolae by completing the square, and so this is the technique
which we will use now. We proceed as we did last year (but now completing the square in both $ x $ and $ y $):
\begin{displaymath}
  ax^2 + bx + dy + ey^2 = a\left(x + \frac{b}{2a}\right)^2 - \frac{b^2}{4a} + e\left(\frac{d}{2e} + y\right)^2 - \frac{d^2}{4e} = 1.
\end{displaymath}
If we now let $ (x, y) \mapsto \left(x - \frac{b}{2a}, y - \frac{d}{2e}\right) $, then we have got a coordinate
transformation that removes all the linear terms and leaves us with something like
\begin{displaymath}
  ax^2 + ey^2 = 1 + \frac{b^2}{4a} + \frac{d^2}{4e}
\end{displaymath}
and upon division of both sides by $ 1 + \frac{b^2}{4a} + \frac{d^2}{4e} $ we end up, as promised, with something of
the form $ px^2 + qy^2 = 1 $. Note that our coordinate system has completely changed: the new curve has the same
shape as our original curve, but sits inside our new coordinate system much more naturally than it sat in our old
coordinate system.

We have thus proved:
\begin{thm}
  The graph of any quadratic function can be obtained by rotating and translating the graph of
  a quadratic function in \df{canonical form}, $ px^2 + qy^2 = 1 $.
\end{thm}

For the remainder of this unit, then, we will concern ourself mainly with studying quadratics in canonical form.

\begin{ex}
  Let us consider the curve defined by $ \frac{10}{4}x^2 + 3xy + \frac{10}{4}y^2 = 1 $. We begin by dealing with the $ xy $ term; according to
  our work on transformations, to get rid of the cross-term we need to rotate our coordinate system
  by $ \theta = \frac{1}{2}\tan^{-1} \frac{c}{a - e} $. Here, $ c = 3 $, and $ a = 10/4 = e $;
  so $ \theta = \frac{1}{2} \tan^{-1} \infty = \frac{1}{2} \frac{\pi}{2} = \frac{\pi}{4} $. It follows that $ \cos \theta = 1/\sqrt{2} = \sin\theta $.

  We are therefore sending
  \begin{equation}
    (x,y) \mapsto \left(\frac{x}{\sqrt{2}} - \frac{y}{\sqrt{2}}, \frac{y}{\sqrt{2}} + \frac{x}{\sqrt{2}}\right)
  \end{equation}
  and our rotated curve is $ x^2 + \frac{y^2}{4} = 1$.
\end{ex}

\begin{ex}
  For an even simpler example, we note that the hyperbola $ y = 1/x $ is actually a quadratic equation: $ 1 = xy $. We
  have $ a = e = 0 $ and $ c = 1 $, so $ \theta = \frac{\pi}{4} $ again. Our rotated curve is $ 1 = \frac{x^2}{2} - \frac{y^2}{2} $.
\end{ex}


\end{document}
