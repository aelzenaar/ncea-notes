\documentclass[answers]{exam}

\usepackage[dvipsnames]{xcolor}
\usepackage{amsmath}
\usepackage{amsfonts}
\usepackage{amsthm}
\usepackage{microtype}
\usepackage{siunitx}
\DeclareSIUnit\year{yr}
\usepackage{pgfplots}
\usepackage{graphicx}
\usepackage{sidecap}
\sidecaptionvpos{figure}{c}
\usepackage{float}
\usepackage{gensymb}
\usepackage{tkz-euclide}
\usetkzobj{all}
\usepackage{commath}
\DeclareMathOperator{\versin}{versin}

\newtheorem*{thm}{Theorem}

% russian integral
\usepackage{scalerel}
\DeclareMathOperator*{\rint}{\scalerel*{\rotatebox{17}{$\!\int\!$}}{\int}}

% \qformat{Question \thequestion: \thequestiontitle\hfill}

\begin{document}

\section*{NCEA Level 3 Trigonometry (exercise set)\\3. Taxonomy of Functions}
\paragraph{Goal} To investigate the relationships between the geometry and algebra of the trig functions.

\begin{questions}
  \question Show that the following hold for all $ \theta $:
    \begin{parts}
      \part $ \sec^2 \theta + \csc^2 \theta = \sec^2 \theta \cos^2 \theta $
      \part $ (\sin \theta + \sec \theta)^2 + (\cos \theta + \csc \theta)^2 = (1 + \sec \theta \csc \theta)^2 $
    \end{parts}
  \question
    \begin{parts}
      \part Draw a graph to check the plausibility of the conjecture that
            \begin{equation}\tag{*}
              \tan \theta = -\cot \left( \theta - \frac{\pi}{2} \right)
            \end{equation}
            holds for all $ \theta $.
      \part Show that (*) holds for all $ \theta $.
      \part Prove that there is \textbf{no} $ \xi $ such that $ \cos \theta = \sec(\theta + \xi) $ holds for all $ \theta $.
    \end{parts}
  \question Let $ O_1 $ and $ O_2 $ be two distinct points, and suppose $ r_1 $ and $ r_2 $ are positive numbers with $ r_1 > r_2 $.
    \begin{parts}
      \part Draw two circles centred at $ O_1 $ and $ O_2 $ with respective radii $ r_1 $ and $ r_2 $.
      \part Suppose $ L $ is a tangent line shared by both circles which does not cross the \emph{segment} $ O_1 O_2 $;
            show that the angle $ \theta $ made by $ L $ with the \emph{line} $ O_1 O_2 $ satisfies
            \begin{displaymath}
              \sin \theta = \frac{r_1 - r_2}{\abs{O_1 O_2}}.
            \end{displaymath}
      \part Find the distance between the points of tangency along $ L $.
    \end{parts}
  \question Complete exercise 3.7 in the notes.
  \question
    \begin{parts}
      \part Illustrate graphically the change in sign and magitude of the functions $ 3\sin x + 4\cos x $, $ e^x \sin x $,
            and $ \sin \left( \frac{\pi}{\sqrt{2}} \sin x \right) $ for all values of $ x $.
      \part Show that $ 2x = (2n + 1)\pi \versin x $ (for $ n $ positive) has \textbf{exactly} $ 2n + 3 $ possible solutions for $ x $;
            indicate roughly their locations. [Hint: approach this problem in the same rough direction as (a) --- i.e. do not try to compute
            the values of $ x $ directly.]
    \end{parts}
\end{questions}

\paragraph{Additional reading} Hobson, chapter III.

\end{document}
