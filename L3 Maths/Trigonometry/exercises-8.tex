\documentclass[answers]{exam}

\usepackage[dvipsnames]{xcolor}
\usepackage{amsmath}
\usepackage{amsfonts}
\usepackage{amssymb}
\usepackage{amsthm}
\usepackage{hyperref}
\usepackage{microtype}
\usepackage{siunitx}
\DeclareSIUnit\year{yr}
\usepackage{pgfplots}
\usepackage{graphicx}
\usepackage{sidecap}
\sidecaptionvpos{figure}{c}
\usepackage{float}
\usepackage{gensymb}
\usepackage{tkz-euclide}
\usetkzobj{all}
\usepackage{commath}
\usepackage{steinmetz}

\newtheorem*{thm}{Theorem}

% russian integral
\usepackage{scalerel}
\DeclareMathOperator*{\rint}{\scalerel*{\rotatebox{17}{$\!\int\!$}}{\int}}
\DeclareMathOperator{\arcsec}{arcsec}
\DeclareMathOperator{\arcversin}{arcversin}

% \qformat{Question \thequestion: \thequestiontitle\hfill}

\begin{document}

\section*{NCEA Level 3 Trigonometry (exercise set)\\8. Periodic Models}
\paragraph{Goal} To study the geometry of the graphs of trigonometric functions.

In this section we completed our goal of fully describing the geometry of the graph of $ y = A \sin(\omega t - \phi) + y_0 $. Since
the graph of cosine is just a shifted sine graph, we know all about cosine graphs as well. We also know quite a bit about the graph
of the tangent function.

In physics, when studying more complicated systems, one often finds that the motion of some object is a \emph{sum} of sine and cosine
functions, combined with an exponential. One example is a mass vibrating back and forwards in the presence of some kind of friction
force.

\paragraph{Question} Systematically study and describe the graph of $ y = e^{-kt} \left(A \sin(\omega t - \phi) + B \cos(\nu t - \psi)\right) $.

\paragraph{Hints}
\begin{itemize}
  \item Periodic?
  \item Maxima and minima?
  \item Zeroes?
  \item Average value?
  \item Behaviour `at infinity'?
\end{itemize}

\paragraph{Additional reading} Hobson chapter VI.

\end{document}
