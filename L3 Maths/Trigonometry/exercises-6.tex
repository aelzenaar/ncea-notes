\documentclass[answers]{exam}

\usepackage[dvipsnames]{xcolor}
\usepackage{amsmath}
\usepackage{amsfonts}
\usepackage{amssymb}
\usepackage{amsthm}
\usepackage{hyperref}
\usepackage{microtype}
\usepackage{siunitx}
\DeclareSIUnit\year{yr}
\usepackage{pgfplots}
\usepackage{graphicx}
\usepackage{sidecap}
\sidecaptionvpos{figure}{c}
\usepackage{float}
\usepackage{gensymb}
\usepackage{tkz-euclide}
\usetkzobj{all}
\usepackage{commath}
\usepackage{steinmetz}

\newtheorem*{thm}{Theorem}

% russian integral
\usepackage{scalerel}
\DeclareMathOperator*{\rint}{\scalerel*{\rotatebox{17}{$\!\int\!$}}{\int}}
\DeclareMathOperator{\arcsec}{arcsec}
\DeclareMathOperator{\arcversin}{arcversin}

% \qformat{Question \thequestion: \thequestiontitle\hfill}

\begin{document}

\section*{NCEA Level 3 Trigonometry (exercise set)\\6. Inverse Functions}
\paragraph{Goal} To continue studying the relationships between the trig functions in the form of identities.

\begin{questions}
  \question
    \begin{parts}
      \part Show that $ \displaystyle \arcsin x = 2\arctan \left( \frac{x}{1 + \sqrt{1 + x^2}} \right) $ for $ -1 \leq x \leq 1 $.
      \part Show that $ \displaystyle \arcversin y = \arctan \left(\frac{\sqrt{2y - y^2}}{1 - y}\right) $ for $ 0 \leq y \leq 2 $.
    \end{parts}
  \question
    \begin{parts}
      \part Use corollary 4.3.3 to show that $ \arctan a + \arctan b = \arctan \left[(a + b)/(1 - ab)\right] $.
      \part Show that $ \arctan 1 + \arctan 2 + \arctan 3 = \pi $.
      \part Show that if $ \tau > 0 $ then $ \arctan \tau + \arctan 1/\tau = \pi/2 $. What if $ \tau < 0 $?
    \end{parts}
  \question Suppose $ p = b/a $ and $ q = y/x $ are rational numbers in simplest form (i.e. $ a, b, x, y $ are integers, $ a $ and $ x $ are nonzero,
            and the fractions cannot be simplified further).

            Define a new operation $ \otimes $ by
            \begin{displaymath}
              p \otimes q = \frac{ay + bx}{ax + by}.
            \end{displaymath}
    \begin{parts}
      \part Suppose $ p $ and $ q $ are rational. Show that $ p \otimes q = q \otimes p $.
      \part Is there a rational number $ \vartheta $ such that $ \vartheta \otimes k = \vartheta $ for all rational numbers $ k $?
      \part Is there a rational number $ \digamma $ such that $ \digamma \otimes k = k $ for all rational numbers $ k $?\footnote{See also \url{https://www.youtube.com/watch?v=GFLkou8NvJo}.} (Hint: yes.)
      \part Fix some rational number $ k $; does there exist a rational number $ k' $ such that $ k \otimes k' = \digamma $?
      \part Show that
            \begin{displaymath}
              \arctan \left( p \otimes q \right) = \arctan\left(p\right) + \arctan\left(q\right).
            \end{displaymath}
            Compare with the identity $ \log ab = \log a + \log b $. (Note: you may wish to complete 2(a) above first.)
    \end{parts}
  \question Find an expression analogous to those in theorem 6.3 for $ \tan(\arcsec x) $.
  \question Find $ a $ such that $ \arcsin a = 2\arccos a $.
\end{questions}

\paragraph{Additional reading} Hobson chapter III.

\end{document}
