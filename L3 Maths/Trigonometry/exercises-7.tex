\documentclass[answers]{exam}

\usepackage[dvipsnames]{xcolor}
\usepackage{amsmath}
\usepackage{amsfonts}
\usepackage{amssymb}
\usepackage{amsthm}
\usepackage{hyperref}
\usepackage{microtype}
\usepackage{siunitx}
\DeclareSIUnit\year{yr}
\usepackage{pgfplots}
\usepackage{graphicx}
\usepackage{sidecap}
\sidecaptionvpos{figure}{c}
\usepackage{float}
\usepackage{gensymb}
\usepackage{tkz-euclide}
\usetkzobj{all}
\usepackage{commath}
\usepackage{steinmetz}

\newtheorem*{thm}{Theorem}

% russian integral
\usepackage{scalerel}
\DeclareMathOperator*{\rint}{\scalerel*{\rotatebox{17}{$\!\int\!$}}{\int}}
\DeclareMathOperator{\arcsec}{arcsec}
\DeclareMathOperator{\arcversin}{arcversin}

% \qformat{Question \thequestion: \thequestiontitle\hfill}

\begin{document}

\section*{NCEA Level 3 Trigonometry (exercise set)\\7. Trigonometric Equations}
\paragraph{Goal} To understand the roots of the trigonometry functions.

\begin{questions}
  \question Let $ a $ and $ b $ be real and define a function $ f $ by $ f(\theta) = a \cos \theta + b \sin \theta $. We will study the behaviour of $ f $.
    \begin{parts}
      \item Show that if $ \alpha $ and $ \beta $ satisfy $ f(\alpha) = f(\beta) $ then $ \frac{b}{a} = \tan \frac{\alpha + \beta}{2} $.
      \item Find all $ \alpha $ such that $ f(\alpha) = 0 $.
      \item Show that $ f(\theta) \leq \sqrt{a^2 + b^2} $ for all $ \theta $. Is equality ever attained?
    \end{parts}
  \question Find all $ \theta $ such that:
    \begin{parts}
      \part $ 64 \sin^7 \theta + \sin 7\theta = 0 $.
      \part $ \cot^2 \theta - 3\cot \theta + 2 = 0 $. (Note: there is no exact value for $ \arctan 1/2 $, so leave your answer in this form.)
      \part $ \frac{\cos 3\theta}{\sec \theta} + \frac{\sin 3\theta}{\csc \theta} = \cos 2\theta $.
    \end{parts}
  \question Find sine and cosine of $ \ang{15} $ and $ \ang{150} $ exactly.
  \fullwidth{We studied exact values of some trig functions in this section; we will now look at some exact formulae for $ \pi $.}
  \question Prove \emph{Machin's formula}, $ \pi/4 = 4\arctan (1/5) - \arctan (1/239) $.
  \question The formula we will consider here was discovered by John Wallis in 1655.
    \begin{parts}
      \part Recall that if a polynomial $ p(x) = p_0 + p_1 x + \cdots + p_n x^n $ has roots $ \alpha_1 , ..., \alpha_n $ then
            we can write $ p(x) = (x - \alpha_1)\cdots(x - \alpha_n) $.

            Convince yourself that the following infinite product (due to Euler) is plausible.
            \begin{displaymath}
              \frac{\sin x}{x} = \left(1 - \frac{x^2}{\pi^2}\right)\left(1 - \frac{x^2}{2^2\pi^2}\right)\left(1 - \frac{x^2}{3^2\pi^2}\right)\cdots
            \end{displaymath}
            [One can prove this rigorously: it is a special case of the so-called \emph{Weierstra\ss\ factorisation theorem}.]
      \part Let $ x = \pi/2 $, and use the formula in (a) to show that
            \begin{displaymath}
              \frac{\pi}{2} = \frac{2}{1} \cdot \frac{2}{3} \cdot \frac{4}{3} \cdot \frac{4}{5} \cdot \frac{6}{5} \cdot \frac{6}{7} \cdots.
            \end{displaymath}
      \part Calculate $ \pi $ by hand to a few decimal places.
    \end{parts}
\end{questions}

\paragraph{Additional reading} Hobson chapter VI.

\end{document}
