\documentclass[a4paper]{article}

\binoppenalty=9999
\relpenalty=9999
\widowpenalty=9999
\clubpenalty=9999

\usepackage{lmodern}
\usepackage[dvipsnames]{xcolor}
\usepackage{amsmath}
\usepackage{amssymb}
\usepackage{amsthm}
\usepackage{mathtools}
\usepackage{commath}
\usepackage{microtype}
\usepackage[parfill]{parskip}
\usepackage{exercises}
\usepackage{pgfplots}
\usepackage{graphicx}
\usepackage{sidecap}
\sidecaptionvpos{figure}{c}
\usepackage{float}
\usepackage{siunitx}
\usepackage[british]{babel}
\usepackage[autostyle, english = british]{csquotes}
\MakeOuterQuote{"}

\usepackage{enumitem}

\usepackage[hang]{footmisc}
\setlength\footnotemargin{10pt}

\usepackage[capitalise,noabbrev]{cleveref}

\usepackage{geometry}
\geometry{
  includeheadfoot,
  margin=2.54cm
}

\renewcommand*{\thefootnote}{\fnsymbol{footnote}}

% russian integral
\usepackage{scalerel}
\DeclareMathOperator*{\rint}{\scalerel*{\rotatebox{17}{$\!\int\!$}}{\int}}

\usetikzlibrary{external}\tikzexternalize

\usepackage{fancyhdr}
\pagestyle{fancy}

\author{Alexander Elzenaar\\Upper Hutt College}
\date{\today}
\title{Assignment: Mathematical Writing Practice III}

\begin{document}

\maketitle

\section{Task}
Suppose that $ | z+w | = | z-w | $. Show that $ \arg z - \arg w = \pm \frac{\pi}{2} $. You
may, if necessary, use the result that $ \arctan u - \arctan v = \arctan\left(\frac{u-v}{1+uv} \right) $.

\textit{Ensure that you write "properly". That means using complete sentences, justifying
all logic, and aiming for clarity!}

\section{Hints}
A list of things to think about:
\begin{itemize}
  \item What can we assume? What are we trying to prove?
  \item How can we simplify the premises and the conclusion?
  \item Why are we given an extra piece of information? Could
        it be useful? Is it a hint about how we should attack the problem?
  \item What is special about $ x $ if $ \arctan x = \frac{\pi}{2} $?
\end{itemize}

\newpage
\section{Example Answer}
This problem is taken from \textit{Solutions} (the Final Exercises).

We first let $ z = a + bi $ and $ w = c + di $. Then $ | z + w | = +\sqrt{(a+c)^2 + (b+d)^2} $,
and $ | z - w | = +\sqrt{(a-c)^2 + (b-d)^2} $. Hence, we can write (by our assumption
$ | z+w | = | z-w | $) that $ (a+c)^2 + (b+d)^2 = (a-c)^2 + (b-d)^2 $, and therefore (by expanding
and simplifying) that
\begin{equation}
  \frac{ac}{bd} = -1.
\end{equation}

We move now to simplify the thing which we are trying to prove: $ \arg z - \arg w = \pm \frac{\pi}{2} $. An
obvious first step is to substitute for $ w $ and $ z $. We therefore obtain (by the definition of the argument
of a complex number) that $ \arctan \frac{b}{a} - \arctan \frac{d}{c} = \pm\frac{\pi}{2} $. Applying the result
given about the difference of arctans, we can write our equation as
\begin{equation}
  \pm\frac{\pi}{2} = \arctan \frac{\frac{b}{a} - \frac{d}{c}}{1 + \frac{bd}{ac}} \Rightarrow \tan\left(\pm\frac{\pi}{2}\right) = \frac{\frac{b}{a} - \frac{d}{c}}{1 + \frac{bd}{ac}}.
\end{equation}

All that remains is to show that (1) implies (2). This can be done by noting that $ \tan\left(\pm\frac{\pi}{2}\right) $
is undefined, and hence that $ \frac{\frac{b}{a} - \frac{d}{c}}{1 + \frac{bd}{ac}} $ being undefined is a sufficient
condition for our conclusion (that $ \arg z - \arg w = \pm \frac{\pi}{2} $) to hold.

We note that by (1), $ 1 + \frac{bd}{ac} = 0 $ and hence the fraction is undefined. $ \square $

\end{document}
