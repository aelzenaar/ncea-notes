\documentclass[a4paper]{article}

\binoppenalty=9999
\relpenalty=9999
\widowpenalty=9999
\clubpenalty=9999

\usepackage{lmodern}
\usepackage[dvipsnames]{xcolor}
\usepackage{amsmath}
\usepackage{amssymb}
\usepackage{amsthm}
\usepackage{mathtools}
\usepackage{commath}
\usepackage{microtype}
\usepackage[parfill]{parskip}
\usepackage{exercises}
\usepackage{pgfplots}
\usepackage{graphicx}
\usepackage{sidecap}
\sidecaptionvpos{figure}{c}
\usepackage{float}
\usepackage{siunitx}
\usepackage[british]{babel}
\usepackage[autostyle, english = british]{csquotes}
\MakeOuterQuote{"}

\usepackage{enumitem}

\usepackage[hang]{footmisc}
\setlength\footnotemargin{10pt}

\usepackage[capitalise,noabbrev]{cleveref}

\usepackage{geometry}
\geometry{
  includeheadfoot,
  margin=2.54cm
}

\renewcommand*{\thefootnote}{\fnsymbol{footnote}}

% russian integral
\usepackage{scalerel}
\DeclareMathOperator*{\rint}{\scalerel*{\rotatebox{17}{$\!\int\!$}}{\int}}

\usetikzlibrary{external}\tikzexternalize

\usepackage{fancyhdr}
\pagestyle{fancy}

\author{Alexander Elzenaar\\Upper Hutt College}
\date{\today}
\title{Assignment: Mathematical Writing Practice}

\begin{document}

\maketitle

\section{Task}
A glucose solution is administered intravenously into the bloodstream at a constant
rate $ r $. As the glucose is added, it is converted into other substances and removed
from the bloodstream at a rate that is proportional to the concentration at that time. Thus
a model for the concentration $ C = C(t) $ of the glucose solution in the bloodstream is
\begin{equation}
    \od{C}{t} = \text{rate in} - \text{rate out} = r - kC
\end{equation}
where $ k $ is a positive constant.

\begin{enumerate}[label=(\alph*)]
  \item Suppose that the concentration at time $ t = 0 $ is $ C_0 $. Determine the
        concentration at any time $ t $ by solving the differential equation.
  \item Assuming that $ C_0 < \frac{r}{k} $, find $ \lim_{t\to\infty} C(t) $ and
        interpret your answer.
\end{enumerate}

\textit{Ensure that you write "properly". That means using complete sentences, justifying
all logic, and aiming for clarity!}

\section{Hints}
A list of things to think about:
\begin{itemize}
  \item What kind of equation is it? What technique(s) are you going to use?
  \item What kind of answer are you expecting for (b)? Why?
  \item What might be the use of finding the limit in (b)?
\end{itemize}

\newpage
\section{Example Answer}
This is a simple separable DE, and so we can write that:
\begin{align*}
                   & \od{C}{t} = r - kC \\
  \Rightarrow\quad & \rint \frac{\dif{C}}{r - kC} = \rint \dif{t} \\
  \Rightarrow\quad & \frac{-1}{k} \ln{\abs{r - kC}} = t + K && \text{(where $ K $ is an arbitrary constant)}\\
  \Rightarrow\quad & \ln{\abs{r - kC}} = -kt + K'           && \text{(where $ K' = -kK $).}\\
  \intertext{Rearranging this to obtain an expression for $ C $, we obtain that}
                   & C = \frac{r - Re^{-kt}}{k} && \text{(where $ R = e^{K'} $).}\\
  \intertext{Using the initial condition that $ C_0 = C(0) $, we can find that $ R = r - kC_0 $. Hence,}
                   & C = \frac{r - (r - kC_0)e^{-kt}}{k} = \frac{r}{k} - \frac{r}{k} e^{-kt} + C_0 e^{-kt}.
\end{align*}

We can therefore see that the limit as $ t $ increases without bound is $ \frac{r}{k} $ (since $ \lim_{t\to\infty} e^{-kt} = 0 $).
This makes sense -- if we let the process carry on for a long time, the concentration tends to the ratio of the input rate
to the output rate. It is also interesting to note that the limit does not depend on the initial concentration. This is because
a higher initial concentration simply leads to a higher initial outflow, and so the total concentration tends to the same ratio at
equilibrium.

\end{document}
