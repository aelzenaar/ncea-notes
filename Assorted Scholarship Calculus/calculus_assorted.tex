\documentclass[addpoints,a4paper]{exam}

\binoppenalty=9999
\relpenalty=9999
\widowpenalty=9999
\clubpenalty=9999

\usepackage{amsmath}
\usepackage{amsfonts}
\usepackage{commath}
\usepackage{graphicx}
\usepackage{tkz-euclide}
\usepackage{caption}

\qformat{Question \thequestion: \thequestiontitle\hfill}
\DeclareMathOperator{\cis}{cis}

\begin{document}

\begin{coverpages}

\begin{center}
  \includegraphics[width=\textwidth]{fermat}

  \vspace{5mm}

  \textbf{\Huge{Scholarship Calculus}}
\end{center}

\vspace{5mm}

\noindent
\large{There are four questions, worth a total of \numpoints\ marks.\\
       Attempt ALL questions, showing all working.\\
       Read questions carefully before attempting them.\\
       Marks are available for partial answers.\\
       The amount of time expected to be spent per question may not necessarily correlate ``nicely'' to the number of marks.\\
       Diagrams may be used to support answers.\\
       Candidates who do not provide diagrams for some questions may be disadvantaged.\\
       Some marks are given for clarity and neatness of solutions or proofs.}
\vspace{2mm}

\begin{tabular}{ll}
  \textbf{Time Allowed:}& Three Hours\\
  \textbf{Scholarship:}& 35 marks\\
  \textbf{Outstanding:}& 56 marks (including at least 10 marks in each question)
\end{tabular}

\vfill

\begin{center}
  \gradetable[h][questions]
  \vspace{2mm}

  \textbf{Available Grades:} \textit{No Scholarship}\quad\textit{Scholarship}\quad\textit{Outstanding}
\end{center}

\end{coverpages}

\begin{questions}
\pointsinrightmargin
\titledquestion{Optimisation}
  \begin{parts}
    \part[6] Find the point on the parabola $ y = 1-x^2 $ at which the tangent line
          cuts from the first quadrant the triangle with the least area. (280/4)
    \part Let ABC be a triangle with right angle $ A $ and hypotenuse $ a = |BC| $. (281/14)
    \begin{subparts}
      \subpart[3] If the inscribed circle touches the hypotenuse at $ D $, show that
        \begin{displaymath}
          |CD| = \frac{1}{2}\left( |BC| + |AC| - |AB| \right)
        \end{displaymath}

      \subpart[4] If $ \theta = \frac{1}{2} \angle C $, express the radius $ r $ of
               the inscribed circle in terms of $ a $ and $ \theta $.

      \subpart[5] If $ a $ is fixed and $ \theta $ varies, find the maximum value
               of $ r $. Leave your answer in terms of $ a $ and $ \tan $.
    \end{subparts}
  \end{parts}

\titledquestion{Conics}
  \begin{parts}
    \part[3] Show that the equation of the tangent line to the parabola $ y^2 = 4px $ at the
          point $ (x_0, y_0) $ can be written as $ y_0 y = 2p(x + x_0) $. (701/56a)
    \part[6] Given an ellipse $ \frac{x^2}{a^2} + \frac{y^2}{b^2} = 1 $ where $ a \neq b $,
          find the equation of the set of all points from which there are two tangents to
          the curve whose curves are (i) reciprocals and (ii) negative reciprocals. (196/20)
    \part[7] A circle $ C $ of radius $ 2r $ has its centre at the origin. A circle of radius
          $ r $ rolls without slipping in the counterclockwise direction around $ C $. A
          point $ P $ is located on a fixed radius of the rolling circle at a distance $ b $
          from its centre such that $ 0 < b < r $; initially, $ P $ lies in line with the
          centre of both circles (on the $ x$-axis). Let $ L $ be the line from the centre of
          $ C $ to the centre of the rolling circle, and let $ \theta $ be the angle that
          $ L $ makes with the positive $ x$-axis. Using $ \theta $ as a parameter, show
          that parametric equations of the path traced by $ P $ are: (712/6a)
          \begin{displaymath}
            x = b \cos 3\theta + 3r \cos \theta \hspace{2cm} y = b \sin 3\theta + 3r \sin \theta
          \end{displaymath}
          \textit{Note: there is a diagram for this question on the next page, although you
                  may wish to include a diagram for $ \theta > 0 $ in your answer.}
  \end{parts}

\titledquestion{Complex Algebra}
  \begin{parts}
    \part[3] $ z = 2-3i $ is one of the roots of the polynomial equation $ z^3 - pz^2 + qz - r = 0 $,
          where $ p $, $ q $, and $ r $ are real numbers. Find the real root in terms of $ p $ or $ r $. (2008)
    \part[3] There are several different ways to calculate the average of two numbers.
          \begin{itemize}
            \item The arithmetic mean, $ A $, is found by halving the sum of the two numbers: $ A = \frac{x+y}{2} $.
            \item The geometric mean, $ G $, is the square root of the product of the two numbers: $ G = \sqrt{xy} $.
            \item The harmonic mean, $ H $, is twice the reciprocal of the sum of the reciprocals
                  of the two numbers: $ H = \frac{2}{\frac{1}{x} + \frac{1}{y}} $
          \end{itemize}
          Show that $ \sqrt{\frac{A \cdot H}{2}} - \sqrt{2} G + \frac{G}{\sqrt{2}} = 0 $ for all $ x $ and $ y $. (2008)
    \part[3] If $ w $ is any real number, the equation $ z^k = w $ has exactly $ k $ roots ($ k \in \mathbb{N} $). Show
          that the sum of all $ k $ roots is zero. Note that $ a + ar + ar^2 + \cdots + ar^{n-1} = a\left(\frac{1-r^n}{1-r} \right) $ if $ |r| \leq 1 $, $ r \neq 1 $. (2008)
    \part[10] Let $ w $ and $ z $ be complex numbers, and let $ u = w + z $ and $ v = w^2 + z^2 $. Prove that $ w $ and $ z $ are real \textbf{if and
          only if} $ u $ and $ v $ are real and $ u^2 \leq 2v$. (Cam15)
  \end{parts}

\titledquestion{Miscellaneous}
  \begin{parts}
    \part[2] Evaluate the Reimann sum for $ f(x) = 3 - 0.5x $ when $ 2 \leq x \leq 14 $
          with six subintervals, taking the sample points to be left endpoints. (306/1)
    \part Euler's formula states that for any real number $ x $, $ e^{ix} = \cis{x} $.
    \begin{subparts}
      \subpart[2] Show that $ \sin x = \frac{e^{ix} - e^{-ix}}{2i} $.
      \subpart[4] Hence show that $ \od[n]{}{x} \left(e^{ax} \sin bx \right) = r^n e^{ax} \sin(bx + n\theta) $
               where $ a $ and $ b $ are positive numbers, $ r^2 = a^2 + b^2 $, and $ \theta = \arctan(b/a) $. (486/5)
    \end{subparts}

    \part[3] Prove that $ \cos^3 x = \frac{1}{4}\left(3\cos x + \cos 3x \right) $, and hence show that $ \cos \alpha $ is
          a root of the equation $ 4x^3 - 3x - \cos 3\alpha = 0 $. (Cam15)
    \part L'Hopital's Rule states that if $ f $ and $ g $ are differentiable functions with $ f(0) = g(0) = 0 $,
          we can write that
          \begin{displaymath}
            \lim_{x \to 0} \frac{f(x)}{g(x)} = \lim_{x \to 0} \frac{f'(x)}{g'(x)}
          \end{displaymath}
    \begin{subparts}
      \subpart[3] Given the added condition that $ g'(0) \neq 0 $, prove, \textit{without assuming L'Hopital's Rule}, the less general statement
                  that $ \lim_{x \to 0} \frac{f(x)}{g(x)} = \frac{f'(0)}{g'(0)} $. (196/18)
      \subpart[3] Use L'Hopital's Rule to evaluate $ \lim_{x \to 0} \frac{\sin(a+2x) - 2 \sin(a+x) + \sin a}{x^2} $. (196/19)
    \end{subparts}
  \end{parts}

\end{questions}

\vfill

\begin{figure}[!h]
  \centering
  \begin{tikzpicture}
    \draw[help lines,ultra thin,dashed,gray!50!white] (-3,-3) grid (5,3);
    \tkzDefPoint(0,0){O}
    \tkzDrawCircle[R](O,2 cm)
    \tkzDefPoint(0:2){T}
    \tkzDefPoint(0:3){X}
    \tkzDrawCircle(X,T)
    \tkzDefPoint(0:3.2){P}
    \tkzDrawSegments(O,T X,T X,P)
    \tkzDrawPoints(P,X,T,O)
    \tkzLabelPoints(P)
    \tkzLabelSegment(X,T){$r$}
    \tkzLabelSegment(O,T){$2r$}
  \end{tikzpicture}
  \caption*{Diagram for Question 2c: $ \theta = 0 $}
\end{figure}

\vfill
\noindent
Some problems taken, either directly or with modification, from: NZQA past external papers; Stewart's \emph{Calculus}~(7th~Ed); University of Cambridge entrance examinations.

\end{document}
