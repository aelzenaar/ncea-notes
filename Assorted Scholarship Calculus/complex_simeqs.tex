\documentclass{exam}

\usepackage{amsmath}
\usepackage{amsfonts}
\usepackage{siunitx}
\DeclareSIUnit\year{yr}

\qformat{Question \thequestion: \thequestiontitle\hfill}

\begin{document}
\begin{questions}

\section*{Scholarship Calculus}

\titledquestion{Complex Numbers}
  \begin{parts}
    \part
      \begin{subparts}
        \subpart
          Find all eighth roots of 256.

        \subpart
          Find the polar form of $ \alpha = 2-3i $.

        \subpart
          If $ \beta = 3+7i $, and $ \gamma = 9+11i $, find the unique
          fourth-degree polynomial with real coefficients that has both
          $ \beta $ and $ \gamma $ as roots.

        \subpart
          Solve for $ x $ if $ x^4 + i = 0 $.
      \end{subparts}

    \part
      If $ \zeta = \sqrt{\frac{1}{2}(a+\sqrt{a^2+b^2})} + i\sqrt{\frac{1}{2}(-a+\sqrt{a^2+b^2})} $
      is a complex number (with $ i = \sqrt{-1} $ and $ b,c \in \mathbb{R} $), find $ \zeta^2 $
      in the form $ p + iq $.
  \end{parts}

\titledquestion{Systems of Equations}
  \begin{parts}
    \part
      Solve the following systems of equations:
      \begin{subparts}
        \subpart
          $ \left\{ \begin{aligned}
            &x + y = 8\\
            &x - y = 0
          \end{aligned} \right. $

        \subpart
          $ \left\{ \begin{aligned}
            &2x-5y = 6\\
            &5x-2y = 12
          \end{aligned} \right. $

        \subpart
          $ \left\{ \begin{aligned}
            &x-2y+z  = 7\\
            &x-y+z   = 4\\
            &2x+y-3z = -4
          \end{aligned} \right. $

        \subpart
          $ \left\{ \begin{aligned}
            &3x+3y              = 2\\
            &\frac{x^2+y^2}{xy} = -2
          \end{aligned} \right. $
      \end{subparts}

      \part
        A teacher sets 99 homework problems for her Calculus class each week, of
        three different types. The number of questions of each type given in
        week $ n $ are represented by $ x_n $, $ y_n $, and $ z_n $ respectively.

        Suppose that this techer uses the following system of linear equations
        to vary the number of questions of each type given each week:
        \begin{align*}
          x_{n+1} &= 0.8x_n + 0.7y_n + 0.6z_n\\
          y_{n+1} &= 0.1x_n + 0.2y_n + 0.4z_n\\
          z_{n+1} &= 0.1x_n + 0.1y_n &
        \end{align*}
        Her class notice that the number of questions of each type stabilises after
        several weeks - in the long run, they notice that $ x_{n+1} = x_n $,
        $ y_{n+1} = y_n $, and $ z_{n+1} = z_n $.

        How many questions of each type will the teacher give each week once the
        numbers stabilise?

      \part
        Find $ c $ such that $ x-y+2=0 $, $ 3x-3y+7=0 $, and $ 3x+2y+x= 0 $ shall
        meet at a single point.
  \end{parts}

\titledquestion{Derivatives}
  \begin{parts}
    \part
      A lighthouse is located on a small island \SI{3}{\kilo\metre} away from the
      nearest point $P$ on a straight shoreline and its light makes four
      revolutions per minute. How fast is the beam of light moving along the
      shoreline when it is \SI{1}{\kilo\metre} away from $P$?

    \part
      If $ f $ and $ g $ are differential functions with $ f(0) = g(0) = 0 $ and
      $ g'(0) \neq 0 $, show that
      \begin{displaymath}
        \lim_{x \to 0} \frac{f(x)}{g(x)} = \frac{f'(0)}{g'(0)}
      \end{displaymath}
      \emph{Note: you may wish to start from the right hand side and expand the derivatives
      using first principles.}

    \part
      Brain weight $ B $ as a function of body weight $ W $ in fish has been
      modelled by the function $ B = 0.007W^{\frac{2}{3}} $, where $ B $ and $ W $
      are measured in grams. A model for body weight as a function of body length
      $ L $ (measured in centimetres) is $ W = 0.12L^{2.53} $. If, over \SI{10}{\mega\year}
      the average length of a certain species of fish evolved from \SI{15}{\centi\metre} to
      \SI{20}{\centi\metre} at a constant rate, how fast has this species' brain
      growing when the average length was \SI{18}{\centi\metre}?
  \end{parts}

\titledquestion{Conic Sections}
  \begin{parts}
    \part
      Determine equations of three different lines, all of which pass through
      the point $ (2, -6) $.
    \part
      Determine the locus of a point which is always as far from the $ x $ axis
      as it is from the point $ (1, 3) $.
    \part
      The earth moves in an elliptical path about the sun with the sun at one focus
      of the ellipse. If the distance of the earth from the sun varies from
      \SI{4.6e7}{\kilo\metre} to \SI{6.98e7}{\kilo\metre}, find the equation of
      the ellipse.
    \part
      A satellite, P, is travelling anticlockwise around an elliptical orbit with
      centre O. The elliptical orbit may be represented by $ \frac{x^2}{a^2} + \frac{y^2}{b^2} = 1 $,
      where the satellite P is represented by the point $ (a \cos \theta, b \sin \theta) $.

      \begin{subparts}
        \subpart
          Draw a diagram showing P, $ \theta $, $ a $, and $ b $, where $ \mathrm{O} = (0,0) $.
        \subpart
          The satellite P shines a microwave beam in the directions perpendicular
          to its direction of motion. Show that, when $ \theta = \frac{\pi}{4} $,
          the beam cuts the vertical plane (the $y$-axis) at a vertical distance
          $ \frac{\sqrt{2}a^2}{2b} $ below the level of the satellite.
      \end{subparts}
  \end{parts}

\titledquestion{Functions}
  \textbf{Definition:} A function $ f $ is called \emph{even} if for all $ x $ in
  the domain of $ f $, $ f(-x) = f(x) $ - i.e. the function is symmetric about the
  $y$-axis. A function $ f $ is called \emph{odd} if for all $ x $ in the domain of
  $ f $, $ f(-x) = -f(x) $.

  \begin{parts}
    \part
      Decide if each of the following function is even, odd, or neither:
      \begin{subparts}
        \subpart
          $ f(x) = x^3 + x $
        \subpart
          $ g(x) = 1-x^2 $
        \subpart
          $ h(x) = 2x-x^2 $
      \end{subparts}
    \part
      \textbf{Definition:} Given some function $ f $ which sends $ x $ to $ y $,
      we can define a function $ f^{-1}(x) $ which sends $ y $ to $ x $. This function
      is the \emph{inverse} of $ f $.

      Prove that the inverse of an odd function is itself odd.
  \end{parts}
\end{questions}
\end{document}
