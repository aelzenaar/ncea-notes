\documentclass[a4paper]{article}

\binoppenalty=9999
\relpenalty=9999
\widowpenalty=9999
\clubpenalty=9999

\usepackage{lmodern}
\usepackage[dvipsnames]{xcolor}
\usepackage{amsmath}
\usepackage{amssymb}
\usepackage{amsthm}
\usepackage{mathtools}
\usepackage{commath}
\usepackage{microtype}
\usepackage[parfill]{parskip}
\usepackage{exercises}
\usepackage{pgfplots}
\usepackage{graphicx}
\usepackage{sidecap}
\sidecaptionvpos{figure}{c}
\usepackage{float}
\usepackage{siunitx}
\usepackage[british]{babel}
\usepackage[autostyle, english = british]{csquotes}
\MakeOuterQuote{"}

\usepackage{enumitem}

\usepackage[hang]{footmisc}
\setlength\footnotemargin{10pt}

\usepackage[capitalise,noabbrev]{cleveref}

\usepackage{geometry}
\geometry{
  includeheadfoot,
  margin=2.54cm
}

\renewcommand*{\thefootnote}{\fnsymbol{footnote}}

% russian integral
\usepackage{scalerel}
\DeclareMathOperator*{\rint}{\scalerel*{\rotatebox{17}{$\!\int\!$}}{\int}}

\usetikzlibrary{external}\tikzexternalize

\usepackage{fancyhdr}
\pagestyle{fancy}

\author{Alexander Elzenaar\\Upper Hutt College}
\date{\today}
\title{Assignment: Mathematical Writing Practice II}

\begin{document}

\maketitle

\section{Task}
A tank contains \SI{1000}{\liter} of pure water. Brine that contains \SI{0.05}{\kilo\gram} of
salt per litre of water enters the tank at a rate of \SI{5}{\liter\per\minute}. Brine that contains
\SI{0.04}{\kilo\gram} of salt per litre of water enters the tank at a rate of \SI{10}{\liter\per\minute}.
The solution is kept thoroughly mixed and drains from the tank at a rate of \SI{15}{\liter\per\minute}.
How much salt is in the tank after one hour?

\textit{Ensure that you write "properly". That means using complete sentences, justifying
all logic, and aiming for clarity!}

\section{Hints}
A list of things to think about:
\begin{itemize}
  \item What information are you given?
  \item How can you model the amount of salt remaining?
  \item What will be your strategy for working out an answer?
\end{itemize}

\newpage
\section{Example Answer}
This problem is very similar to a problem that appeared in the 2015 scholarship paper.

We begin by attempting to write a DE to model the situation. We will use $ t $ to denote
the time since initial conditions (\SI{0}{\kilo\gram} of salt in the tank), $ S $ to denote
the mass of salt in the tank, and $ C $ to denote the concentration of salt in the tank.
Immediately, we see that $ C = \frac{S}{1000} $.

We notice that we have two inputs of salt: an input of $ \SI{0.05}{\kilo\gram\per\liter} \times \SI{5}{\liter\per\minute} = \SI{0.25}{\kilo\gram\per\minute} $,
and an input of $ \SI{0.04}{\kilo\gram\per\liter} \times \SI{10}{\liter\per\minute} = \SI{0.4}{\kilo\gram\per\minute} $. We have a single output of
$ C\,\si{\kilo\gram\per\liter} \times \SI{15}{\liter\per\minute} = 15C\,\si{\kilo\gram\per\minute} $.

Since $ \od{S}{t} = \text{rate in} - \text{rate out} $, we have that
\begin{displaymath}
  \od{S}{t} = 0.25 + 0.4 - 15C = 0.65 - 15C.
\end{displaymath}

From the result above, we can substitute in order to obtain a separable DE in $ S $,
\begin{displaymath}
  \od{S}{t} = 0.65 - \frac{15S}{1000}.
\end{displaymath}

Solving this for $ S $, we find that
\begin{align*}
              && \rint \frac{\dif{S}}{0.65 - \frac{15S}{1000}}      &= \rint \dif{t} &&\\
  \Rightarrow &&-\frac{1000}{15} \ln{\abs{0.65 - \frac{15S}{1000}}} &= t + K && \text{(where $ K $ is an arbitrary constant)}  \\
  \Rightarrow &&\ln{\abs{0.65 - \frac{15S}{1000}}}                  &= -\frac{15t}{1000} + K' && \text{(where $ K' = -\frac{15K}{1000}$)} \\
  \Rightarrow && S                                                  &= \abs{Re^{-\frac{15t}{1000}} - \frac{650}{15}} && \text{(where $ R = \frac{1000}{15}e^{K'} $)}
\end{align*}

Remembering that our initial condition was that $ S(0) = \SI{0}{\kilo\gram} $, we see that $ R = \frac{650}{15} $. Hence, after \SI{60}{\minute},
we have $ S(60) = \abs{\frac{650}{15}e^{-\frac{15\times 60}{1000}} - \frac{650}{15}} = \SI{25.715}{\kilo\gram}$.

\end{document}
