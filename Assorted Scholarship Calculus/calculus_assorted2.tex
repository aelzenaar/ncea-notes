\documentclass[addpoints,a4paper]{exam}

\binoppenalty=9999
\relpenalty=9999
\widowpenalty=9999
\clubpenalty=9999

\usepackage{amsmath}
\usepackage{amsfonts}
\usepackage{commath}
\usepackage{graphicx}
\usepackage{tkz-euclide}
\usepackage{caption}
\usepackage{siunitx}
\usepackage[inline]{enumitem}
\usepackage[linewidth=1pt]{mdframed}
\usepackage{multirow}

\qformat{Question \thequestion: \thequestiontitle\hfill}
\DeclareMathOperator{\cis}{cis}

\begin{document}

\begin{coverpages}

\begin{center}
  \includegraphics[width=0.7\textwidth]{goldbach}

  \vspace{5mm}

  \textbf{\Huge{Scholarship Calculus}}
\end{center}

\vspace{5mm}

\noindent
\large{There are five questions, worth a total of \numpoints\ marks.\\
       Attempt ALL questions, showing all working.\\
       Read questions carefully before attempting them.\\
       Marks are available for partial answers.\\
       The amount of time expected to be spent per question may not necessarily correlate ``nicely'' to the number of marks.\\
       Diagrams may be used to support answers.\\
       Candidates who do not provide diagrams for some questions may be disadvantaged.\\
       Some marks are given for clarity and neatness of solutions or proofs.}
\vspace{2mm}

\begin{tabular}{ll}
  \textbf{Time Allowed:}& Three Hours\\
  \textbf{Scholarship:}& 35 marks\\
  \textbf{Outstanding:}& 56 marks (including at least 10 marks in each question)
\end{tabular}

\vfill

\begin{center}
  \gradetable[h][questions]
  \vspace{2mm}

  \textbf{Available Grades:} \textit{No Scholarship}\quad\textit{Scholarship}\quad\textit{Outstanding}
\end{center}

\end{coverpages}

\begin{questions}
\pointsinrightmargin
\titledquestion{Aircraft}
  \begin{parts}
    \part Consider an aircraft coming in to land at an airport. The approach path of
          the aircraft satisfies three conditions: (156/ap)
          \begin{mdframed}\begin{enumerate}
            \item The cruising altitude is $ h $ when descent starts, at a horizontal
                  distance $ \ell $ from touchdown at the origin.
            \item The pilot must maintain a constant horizontal speed of $ v $ throughout
                  descent.
            \item The absolute value of the vertical acceleration should not exceed a
                  constant $ k $, which is much less than the acceleration due to gravity.
          \end{enumerate}\end{mdframed}
      \begin{subparts}
        \subpart[4] Find a cubic polynomial $ P(x) = ax^3 + bx^2 + cx + d $ that satisfies
                 the first condition, by imposing suitable restrictions on $ P(x) $ and
                 $ P'(x) $ at the start of descent and at touchdown.
        \subpart Use the second and third conditions to show that
          \begin{displaymath}
            \frac{6hv^2}{\ell^2} \leq k
          \end{displaymath}
        \subpart Suppose that an airline decides not to allow the vertical acceleration
                 of a plane to exceed $ k = \SI{1385}{\kilo\meter\per\hour^2} $. If the
                 cruising altitude of a plane is \SI{11 000}{\meter^2} and the speed
                 is \SI{480}{\kilo\meter\per\hour}, how far away from the airport should
                 the pilot start descent?
      \end{subparts}

      \part It so happens that the vertical flight profile of the aircraft during the whole of
            its flight can be modelled by the equation
              $$ y = 6 - (0.01x)^2 - (0.01x)^4 + \sin(0.09x) $$
            where $ -1.40721 \leq x \leq 1.44154 $.
        \begin{subparts}
            \subpart Find the maximum altitude of the aircraft.
            \subpart Approximate the total distance travelled by the aircraft
                     using Simpson's rule with $ n = 10 $.
              \begin{mdframed}
                If $ f(x) $ and $ f'(x) $ are both continuous over the interval $ a \leq x \leq b $,
                the arc length of the curve $ y = f(x) $ over that interval is given by
                \begin{displaymath}
                  L = \int_a^b \sqrt{1+ [f'(x)]^2} \dif{x}
                \end{displaymath}
              \end{mdframed}
        \end{subparts}
  \end{parts}

\titledquestion{Zeno's Limits}
  \begin{parts}
    \part Zeno, an ancient Greek philosopher, examined the following scenario in around
          450~BC. Use your knowledge of limits to explain why it is possible for Homer
          to walk to the end of the path.
      \begin{mdframed}
        \textbf{Zeno's Dichotomy Paradox:} Suppose Homer wishes to walk to the end of a path. Before he can get
        there, he must get halfway there. Before he can get halfway there, he must get a quarter of
        the way there. Before travelling a quarter, he must travel one-eighth; before an eighth, one-sixteenth;
        and so on. This requires Homer to complete an infinite number of tasks -- an impossibility.

        In fact, Homer cannot even begin to walk - since any possible (finite) first distance could be divided
        in half, and hence would not be first after all. Hence travel over any finite distance can neither be
        completed nor begun.
      \end{mdframed}
    \part Prove that $ \od{}{x} \sin x + \cos x = \cos x - \sin x $. You may assume the following limits:
          \begin{displaymath}
            \lim_{x \to 0} \frac{\sin x}{x} = 1 \hspace{0.5in} \lim_{x \to 0} \frac{\cos x - 1}{x} = 0
          \end{displaymath}
    \part Evaluate $ \lim_{x \to -\infty} \frac{e^{2x}+e^x + 4}{7e^{2x} + e^x - 4} $.
  \end{parts}

\titledquestion{Trig Integrals}
  \begin{parts}
    \part[3] Evaluate $ \int \cos^3 x \sin x \dif{x} $.
    \part[3] Differentiate $ 2\theta \sin \theta $, and hence find $ \int \cos \sqrt{x} \dif{x} $.
    \part[4] Prove, by integration, that the area of a circle of radius $ r $ is $ A = \pi r^2 $. You
          may find the substitution $ x = r \sin \theta $ useful.
    \part A useful result is that $ \od{}{x} \arcsin x = \frac{1}{\sqrt{1-x^2}} $. (Cam03)
      \begin{subparts}
        \subpart Given that $ x + a > 0 $ and $ x + b > 0 $, and that $ b > a $, show that
          \begin{displaymath}
            \od{}{x} \arcsin \left( \frac{x+a}{x+b} \right) = \frac{\sqrt{b-a}}{(x+b)\sqrt{a+b+2x}}
          \end{displaymath}
        \subpart Hence, or otherwise, find the following integral (for $ x > -1 $):
          $$ \int \frac{1}{(x+3) \sqrt{x+1}} \dif{x} $$
      \end{subparts}
  \end{parts}

\titledquestion{Functions}
  \begin{parts}
    \part For what values of $ k > 0 $ does the function $ f $ defined by
      \begin{displaymath}
        f(x) = \frac{\ln x}{k} - \frac{kx}{x+1}
      \end{displaymath}
      have local extrema? For each such $ k $ locate and classify the extrema, and
      explain the reasons for your conclusions carefully.
    \part[10] Prove (without explicit calculation) that $ e^\pi > \pi^e $.

          \textit{Hint: Begin by taking the natural log of both sides, and try to define
          a suitable function that has the essential properties that yield the inequality.}
    \part[4] Let $ f(x) = x^5 + 3x^3 + x - 10 $. Find $ a \in \mathbb{Z} $ such that
          $ \od{}{x} f^{-1}(48) = \frac{1}{a} $.
  \end{parts}

\titledquestion{Accounting}
A company produces three items, A, B, and C. The company has three factories, each
of which produces the three items in the quantities per hour indicated in the following table:
\begin{center}
  \renewcommand{\arraystretch}{1.5}
  \begin{tabular}{  c | c | c | c | c | }
    \multicolumn{1}{r}{} & \multicolumn{1}{r}{} & \multicolumn{3}{c}{Plant} \\ \cline{3-5}
    \multicolumn{1}{r}{} & & \textit{I} & \textit{II} & \textit{III} \\ \cline{2-5}
    \multirow{3}{*}{Item} & \textit{A} & 1 & 2 & 3 \\ \cline{2-5}
    & \textit{B} & 2 & 1 & 4 \\ \cline{2-5}
    & \textit{C} & 3 & 1 & 1 \\ \cline{2-5}
  \end{tabular}
\end{center}
It costs \SI{1000}[\$] per hour to operate plant I, \SI{400}[\$] per hour
to operate plant II, and \SI{2400}[\$] per hour to operate plant III.
\begin{parts}
  \part An order is placed with the company for three units of item A, five of item B,
        and six of item C. Determine the number of hours each plant should be operated
        to produce at least the required number of items for the order at minimum cost.
  \part The accounting department wishes to assign values to each item produced as a
        measure of their respective contributions to company profits. If we let $ x_1 $
        be the value per unit of item A, $ x_2 $ the value per unit of B, and $ x_3 $ the
        value per unit of C, then what we are trying to do is determine values of $ x_1 $,
        $ x_2 $, $ x_3 $ that will maximize $ R = 3x_1 + 5x_2 + 6x_3 $ (recalling that the
        order was three units of A, two of B, and six of C).
    \begin{subparts}
      \subpart Set up and solve the linear programming problem to maximize $ R $.
      \subpart If we allow our cost of operating plant II to increase to \SI{1000}[\$]
               per hour, what effect will this have on $ R $?
    \end{subparts}

\end{parts}

\end{questions}

\vfill
\noindent
Some problems taken, either directly or with modification, from:
Stewart's \emph{Calculus}~(7th~Ed); University of Cambridge entrance examinations; Erdman's
\emph{Exercises and Problems in Calculus}; Rammaha's \emph{Challenging Problems for Calculus Students};
Whipkey's \emph{The Power of Mathematics}.

\end{document}
