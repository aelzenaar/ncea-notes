\documentclass[answers]{exam}

\usepackage[dvipsnames]{xcolor}
\usepackage{amsmath}
\usepackage{amsfonts}
\usepackage{amsthm}
\usepackage{microtype}
\usepackage{siunitx}
\DeclareSIUnit\year{yr}
\usepackage{pgfplots}
\usepackage{graphicx}
\usepackage{sidecap}
\sidecaptionvpos{figure}{c}
\usepackage{float}
\usepackage{gensymb}
\usepackage{tkz-euclide}
\usetkzobj{all}
\usepackage{commath}

\newtheorem*{thm}{Theorem}

% russian integral
\usepackage{scalerel}
\DeclareMathOperator*{\rint}{\scalerel*{\rotatebox{17}{$\!\int\!$}}{\int}}

% \qformat{Question \thequestion: \thequestiontitle\hfill}

\begin{document}

\section*{NCEA Level 2 Physics (waves)\\Assignment W1: Waves in one dimension}

\begin{questions}
  \question A tuba is heard to produce a sound with frequency \SI{32}{\hertz}. Given that
            the speed of sound in air is \SI{330}{\metre\per\second}, draw a diagram showing
            the wave frozen in time as the sound travels away from the tuba. Label any important
            quantities (like the wavelength) on your diagram.
  \question Suppose a string has a density $ \mu $ (measured in kg/m), and is under a tension $ T $
            (measured in newtons). It can be shown that any wave travelling along the string
            will move at the same speed, $ v = \sqrt{T/\mu} $
    \begin{parts}
      \part Show that the units of $ v $ are indeed metres per second. (Hint: write newtons in terms of
            kilograms, seconds, and metres.)
      \part Suppose one particular string has density $ \mu = \SI{0.010}{\kilo\gram\per\metre} $
            and is under a tension of $ T = \SI{4.0}{\newton} $; so the speed of waves in the string
            is $ \sqrt{4/0.01} = \SI{20}{\metre\per\second} $. Suppose the amplitude of these waves
            is \SI{1.0}{\centi\metre}.
    \end{parts}
\end{questions}

\end{document}
