\documentclass[answers]{exam}

\usepackage[dvipsnames]{xcolor}
\usepackage{amsmath}
\usepackage{amsfonts}
\usepackage{amsthm}
\usepackage{microtype}
\usepackage{siunitx}
\DeclareSIUnit\year{yr}
\usepackage{pgfplots}
\usepackage{graphicx}
\usepackage{sidecap}
\sidecaptionvpos{figure}{c}
\usepackage{float}
\usepackage{gensymb}
\usepackage{tkz-euclide}
\usetkzobj{all}
\usepackage{commath}

\newtheorem*{thm}{Theorem}

% russian integral
\usepackage{scalerel}
\DeclareMathOperator*{\rint}{\scalerel*{\rotatebox{17}{$\!\int\!$}}{\int}}

% \qformat{Question \thequestion: \thequestiontitle\hfill}

\begin{document}

\section*{NCEA Level 2 Physics, assignment on kinematics and momentum}

The acceleration due to gravity on the moon is \SI{1.62}{\metre\per\second\squared}.

Suppose I have two identical cannons, one on the moon and one on the earth, both orientated to shoot at an angle \ang{30}
to the horizontal. The barrel of each cannon imparts an impulse of \SI{60}{\newton\second}
on the \SI{1}{\kilo\gram} cannonball, over a time period of \SI{0.01}{\second}.

\begin{questions}
  \question What is the average force imparted on the cannonball by the firing mechanism while it is in the barrel?
  \question If the acceleration of the cannonball is constant while it is in the barrel,
            calculate this acceleration; use this acceleration to find the length of the barrel.
  \question Now assume that the cannonball has left the cannon, at a height of \SI{1}{\metre}
            above the (flat) ground.
    \begin{parts}
      \part Explain why we can model the cannonball as a projectile. Draw a force diagram in your answer.
      \part Calculate the following, for both the earth and the moon:
        \begin{itemize}
          \item The maximum height reached by the cannonball.
          \item The time taken for the cannonball to hit the ground.
          \item The distance travelled, measured along the ground, by the cannonball.
        \end{itemize}
      \part Compare and contrast the paths travelled by the cannonballs on the earth and the moon. Include
            a diagram of each path.
    \end{parts}
  \question Suppose now I place one of my cannons in space, so no external forces are acting on it; assume
            that I have parked my space ship next to the cannon so that it is stationary with respect to me.

            Again, the cannon fires an \SI{1}{\kilo\gram} cannonball, imparting an impulse of \SI{60}{\newton\second}
            for \SI{0.01}{\second}.
    \begin{parts}
      \part What average force is imparted on the \emph{cannon} during the firing? Justify your answer.
      \part If the cannon has a mass of \SI{0.5}{\tonne} (\SI{500}{\kilo\gram}), with what speed will
            it now be flying in the opposite direction to the cannon ball?
    \end{parts}
\end{questions}

\subsection*{Guidelines for writing physics in general}
\begin{itemize}
  \item Use full sentences, note any assumptions you make, and write a couple of words to justify each step.
  \item Feel free to draw diagrams or pictures, even if nothing explicitly asks you to.
\end{itemize}

\end{document}
