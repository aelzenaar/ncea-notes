\documentclass[answers]{exam}

\usepackage[dvipsnames]{xcolor}
\usepackage{amsmath}
\usepackage{amsfonts}
\usepackage{amsthm}
\usepackage{microtype}
\usepackage{siunitx}
\DeclareSIUnit\year{yr}
\usepackage{pgfplots}
\usepackage{graphicx}
\usepackage{sidecap}
\sidecaptionvpos{figure}{c}
\usepackage{float}
\usepackage{gensymb}
\usepackage{tkz-euclide}
\usetkzobj{all}
\usepackage{commath}

\newtheorem*{thm}{Theorem}

% russian integral
\usepackage{scalerel}
\DeclareMathOperator*{\rint}{\scalerel*{\rotatebox{17}{$\!\int\!$}}{\int}}

% \qformat{Question \thequestion: \thequestiontitle\hfill}

\begin{document}

\section*{NCEA Level 2 Physics (waves)\\Assignment W3: Optics}

\begin{questions}
\end{questions}

\clearpage
\subsection*{Some thoughts}
In this part of the standard, there are two main topics:
\begin{itemize}
  \item Behaviour of light with lenses and mirrors.
  \item Behaviour of light as it enters and leaves media.
\end{itemize}

The important ideas and mathematical relationships are:
\begin{itemize}
  \item Light always travels in a straight line.
  \item The angle of incidence is equal to the angle of reflection.
  \item Concave mirrors bend light towards the focus, and convex mirrors bend light away from the focus.
  \item The image of an object can be found by simply drawing light rays.
  \item The focal length $ f = \frac{r}{2} $ for all mirrors.
  \item The ratios that relate distances and heights of images to those of the reflected objects are simply $ \frac{H_i}{H_o} = \frac{D_i}{D_o} $.
  \item The focal length is also related to the apparent distance of the reflected image by $ \frac{1}{f} = \frac{1}{D_i} + \frac{1}{D_o} $.
  \item For a VIRTUAL IMAGE, $ D_i $ and $ H_i $ are negative. For a convex mirror, $ f $ is negative.
  \item The refractive index of a medium is simply $ n = \frac{c}{v} $, where $ c $ is the speed of light in a vacuum and $ v $ is the speed
        of light in the medium.
  \item If a light ray travels from one medium to another, we have
        that $ \frac{\lambda_1}{\lambda_2} = \frac{v_1}{v_2} = \frac{\sin \theta_1}{\sin \theta_2} = \frac{n_2}{n_1} $.
  \item Snell's Law is simply a restatement of the above: $ n_1 \sin \theta_1 = n_2 \sin\theta_2 $.
  \item When a light ray moves from one medium to another, some light is always reflected back into the originating
        medium.
  \item When $ \theta_2 = \SI{90}{\degree} $, total internal reflection occurs and no light is refracted.
\end{itemize}

\end{document}
