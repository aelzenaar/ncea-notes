\documentclass[answers]{exam}

\usepackage[dvipsnames]{xcolor}
\usepackage{amsmath}
\usepackage{amsfonts}
\usepackage{amsthm}
\usepackage{microtype}
\usepackage{siunitx}
\DeclareSIUnit\year{yr}
\usepackage{pgfplots}
\usepackage{graphicx}
\usepackage{sidecap}
\sidecaptionvpos{figure}{c}
\usepackage{float}
\usepackage{gensymb}
\usepackage{tkz-euclide}
\usetkzobj{all}
\usepackage{commath}

\newtheorem*{thm}{Theorem}

\renewcommand*{\thefootnote}{\fnsymbol{footnote}}

% russian integral
\usepackage{scalerel}
\DeclareMathOperator*{\rint}{\scalerel*{\rotatebox{17}{$\!\int\!$}}{\int}}

% \qformat{Question \thequestion: \thequestiontitle\hfill}

\begin{document}

\section*{NCEA Level 2 Physics\\Assignment M3: Rotation}

\begin{questions}
  \question I have a rigid vertical rod of length \SI{0.5}{\metre}, pivoted at one end so that
            the other end is free to swing in a circle. I put a mass of \SI{300}{\gram}
            at the free end. This is visualised in the following diagram, where gravity is pointing
            directly down the page.

            \includegraphics[width=0.4\textwidth]{pendulum}
            \vspace*{3em}
    \begin{parts}
      \part Suppose that at the instant of time depicted, the pendulum is swinging down. On the
            diagram above, draw in both the forces acting on the mass, and the net force acting
            on the mass (i.e. their vector sum).
      \part The pendulum is pulled up to a height of \SI{0.5}{\metre} (i.e. the pendulum rod is
            horizontal) and then the mass is released. Assuming there is no energy loss due to friction,
            calculate:
        \begin{subparts}
          \item The speed the mass will be travelling around the circle when it reaches the bottom.
          \item The tension force in the rod as the mass reaches the bottom. (What is providing the tension force?)
          \item The maximum height the mass will reach on the other side of its swing.
        \end{subparts}
    \end{parts}
  \question Earth is at a distance \SI{1.496e11}{\metre} from the sun. It takes one year for the
            earth to complete a single orbit. We will use this information (and only this information!) to calculate the mass of the sun.
    \begin{parts}
      \part A year is 365.25 days; how many seconds is this? (Just work it out to 2 or 3 significant figures.)
      \part Assuming the orbit of the earth is circular, a top-down view looks like this:

            \includegraphics[width=0.3\textwidth]{earth}
        \begin{subparts}
          \item Calculate the circumference of the orbit of the earth.
          \item From (i) we know how far the earth travels each year; from (a) we know how long it takes
                to travel that distance. What is the average speed of the earth around its orbit?
          \item Hence calculate the centripetal acceleration $ a_{\text{centripetal}} $ felt by the earth in orbit.
        \end{subparts}
      \part The centripetal force on the earth is just the gravitational force exerted on the earth by the sun.
            The law of Newtonian gravity tells us that if two objects of mass $ m $ and $ M $ are a distance $ r $
            apart, then the gravitational force between them has magnitude
            \begin{displaymath}
              F_\text{grav} = \frac{GMm}{r^2}
            \end{displaymath}
            where $ G \approx \SI{6.67e-11}{\metre\cubed\per\kilogram\per\second\squared} $ is some constant that is the same everywhere
            in the universe. Let $ m $ be the mass of the earth, and $ M $ be the mass of the sun. We have
            \begin{displaymath}
              F_{\text{grav}} = F_{\text{centripetal}} \implies \frac{GMm}{r^2} = m a_\text{centripetal}.
            \end{displaymath}
            Note that the mass $ m $ of the earth cancels from both sides. Using your value of $ a_\text{centripetal} $ from (b)
            above, substitute everything in and work out the value of $ M $.\footnote{You should get something like \SI{1.99e30}{\kilo\gram}.}
    \end{parts}
  \question A child (at school on Earth, so they feel gravity) swings a ball on a string in a horizontal circle around their head, in
            such a way that the speed of the ball around the circle is constant. Nothing else is touching the ball.
    \begin{parts}
      \part State, with \emph{detailed} reasoning, whether the following statement is true or false.

            \emph{The ball is moving at a constant speed, so it is not accelerating.}

      \part Explain why the string can \emph{never} be horizontal in this situation, no matter how fast the ball is swung or how tightly
            the child is gripping the string. (You may find it useful to draw the forces acting on the ball.)
    \end{parts}
  \question Consider the pictured scenario:- an \SI{80.0}{\kilo\gram} construction worker sits down \SI{2.0}{\metre} from the
            end of a \SI{6.0}{\metre}, \SI{1450}{\kilo\gram} steel beam to eat his lunch. (Hint: assume the centre of mass
            of the beam is halfway along.)

            \includegraphics[width=0.3\textwidth]{construction}

    \begin{parts}
      \part Show that the total torque on the beam about the fixed end is \SI{45813}{\newton\metre}. (Note that we are giving more
            significant figures than we should justifiably use; this is only an intermediate calculation, and we will round at the end.)
      \part What is the tension in the cable, given that the beam is stationary?
      \part Ensure you round your answer to (ii) to the correct number of significant figures; briefly justify the number of significant
            figures you chose.
      \part If the maximum tension in the cable is \SI{17000}{\newton\metre}, what is the greatest mass that could
            be placed at the position of the worker before the cable breaks? (Assume the worker is no longer on the
            beam, so their mass does not contribute and they do not fall to their demise.)
    \end{parts}
\end{questions}

\vspace*{\fill}
This version: \today

\end{document}
