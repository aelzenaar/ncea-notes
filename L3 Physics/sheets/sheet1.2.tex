\documentclass[a4paper]{exam}

\usepackage{siunitx}
\DeclareSIUnit{\revolution}{rev}
\DeclareSIUnit{\rpm}{\revolution\per\minute}
\DeclareSIUnit{\lightyear}{ly}

\begin{document}
  \section*{L3 Physics: Questions for 1.2 (Mechanics, rotation)}
  Majority taken from Knight chap. 8, 12.
  \begin{questions}
    \question A \SI{1350}{\kilo\gram} car drives around a flat \SI{150}{\metre} diameter circular track at \SI{20}{\metre\per\second}.
              What is the magnitude and direction of the net force on the car? What causes this force?
    \question A car drives over the top of a circular-section hill with a radius of \SI{50}{\metre}. Draw a force diagram, and calculate
              the maximum speed the car can drive without taking flight.
    \question A new car is tested on a \SI{200}{\metre} diameter circular track. If the car speeds up at a steady \SI{1.5}{\metre\per\second},
              how long after starting is its centripetal acceleration equal to the tangential acceleration?
    \question A \SI{300}{\gram} ball and a \SI{600}{\gram} ball are connected by a \SI{40}{\centi\metre} rigid massless rod. The structure rotates
              about its centre of mass at \SI{100}{\rpm}. What is its rotational kinetic energy?
    \question A ball of radius $ R $ is placed at a height of \SI{30}{\centi\metre} on a \SI{15}{\degree} slope. It is released and rolls
              without slipping to the bottom. From what height should a circular hoop of radius $ R $ be released in order to equal the
              ball's speed at the bottom?
    \question A steel beam is \SI{5}{\metre} long and has a mass of \SI{400}{\kilo\gram}. It extends horizontally
              from the point to which it has been bolted to the framework of a building. A construction worker of mass \SI{100}{\kg}, unworried
              about health and safety regulations, stands on the end of the beam. What is the total torque about the bolt due to the worker
              and the beam?
    \question A \SI{2.0}{\kilo\gram}, \SI{30}{\centi\metre} diameter disc is spinning at \SI{300}{\rpm}. How much friction force must a brake
              apply to the rim to bring the disc to a halt in \SI{3.0}{\second}?
    \question A \SI{200}{\gram} toy car is placed on a narrow \SI{60}{\centi\metre} diameter track (mass of track ring \SI{1.0}{\kilo\gram},
              massless interior) that is free to rotate on a frictionless vertical axis. When the car is started, it soon reaches a steady
              speed of \SI{0.75}{\metre\per\second} relative to the track. At this point, what is the track's angular velocity in rpm?
    \question A flywheel is a large, massive wheel used to store energy. They can be spun up slowly, then the wheel's energy can be released
              quickly to accomplish a task that demands high power. An industrial flywheel has diameter \SI{1.5}{\metre} and mass \SI{250}{\kilo\gram}.
              It has maximum angular velocity \SI{1200}{\rpm}.
      \begin{parts}
        \part A motor spins up the flywheel with a constant torque of \SI{50}{\newton\metre}. How long does it take for the wheel to reach maximum speed?
        \part How much energy is stored in the flywheel?
        \part The flywheel is disconnected from the motor and connected to a machine to which it will deliver energy. Half the energy stored in the flywheel
              is delivered in \SI{2.0}{\second}. What is the average power delivered to the machine?
        \part How much torque does the flywheel exert on the machine?
      \end{parts}
    \question A piece of modern sculpture consists of an \SI{8.0}{\metre} long, \SI{150}{\kilo\gram} stainless steel bar passing diametrically through
              a \SI{50}{\kilo\gram} copper sphere. The centre of the sphere is \SI{2.0}{\metre} from the end of the bar. To be mounted for display, the
              bar is oriented vertically with the copper sphere at the lower end, then tilted \ang{35} from the vertical and held in plce by one horizontal
              steel cable attached to the bar \SI{2.0}{\metre} from the top end. What is the tension in the cable?
    \question {[Difficult]} If a vertical cylinder of some liquid rotates about its axis, the surface forms a smooth curve. Assuming that the water
              moves as a unit (i.e. all the water moves with the same angular velocity), show that the shape of the surface is a parabola described
              by the equation
              \begin{equation}
                z = \frac{\omega^2}{2g} r^2.
              \end{equation}
              [Hint: Each particle of water on the surface is subject to only two forces: gravity, and the normal force due to the water underneath
              it. The normal force, as always, acts perpendicular to the surface.]
  \end{questions}

\vspace*{\fill}
This version: \today

\end{document}
